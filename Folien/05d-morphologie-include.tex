%%%%%%%%%%%%%%%%%%%%%%%%%%%%%%%%%%%%%%%%%%%%%%%%%%%%
%%%             Metadata                         %%%
%%%%%%%%%%%%%%%%%%%%%%%%%%%%%%%%%%%%%%%%%%%%%%%%%%%%      

\title{Grundkurs Linguistik}

\subtitle{Morphologie IV: Typologie}

\author[aMyP]{
	{\small Antonio Machicao y Priemer}
%	\\
%	{\footnotesize \url{http://www.linguistik.hu-berlin.de/staff/amyp}\\
%	\href{mailto:mapriema@hu-berlin.de}{mapriema@hu-berlin.de}}
}

\institute{Institut für deutsche Sprache und Linguistik}

%%%%%%%%%%%%%%%%%%%%%%%%%      
\date{ }
%\publishers{\textbf{6. linguistischer Methodenworkshop \\ Humboldt-Universität zu Berlin}}

%\hyphenation{nobreak}


%%%%%%%%%%%%%%%%%%%%%%%%%%%%%%%%%%%%%%%%%%%%%%%%%%%%
%%%             Preamble's End                   %%%
%%%%%%%%%%%%%%%%%%%%%%%%%%%%%%%%%%%%%%%%%%%%%%%%%%%%      


%%%%%%%%%%%%%%%%%%%%%%%%%      
\huberlintitlepage
\iftoggle{toc}{
\frame{
\begin{multicols}{2}
	\frametitle{Inhaltsverzeichnis}\tableofcontents
	%[pausesections]
\end{multicols}
	}
	}


%%%%%%%%%%%%%%%%%%%%%%%%%%%%%%%%%%%
%%%%%%%%%%%%%%%%%%%%%%%%%%%%%%%%%%

%\nocite


%%%%%%%%%%%%%%%%%%%%%%%%%%%%%%%%%%
%%%%%%%%%%%%%%%%%%%%%%%%%%%%%%%%%%
\section{Einführung}
%\frame{
%\frametitle{~}
%	\tableofcontents[currentsection]
%}


%%%%%%%%%%%%%%%%%%%%%%%%%%%%%%%%%%
\begin{frame}
\frametitle{Einführung}

\begin{itemize}
	\item Unterscheidung von Sprachtypen nach der Art der Realisierung der Flexion (welche Flexionskategorien und wie werden angezeigt):
	
	\begin{itemize}
		\item analytische:
		
		\begin{itemize}
			\item \textbf{isolierend}: jedes morphologische Merkmal wird durch ein separates freies Morphem realisiert
		\end{itemize}
		
		\item[]
		\item synthetische:
		
		\begin{itemize}
			\item \textbf{agglutinierend}
			\item \textbf{fusionierend} (flektierend)
			\item \textbf{polysynthetisch} (inkorporierend)
		\end{itemize}
		
		\item[]
		\item Mischformen (die meisten Sprachen)
	\end{itemize}
\end{itemize}


\end{frame}


%%%%%%%%%%%%%%%%%%%%%%%%%%%%%%%%%%
%%%%%%%%%%%%%%%%%%%%%%%%%%%%%%%%%%
\section{Isolierende Sprachen}
%\frame{
%\frametitle{~}
%	\tableofcontents[currentsection]
%}


%%%%%%%%%%%%%%%%%%%%%%%%%%%%%%%%%%
\begin{frame}
\frametitle{Isolierende Sprachen}

\begin{itemize}
	\item Grammatische Beziehungen zwischen Wörtern im Satz durch selbständige, \textbf{syntaktische Formenelemente} realisiert
	\item[] \ras keine gebundenen Morpheme
	\item Vietnamesisch, Chinesisch, westafrikanische Sprachen
	
	\item[] Vietnamesisch:
	
\ea
\gll \foreignlanguage{vietnamese}{khi} \foreignlanguage{vietnamese}{tôi} \foreignlanguage{vietnamese}{đến} \foreignlanguage{vietnamese}{nhà} \foreignlanguage{vietnamese}{bạn} \foreignlanguage{vietnamese}{tôi} \foreignlanguage{vietnamese}{chúng} \foreignlanguage{vietnamese}{tôi} \foreignlanguage{vietnamese}{bắt đấu} \foreignlanguage{vietnamese}{làm} \foreignlanguage{vietnamese}{bài} \\
als 1P komm Haus Freund 1P PL 1P anfangen tun Übung \\
\mytrans{Als ich zum Haus meines Freundes kam, begannen wir, Übungen zu machen.}
\z

\end{itemize}


\end{frame}

%%%%%%%%%%%%%%%%%%%%%%%%%%%%%%%%%%
\begin{frame}
\frametitle{Isolierende Sprachen}

\begin{itemize}
	\item Auch im Deutschen oder Englischen gibt es Formen der Isolation, etwa Auxiliare
	
	\begin{itemize}
		\item[]
		\item im Deutschen aber mit Flexion verbunden, im Englischen oft ohne Flexion
		
		\eal 
			\ex Ich werd-e gehen.
			\ex Wir werd-en gehen.
			\ex I/you/(s)he/we/they will go.
		\zl
		
	\end{itemize}
\end{itemize}


\end{frame}


%%%%%%%%%%%%%%%%%%%%%%%%%%%%%%%%%%
%%%%%%%%%%%%%%%%%%%%%%%%%%%%%%%%%%
\section{Agglutinierende Sprachen}
%\frame{
%\frametitle{~}
%	\tableofcontents[currentsection]
%}


%%%%%%%%%%%%%%%%%%%%%%%%%%%%%%%%%%
\begin{frame}
\frametitle{Agglutinierende Sprachen}

\begin{itemize}
	\item Grammatische und lexikalische Morpheme mit jeweils einfachen Bedeutungen werden \textbf{aneinandergereiht}
	\item[]
	\item \textbf{1:1-Zuordnung} von Morphem und Bedeutung/Funktion
	\item[]
	\item Resultat \ras hochkomplexe Wörter mit zahlreichen Morphemen
	\item[]
	\item Türkisch, Finnisch, Ungarisch, Bantu-Sprachen
	
	\ea
	\gll	çalış - tIr - Il - mA - mAlI - ymIş\\
			arbeit - Verursachung - Passiv - Negation - Obligation - Evidenz \\
			çalıştırılmamalıymış \\
			\mytrans{anscheinend sollte man ihn nicht zur Arbeit veranlassen.} \\
	\z
			
	
\end{itemize}


\end{frame}


%%%%%%%%%%%%%%%%%%%%%%%%%%%%%%%%%%
%%%%%%%%%%%%%%%%%%%%%%%%%%%%%%%%%%
\section{Fusionierende Sprachen}
%\frame{
%\frametitle{~}
%	\tableofcontents[currentsection]
%}


%%%%%%%%%%%%%%%%%%%%%%%%%%%%%%%%%%
\begin{frame}
\frametitle{Fusionierende Sprachen}

\begin{itemize}
	\item auch \textbf{flektierende} Sprachen genannt
	\item[]
	\item Die Morpheme oft \textbf{polysem} (ein Flexionsmorphem trägt verschiedene grammatische Informationen)
	\item[]
	\item Darüber hinaus kann ein Flexionsmorphem gleichlautend mit einem funktional anderen sein (\zB -en), d.\,h. es kommt zu \textbf{Allomorphie}.
	\item[]
	\item Bestimmte morphologische Prozesse werden mehrfach markiert (\zB bei der Pluralbildung: Affigierung plus Stammvokaländerung).
	\item[]
	\item Zu den flektierenden Sprachen gehören die indogermanischen Sprachen.
\end{itemize}


\end{frame}


%%%%%%%%%%%%%%%%%%%%%%%%%%%%%%%%%%
%%%%%%%%%%%%%%%%%%%%%%%%%%%%%%%%%%
\section{Polysynthetische Sprachen}
%\frame{
%\frametitle{~}
%	\tableofcontents[currentsection]
%}


%%%%%%%%%%%%%%%%%%%%%%%%%%%%%%%%%%
\begin{frame}
\frametitle{Polysynthetische Sprachen}

\begin{itemize}
	\item auch \textbf{inkorporierende} Sprachen genannt
	\item[]
	\item syntaktische Beziehungen im Satz durch \textbf{Aneinanderreihen} und \textbf{Ineinanderfügen} lexikalischer und grammatischer Morpheme realisiert
	\item[]
	\item \zB Subjekt- und Objektverhältnisse im Verb ausdrücken
	\item[]
	\item Inuit, Irokesisch, Maya-Sprachen, Nahuatl, Sprachen im Pazifik-Raum wie Samoanisch, Tonga, Maori
	
	\ea
	\glll	ni kin ita k \\
			{Subj-1.Ps-Sing} {Obj.-3.Ps-Plur} seh Prät \\
			{nikinitak (Aztekisch (Zacapoaxatla))} \\
	\mytrans{\gqq{ich sah sie (pl)}}
	
	
\end{itemize}


\end{frame}


%%%%%%%%%%%%%%%%%%%%%%%%%%%%%%%%%
\begin{frame}
\frametitle{Polysynthetische Sprachen}

\begin{itemize}
	\item Auch im Deutschen kann man bestimmte Konstruktionen als Inkorporationen analysieren
	
	\begin{itemize}
		\item die Kombination Verb + artikelloses Nomen
		\item das artikellose Nomen weist andere Eigenschaften auf als sein Gegenstück mit Artikel:
		
		\ea Andrea liest die Zeitung. \\
			 Andrea liest Zeitung.
		\z
			 
		\ea Andrea liest die informative Zeitung. \\
			 *Andrea liest informative Zeitung.
		\z
			 
		\ea Andrea liest eine Zeitung. Sie ist informativ. \\
			 Andrea liest Zeitung\textsubscript{i}. *Sie\textsubscript{i} ist informativ.
		\z
			 
	\end{itemize}
	\item Mit anderen Worten: \\
		Die meisten Sprachen sind Mischformen der vier Typen.
\end{itemize}


\end{frame}



