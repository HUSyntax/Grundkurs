%%%%%%%%%%%%%%%%%%%%%%%%%%%%%%%%%%%%%%%%%%%%%%
%% Compile: XeLaTeX BibTeX XeLaTeX XeLaTeX
%% Loesung-Handout: Antonio Machicao y Priemer
%% Course: GK Linguistik
%%%%%%%%%%%%%%%%%%%%%%%%%%%%%%%%%%%%%%%%%%%%%%

%\documentclass[a4paper,10pt, bibtotoc]{beamer}
\documentclass[10pt,handout]{beamer}

%%%%%%%%%%%%%%%%%%%%%%%%
%%     PACKAGES      
%%%%%%%%%%%%%%%%%%%%%%%%

%%%%%%%%%%%%%%%%%%%%%%%%
%%     PACKAGES       %%
%%%%%%%%%%%%%%%%%%%%%%%%



%\usepackage[utf8]{inputenc}
%\usepackage[vietnamese, english,ngerman]{babel}   % seems incompatible with german.sty
%\usepackage[T3,T1]{fontenc} breaks xelatex
\usepackage{lmodern}

\usepackage{amsmath}
\usepackage{amsfonts}
\usepackage{amssymb}
%% MnSymbol: Mathematische Klammern und Symbole (Inkompatibel mit ams-Packages!)
%% Bedeutungs- und Graphemklammern: $\lsem$ Tisch $\rsem$ $\langle TEXT \rangle$ $\llangle$ TEXT $\rrangle$ 
\usepackage{MnSymbol}
%% ulem: Strike out
\usepackage[normalem]{ulem}  

%% Special Spaces (s. Commands)
\usepackage{xspace}				
\usepackage{setspace}
%	\onehalfspacing

%% mdwlist: Special lists
\usepackage{mdwlist}	

\usepackage[noenc,safe]{tipa}

% maybe define \textipa to use \originalTeX to avoid problems with `"'.
%
%	\ex \textipa{\originalTeX [pa.pa."g\t{aI}]}

%

\usepackage{etex}		%For Forest bug

%
%\usepackage{jambox}
%


%\usepackage{forest-v105}
%\usepackage{modified-langsci-forest-setup}

\usepackage{xeCJK}
\setCJKmainfont{SimSun}


%\usepackage{natbib}
%\setcitestyle{notesep={:~}}




% for toggles
\usepackage{etex}



% Fraktur!
\usepackage{yfonts}

\usepackage{url}

% für UDOP
\usepackage{adjustbox}


%% huberlin: Style sheet
%\usepackage{huberlin}
\usepackage{hu-beamer-includes-pdflatex}
\huberlinlogon{0.86cm}


%% Last Packages
%\usepackage{hyperref}	%URLs
%\usepackage{gb4e}		%Linguistic examples

% sorry this was incompatible with gb4e and had to go.
%\usepackage{linguex-cgloss}	%Linguistic examples (patched version that works with jambox

\usepackage{multirow}  %Mehrere Zeilen in einer Tabelle
%\usepackage{array}
\usepackage{marginnote}	%Notizen




%%%%%%%%%%%%%%%%%%%%%%%%%%%%%%%%%%%%%%%%%%%%%%%%%%%%
%%%          Commands                            %%%
%%%%%%%%%%%%%%%%%%%%%%%%%%%%%%%%%%%%%%%%%%%%%%%%%%%%

%%%%%%%%%%%%%%%%%%%%%%%%%%%%%%%%
% German quotation marks:
\newcommand{\gqq}[1]{\glqq{}#1\grqq{}}		%double
\newcommand{\gq}[1]{\glq{}#1\grq{}}			%simple


%%%%%%%%%%%%%%%%%%%%%%%%%%%%%%%%
% Abbreviations in German
% package needed: xspace
% Short space in German abbreviations: \,	
\newcommand{\idR}{\mbox{i.\,d.\,R.}\xspace}
\newcommand{\su}{\mbox{s.\,u.}\xspace}
%\newcommand{\ua}{\mbox{u.\,a.}\xspace}       % in abbrev
%\newcommand{\zB}{\mbox{z.\,B.}\xspace}       % in abbrev
%\newcommand{\s}{s.~}
%not possibel: \dh --> d.\,h.


%%%%%%%%%%%%%%%%%%%%%%%%%%%%%%%%
%Abbreviations in English
\newcommand{\ao}{a.o.\ }	% among others
\newcommand{\cf}[1]{(cf.~#1)}	% confer = compare
\renewcommand{\ia}{i.a.}	% inter alia = among others
\newcommand{\ie}{i.e.~}	% id est = that is
\newcommand{\fe}{e.g.~}	% exempli gratia = for example
%not possible: \eg --> e.g.~
\newcommand{\vs}{vs.\ }	% versus
\newcommand{\wrt}{w.r.t.\ }	% with respect to


%%%%%%%%%%%%%%%%%%%%%%%%%%%%%%%%
% Dash:
\newcommand{\gs}[1]{--\,#1\,--}


%%%%%%%%%%%%%%%%%%%%%%%%%%%%%%%%
% Rightarrow with and without space
\def\ra{\ensuremath\rightarrow}			%without space
\def\ras{\ensuremath\rightarrow\ }		%with space


%%%%%%%%%%%%%%%%%%%%%%%%%%%%%%%%
%% X-bar notation

%% Notation with primes (not emphasized): \xbar{X}
\newcommand{\MyPxbar}[1]{#1$^{\prime}$}
\newcommand{\xxbar}[1]{#1$^{\prime\prime}$}
\newcommand{\xxxbar}[1]{#1$^{\prime\prime\prime}$}

%% Notation with primes (emphasized): \exbar{X}
\newcommand{\exbar}[1]{\emph{#1}$^{\prime}$}
\newcommand{\exxbar}[1]{\emph{#1}$^{\prime\prime}$}
\newcommand{\exxxbar}[1]{\emph{#1}$^{\prime\prime\prime}$}

% Notation with zero and max (not emphasized): \xbar{X}
\newcommand{\zerobar}[1]{#1$^{0}$}
\newcommand{\maxbar}[1]{#1$^{\textsc{max}}$}

% Notation with zero and max (emphasized): \xbar{X}
\newcommand{\ezerobar}[1]{\emph{#1}$^{0}$}
\newcommand{\emaxbar}[1]{\emph{#1}$^{\textsc{max}}$}

%% Notation with bars (already implemented in gb4e):
% \obar{X}, \ibar{X}, \iibar{X}, \mbar{X} %Problems with \mbar!
%
%% Without gb4e:
\newcommand{\overbar}[1]{\mkern 1.5mu\overline{\mkern-1.5mu#1\mkern-1.5mu}\mkern 1.5mu}
%
%% OR:
\newcommand{\MyPibar}[1]{$\overline{\textrm{#1}}$}
\newcommand{\MyPiibar}[1]{$\overline{\overline{\textrm{#1}}}$}
%% (emphasized):
\newcommand{\eibar}[1]{$\overline{#1}$}
\newcommand{\eiibar}[1]{\overline{$\overline{#1}}$}

%%%%%%%%%%%%%%%%%%%%%%%%%%%%%%%%
%% Subscript & Superscript: no italics
\newcommand{\MyPdown}[1]{$_{\textrm{#1}}$}
\newcommand{\MyPup}[1]{$^{\textrm{#1}}$}


%%%%%%%%%%%%%%%%%%%%%%%%%%%%%%%%
% Objekt language marking:
%\newcommand{\obj}[1]{\glqq{}#1\grqq{}}	%German double quotes
%\newcommand{\obj}[1]{``#1''}			%English double quotes
\newcommand{\MyPobj}[1]{\emph{#1}}		%Emphasising


%%%%%%%%%%%%%%%%%%%%%%%%%%%%%%%%
%% Semantic types (<e,t>), features, variables and graphemes in angled brackets 

%%% types and variables, in math mode: angled brackets + italics + no space
%\newcommand{\type}[1]{$<#1>$}

%%% OR more correctly: 
%%% types and variables, in math mode: chevrons! + italics + no space
\newcommand{\MyPtype}[1]{$\langle #1 \rangle$}

%%% features and graphemes, in math mode: chevrons! + italics + no space
\newcommand{\abe}[1]{$\langle #1 \rangle$}


%%% features and graphemes, in math mode: chevrons! + no italics + space
\newcommand{\ab}[1]{$\langle$#1$\rangle$}  %%same as \abu  
\newcommand{\abu}[1]{$\langle$#1$\rangle$} %%Umlaute

%%% Notizen
\renewcommand{\marginfont}{\singlespacing}
\renewcommand{\marginfont}{\footnotesize}
\renewcommand{\marginfont}{\color{black}}

\newcommand{\myp}[1]{%
	\marginnote{%
		\begin{spacing}{1}
			\vspace{-\baselineskip}%
			\color{red}\footnotesize#1
		\end{spacing}
	}
}
%%%%%%%%%%%%%%%%%%%%%%%%%%%%%%%%
%% Outputbox
\newcommand{\outputbox}[1]{\noindent\fbox{\parbox[t][][t]{0.98\linewidth}{#1}}\vspace{0.5em}}

%%%%%%%%%%%%%%%%%%%%%%%%%%%%%%%%
%% (Syntactic) Trees
% package needed: forest
%
%% Setting for simple trees
\forestset{
	MyP edges/.style={for tree={parent anchor=south, child anchor=north}}
}

%% this is taken from langsci-setup file
%% Setting for complex trees
%% \forestset{
%% 	sn edges/.style={for tree={parent anchor=south, child anchor=north,align=center}}, 
%% background tree/.style={for tree={text opacity=0.2,draw opacity=0.2,edge={draw opacity=0.2}}}
%% }

\newcommand\HideWd[1]{%
	\makebox[0pt]{#1}%
}


%%%%%%%%%%%%%%%%%%%%%%%%%%%%%%%%%%%%%%%%%%%%%%%%%%%%
%%%          Useful commands                     %%%
%%%%%%%%%%%%%%%%%%%%%%%%%%%%%%%%%%%%%%%%%%%%%%%%%%%%

%%%%%%%%%%%%%%%%%%%%%
%% FOR ITEMS:
%\begin{itemize}
%  \item<2-> from point 2
%  \item<3-> from point 3 
%  \item<4-> from point 4 
%\end{itemize}
%
% or: \onslide<2->
% or: \pause

%%%%%%%%%%%%%%%%%%%%%
%% VERTICAL SPACE:
% \vspace{.5cm}
% \vfill

%%%%%%%%%%%%%%%%%%%%%
% RED MARKING OF TEXT:
%\alert{bis spätestens Mittwoch, 18 Uhr}

%%%%%%%%%%%%%%%%%%%%%
%% RESCALE BIG TABLES:
%\scalebox{0.8}{
%For Big Tables
%}

%%%%%%%%%%%%%%%%%%%%%
%% BLOCKS:
%\begin{alertblock}{Title}
%Text
%\end{alertblock}
%
%\begin{block}{Title}
%Text
%\end{block}
%
%\begin{exampleblock}{Title}
%Text
%\end{exampleblock}


\newtoggle{uebung}
\newtoggle{loesung}
\newtoggle{toc}

% The toc is not needed on Handouts. Safe trees.
\mode<handout>{
\togglefalse{toc}
}

\newtoggle{hpsgvorlesung}\togglefalse{hpsgvorlesung}
\newtoggle{syntaxvorlesungen}\togglefalse{syntaxvorlesungen}

%\includecomment{psgbegriffe}
%\excludecomment{konstituentenprobleme}
%\includecomment{konstituentenprobleme-hinweis}

\newtoggle{konstituentenprobleme}\togglefalse{konstituentenprobleme}
\newtoggle{konstituentenprobleme-hinweis}\toggletrue{konstituentenprobleme-hinweis}

%\includecomment{einfsprachwiss-include}
%\excludecomment{einfsprachwiss-exclude}
\newtoggle{einfsprachwiss-include}\toggletrue{einfsprachwiss-include}
\newtoggle{einfsprachwiss-exclude}\togglefalse{einfsprachwiss-exclude}

\newtoggle{psgbegriffe}\toggletrue{psgbegriffe}

\newtoggle{gb-intro}\togglefalse{gb-intro}



%%%%%%%%%%%%%%%%%%%%%%%%%%%%%%%%%%%%%%%%%%%%%%%%%%%%
%%%             Preamble's End                   
%%%%%%%%%%%%%%%%%%%%%%%%%%%%%%%%%%%%%%%%%%%%%%%%%%%% 

\begin{document}
	
	
%%%% ue-loesung
%%%% true: Übung & Lösungen (slides) / false: nur Übung (handout)
%	\toggletrue{ue-loesung}

%%%% ha-loesung
%%%% true: Hausaufgabe & Lösungen (slides) / false: nur Hausaufgabe (handout)
%	\toggletrue{ha-loesung}

%%%% toc
%%%% true: TOC am Anfang von Slides / false: keine TOC am Anfang von Slides
\toggletrue{toc}

%%%% sectoc
%%%% true: TOC für Sections / false: keine TOC für Sections (StM handout)
%	\toggletrue{sectoc}

%%%% gliederung
%%%% true: Gliederung für Sections / false: keine Gliederung für Sections
%	\toggletrue{gliederung}
	
	
%%%%%%%%%%%%%%%%%%%%%%%%%%%%%%%%%%%%%%%%%%%%%%%%%%%%
%%%             Metadata                         
%%%%%%%%%%%%%%%%%%%%%%%%%%%%%%%%%%%%%%%%%%%%%%%%%%%%      

\title{Grundkurs Linguistik}

\subtitle{Lösungen -- Morphologie}

\author[A. Machicao y Priemer]{
	{\small Antonio Machicao y Priemer}
	\\
	{\footnotesize \url{http://www.linguistik.hu-berlin.de/staff/amyp}}
	%	\\
	%	\href{mailto:mapriema@hu-berlin.de}{mapriema@hu-berlin.de}}
}

\institute{Institut für deutsche Sprache und Linguistik}


% bitte lassen, sonst kann man nicht sehen, von wann die PDF-Datei ist.
%\date{ }

%\publishers{\textbf{6. linguistischer Methodenworkshop \\ Humboldt-Universität zu Berlin}}

%\hyphenation{nobreak}


%%%%%%%%%%%%%%%%%%%%%%%%%%%%%%%%%%%%%%%%%%%%%%%%%%%%
%%%             Preamble's End                  
%%%%%%%%%%%%%%%%%%%%%%%%%%%%%%%%%%%%%%%%%%%%%%%%%%%%      


%%%%%%%%%%%%%%%%%%%%%%%%%      
\huberlintitlepage[22pt]
\iftoggle{toc}{
	\frame{
		\begin{multicols}{2}
			\frametitle{Inhaltsverzeichnis}
			\tableofcontents
			%[pausesections]
			\columnbreak
% muss noch hinzugefügt werden, wenn in include auch \refs und \exrs angewendet sind
%				\textcolor{white}{
%					\ea \label{ex:05aHA1}
%					\ex \label{ex:05aHA5}
%					\z
%				}
		\end{multicols}
	}
}


%%%%%%%%%%%%%%%%%%%%%%%%%%%%%%%%%%%
%%%%%%%%%%%%%%%%%%%%%%%%%%%%%%%%%%%
\section{Übungen}

%%%%%%%%%%%%%%%%%%%%%%%%%%%%%%%%%%
%% UE 1 - 05b Morphologie Stefan
%%%%%%%%%%%%%%%%%%%%%%%%%%%%%%%%%%

\frame{
\frametitle{Lösung: Vorlesungsankündigung}

\centerline{
\begin{forest}
sm edges
[N
  [N
    [N 
      [V 
        [Part [vor]]
        [V [les]]]
      [N-Aff [ung-s]]]
    [N [V [Part [an]]
          [V [kündig]]]
       [N-Aff [ung]]]]
  [Flex [$\varnothing$]]]
\end{forest}
}

\emph{kündigen}: mhd. für `mitteilen, künden'

}


\frame{
\frametitle{Lösung: Straßenbahnhaltestelle}

\centerline{%
\begin{forest}
sm edges
[N 
  [N
    [N
      [N [Straße-n]]
      [N [bahn]]]
    [N [V [halt-e]]
       [N [stelle]]]]
  [Flex [$\varnothing$]]]
\end{forest}}

}

\frame{
\frametitle{Lösung: Kinderschlafsack}

\centerline{%
\begin{forest}
sm edges
[N
  [N [N [kind-er]]
     [N
       [V [schlaf]]
       [N [sack]]]]
  [Flex [$\varnothing$]]]
\end{forest}
}

}

\frame{
\frametitle{Lösung: Kinderschreibtische}

\centerline{%
\begin{forest}
sm edges
[N
  [N [N [kind-er]]
     [N
       [V [schreib]]
       [N [tisch]]]]
  [Flex [e]]]
\end{forest}
}

}




%%%%%%%%%%%%%%%%%%%%%%%%%%%%%%%%%%%
%%%%%%%%%%%%%%%%%%%%%%%%%%%%%%%%%%%
\section{Hausaufgaben}
	
%%%%%%%%%%%%%%%%%%%%%%%%%%%%%%%%%%
%% HA 1 - 05c Morphologie Stefan
%%%%%%%%%%%%%%%%%%%%%%%%%%%%%%%%%%


\begin{frame}
	\frametitle{Lösungen}

\begin{enumerate}
	\item Kreuzen Sie die korrekten Aussagen an: %\hfill(0,5 Punkte pro Aussage)\\
	
	\begin{itemize}
	\item[$\circ$] Die Graphemkette abarbeiten ist ein einzelnes phonologisches Wort im Deutschen.
	\item[$\circ$] \emph{Morphologieeinführungsbuch} ist ein orthographisch-graphemisches Wort des Deutschen, sowie \emph{introductory morphology book} ein orthographisch-graphemisches Wort des Englischen ist.
	\item[$\circ$] Ein Morphem ist die kleinste bedeutungsunterscheidende Einheit in einem bestimmten Sprachsystem.
	\item[\textcolor{red}{$\checkmark$}] \textcolor{red}{\ab{Brot} und \ab{Bröt} sind Allomorphe eines einzelnen Morphems.}
\end{itemize}
	
	\item Erklären Sie das Prinzip der Rechtsköpfigkeit in der Morphologie des Deutschen. Verwenden Sie bei Ihrer Erklärung die unten angegebenen Beispiele.%\hfill(4 Punkte)\\
	
	\settowidth\jamwidth{XXXXXXXXXXXXXXXXXXXXXXt}
		\eal
			\ex lichtblau, Blaulicht \loesung{2}{\ras Wortart}
			\ex die Fotowelt, das Weltfoto \loesung{3}{\ras Genus und Semantik}
			\ex der Bücherrücken/die Bücherrücken, das Rückenbuch/die Rückenbücher \loesung{4}{\ras Pluralflexion}
		\zl		
\end{enumerate}

\end{frame}

%%%%%%%%%%%%%%%%%%%%%%%%%%%%%%%%%%%%%%%%%%%%%%%%%%%%%%%%%%%%

\begin{frame}
\frametitle{Lösungen}
\begin{itemize}
	\item[3.] Geben Sie Argumente für oder gegen die Behandlung von \emph{ver-} in den folgenden Wörtern als Morphem an. Wenn es sich um ein Morphem handelt, ist das immer das gleiche Morphem? %(4 Punkte)
	
	\eal \label{ver}
	\ex\label{vera}  \emph{Ver}zweiflung
	\ex\label{verb} \emph{Ver}s
	\ex \label{verc} \emph{ver}kaufen
	\ex\label{verd}  \emph{ver}schreiben
	\ex\label{vere} \emph{ver}fahren
	\zl

	\begin{itemize}

		\item[] \alertred{Morphem: Kleinste bedeutungstragende Einheit im Sprachsystem.}

		\begin{itemize}{\color{red}
			\item[] \gqq{ver} in (\ref{verb}) ist kein Morphem, sondern Bestandteil des Stammes.
			\item[] \gqq{ver} in (\ref{ver} a,c,d,e) sind Morpheme, aber unterschiedliche Morpheme, weil sie unterschiedliche Bedeutungen tragen
			\item[] \gqq{ver} in (\ref{ver} d,e) trägt die Bedeutung \gq{X falsch machen} (d.h. \gq{falsch schreiben/fahren})
			\item[] \gqq{ver} in (\ref{verc}) kehrt die Bedeutung von X um (kaufen \ras verkaufen)
			\item[] \gqq{ver} in (\ref{vera}) trägt eine intensivierende(?) Bedeutung
		}
		\end{itemize}
	
	\end{itemize}


\end{itemize}
\end{frame}



%%%%%%%%%%%%%%%%%%%%%%%%%%%%%%%%%%%%%%%%%%%%%%%%%%%%%%%%%%%%

\begin{frame}
\frametitle{Lösungen}

\begin{itemize}
	\item[4.] Ordnen Sie die Wortbildungsprozesse links den passenden Beispielen rechts zu (dazu müssen Sie nur den entsprechenden Buchstaben neben das passende Beispiel schreiben). %(0,5 Punkte pro Aussage)
\end{itemize}

	\begin{table}[h!]
	\begin{minipage}{0.4\linewidth}
		\centering
		\begin{tabular}{|l|p{0.1\textwidth}|}
			\hline 
			Determinativkompositum & (A)\\
			\hline
			Konversion & (B)\\
			\hline
			Zirkumfigierung (Derivation) & (C)\\
			\hline
			Rektionskompositum & (D)\\
			\hline
			Possessivkompositum & (E)\\
			\hline 
		\end{tabular}
		
	\end{minipage}\hfill%
	\begin{minipage}{0.4\linewidth}
		\centering
		\begin{tabular}{|p{0.1\textwidth}|r|}
			\hline 
			\textcolor{red}{C} & \emph{Gerede} \\
			\hline
			\textcolor{red}{E} & \emph{Milchgesicht}\\
			\hline
			\textcolor{red}{B} & \emph{Lauf} \\
			\hline
			\textcolor{red}{A} & \emph{Kettenraucher}  \\
			\hline
			\textcolor{red}{D} & \emph{Klausurbesprechung}  \\
			\hline 
		\end{tabular}
	\end{minipage}
\end{table}
\end{frame}



%%%%%%%%%%%%%%%%%%%%%%%%%%%%%%%%%%%%%%%%%%%%%%%%%%%%%%%%%%%

\begin{frame}
\frametitle{Lösungen}
\begin{itemize}
	\item[5.] Warum sind die Wörter unter (\ref{kauf}) grammatisch und die unter (\ref{fenster}) ungrammatisch? %(4 Punkte)
	\eal
	\ex\label{kauf} kaufbar, trinkbar
	\ex\label{fenster} *fensterbar, *helfbar, *schönbar
	\zl
	
	\textcolor{red}{
		Das Suffix \gqq{-bar} hat die folgenden Beschränkungen bzgl. der Basis X, mit der es sich verbindet:
	}

		\begin{itemize}{\color{red}
			\item[] X muss ein Verb sein (nicht Nomen oder Adjektiv)
			\item[] X muss transitiv sein (nicht wie \gqq{helfen})
		}
		\end{itemize}

\end{itemize}

\end{frame}


\begin{frame}
\frametitle{Lösungen}
\begin{itemize}
	\item [6.] Sind die folgenden Verben Präfixverben oder Partikelverben? Begründen Sie Ihre Entscheidungen. %(3 Punkte)
	
	\eal
	\ex auskennen
	\ex erkennen
	\ex aberkennen
	\zl
	
	\textcolor{red}{
		Partikelverb: 1) morphologisch trennbar (\emph{aus-ge-kannt}, \emph{ab-zu-erkennen}), 2) syntaktisch trennbar (\gqq{Peter \emph{kennt} sich \emph{aus}}, \gqq{Die Frau \emph{erkennt} die Urkunde \emph{ab}}) und 3) die Partikel trägt die Hauptbetonung (\emph{AUSkennen} und \emph{ABerkennen}).
		Präfixverb: weder morphologisch noch syntaktisch trennbar (*\emph{ergekannt}, \gqq{*\emph{Peter kannte ihn er}}), Hauptbetonung liegt auf der Basis (\emph{erKENnen}).\\
		\bigskip
		EXTRA: \gqq{aberkennen} ist ein Partikelverb, welches aus einem Präfixverb und einer Partikel besteht (ab+erkennen).\\
	}
\end{itemize}

\end{frame}

%%%%%%%%%%%%%%%%%%%%%%%%%%%%%%%%%%%%%%%%%%%%%%%%%%%%%%%%%%%

\begin{frame}
	\frametitle{Lösungen}
\begin{itemize}
	\item [7.] Geben Sie für das folgende Wort eine morphologische Konstituentenstruktur (inklusive Konstituentenkategorien (N, N\textsuperscript{af}, V, V\textsuperscript{af}, \dots)) an, und bestimmen Sie für jeden Knoten den Wortbildungstyp. %(6,5 Punkte)
\ea
Wahlkampfberaterinnen
\z

\scalebox{.6}{
\textcolor{red}{
	%	\begin{figure}[h]
	\begin{forest} MyP edges,
		[N, name=N1
		[N, name=N2
		[N, name=N3
		[N, name=N4
		[N, name=N6 [V[wahl/wähl]]]
		[N, name=N7 [V[kampf/kämpf]]]]
		[N, name=N5[V, name=V1	[V\textsubscript{af}[be-]]
		[V[rat]]]
		[N\textsuperscript{af}[-er]]]]
		[N\textsuperscript{af}[-in]]]
		[Fl[-nen]]]	
		{
			\draw[<-, red] (N1.west)--++(-10em,0pt)
			node[anchor=east,align=center]{Flexion (KEIN Wortbildungsporzess)};
			\draw[<-, red] (N2.west)--++(-12em,0pt)
			node[anchor=east,align=center]{Derivation (Movierung)};
			\draw[<-, red] (N3.west)--++(-8em,0pt)
			node[anchor=east,align=center]{Determinativkompositum};
			\draw[<-, red] (N4.west)--++(-3em,0pt)
			node[anchor=east,align=center]{Determinativkompositum};
			\draw[<-, red] (N5.west)--++(-2em,0pt)
			node[anchor=east,align=center]{Derivation};
			\draw[<-, red] (N6.west)--++(-2em,0pt)
			node[anchor=east,align=center]{Implizite Derivation};
			\draw[<-, red] (N7.east)--++(2.5em,0pt)--++(0em,-18ex)%--++(2em,0pt)
			node[anchor=north,align=center]{Implizite Derivation};
			\draw[<-, red] (V1.east)--++(1.5em,0pt)--++(0em,-14ex)--++(2em,0pt)
			node[anchor=west,align=center]{Derivation};
		}	
	\end{forest}	
	%	\end{figure}
}
}
\end{itemize}
\end{frame}


%%%%%%%%%%%%%%%%%%%%%%%%%%%%%%%%%%%%%%%%%%%%%%%%%%%%%%%%%%%

\begin{frame}
\frametitle{Lösungen}

\begin{itemize}
	
	\item [8.] Paraphrasieren Sie das folgende komplexe Wort so, dass es der angegebenen Struktur entspricht (auch wenn Sie selbst eine andere Struktur plausibler finden sollten). %(2 Punkte)
	
	\begin{forest}sn edges,
		[N
		[N[N[Reserve]]
		[N[V[lehr]][N\textsuperscript{af}[-er]]]]
		[N[zimmer]]
		]
	\end{forest}
	
	\item[] \textcolor{red}{
		ein Zimmer für Reservelehrer
	}
\end{itemize}

\end{frame}


%%%%%%%%%%%%%%%%%%%%%%%%%%%%%%%%%%%%%%%%%%%%%%%%%%%%%%%%%%%

\begin{frame}
\frametitle{Lösungen}

\begin{itemize}

\item [9.] Geben Sie für die folgende Wortform die Flexionskategorien an, nach denen sie flektiert ist.%\\
%\hfill(3 Punkte)\\
\ea
bestehe
\z

\item [] \textcolor{red}{
	1. \ras 1.P. / Sg. / Präsens / Indikativ / Aktiv\\
	2. \ras 1.P. / Sg. / Präsens / Konjunktiv I / Aktiv\\
	3. \ras 3.P. / Sg. / Präsens / Konjunktiv I / Aktiv\\
	4. \ras 2.P. / Sg. / Präsens / Imperativ / Aktiv\\
}
\end{itemize}

\end{frame}



	
%% -*- coding:utf-8 -*-

%%%%%%%%%%%%%%%%%%%%%%%%%%%%%%%%%%%%%%%%%%%%%%%%%%%%%%%%%


\def\insertsectionhead{\refname}
\def\insertsubsectionhead{}

\huberlinjustbarfootline


\ifpdf
\else
\ifxetex
\else
\let\url=\burl
\fi
\fi
\begin{multicols}{2}
{\tiny
%\beamertemplatearticlebibitems

\bibliography{gkbib,bib-abbr,biblio}
\bibliographystyle{unified}
}
\end{multicols}





%% \section{Literatur}
%% \begin{frame}[allowframebreaks]
%% \frametitle{Literatur}
%% 	\footnotesize

%% \bibliographystyle{unified}

%% 	%German
%% %	\bibliographystyle{deChicagoMyP}

%% %	%English
%% %	\bibliographystyle{chicago} 

%% 	\bibliography{gkbib,bib-abbr,biblio}
	
%% \end{frame}

	
	
\end{document}
