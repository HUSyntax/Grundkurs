%%%%%%%%%%%%%%%%%%%%%%%%%%%%%%%%%%%%%%%%%%%%%%%%%%%%
%%%             Metadata                         %%%
%%%%%%%%%%%%%%%%%%%%%%%%%%%%%%%%%%%%%%%%%%%%%%%%%%%%      

\title{Grundkurs Linguistik}

\subtitle{Ende\dots}

\author[aMyP]{
	{\small Antonio Machicao y Priemer}
%	\\
%	{\footnotesize \url{http://www.linguistik.hu-berlin.de/staff/amyp}\\
%	\href{mailto:mapriema@hu-berlin.de}{mapriema@hu-berlin.de}}
}

\institute{Institut für deutsche Sprache und Linguistik}

%%%%%%%%%%%%%%%%%%%%%%%%%      
\date{ }
%\publishers{\textbf{6. linguistischer Methodenworkshop \\ Humboldt-Universität zu Berlin}}

%\hyphenation{nobreak}


%%%%%%%%%%%%%%%%%%%%%%%%%%%%%%%%%%%%%%%%%%%%%%%%%%%%
%%%             Preamble's End                   %%%
%%%%%%%%%%%%%%%%%%%%%%%%%%%%%%%%%%%%%%%%%%%%%%%%%%%%      


%%%%%%%%%%%%%%%%%%%%%%%%%      
\huberlintitlepage
\iftoggle{toc}{
\frame{
%\begin{multicols}{2}
	\frametitle{Inhaltsverzeichnis}\tableofcontents
	%[pausesections]
%\end{multicols}
	}
	}


%%%%%%%%%%%%%%%%%%%%%%%%%%%%%%%%%%
%%%%%%%%%%%%%%%%%%%%%%%%%%%%%%%%%%
%%%%%LITERATURE:

\nocite{Altmann&Hofmann08a}
\nocite{Altmann93a}
\nocite{Brandt&Co06a}
\nocite{Glueck05a} 
\nocite{Grewendorf&Co91a} 
\nocite{Luedeling2009a} 
\nocite{Meibauer&Co07a}
\nocite{MuellerS13f} 
\nocite{MuellerS15b}
\nocite{Repp&Co15a} 
\nocite{Stechow&Sternefeld88a}
\nocite{Woellstein10a}


%%%%%%%%%%%%%%%%%%%%%%%%%%%%%%%%%%
%%%%%%%%%%%%%%%%%%%%%%%%%%%%%%%%%%
\section{Anmerkungen zur Klausur}
%\frame{
%\frametitle{~}
%	\tableofcontents[currentsection]
%}

%%%%%%%%%%%%%%%%%%%%%%%%%%%%%%%%%%
\begin{frame}
\frametitle{Anmerkungen zur Klausur}

\begin{itemize}
	\item \textbf{Termin:} Mo. 25.07.2016, 14--16 Uhr
	\item[]
	
	\item \textbf{Raum:} DOR 24, 1.101
	\item[]

	\item \textbf{Punkte:} 70
	\begin{itemize}
		\item GK: 50, UE: 20
		\item Bestanden: ab 35
	\end{itemize}

	\item[]
	\item \textbf{Zeit:} \MyPxbar{90}
	\begin{itemize}
		\item Empfehlung GK: \MyPxbar{60}
		\item Empfehlung UE: \MyPxbar{25}
		\item Empfehlung Reserve: \MyPxbar{5}
	\end{itemize}		
\end{itemize}

\end{frame}


%%%%%%%%%%%%%%%%%%%%%%%%%%%%%%%%%
\begin{frame}
\frametitle{Anmerkungen zur Klausur}

\begin{itemize}

	\item \textbf{Sitzanordnung}
	\begin{itemize}
		\item Bitte lassen Sie immer einen Platz frei zwischen Ihnen und Ihrem Nachbarn!
	\end{itemize}
	\item[]
	\item \textbf{Personal-} und \textbf{Studentenausweis} mitbringen
	\item[]
	\item \textbf{Lehrveranstaltungs-/Arbeitsnachweise}
	\begin{itemize}
	
	 	\item \dots werden mit der Klausur abgegeben, sofern die Angaben und die Unterschrift des Dozenten der UE deutsche Grammatik vorhanden ist.
	 	\item[]
	 	\item Ich unterschreibe den Schein bei der Korrektur.
	 	\item[]
	 	\item Wenn Sie den Schein vergessen haben, bzw. eine Unterschrift fehlt, können Sie den Schein direkt ins Prüfungsbüro bringen.
	
	\end{itemize}
	
\end{itemize}

\end{frame}


%%%%%%%%%%%%%%%%%%%%%%%%%%%%%%%%%%
\begin{frame}
\frametitle{Anmerkungen zur Klausur}

\begin{itemize}
	\item Sie erhalten die Klausur in einem Umschlag. Beschädigen Sie den Umschlag bitte nicht. Denken Sie an die Umwelt!
	\item[]
	\item \textbf{Handys} sind auszuschalten. 
	\item[]
	\item \textbf{Toilettengang} einzeln
	\item[]
	\item Sofortige Abgabe der Klausur bei \textbf{Täuschungsversuch} $+$ Meldung im Prüfungsbüro
	\item[]
	\item \textbf{Hilfsmittel}: DaF-Wörterbuch wird bereitgestellt, andere Hilfsmittel sind nicht erlaubt!
	\item[]
	\item Alle \textbf{Schmierblätter} sind ebenfalls abzugeben.
	
\end{itemize}

\end{frame}

%%%%%%%%%%%%%%%%%%%%%%%%%%%%%%%%%%
%%%%%%%%%%%%%%%%%%%%%%%%%%%%%%%%%%
\section{Übungen}
%\frame{
%\frametitle{~}
%	\tableofcontents[currentsection]
%}

%%%%%%%%%%%%%%%%%%%%%%%%%%%%%%%%%%
%%%%%%%%%%%%%%%%%%%%%%%%%%%%%%%%%%
\subsection{Phonetik/Phonologie}
%\frame{
%\frametitle{~}
%	\tableofcontents[currentsection]
%}
%%%%%%%%%%%%%%%%%%%%%%%%%%%%%%%%%%
\begin{frame}
\frametitle{Übungen: Phonetik/Phonologie}

\begin{itemize}
	\item Erläutern Sie den Unterschied zwischen Phonem, Phon und Allophon.
	\item[]	
	\item Geben sie die artikulatorischen Eigenschaften der folgenden Laute an.
	
	\eal
	\ex \textipa{[r]}
	\ex \textipa{[P]}
	\ex \textipa{[b]}
	\ex \textipa{[x]}
	\zl

\end{itemize}

\end{frame}


%%%%%%%%%%%%%%%%%%%%%%%%%%%%%%%%%%
\begin{frame}
\frametitle{Übungen: Phonetik/Phonologie}

\begin{itemize}
	\item Geben Sie eine phonetische standarddeutsche Transkription der folgenden Wörter mit Silbenstruktur und X-Skelettschicht an
	
	\eal
	\ex Spitzenschuhe
	\ex Zwischendinger
	\ex königlich	
	\zl

	\item Benennen Sie die phonetisch/phonologischen Prozesse, die stattfinden, bei der Aussprache der folgenden Wörter:
	
	\eal
	\ex mild
	\ex ungelenkig
	\ex süchtig
	\ex Haken
	\zl


\end{itemize}

\end{frame}


%%%%%%%%%%%%%%%%%%%%%%%%%%%%%%%%%%
\begin{frame}
\frametitle{Übungen: Phonetik/Phonologie}

\begin{itemize}
	\item Sind die folgenden Segmentfolgen mögliche phonetische Wörter des Standarddeutschen?
	\a. \textipa{[\textprimstress On.tIpl]}	
	\b. \textipa{[Ne:."nt@g]}
	
	
\end{itemize}

\end{frame}


%%%%%%%%%%%%%%%%%%%%%%%%%%%%%%%%%%
%%%%%%%%%%%%%%%%%%%%%%%%%%%%%%%%%%
\subsection{Graphematik}
%\frame{
%\frametitle{~}
%	\tableofcontents[currentsection]
%}

%%%%%%%%%%%%%%%%%%%%%%%%%%%%%%%%%%
\begin{frame}
\frametitle{Übungen: Graphematik}

\begin{itemize}
	\item Geben Sie Beispiele für die Anwendung der folgenden graphematischen Prinzipien an:
	
	\eal 
	\ex Prinzip der Morphemkonstanz
	\ex Homonymieprinzip
	\ex Silbisches Prinzip
	\zl
	
	\item Geben Sie die rein phonographische Schreibung des folgenden Wortes an:
	\ea sprachbegabt
	\z
	
	
\end{itemize}

\end{frame}

%%%%%%%%%%%%%%%%%%%%%%%%%%%%%%%%%%
%%%%%%%%%%%%%%%%%%%%%%%%%%%%%%%%%%
\subsection{Morphologie}
%\frame{
%\frametitle{~}
%	\tableofcontents[currentsection]
%}

%%%%%%%%%%%%%%%%%%%%%%%%%%%%%%%%%%
\begin{frame}
\frametitle{Übungen: Morphologie}

\begin{itemize}
	\item Welche Wortbildungsprozesse haben hier stattgefunden?
	\a. übersetzen
	\b. bleifrei
	\b. Lauf
	\b. Bearbeitung

	\item Geben Sie die Konstituentenstruktur der folgenden Wörter an und bestimmen sie die Wortbildungstypen an jedem Knoten des Baumes so genau wie möglich.
	\a. Unbeweisbarkeitsannahmen
	\b. Blickbewegunsaufnahmen
	
\end{itemize}

\end{frame}


%%%%%%%%%%%%%%%%%%%%%%%%%%%%%%%%%%
%%%%%%%%%%%%%%%%%%%%%%%%%%%%%%%%%%
\subsection{Syntax}
%\frame{
%\frametitle{~}
%	\tableofcontents[currentsection]
%}

%%%%%%%%%%%%%%%%%%%%%%%%%%%%%%%%%%
\begin{frame}
\frametitle{Übungen: Syntax}

\begin{itemize}
	\item Ordnen Sie die folgenden Matrixsätze und ihre Nebensätze in das topologische Feldermodell ein.
	
	\eal
	\ex Petra sieht müde aus, obwohl sie nicht viel getanzt hat.
	\ex Wenn ich im Konzert bin, höre ich der Musik zu.
	\ex Die Frau, die hier arbeitet, obwohl die Heizung ausgeschaltet ist, ist leider krank geworden.
	\ex Anke hat gemerkt, dass Maria trotz der Erkältung arbeiten gegangen ist.
	\zl
	
\end{itemize}

\end{frame}


%%%%%%%%%%%%%%%%%%%%%%%%%%%%%%%%%%
\begin{frame}
\frametitle{Übungen: Syntax}

\begin{itemize}
	\item Analysieren Sie die folgenden Sätze nach dem X-Bar-Schema.
	\a. Maria schlägt Peter.
	\b. Maria schlug gestern Peter.
	\b. Obwohl Helga morgens gearbeitet hat, hat sie am Nachmittag Peter getroffen.
	\b. Über die Behandlung von zwei Patienten haben bis zum Morgengrauen die Ärzte diskutiert.
	
\end{itemize}

\end{frame}


%%%%%%%%%%%%%%%%%%%%%%%%%%%%%%%%%%%
%%%%%%%%%%%%%%%%%%%%%%%%%%%%%%%%%%
\subsection{Semantik/Pragmatik}
%\frame{
%\frametitle{~}
%	\tableofcontents[currentsection]
%}

%%%%%%%%%%%%%%%%%%%%%%%%%%%%%%%%%%
\begin{frame}
\frametitle{Übungen: Semantik/Pragmatik}

\begin{itemize}
	\item Geben Sie die Bedeutungsrelationen zwischen den folgenden Wörtern an.
	\eal
	\ex betrunken -- nüchtern
	\ex Orange -- Apfelsine
	\ex Vogel -- Feder
	\ex verheiratet -- ledig
	\ex mehr -- Meer
	\ex Veilchen -- Lilie
	\ex fruchtbar -- unfruchtbar
	\zl
	
	\item Illustrieren Sie die Begriffe Satzbedeutung, Äußerungsbedeutung und Sprecherbedeutung mithilfe des folgenden Satzes.
	
	\ea Ich glaube, du gehst jetzt!
	\z
	
\end{itemize}

\end{frame}


%%%%%%%%%%%%%%%%%%%%%%%%%%%%%%%%%%
\begin{frame}
\frametitle{Übungen: Semantik/Pragmatik}

\begin{itemize}
	\item Geben Sie die Bedeutungsrelationen zwischen den folgenden Sätzen an.
	
	\eal 
	\ex Auf dem Tisch liegt eine Rose.
	\ex Auf dem Tisch liegt eine Blume.
	\zl
	
	\eal 
	\ex Alle Vögel können fliegen.
	\ex Kein Vogel kann nicht fliegen.
	\zl
	
	\eal 
	\ex Einige Tiere haben Federn.
	\ex Alle Tiere haben Federn.
	\zl
	
	\eal 
	\ex Mario ist nicht tot.
	\ex Mario ist nicht lebendig.
	\zl
	
	\eal 
	\ex Ines mag keine Orangen.
	\ex Ines mag keine Apfelsinen.
	\zl
	
	\eal 
	\ex Ich lese ein Buch.
	\ex Ich lese eine Zeitschrift.
	\zl
	
\end{itemize}

\end{frame}


%%%%%%%%%%%%%%%%%%%%%%%%%%%%%%%%%%
\begin{frame}
\frametitle{Übungen: Semantik/Pragmatik}

\begin{itemize}
	\item Markieren Sie alle deiktischen und anaphorischen Elemente in den folgenden Sätzen
	
	\eal
	\ex Sie haben diese Tür nicht geschlossen.
	\ex Gestern war mir das Wetter echt zu kalt!
	\ex \gqq{Ich bin sehr glücklich, mich wieder für das WTA-Finale qualifiziert zu haben. Ich freue mich darauf, dort anzutreten und gegen die Besten der Welt zu spielen}, sagte die 27-Jährige, die im vergangenen Jahr nur als Ersatzspielerin mitfahren durfte.
	\zl
	
\end{itemize}

\end{frame}


%%%%%%%%%%%%%%%%%%%%%%%%%%%%%%%%%%
\begin{frame}
\frametitle{Übungen: Semantik/Pragmatik}

\begin{itemize}
	\item Bestimmen Sie die Art von Folgerung, die zwischen dem ersten und den folgenden Sätzen besteht:
	
	\eal 
	\ex Sogar Peter hat zwei Kinder. 
	\ex Peter hat nicht mehr als zwei Kinder.
	\ex Es gibt ein Individuum namens Peter.
	\ex Peter ist Vater.
	\ex Peter hat vier Kinder.
	\zl
	
	\item Bestimmen Sie jeweils eine semantische Implikation aus dem folgenden Sätzen:
	
	\eal
	\ex In einem Schuhkarton gibt es Platz für zwei Schuhe.
	\ex Greg heiratete eine Norwegerin.
	\zl
		
\end{itemize}

\end{frame}


%%%%%%%%%%%%%%%%%%%%%%%%%%%%%%%%%%
\begin{frame}
\frametitle{Übungen: Semantik/Pragmatik}

\begin{itemize}
	
	\item Bestimmen Sie jeweils eine Präsupposition aus dem folgenden Sätzen:
	\eal
	\ex Ich freue mich darüber, dass wir die Klausur bestanden haben.
	\ex Maria ist auch schwanger.
	\ex Alle Banken wurden mit unseren Steuergeldern gerettet.
	\ex Sie lieben Syntax immer noch nicht.
	\zl

	\item Geben Sie an, ob eine Maxime (scheinbar) verletzt oder befolgt wurde und um welche es sich handelt, um die angegebene Implikatur zu erhalten.
	
	\ea Wir haben einige Personen entlassen.\\
	$+>$ Es wurden nicht alle entlassen.
	\z
	
	\ea A: Wie war das Bewerbungsgespräch?\\
	B: Das Wetter ist ja super heute!\\
	$+>$ Es war furchtbar!
	\z
		
\end{itemize}

\end{frame}
