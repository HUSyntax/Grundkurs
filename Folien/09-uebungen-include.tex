%%%%%%%%%%%%%%%%%%%%%%%%%%%%%%%%%%%%%%%%%%%%%%%%
%% Compile the master file!
%% 		Slides: Antonio Machicao y Priemer
%% 		Course: GK Linguistik
%%%%%%%%%%%%%%%%%%%%%%%%%%%%%%%%%%%%%%%%%%%%%%%%


%%%%%%%%%%%%%%%%%%%%%%%%%%%%%%%%%%%%%%%%%%%%%%%%%%%%
%%%             Metadata                         
%%%%%%%%%%%%%%%%%%%%%%%%%%%%%%%%%%%%%%%%%%%%%%%%%%%%      

\title{Grundkurs Linguistik}

\subtitle{Übungen}

\author[A. Machicao y Priemer]{
	{\small Antonio Machicao y Priemer}
	\\
	{\footnotesize \url{http://www.linguistik.hu-berlin.de/staff/amyp}}\\
	%	\href{mailto:mapriema@hu-berlin.de}{mapriema@hu-berlin.de}}
}

\institute{Institut für deutsche Sprache und Linguistik}
  
%\date{ }
%\publishers{\textbf{6. linguistischer Methodenworkshop \\ Humboldt-Universität zu Berlin}}


%%%%%%%%%%%%%%%%%%%%%%%%%%%%%%%%%%%%%%%%%%%%%%%%%%%%
%%%             Preamble's End                   %%%
%%%%%%%%%%%%%%%%%%%%%%%%%%%%%%%%%%%%%%%%%%%%%%%%%%%%      


%%%%%%%%%%%%%%%%%%%%%%%%%      
\huberlintitlepage[22pt]
\iftoggle{toc}{
	\frame{
%		\begin{multicols}{2}
			\frametitle{Inhaltsverzeichnis}\tableofcontents
			%[pausesections]
%		\end{multicols}
	}
}



%%%%%%%%%%%%%%%%%%%%%%%%%%%%%%%%%%
%%%%%%%%%%%%%%%%%%%%%%%%%%%%%%%%%%
%%%%%LITERATURE:

\nocite{Altmann&Hofmann08a}
\nocite{Altmann93a}
\nocite{Brandt&Co06a}
%\nocite{Glueck05a} 
\nocite{Glueck&Roedel16a}
\nocite{Grewendorf&Co91a} 
\nocite{Luedeling2009a} 
\nocite{MyP18a} %Konstituententests
\nocite{MyP18c} %Phrase
\nocite{MyP18b} %Kopf
\nocite{Meibauer&Co07a}
%\nocite{MuellerS13f} 
%\nocite{MuellerS15b}
\nocite{Repp&Co15a} 
%\nocite{Stechow&Sternefeld88a}
\nocite{Woellstein10a}


%%%%%%%%%%%%%%%%%%%%%%%%%%%%%%%%%%
%%%%%%%%%%%%%%%%%%%%%%%%%%%%%%%%%%
\section{Wiederholungsstunde}

%%%%%%%%%%%%%%%%%%%%%%%%%%%%%%%%%%
\begin{frame}
	\frametitle{Begleitlektüre}
	
	\begin{itemize}
		
		\item \textbf{obligatorisch:}
			
			\begin{itemize}
				\item  \citet{Meibauer&Co07a}: Kapitel 6 (S.~210--240)
			\end{itemize}
			
	\end{itemize}

\end{frame}


%%%%%%%%%%%%%%%%%%%%%%%%%%%%%%%%%%
%%%%%%%%%%%%%%%%%%%%%%%%%%%%%%%%%%
\subsection{Anmerkungen zur Klausur}
\iftoggle{sectoc}{
	\frame{
		%\begin{multicols}{2}
		\frametitle{~}
		\tableofcontents[currentsubsection,subsubsectionstyle=hide]
		%\end{multicols}
	}
}
%%%%%%%%%%%%%%%%%%%%%%%%%%%%%%%%%%

\begin{frame}
\frametitle{Anmerkungen zur Klausur}

\begin{itemize}
	\item \textbf{Termin:} Mo. 17.02.2020, 12--14 Uhr
	\medskip 
	
	\item \textbf{Raum:} DOR 24, 1.101
	\medskip 

	\item \textbf{Punkte:} 70
	\begin{itemize}
		\item GK: 50, UE: 20
		\item Bestanden: ab 35
	\end{itemize}
	\medskip 

	\item \textbf{Zeit:} \MyPxbar{90}
	\begin{itemize}
		\item Empfehlung GK: \MyPxbar{60}
		\item Empfehlung UE: \MyPxbar{25}
		\item Empfehlung Reserve: \MyPxbar{5}
	\end{itemize}		
\end{itemize}

\end{frame}


%%%%%%%%%%%%%%%%%%%%%%%%%%%%%%%%%
\begin{frame}
%\frametitle{Anmerkungen zur Klausur}

\begin{itemize}

	\item \textbf{Sitzanordnung}
	\begin{itemize}
		\item Bitte lassen Sie immer einen Platz frei zwischen Ihnen und Ihrem Nachbarn.
		\item Nehmen Sie so Platz, dass jemand direkt vor Ihnen sitzt.
	\end{itemize}
%	\medskip   

	\item \textbf{Personal-} und \textbf{Studentenausweis} mitbringen.
%	\medskip 

	\item \textbf{Lehrveranstaltungs-/Arbeitsnachweise} werden mit der Klausur ausgegeben.
%	\begin{itemize}
%	 	\item \dots werden mit der Klausur abgegeben, sofern die Angaben und die Unterschrift des Dozenten der UE deutsche Grammatik vorhanden ist.

%	 	\item Die Scheine werden bei der Korrektur unterschrieben.
	 	
%	 	\item Wenn Sie den Schein vergessen haben, bzw. eine Unterschrift fehlt, können Sie den Schein direkt ins Prüfungsbüro bringen.
	
%	\end{itemize}
	
%\end{itemize}
%
%\end{frame}
%
%
%%%%%%%%%%%%%%%%%%%%%%%%%%%%%%%%%%%
%\begin{frame}
%\frametitle{Anmerkungen zur Klausur}
%
%\begin{itemize}
	\item Sie erhalten die Klausur in einem Umschlag. Beschädigen Sie den Umschlag bitte nicht. Denken Sie an die Umwelt!

	\item Benutzen Sie \textbf{keinen Bleistift}. Antworten mit Bleistift werden nicht berücksichtigt.
	
	\item \textbf{Handys} sind auszuschalten. 

	\item \textbf{Toilettengang} einzeln

	\item Sofortige Abgabe der Klausur bei Täuschungsversuch $+$ Meldung im Prüfungsbüro

	\item \textbf{Hilfsmittel}: DaF-Wörterbuch wird bereitgestellt, andere Hilfsmittel sind nicht erlaubt!

	\item Alle \textbf{Schmierblätter} sind ebenfalls abzugeben.
	
\end{itemize}

\end{frame}


%%%%%%%%%%%%%%%%%%%%%%%%%%%%%%%%%%
%%%%%%%%%%%%%%%%%%%%%%%%%%%%%%%%%%
\subsection{Übungen: Phonetik/Phonologie}
\iftoggle{sectoc}{
	\frame{
		%\begin{multicols}{2}
		\frametitle{~}
		\tableofcontents[currentsubsection,subsubsectionstyle=hide]
		%\end{multicols}
	}
}
%%%%%%%%%%%%%%%%%%%%%%%%%%%%%%%%%%

\begin{frame}
\frametitle{Übungen: Phonetik/Phonologie}

\begin{itemize}
	\item[1.] Erläutern Sie den Unterschied zwischen Phon, Phonem und Allophon.
	\item[]	
	\item[2.] Geben sie die artikulatorischen Eigenschaften der folgenden Laute an.
	
	\eal \label{ex:01}
	\ex \textipa{[r]}
	\ex \textipa{[P]}
	\ex \textipa{[b]}
	\ex \textipa{[f]}
	\ex \textipa{[I]}
	\ex \textipa{[u:]}
	\zl

\end{itemize}

\end{frame}


%%%%%%%%%%%%%%%%%%%%%%%%%%%%%%%%%%
\begin{frame}
%\frametitle{Übungen: Phonetik/Phonologie}

\begin{itemize}
	\item[3.] Geben Sie die phonologische Repräsentation und die phonetische standarddeutsche Transkription der folgenden Wörter mit Silbenstruktur und X-Skelettschicht an.
	
	\eal \label{ex:02}
	\ex Näherinnen
	\ex Zwischendinger
	\ex königlich
	\zl

	\item[4.] Benennen Sie die phonetisch/phonologischen Prozesse, die stattfinden, bei der Aussprache der folgenden Wörter:
	
	\eal \label{ex:03}
	\ex mild
	\ex ungelenkig
	\ex süchtig
	\ex Kraken
	\zl


\end{itemize}

\end{frame}


%%%%%%%%%%%%%%%%%%%%%%%%%%%%%%%%%%
\begin{frame}
%\frametitle{Übungen: Phonetik/Phonologie}

\begin{itemize}
	\item[5.] Sind die folgenden Segmentfolgen mögliche phonetische Wörter des Standarddeutschen?
	\ea \label{ex:04} \textipa{[p@:kl.\textprimstress Ipl]}	
	\ex \label{ex:05} \textipa{[\textprimstress Na:h.i:ltd]}
	\z 
	
\end{itemize}

\end{frame}

%%%%%%%%%%%%%%%%%%%%%%%%%%%%%%%%%%

\iftoggle{ue-loesung}{
%%%%%%%%%%%%%%%%%%%%%%%%%%%%%%%%%%
%% UE 1 - 09 Übungen
%%%%%%%%%%%%%%%%%%%%%%%%%%%%%%%%%%

\begin{frame}
\frametitle{Übung: Phonetik/Phonologie -- Lösung}

\begin{itemize}
	\item[1.] Erläutern Sie den Unterschied zwischen Phon, Phonem und Allophon.
	
	\item Phon:
		
		\only<2->{
		\begin{itemize}
			\item \alertgreen{Minimaleinheit der Phonetik}
			\item \alertgreen{physikalisch messbare lautliche Einheit einer Sprache} \pause
		\end{itemize}
	}
		
	\item Phonem:
		
		\only<3->{
		\begin{itemize}
			\item \alertgreen{Minimaleinheit der Phonologie}
			\item \alertgreen{abstraktes Konstrukt, steht für eine Menge von möglichen Phonen}
			\item \alertgreen{ermittelbar durch Minimalpaarbildung (strukturalistisches Kriterium)} \pause
		\end{itemize}
	}
		
	\item Allophon:
		
		\only<4->{
		\begin{itemize}
			\item \alertgreen{phonetische Realisierungsvariante eines Phonems}
			\item \alertgreen{Untertypen: komplementäre und freie Allophonie, regionale und soziale Variation}
		\end{itemize}
		}

\end{itemize}

\end{frame}

%%%%%%%%%%%%%%%%%%%%%%%%%%%%%%%%%%%

\begin{frame}
	
\begin{itemize}
	\item[2.] Geben sie die artikulatorischen Eigenschaften der folgenden Laute an.
	
	\settowidth \jamwidth{\alertgreen{halbhoher fast vorderer ungerundeter ungespannter Vokal}}
	
	\eal
	\ex \textipa{[r]}	\only<2->{\jambox{\alertgreen{alveolarer stimmhafter Vibrant}}}
	\ex \textipa{[P]}	\only<3->{\jambox{\alertgreen{glottaler stimmloser Plosiv}}}
	\ex \textipa{[b]}	\only<4->{\jambox{\alertgreen{bilabialer stimmhafter Plosiv}}}
	\ex \textipa{[f]}	 \only<5->{\jambox{\alertgreen{labiodentaler stimmloser Frikativ}}}
	\ex \textipa{[I]}	 \only<6->{\jambox{\alertgreen{halbhoher fast vorderer ungerundeter ungespannter Vokal}}}
	\ex \textipa{[u:]}	\only<7->{\jambox{\alertgreen{hoher hinterer gerundeter gespannter (langer) Vokal}}}
	\zl
	
\end{itemize}

\end{frame}

%%%%%%%%%%%%%%%%%%%%%%%%%%%%%%%%%%%

\begin{frame}
	
\begin{itemize}
		\item[3.] Geben Sie die phonologische Repräsentation und die phonetische standarddeutsche Transkription der folgenden Wörter mit Silbenstruktur und X-Skelettschicht an.
	
	\eal
	\ex \label{ex:Skelett1}{Näherinnen}
	\ex \label{ex:Skelett2}{Zwischendinger}
	\ex \label{ex:Skelett3}{königlich}
	\zl
	
\end{itemize}

\end{frame}

%%%%%%%%%%%%%%%%%%%%%%%%%%%%%%%%%%%	
	
\begin{frame}

(\ref{ex:Skelett1}): Näherinnen \\

\medskip

	\alertgreen{\textipa{/ne:.@.KI\d{n}@n/} \ras \textipa{[\textprimstress ne:.@.KI\d{n}@n]} }
	
	\centering
	\alertgreen{
	\scalebox{1}{
	\begin{forest} MyP edges, [,phantom
		[$\sigma$
		[O
			[x, tier=word[\textipa{n}]]
		]
		[R
			[N
				[x, tier=word[\textipa{E:}]]
			]
			[K]
		]
		]
		[$\sigma$
		[O]
		[R
			[N
				[x, tier=word[\textipa{@}]]
			]
			[K]
		]
		]
		[$\sigma$
		[O
			[x, tier=word[\textipa{K}]]+
		]
		[R
			[N
				[x, tier=word[\textipa{I}]]
			]
			[K
				[x, tier=word, name=x[\textipa{n}]]
			]
		]
		]
		[$\sigma$
		[O, name=O]
		[R
			[N
				[x, tier=word[\textipa{@}]]
			]
			[K
				[x, tier=word[\textipa{n}]]
			]
		]
		]
	]
	\draw[HUgreen] (O.south)--(x.north);
	\end{forest}
	}}

\end{frame}

%%%%%%%%%%%%%%%%%%%%%%%%%%%%%%%%%%%

\begin{frame}

(\ref{ex:Skelett2}): Zwischendinger	\\

\medskip

	\alertgreen{\textipa{/\t{ts}vI\d{S}@n.dIn.g@\textscr/} \ras \textipa{[\textprimstress \t{ts}vI\d{S}@n.dI\d{N}5]}}
	
	\centering
	\alertgreen{
	\scalebox{1}{
	\begin{forest} MyP edges, [,phantom
		[$\sigma$
		[O
			[x, tier=word[\textipa{\t{ts}}] ]
			[x, tier=word[\textipa{v}] ]
		]
		[R
			[N
				[x, tier=word[\textipa{I}]]
			]
			[K
				[x, tier=word, name=x[\textipa{S}]]
			]
		]
		]
		[$\sigma$
		[O, name=O]
		[R
			[N
				[x, tier=word[\textipa{@}] ]
			]
			[K
				[x, tier=word[\textipa{n}]]
			]
		]
		]
		[$\sigma$
		[O
			[x, tier=word[\textipa{d}]]
		]
		[R
			[N
				[x, tier=word[\textipa{I}]]
			]
			[K
				[x, tier=word, name=x2[\textipa{N}]]
			]
		]
		]
		[$\sigma$
		[O, name=O2]
		[R
			[N
				[x, tier=word[\textipa{5}]]
			]
			[K]	
		]
		]
	]
	\draw[HUgreen] (O.south)--(x.north);
	\draw[HUgreen] (O2.south)--(x2.north);
	\end{forest}
	} }

\end{frame}

%%%%%%%%%%%%%%%%%%%%%%%%%%%%%%%%%%%

\begin{frame}

(\ref{ex:Skelett3}): königlich \\

\medskip

	\alertgreen{\textipa{/k\o:.nIg.lI\c{c}/} \ras \textipa{[\textprimstress k\o:.nIk.lI\c{c}]}} ~\\
	
	\centering
	\alertgreen{
	\scalebox{1}{
	\begin{forest} MyP edges, [,phantom
		[$\sigma$
		[O
			[x, tier=word[\textipa{k}]]
		]
		[R
			[N
				[x, tier=word[\textipa{\o:}, name=o]]
				[x, tier=word, name=x]		
			]
			[K]
		]
		]
		[$\sigma$
		[O
			[x, tier=word[\textipa{n}]]
		]
		[R
			[N
				[x, tier=word[\textipa{I}]]
			]
			[K
				[x, tier=word[\textipa{k}]]
			]
		]
		]
		[$\sigma$
		[O
			[x, tier=word[\textipa{l}]]
		]
		[R
			[N
				[x, tier=word[\textipa{I}]]
			]
			[K
				[x, tier=word[\textipa{\c{c}}]]
			]
		]
		]
	]
	\draw[HUgreen] (x.south)--(o.north);		
	\end{forest}
	} }

\end{frame}

%%%%%%%%%%%%%%%%%%%%%%%%%%%%%%%%%%%
\begin{frame}

\begin{itemize}
	\item[4.] Benennen Sie die phonetisch/phonologischen Prozesse, die stattfinden, bei der Aussprache der folgenden Wörter:
	
	\settowidth \jamwidth{\alertgreen{regressive velare Nasalassimilation: \textipa{/n/} \ras \textipa{[N]}},}
	
	\eal
	\ex mild	\only<2->{\jambox{\alertgreen{Auslautverhärtung: \textipa{/d/} \ras \textipa{[t]}}}}
\medskip
	\ex ungelenkig	\only<3->{\jambox{\alertgreen{Knacklauteinsetzung: $\emptyset$ \ras \textipa{[P]}},}}
	\only<3->{\jambox{\alertgreen{regressive velare Nasalassimilation: \textipa{/n/} \ras \textipa{[N]}},}} 
	\only<3->{\jambox{\alertgreen{g-Spirantisierung: \textipa{/g/} \ras \textipa{[\c{c}]}}}}
\medskip
	\ex süchtig	\only<4->{\jambox{\alertgreen{g-Spirantisierung: \textipa{/g/} \ras \textipa{[\c{c}]}}}}
\medskip
	\ex Kraken	\only<5->{\jambox{\alertgreen{Schwa-Tilgung: \textipa{/@/} \ras $\emptyset$},}}
	\only<5->{\jambox{\alertgreen{progressive Ortsassimilation: \textipa{/kn/} \ras \textipa{[kN]}}}}
	\zl
	
\end{itemize}

\end{frame}

%%%%%%%%%%%%%%%%%%%%%%%%%%%%%%%%%%%
\begin{frame}

\begin{itemize}
	\item[5.] Sind die folgenden Segmentfolgen mögliche phonetische Wörter des Standarddeutschen?
	
	\ea \label{ex:Segment} \textipa{[p@:kl.\textprimstress Ipl]} 
	\ex \label{ex:Segment2}\textipa{[\textprimstress Na:h.i:ltd]}
	\z 
	
	\only<2->{\item \alertgreen{Beispiel (\ref{ex:Segment}) kann kein phonetisches Wort des Standarddeutschen sein, denn: gespanntes \textipa{[@]}, Verletzung der Sonoritätshierarchie in der Koda oder Onset-Maximierung in der folgenden Silbe \textipa{[kl]}, keine Knacklauteinsetzung in betonter Silbe, Verletzung der Sonoritätshierarchie \textipa{[pl]}}}
	
	\only<3->{\item \alertgreen{Beispiel (\ref{ex:Segment2}) kann kein phonetisches Wort des Standarddeutschen sein, denn: \textipa{[N]} am Wortanfang, \textipa{[h]} wird wortintern nicht realisiert, \textipa{[i:]} muss mit daraffolgendem Konsonanten kurz sein, fehlende Auslautverhärtung \textipa{[d]}, anschließende fehlende Geminatenreduktion \textipa{[tt]} }}
\end{itemize}

\end{frame}
%%%%%%%%%%%%%%%%%%%%%%%%%%%%%%%%%%%
}

%%%%%%%%%%%%%%%%%%%%%%%%%%%%%%%%%%
%%%%%%%%%%%%%%%%%%%%%%%%%%%%%%%%%%
\subsection{Übungen: Graphematik}
\iftoggle{sectoc}{
	\frame{
		%\begin{multicols}{2}
		\frametitle{~}
		\tableofcontents[currentsubsection,subsubsectionstyle=hide]
		%\end{multicols}
	}
}
%%%%%%%%%%%%%%%%%%%%%%%%%%%%%%%%%%

\begin{frame}
\frametitle{Übungen: Graphematik}

\begin{itemize}
	\item[6.] Geben Sie Beispiele für die Anwendung der folgenden graphematischen Prinzipien an:
	
	\eal \label{ex:06}
	\ex Prinzip der Morphemkonstanz
	\ex Homonymieprinzip
	\ex Silbisches Prinzip
	\zl
	
	\item[7.] Geben Sie die rein phonographische Schreibung der folgenden Wörter an:
	\eal \label{ex:07}
	\ex sprachbegabt
	\ex Sträuchersee
	\zl
	
\end{itemize}

\end{frame}

%%%%%%%%%%%%%%%%%%%%%%%%%%%%%%%%%%

\iftoggle{ue-loesung}{
	%%%%%%%%%%%%%%%%%%%%%%%%%%%%%%%%%%
%% UE 2 - 09 Übungen
%%%%%%%%%%%%%%%%%%%%%%%%%%%%%%%%%%

\begin{frame}
\frametitle{Übung: Graphematik -- Lösung}

\begin{itemize}
	\item[6.] Geben Sie Beispiele für die Anwendung der folgenden graphematischen Prinzipien an:
	
	\eal 
	\ex \only<1->{Prinzip der Morphemkonstanz} \\
	\alertgreen{\only<2->{\item[-] Silbengelenke, wegen zugehöriger Pluralformen: \zB \textit{Ba\underline{ll}},}}
	\alertgreen{\only<2->{\item[-] Dehnungs-h, wegen zugehöriger Flexionsformen: \zB \textit{de\underline{h}nen, weil du dehnst}}}
	
	\ex \only<1->{Homonymieprinzip} \\
	\alertgreen{\only<3->{\item[-] Differenzierung homophoner Formen: \zB \textit{L\underline{ee}re vs. L\underline{eh}re}}}
	
	\ex \only<1->{Silbisches Prinzip} \\
	\alertgreen{\only<4->{\item[-] Silbengelenk: \zB \textit{Wa\underline{ss}er},}}
	\alertgreen{\only<4->{\item[-] Silbentrennendes h: \zB S\textit{chu\underline{h}e},}}
	\alertgreen{\only<4->{\item[-] Dehnungs-h: \zB \textit{Sa\underline{h}ne},}}
	\alertgreen{\only<4->{\item[-] Gespanntheit: \zB \textit{M\underline{oo}s},}}
	\zl
		
\end{itemize}

\end{frame}

%%%%%%%%%%%%%%%%%%%%%%%%%%%%%%%%%%

\begin{frame}
	
\begin{itemize}
	
	\item[7.] Geben Sie die rein phonographische Schreibung der folgenden Wörter an:	
	
	\eal
	\ex sprachbegabt \loesung{2}{\ab{schprachbegabt}}
	
	\ex Sträuchersee  \loesung{3}{\ab{schtreucherse}}
	\zl
	
\end{itemize}

\end{frame}
%%%%%%%%%%%%%%%%%%%%%%%%%%%%%%%%%%
}

%%%%%%%%%%%%%%%%%%%%%%%%%%%%%%%%%%
%%%%%%%%%%%%%%%%%%%%%%%%%%%%%%%%%%
\subsection{Übungen: Morphologie}
\iftoggle{sectoc}{
	\frame{
		%\begin{multicols}{2}
		\frametitle{~}
		\tableofcontents[currentsubsection,subsubsectionstyle=hide]
		%\end{multicols}
	}
}
%%%%%%%%%%%%%%%%%%%%%%%%%%%%%%%%%%

\begin{frame}
\frametitle{Übungen: Morphologie}

\begin{itemize}
	\item[8.] Welche Wortbildungsprozesse haben hier stattgefunden?

	\eal \label{ex:08}
	\ex übersetz(-en)
	\ex bleifrei
	\ex Tanz
	\ex Bearbeitung
	\zl
	
	\item[9.] Geben Sie die Konstituentenstruktur der folgenden Wörter an und bestimmen Sie die Wortbildungstypen an jedem Knoten des Baumes so genau wie möglich.

	\eal \label{ex:09}
	\ex Unbeweisbarkeitsannahmen
	\ex (mit den) Blickbewegungsmessern
	\zl

\end{itemize}

\end{frame}


%%%%%%%%%%%%%%%%%%%%%%%%%%%%%%%%%%

\begin{frame}
%	\frametitle{Übungen: Morphologie}
	
\begin{itemize}	
	\item[10.] Geben Sie Beispiele für die folgenden Kompositionsarten an:
	
	\eal \label{ex:10}
	\ex Determinativkomposition
	\ex Rektionskomposition
	\ex Possessivkomposition
	\ex Kopulativkomposition 
	\zl

	\item[11.] Geben Sie je ein Beispiel für einen Stamm, für eine Wurzel, für eine Basis und für ein unikales Morphem an.
	
	\eal \label{ex:11}
	\ex Stamm:
	\ex Wurzel:
	\ex Basis:
	\ex unikales Morphem:
	\zl
	
\end{itemize}

\end{frame}

%%%%%%%%%%%%%%%%%%%%%%%%%%%%%%%%%%

\iftoggle{ue-loesung}{
	%%%%%%%%%%%%%%%%%%%%%%%%%%%%%%%%%%
%% UE 3 - 09 Übungen
%%%%%%%%%%%%%%%%%%%%%%%%%%%%%%%%%%

\begin{frame}
\frametitle{Übung: Morphologie -- Lösung}

\begin{itemize}
	\item[8.] Welche Wortbildungsprozesse haben hier stattgefunden?
	
	\settowidth \jamwidth{\alertgreen{\only<2->{Derivation (Präfigierung) oder Partikelverbbildung}}}
	
	\eal
	\ex übersetz(-en)  \jambox{\alertgreen{\only<2->{Derivation (Präfigierung) oder Partikelverbbildung}}}
	\ex bleifrei \jambox{\alertgreen{\only<3->{Rektionskompositum}}}
	\ex Tanz \jambox{\alertgreen{\only<4->{Konversion}}}
	\ex Bearbeitung \jambox{\alertgreen{\only<5->{1. Derivation (Präfigierung),}}}
			\jambox{\alertgreen{\only<5->{2. Derivation (Suffigierung)}}}
	\zl
	
\end{itemize}

\end{frame}

%%%%%%%%%%%%%%%%%%%%%%%%%%%%%%%%%%
\begin{frame}
	
\begin{itemize}
	
%	\only<1-1>{
		\item[9.] Geben Sie die Konstituentenstruktur der folgenden Wörter an und bestimmen Sie die Wortbildungstypen an jedem Knoten des Baumes so genau wie möglich.
%	}
	
%	\only<1-1>{
	\eal
	\ex \label{ex:Konstituente1}{Unbeweisbarkeitsannahmen}
	\ex \label{ex:Konstituente2}{(mit den) Blickbewegungsmessern}
	\zl
%	}

\end{itemize}

\end{frame}

%%%%%%%%%%%%%%%%%%%%%%%%%%%%%%%%%%

\begin{frame}

\begin{itemize}
	\item Analyse mit Fugenelement \\
	(\ref{ex:Konstituente1}): Unbeweisbarkeitsannahmen

	\scalebox{.68}{
	\alertgreen{
	\begin{forest} MyP edges,
		[N, name=N1
			[N, name=N2
				[N
					[N, name=N3
						[A, name=A1
							[A\MyPup{af} [un-, tier=word]]
							[A, name=A2
								[V, name=V3
									[A\MyPup{af} [be-, tier=word]]
									[V [weis, tier=word]]
								]
								[A\MyPup{af} [-bar, tier=word]]
							]
						]
						[N\MyPup{af} [-keit, tier=word]]
					]
					[FE [-s, tier=word]]
				]
				[N, name=N4
					[V, name=V1
						[V\MyPup{af} [an-, tier=word]]
						[V, name=V2 [nahm/nehm, tier=word]]
					]
					[N [-e, tier=word]]
				]
			]
			[FI [-n, tier=word]]
		]
	\draw[<-, HUgreen] (N1.west)--++(-12.5em,0pt)
	node[anchor=east,align=center]{Flexion (KEIN Wortbildungsprozess)};
	\draw[<-, HUgreen] (N2.west)--++(-13.5em,0pt)
	node[anchor=east,align=center]{Rektionskompositum};
	\draw[<-, HUgreen] (N3.west)--++(-9em,0pt)
	node[anchor=east,align=center]{Derivation};
	\draw[<-, HUgreen] (A1.west)--++(-4.7em,0pt)
	node[anchor=east,align=center]{Derivation};
	\draw[<-, HUgreen] (A2.east)--++(2.5em,0pt)--++(0pt,-22.5ex)
	node[anchor=north,align=center]{Derivation};
	\draw[<-, HUgreen](N4.east)--++(3.5em,0pt)--++(0pt,-43.5ex)
	node[anchor=north,align=center]{Derivation};
	\draw[<-, HUgreen](V1.west)--++(-2.5em,0pt)--++(0pt,-36.7ex)
	node[anchor=north,align=center]{Derivation};
	\draw[<-, HUgreen](V2.west)--++(-2em,0pt)--++(0pt,-32ex)
	node[anchor=north,align=center]{implizite Derivation};
	\draw[<-, HUgreen](V3.west)--++(-2em,0pt)--++(0pt,-15.5ex)
	node[anchor=north,align=center]{Derivation};
	\end{forest} 
	} }
	
\end{itemize}

\end{frame}

%%%%%%%%%%%%%%%%%%%%%%%%%%%%%%%%%%

\begin{frame}

\begin{itemize}
	\item Analyse mit Kompositionsstammform \\
	(\ref{ex:Konstituente1}): Unbeweisbarkeitsannahmen
	
	\scalebox{.75}{
	\alertgreen{
	\begin{forest} MyP edges,
		[N, name=N1
			[N, name=N2
				[N, name=N3
					[A, name=A1
						[A\MyPup{af} [un-, tier=word]]
						[A, name=A2
							[V, name=V3
								[A\MyPup{af} [be-, tier=word]]
								[V [weis, tier=word]]
							]
							[A\MyPup{af} [-bar, tier=word]]
						]
					]
					[N\MyPup{af} [-keit(-s), tier=word]]
				]
				[N, name=N4
					[V, name=V1
						[V\MyPup{af} [an-, tier=word]]
						[V, name=V2 [nahm/nehm, tier=word]]
					]
					[N [-e, tier=word]]
				]
			]
			[FI [-n, tier=word]]
		]
	\draw[<-, HUgreen] (N1.west)--++(-13em,0pt)
	node[anchor=east,align=center]{Flexion (KEIN Wortbildungsprozess)};
	\draw[<-, HUgreen] (N2.west)--++(-13.5em,0pt)
	node[anchor=east,align=center]{Rektionskompositum};
	\draw[<-, HUgreen] (N3.west)--++(-11em,0pt)
	node[anchor=east,align=center]{Derivation};
	\draw[<-, HUgreen] (A1.west)--++(-6.5em,0pt)
	node[anchor=east,align=center]{Derivation};
	\draw[<-, HUgreen] (A2.east)--++(2.5em,0pt)--++(0pt,-22.5ex)
	node[anchor=north,align=center]{Derivation};
	\draw[<-, HUgreen](N4.east)--++(3.5em,0pt)--++(0pt,-36.5ex)
	node[anchor=north,align=center]{Derivation};
	\draw[<-, HUgreen](V1.west)--++(-2.5em,0pt)--++(0pt,-29.7ex)
	node[anchor=north,align=center]{Derivation};
	\draw[<-, HUgreen](V2.west)--++(-2em,0pt)--++(0pt,-25ex)
	node[anchor=north,align=center]{implizite Derivation};
	\draw[<-,HUgreen](V3.west)--++(-2em,0pt)--++(0pt,-15.5ex)
	node[anchor=north,align=center]{Derivation};
	\end{forest} 
	} }

\end{itemize}

\end{frame}

%%%%%%%%%%%%%%%%%%%%%%%%%%%%%%%%%%

\begin{frame}

\begin{itemize}
	\item Analyse mit Fugenelement \\
	(\ref{ex:Konstituente2}): (mit den) Blickbewegungsmessern
			
	\scalebox{.85}{
	\alertgreen{
	\begin{forest} MyP edges,
		[N, name=N1
			[N, name=N2
				[N
					[N, name=N3
						[N [blick, tier=word]]
						[N, name=N4
							[V [beweg, tier=word]]
							[N\MyPup{af} [-ung, tier=word]]
						]
					]
					[FE [-s, tier=word]]
				]
				[N, name=N5
					[V [mess, tier=word]]
					[N\MyPup{af} [-er, tier=word]]
				]
			]
			[FI [-n, tier=word]]
		]
	\draw[<-, HUgreen](N1.west)--++(-11.8em,0pt)
	node[anchor=east,align=center]{Flexion (KEIN Wortbildungsprozess)};
	\draw[<-, HUgreen](N2.west)--++(-14.5em,0pt)
	node[anchor=east,align=center]{Rektionskompositum};
	\draw[<-, HUgreen](N3.west)--++(-7em,0pt)
	node[anchor=east,align=center]{Rektionskompositum};
	\draw[<-, HUgreen](N4.east)--++(2.5em,0pt)--++(0pt,-16.2ex)
	node[anchor=north,align=center]{Derivation};
	\draw[<-, HUgreen](N5.east)--++(2em,0pt)--++(0pt,-30ex)
	node[anchor=north,align=center]{Derivation};
	\end{forest}
	} }

\end{itemize}

\end{frame}	

%%%%%%%%%%%%%%%%%%%%%%%%%%%%%%%%%%

\begin{frame}

\begin{itemize}
	\item Analyse mit Kompositionsstammform \\
	(\ref{ex:Konstituente2}): (mit den) Blickbewegungsmessern
	
	\scalebox{.9}{
	\alertgreen{
	\begin{forest} MyP edges,
		[N, name=N1
			[N, name=N2
				[N, name=N3
					[N [blick, tier=word]]
					[N, name=N4
						[V [beweg, tier=word]]
						[N\MyPup{af} [-ung(-s), tier=word]]
					]
				]
				[N, name=N5
					[V [mess, tier=word]]
					[N\MyPup{af} [-er, tier=word]]
				]
			]
			[FI [-n, tier=word]]
		]
	\draw[<-, HUgreen](N1.west)--++(-11em,0pt)
	node[anchor=east,align=center]{Flexion (KEIN Wortbildungsprozess)};
	\draw[<-, HUgreen](N2.west)--++(-13em,0pt)
	node[anchor=east,align=center]{Rektionskompositum};
	\draw[<-, HUgreen](N3.west)--++(-7.7em,0pt)
	node[anchor=east,align=center]{Rektionskompositum};
	\draw[<-, HUgreen](N4.east)--++(3.5em,0pt)--++(0pt,-16.2ex)
	node[anchor=north,align=center]{Derivation};
	\draw[<-, HUgreen](N5.east)--++(2em,0pt)--++(0pt,-23ex)
	node[anchor=north,align=center]{Derivation};
	\end{forest}
	} }
	
\end{itemize}

\end{frame}
%%%%%%%%%%%%%%%%%%%%%%%%%%%%%%%%%%

\begin{frame}
	
\begin{itemize}
	
	\item[10.] Geben Sie Beispiele für die folgenden Kompositionsarten an:
	
	\settowidth \jamwidth{\alertgreen{Romanleser, Tierkennerin, Konfliktbewältigung, \dots}}
	
	\eal
	\ex Determinativkomposition \\ 
	\only<2->{\jambox{\alertgreen{Apfelsaft, Taschenlampe, Hundeleine, \dots}}}
\medskip	
	\ex Rektionskomposition \\ 
	\only<3->{\jambox{\alertgreen{Romanleser, Tierkennerin, Konfliktbewältigung, \dots}}}
\medskip	
	\ex Possessivkomposition \\ 
	\only<4->{\jambox{\alertgreen{Dickkopf, Grünschnabel, Milchgesicht, \dots}}}
\medskip	
	\ex Kopulativkomposition \\ 
	\only<5->{\jambox{\alertgreen{nordost, Berlin-Brandenburg, blaugrau, \dots}}}
	\zl
	
\end{itemize}

\end{frame}
%%%%%%%%%%%%%%%%%%%%%%%%%%%%%%%%%%

\begin{frame}

\begin{itemize}
	\item[11.] Geben Sie je ein Beispiel für einen Stamm, für eine Wurzel, für eine Basis und für ein unikales Morphem an.
	
	\settowidth \jamwidth{\only<2->{\alertgreen{Fehlerkorrekturs}-stift}}
	
	\eal
	\ex Stamm: \jambox{\only<2->{\alertgreen{Fehlerkorrekturs}-stift}}
	
	\ex Wurzel: \jambox{\only<3->{Mittags-\alertgreen{schläf}-chen}}
	
	\ex Basis: \jambox{\only<4->{\alertgreen{Verständ}-nis}}
	
	\ex unikales Morphem: \jambox{\only<5->{ver-\alertgreen{letz}-en}}
	\zl
	
\end{itemize}

\end{frame}
%%%%%%%%%%%%%%%%%%%%%%%%%%%%%%%%%%
}

%%%%%%%%%%%%%%%%%%%%%%%%%%%%%%%%%%
%%%%%%%%%%%%%%%%%%%%%%%%%%%%%%%%%%
\subsection{Übungen: Syntax}
\iftoggle{sectoc}{
	\frame{
		%\begin{multicols}{2}
		\frametitle{~}
		\tableofcontents[currentsubsection,subsubsectionstyle=hide]
		%\end{multicols}
	}
}
%%%%%%%%%%%%%%%%%%%%%%%%%%%%%%%%%%

\begin{frame}
\frametitle{Übungen: Syntax}

\begin{itemize}
	\item[12.] Ordnen Sie die folgenden Matrixsätze und ihre Nebensätze in das topologische Feldermodell ein.
	
	\eal \label{ex:12}
	\ex Petra wirkt müde, obwohl sie nicht viel getanzt hat.
	\ex Wenn ich im Konzert bin, höre ich der Musik zu.
	\ex Die Frau, die hier arbeitet, obwohl die Heizung ausgeschaltet ist, ist leider krank geworden.
	\ex Anke hat gemerkt, dass Maria trotz der Erkältung arbeiten gegangen ist.
	\zl
	
\end{itemize}

\end{frame}


%%%%%%%%%%%%%%%%%%%%%%%%%%%%%%%%%%
\begin{frame}
%\frametitle{Übungen: Syntax}

\begin{itemize}
	\item[13.] Testen Sie anhand von jeweils zwei Konstituententests, ob die kursiv gesetzte Wortfolge eine Konstituente des Satzes bildet.
	
	\eal \label{ex:13}
	\ex Am Ende bekam Jakob das \emph{für Luise vorbereitete} Kostüm.
	\ex Er hatte sich das überlegt, \emph{weil Jakob wieder krank war}.
	\zl
	
	\item[14.] Analysieren Sie die folgenden Sätze nach dem X-Bar-Schema.
	\eal \label{ex:14}
	\ex Maria sieht Peter.
%	\ex Maria schlug gestern Peter.
	\ex Weil der nette Nachbar konzentriert gearbeitet hat, hat er erst an dem Abend Peter getroffen.
	\ex Über die Behandlung der zwei Patienten haben bis gestern die Ärzte diskutiert.
	\ex Luise hat gefragt, ob Jakob kommt.
	\zl
	
\end{itemize}

\end{frame}

%%%%%%%%%%%%%%%%%%%%%%%%%%%%%%%%%%

\iftoggle{ue-loesung}{
	%%%%%%%%%%%%%%%%%%%%%%%%%%%%%%%%%%
%% UE 4 - 09 Übungen
%%%%%%%%%%%%%%%%%%%%%%%%%%%%%%%%%%

\begin{frame}
\frametitle{Übung: Syntax -- Lösung}

\begin{itemize}
	\item[12.] Ordnen Sie die folgenden Matrixsätze und ihre Nebensätze in das topologische Feldermodell ein.
	
	\eal
	\ex Petra wirkt müde, obwohl sie nicht viel getanzt hat.
	\ex Wenn ich im Konzert bin, höre ich der Musik zu.
	\ex Die Frau, die hier arbeitet, obwohl die Heizung ausgeschaltet ist, ist leider krank geworden.
	\ex Anke hat gemerkt, dass Maria trotz der Erkältung arbeiten gegangen ist.
	\zl
	
\end{itemize}

\end{frame}
%%%%%%%%%%%%%%%%%%%%%%%%%%%%%%%%%%
\begin{frame}
	
	\begin{table}
		\centering
		\scalebox{.75}{
		\alertgreen{
		\begin{tabular}{p{3.5cm}|l|p{4cm}|p{2cm}|p{3cm}}
			\textbf{VF} & \textbf{LSK} & \textbf{MF} & \textbf{RSK} & \textbf{NF} \\
			\hline
			Petra & wirkt & müde, & & obwohl sie nicht viel getanzt hat.\\
			\hline
			& obwohl & sie nicht viel & getanzt hat. & \\
			\hline
			\hline
			Wenn ich im Konzert bin, & höre & ich der Musik & zu. & \\
			\hline
			& Wenn & ich im Konzert & bin, & \\
			\hline
			\hline
			Die Frau, die hier arbeitet, obwohl die Heizung ausgeschaltet ist, & ist & leider krank & geworden. & \\
			\hline
			die & & hier & arbeitet, & obwohl die Heizung ausgeschaltet ist, \\
			\hline
			& obwohl & die Heizung & ausgeschaltet ist, & \\
			\hline
			\hline
			Anke & hat & & gemerkt, & dass Maria trotz der Erkältung arbeiten gegangen ist. \\
			\hline
			& dass & Maria trotz der Erkältung & arbeiten gegangen ist. & \\
		\end{tabular}
		} }
	\end{table}

\end{frame}
%%%%%%%%%%%%%%%%%%%%%%%%%%%%%%%%%%

\begin{frame}

\begin{itemize}
	
	\item[13.] Testen Sie anhand von jeweils zwei Konstituententests, ob die kursiv gesetzte Wortfolge eine Konstituente des Satzes bildet.
	
	\eal
	\ex Am Ende bekam Jakob das \emph{für Luise vorbereitete} Kostüm.

	\only<2->{
		\begin{itemize}
			\item \alertgreen{Vorfeldtest: *Für Luise vorbereitete bekam am Ende Jakob das Kostüm.}
			\item \alertgreen{Fragetest: Was bekam Jakob am Ende? \textit{*Für Luise vorbereitete.}}
		\end{itemize}
	}
	
	\alertgreen{\only<2->{$\rightarrow$ [für Luise vorbereitete] ist keine Konstituente}}
	
\medskip
	
	\ex Er hatte sich das überlegt, \emph{weil Jakob wieder krank war}.

	\only<3->{
		\begin{itemize}
			\item \alertgreen{Vorfeldtest: Weil Jakob wieder krank war, hatte er sich das überlegt.}
			\item \alertgreen{Fragetest: Warum hatte er sich das überlegt? \textit{Weil Jakob wieder krank war.}}
		\end{itemize}
	}
	
	\alertgreen{\only<3->{$\rightarrow$ [weil Jakob wieder krank war] ist eine Konstituente}}
	\zl
		
\end{itemize}

\end{frame}
%%%%%%%%%%%%%%%%%%%%%%%%%%%%%%%%%%

\begin{frame}
	
	\begin{itemize}
		\item[14.] Analysieren Sie die folgenden Sätze nach dem X-Bar-Schema.
		\eal
		\ex \label{ex:XBar1}{Maria sieht Peter.}
		\ex \label{ex:XBar2}{Weil der nette Nachbar konzentriert gearbeitet hat, hat er erst an dem Abend Peter getroffen.}
		\ex \label{ex:XBar3}{Über die Behandlung der zwei Patienten haben bis gestern die Ärzte diskutiert.}
		\ex \label{ex:XBar4}{Luise hat gefragt, ob Jakob kommt.}
		\zl
		
	\end{itemize}

\end{frame}
%%%%%%%%%%%%%%%%%%%%%%%%%%%%%%%%%%

\begin{frame}
	
(\ref{ex:XBar1}): Maria sieht Peter.
		
	\centering
	\scalebox{.56}{
			\alertgreen{
					
				\begin{forest}
					sm edges, empty nodes
					[CP
						[DP$_{i}$
							[\MyPxbar{D}
								[\zerobar{D} [$\emptyset$]]
								[NP
									[\MyPxbar{N}
										[\zerobar{N} [Maria]]
									]
								]
							]
						]
						[\MyPxbar{C}
							[\zerobar{C} [sieht$_{ii}$]]
							[IP
								[[t$_{i}$]]
								[\MyPxbar{I}
									[VP
										[\MyPxbar{V}
											[DP
												[\MyPxbar{D}
													[\zerobar{D} [$\emptyset$]]
													[NP
														[\MyPxbar{N}
															[\zerobar{N} [Peter]]
														]
													]
												]
											]
											[\zerobar{V} [t$_{ii}$]]
										]
									]
									[\zerobar{I} [t$_{ii}$]]
								]
							]
						]
					]
				\end{forest}
				}}

\end{frame}

%%%%%%%%%%%%%%%%%%%%%%%%%%%%%%%%%%

\begin{frame}

(\ref{ex:XBar2}): Weil der nette Nachbar konzentriert gearbeitet hat, hat er erst an dem Abend Peter getroffen.
			
	\centering
	\scalebox{.45}{
			\alertgreen{
				\begin{forest}
					sm edges, empty nodes
					[CP
						[CP$_{i}$
							[\MyPxbar{C}
								[\zerobar{C} [weil]]
								[IP
									[DP
										[\MyPxbar{D}
											[\zerobar{D} [der]]
											[NP
												[AP
													[\MyPxbar{A}
														[\zerobar{A} [nette]]
													]
												]
												[NP
													[\MyPxbar{N}
														[\zerobar{N} [Nachbar]]
													]
												]
											]
										]
									]
									[\MyPxbar{I}
										[VP
											[AdvP
												[\MyPxbar{Adv}
													[\zerobar{Adv} [konzentriert]]
												]
											]
											[VP
												[\MyPxbar{V}
													[\zerobar{V} [gearbeitet]]
												]
											]
										]
										[\zerobar{I} [hat]]
									]
								]
							]
						]
						[\MyPxbar{C}
							[\zerobar{C} [hat$_{ii}$]]
							[IP
								[DP
									[\MyPxbar{D}
										[\zerobar{D} [er]]
									]
								]
								[\MyPxbar{I}
									[VP
										[PP
											[AdvP
												[\MyPxbar{Adv}
													[\zerobar{Adv} [erst]]
												]
											]
											[PP
												[\MyPxbar{P}
													[\zerobar{P} [an]]
													[DP
														[\MyPxbar{D}
															[\zerobar{D} [dem]]
															[NP
																[\MyPxbar{N}
																	[\zerobar{N} [Abend]]
																]
															]
														]
													]
												]
											]
										]
										[VP
											[[t$_{i}$]]
											[VP
												[\MyPxbar{V}
													[DP
														[\MyPxbar{D}
															[\zerobar{D} [$\emptyset$]]
															[NP
																[\MyPxbar{N}
																	[\zerobar{N} [Peter]]
																]
															]		
														]
													]
												[\zerobar{V} [getroffen]]
												]
											]
										]
									]
									[\zerobar{I} [t$_{ii}$]]
								]
							]
						]
					]			
				\end{forest}
				}}
			
\end{frame}	

%%%%%%%%%%%%%%%%%%%%%%%%%%%%%%%%%%

\begin{frame}

(\ref{ex:XBar3}): Über die Behandlung der zwei Patienten haben bis gestern die Ärzte diskutiert.

	\centering
	\scalebox{.47}{
			\alertgreen{
				\begin{forest}
					sm edges, empty nodes
					[CP
						[PP$_{ii}$
							[\MyPxbar{P}
								[\zerobar{P} [über]]
								[DP
									[\MyPxbar{D}
										[\zerobar{D} [die]]
										[NP
											[\MyPxbar{N}
												[\zerobar{N} [Behandlung]]
												[DP
													[\MyPxbar{D}
														[\zerobar{D} [der]]
														[NP
															[AP
																[\MyPxbar{A}
																	[\zerobar{A} [zwei]]
																]
															]
															[NP
																[\MyPxbar{N}
																	[\zerobar{N} [Patienten]]
																]
															]
														]
													]
												]
											]
										]
									]
								]
							]
						]
						[\MyPxbar{C}
							[\zerobar{C} [haben$_{i}$]]
							[IP
								[PP
									[\MyPxbar{P}
										[\zerobar{P} [bis]]
									[AdvP
										[\MyPxbar{Adv}
											[\zerobar{Adv} [gestern]]
										]
									]
									]
								]
								[IP
									[DP
										[\MyPxbar{D}
											[\zerobar{D} [die]]
											[NP
												[\MyPxbar{N}
													[\zerobar{N} [Ärzte]]
												]
											]
										]
									]
									[\MyPxbar{I}
										[VP
											[\MyPxbar{V}
												[[t$_{ii}$]]
												[\zerobar{V} [diskutiert]]
											]
										]
										[\zerobar{I} [t$_{i}$]]
									]
								]
							]
						]
					]
				\end{forest}
				}}
			
\end{frame}

%%%%%%%%%%%%%%%%%%%%%%%%%%%%%%%%%%

\begin{frame}

(\ref{ex:XBar4}): Luise hat gefragt, ob Jakob kommt.
			
	\centering
	\scalebox{.63}{
			\alertgreen{
				\begin{forest}
					sm edges, empty nodes
					[CP
						[DP$_{i}$
							[\MyPxbar{D}
								[\zerobar{D} [$\emptyset$]]
								[NP
									[\MyPxbar{N}
										[\zerobar{N} [Luise]]
									]
								]
							]
						]
						[\MyPxbar{C}
							[\zerobar{C} [hat$_{ii}$]]
							[IP
								[IP
									[[t$_{i}$]]
									[\MyPxbar{I}
										[VP
											[\MyPxbar{V}
												[[t$_{iii}$]]
												[\zerobar{V} [gefragt]]
											]
										]
										[\zerobar{I} [t$_{ii}$]]
									]
								]
								[CP$_{iii}$
									[\MyPxbar{C}
										[\zerobar{C} [ob]]
										[IP
											[DP
												[\MyPxbar{D}
													[\zerobar{D} [er]]
												]
											]
											[\MyPxbar{I}
												[VP
													[\MyPxbar{V}
														[\zerobar{V} [t$_{iv}$]]
													]
												]
												[\zerobar{I} [kommt$_{iv}$]]
											]
										]
									]
								]
							]
						]
					]
				\end{forest}
				}}			

\end{frame}

%%%%%%%%%%%%%%%%%%%%%%%%%%%%%%%%%%
}

%%%%%%%%%%%%%%%%%%%%%%%%%%%%%%%%%%
%%%%%%%%%%%%%%%%%%%%%%%%%%%%%%%%%%
\subsection{Übungen: Semantik/Pragmatik}
\iftoggle{sectoc}{
	\frame{
		%\begin{multicols}{2}
		\frametitle{~}
		\tableofcontents[currentsubsection,subsubsectionstyle=hide]
		%\end{multicols}
	}
}
%%%%%%%%%%%%%%%%%%%%%%%%%%%%%%%%%%

\begin{frame}
\frametitle{Übungen: Semantik/Pragmatik}

\begin{itemize}
	\item[15.] Geben Sie die Bedeutungsrelationen (so genau wie möglich) zwischen den folgenden Wörtern an.
	\eal \label{ex:15}
	\ex satt -- hungrig
	\ex erwerben -- kaufen
	\ex Haare -- Kopf
	\ex schuldig -- nicht schuldig
	\ex heute -- Häute
	\ex Tiger -- Katze
	\ex fruchtbar -- unfruchtbar
	\zl
	
	\item[16.] Illustrieren Sie die Begriffe Satzbedeutung, Äußerungsbedeutung und Sprecherbedeutung mithilfe des folgenden Satzes.
	
	\ea \label{ex:16} Ich glaube, du gehst jetzt! \\
	\ras Peter zu Klaus am 25. August 2020 um 20:30 Uhr.
	\z
	
\end{itemize}

\end{frame}


%%%%%%%%%%%%%%%%%%%%%%%%%%%%%%%%%%
\begin{frame}
%\frametitle{Übungen: Semantik/Pragmatik}

\begin{itemize}
	\item[17.] Geben Sie die Bedeutungsrelationen zwischen den folgenden Sätzen an.
	
	\eal \label{ex:17}
	\ex Hinter dem Baum steht ein Bär.
	\ex Hinter dem Baum steht ein Tier.
	\zl
	
	\eal \label{ex:18}
	\ex Peter fängt an zu arbeiten.
	\ex Peter nimmt die Arbeit auf.
	\zl
	
	\eal \label{ex:19}
	\ex Sandra ist groß.
	\ex Sandra ist nicht-groß.
	\zl
	
	\eal \label{ex:20}
	\ex Ich habe alle Studenten gesehen.
	\ex Ich habe nicht einen Studenten nicht gesehen.
	\zl
	
	\eal \label{ex:21}
	\ex Maria geht wandern.
	\ex Maria macht eine Kreuzfahrt.
	\zl
	
	\eal \label{ex:22}
	\ex Gert ist verletzt.
	\ex Gert hat ein gebrochenes Bein.
	\zl
	
\end{itemize}

\end{frame}


%%%%%%%%%%%%%%%%%%%%%%%%%%%%%%%%%%
\begin{frame}
%\frametitle{Übungen: Semantik/Pragmatik}

\begin{itemize}
	
	\item[18.] Geben Sie eine Wahrheitswerttabelle für den folgenden aussagenlogischen Ausdruck an und bestimmen Sie, ob es sich dabei um eine tautologische, eine kontradiktorische oder eine kontingente Aussage handelt.
	\ea \label{ex:23} $((p \rightarrow q) \lor q)$
	\z 
	
	\item[19.] Markieren Sie alle deiktischen und anaphorischen Elemente in den folgenden Sätzen und spezifizieren Sie diese.
	
	\eal \label{ex:24}
	\ex Sie haben diese Tür nicht geschlossen.
	\ex Gestern war mir das Wetter echt zu kalt!
	\ex Peter wusste, dass er es sich dort gemütlich machen würde. 
%	\ex \gqq{Ich bin sehr glücklich, mich wieder für das WTA-Finale qualifiziert zu haben. Ich freue mich darauf, dort anzutreten und gegen die Besten der Welt zu spielen}, sagte die 27-Jährige, die im vergangenen Jahr nur als Ersatzspielerin mitfahren durfte.
	\zl
	
\end{itemize}

\end{frame}


%%%%%%%%%%%%%%%%%%%%%%%%%%%%%%%%%%
\begin{frame}
%\frametitle{Übungen: Semantik/Pragmatik}

\begin{itemize}
	\item[20.] Bestimmen Sie die Art von Folgerung (Implikation, Präsupposition, Implikatur), die zwischen dem ersten und den folgenden Sätzen besteht:
	
	\ea \label{ex:25} Sogar Peter hat zwei Kinder. 
	\ea Peter hat nicht mehr als zwei Kinder.
	\ex Es gibt ein Individuum namens Peter.
	\ex Peter ist Vater.
	\ex Peter hat vier Kinder.
	\ex Überraschenderweise hat Peter Kinder.
	\z
	\z 
	
	\item[21.] Bestimmen Sie jeweils eine semantische Implikation aus den folgenden Sätzen:
	
	\eal \label{ex:26}
	\ex In einem Schuhkarton gibt es Platz für zwei Schuhe.
	\ex Saskia hat eine Schwedin geheiratet.
	\zl
		
\end{itemize}

\end{frame}


%%%%%%%%%%%%%%%%%%%%%%%%%%%%%%%%%%
\begin{frame}
%\frametitle{Übungen: Semantik/Pragmatik}

\begin{itemize}
	
	\item[22.] Bestimmen Sie jeweils eine Präsupposition aus den folgenden Sätzen:
	\eal \label{ex:27}
	\ex Ich freue mich darüber, dass wir die Klausur bestanden haben.
	\ex Auch Maria ist schwanger.
	\ex Alle Geiseln wurden gerettet.
	\ex Sie mögen immer noch Syntax.
	\zl

	\item[23.] Geben Sie an, ob eine Maxime (scheinbar) verletzt oder befolgt wurde und um welche es sich handelt, um die angegebene Implikatur zu erhalten.
	
	\ea \label{ex:28} Wir haben einige Personen entlassen.\\
	$+>$ Es wurden nicht alle entlassen.
	\z
	
	\ea \label{ex:29} A: Wie war das Bewerbungsgespräch?\\
	B: Das Wetter ist ja super heute!\\
	$+>$ Es war furchtbar!
	\z
		
\end{itemize}

\end{frame}

%%%%%%%%%%%%%%%%%%%%%%%%%%%%%%%%%%

\iftoggle{ue-loesung}{
	%%%%%%%%%%%%%%%%%%%%%%%%%%%%%%%%%%
%% UE 5 - 09 Übungen
%%%%%%%%%%%%%%%%%%%%%%%%%%%%%%%%%%

\begin{frame}
\frametitle{Übung: Semantik/Pragmatik -- Lösung}

\begin{itemize}
	\item[15.] Geben Sie die Bedeutungsrelationen (so genau wie möglich) zwischen den folgenden Wörtern an.
	
	\begin{exe}
		\exr{ex:15}
	\settowidth \jamwidth{\alertgreen{kontradiktorische Antonymie}}
	
		\begin{xlist}
			\ex satt -- hungrig \only<2->{\jambox{\alertgreen{konträre Antonymie}}}
			\ex erwerben -- kaufen \only<3->{\jambox{\alertgreen{(partielle) Synonymie}}}
			\ex Haare -- Kopf \only<4->{\jambox{\alertgreen{Meronymie}}}
			\ex schuldig -- nicht schuldig \only<5->{\jambox{\alertgreen{kontradiktorische Antonymie}}}
			\ex heute -- Häute \only<6->{\jambox{\alertgreen{Homophonie}}}
			\ex Tiger -- Katze \only<7->{\jambox{\alertgreen{Hyponymie}}}
			\ex fruchtbar -- unfruchtbar \only<8->{\jambox{\alertgreen{kontradiktorische Antonymie}}}
		\end{xlist}
	
	\end{exe}
	
\end{itemize}

\end{frame}

%%%%%%%%%%%%%%%%%%%%%%%%%%%%%%%%%%

\begin{frame}
	
\begin{itemize}
	\item[16.] Illustrieren Sie die Begriffe Satzbedeutung, Äußerungsbedeutung und Sprecherbedeutung mithilfe des folgenden Satzes.
	
	\begin{exe}
		\exr{ex:16} Ich glaube, du gehst jetzt! \\
		\ras Peter zu Klaus am 25. August 2020 um 20:30 Uhr.
	\end{exe}
	
	\item Satzbedeutung: \\ \only<2->{\alertgreen{Der Sprecher des Satzes glaubt (zum Zeitpunkt der Äußerung), dass der Adressat der Äußerung geht.}}
	\item Äußerungsbedeutung: \\ \only<3->{\alertgreen{Klaus glaubt, dass Peter am 25. August 2020 um 20:30 Uhr geht.}}
	\item Sprecherbedeutung: \\ \only<4->{\alertgreen{Peter fordert Klaus bestimmt auf (\zB nach einer Auseinandersetzung) sofort zu gehen.}}
	
\end{itemize}

\end{frame}

%%%%%%%%%%%%%%%%%%%%%%%%%%%%%%%%%%

\begin{frame}

\begin{itemize}
	\item[17.] Geben Sie die Bedeutungsrelationen zwischen den folgenden Sätzen an.
	
	\begin{exe}
		\exr{ex:17}
	\settowidth \jamwidth{\only<2->{\alertgreen{\ras a impliziert b}}}
	
		\begin{xlist}
			\ex Hinter dem Baum steht ein Bär. \jambox{\only<2->{\alertgreen{Implikation}}}
			\ex Hinter dem Baum steht ein Tier. \jambox{\only<2->{\alertgreen{\ras a impliziert b}}}
		\end{xlist}
	
		\exr{ex:18}
		
		\begin{xlist}
			\ex Peter fängt an zu arbeiten. \jambox{\only<3->{\alertgreen{Paraphrase}}}
			\ex Peter nimmt die Arbeit auf.
		\end{xlist}
		
		\exr{ex:19}
		
		\begin{xlist}
			\ex Sandra ist groß. \jambox{\only<4->{\alertgreen{Kontradiktion}}}
			\ex Sandra ist nicht-groß.
		\end{xlist}
		
		\exr{ex:20}
		
		\begin{xlist}
			\ex Ich habe alle Studenten gesehen. \jambox{\only<5->{\alertgreen{Paraphrase}}}
			\ex Ich habe nicht einen Studenten nicht gesehen.
		\end{xlist}
	
		\exr{ex:21}
		
		\begin{xlist}
			\ex Maria geht wandern. \jambox{\only<6->{\alertgreen{Inkompatibilität}}}
			\ex Maria macht eine Kreuzfahrt.
		\end{xlist}
		
		\exr{ex:22}
		
		\begin{xlist}
			\ex Gert ist verletzt. \jambox{\only<7->{\alertgreen{Implikation}}}
			\ex Gert hat ein gebrochenes Bein. \jambox{\only<6->{\alertgreen{\ras b impliziert a}}}
		\end{xlist}
		
	\end{exe}

\end{itemize}
	
\end{frame}

%%%%%%%%%%%%%%%%%%%%%%%%%%%%%%%%%%

\begin{frame}
	
\begin{itemize}
		
	\item[18.] Geben Sie eine Wahrheitswerttabelle für den folgenden aussagenlogischen Ausdruck an und bestimmen Sie, ob es sich dabei um eine tautologische, eine kontradiktorische oder eine kontingente Aussage handelt.
	
	\begin{exe}
		\exr{ex:23} $((p \rightarrow q) \lor q)$
	\end{exe}
		
		\begin{table}
			\centering
			\scalebox{.95}{
				\only<2->{\alertgreen{
					\begin{tabular}{c|c|c|c}
					p & q & $(p \rightarrow q)$ & $((p \rightarrow q) \lor q)$ \\
					\hline
					1 & 1 & 1 & 1 \\
					\hline
					1 & 0 & 0 & 0 \\
					\hline
					0 & 1 & 1 & 1 \\
					\hline
					0 & 0 & 1 & 1 \\
					\end{tabular}
					}}}
		\end{table}

\medskip
	
	\only<2->{\item \alertgreen{Bei dem vorangehenden aussagenlogischen Ausdruck handelt es sich um eine kontingente Aussage.}}
				
\end{itemize}
	
\end{frame}

%%%%%%%%%%%%%%%%%%%%%%%%%%%%%%%%%%

\begin{frame}

\begin{itemize}
	\item[19.] Markieren Sie alle deiktischen und anaphorischen Elemente in den folgenden Sätzen und spezifizieren Sie diese.
		
	\begin{exe}
		\exr{ex:24}
		
		\begin{xlist}
			\ex Sie haben diese Tür nicht geschlossen.
			\ex Gestern war mir das Wetter echt zu kalt!
			\ex Peter wusste, dass er es sich dort gemütlich machen würde.
		\end{xlist}
	
	\end{exe}

\end{itemize}

\end{frame}

%%%%%%%%%%%%%%%%%%%%%%%%%%%%%%%%%%

\begin{frame}

\begin{itemize}
	\item[19.] Markieren Sie alle deiktischen und anaphorischen Elemente in den folgenden Sätzen und spezifizieren Sie diese.
	
	\begin{exe}
		\exr{ex:24}
		
		\begin{xlist}
			\ex \alertgreen{Sie} haben \alertgreen{diese} Tür nicht geschlossen.
			\ex \alertgreen{Gestern} war \alertgreen{mir} das Wetter echt zu kalt!
			\ex Peter wusste, dass \alertblue{er} es \alertblue{sich} \alertgreen{dort} gemütlich machen würde. 
			%	\ex \gqq{Ich bin sehr glücklich, mich wieder für das WTA-Finale qualifiziert zu haben. Ich freue mich darauf, dort anzutreten und gegen die Besten der Welt zu spielen}, sagte die 27-Jährige, die im vergangenen Jahr nur als Ersatzspielerin mitfahren durfte.
		\end{xlist}
	
	\end{exe}

\end{itemize}	
	
	\begin{minipage}[t]{0.35\textwidth}

	\begin{itemize}
		\item \alertgreen{Deiktische Elemente:}
		\only<2->{
		
			\begin{itemize}
				 \item \only<2->{\alertgreen{Sie: Sozialdeixis}}
				 \item \only<2->{\alertgreen{diese: Objektdeixis}}
				 \item \only<2->{\alertgreen{gestern: Temporaldeixis}}
				 \item \only<2->{\alertgreen{mir: Personaldeixis}}
				 \item \only<2->{\alertgreen{dort: Lokaldeixis}} 
			\end{itemize}		
			}

	\end{itemize}
	
	\end{minipage}
	\begin{minipage}[t]{0.60\textwidth}
	
	\begin{itemize}
		\item \alertblue{Anaphorische Elemente:}
		\only<2->{
		
			\begin{itemize}
				\item \only<2->{\alertblue{er: anaphorischer Ausdruck; Antezedens: \textit{Peter}}}
				\item \only<2->{\alertblue{sich: anaphorischer Ausdruck; Antezedens: \textit{er}}}
			\end{itemize}
			}
		
	\end{itemize}

	\end{minipage}


\end{frame}

%%%%%%%%%%%%%%%%%%%%%%%%%%%%%%%%%%

\begin{frame}
	
\begin{itemize}
	\item[20.] Bestimmen Sie die Art von Folgerung (Implikation, Präsupposition, Implikatur), die zwischen dem ersten und den folgenden Sätzen besteht:
	
	\begin{exe}
		\exr{ex:25} Sogar Peter hat zwei Kinder.
	\settowidth \jamwidth{\only<2->{\alertgreen{konversationalle Implikatur}}}
		
		\begin{xlist}
			\ex Peter hat nicht mehr als zwei Kinder. \jambox{\only<2->{\alertgreen{konversationalle Implikatur}}}
			\ex Es gibt ein Individuum namens Peter. \jambox{\only<3->{\alertgreen{Präsupposition}}}
			\ex Peter ist Vater. \jambox{\only<4->{\alertgreen{semantische Implikation}}}
			\ex Peter hat vier Kinder. \jambox{\only<5->{\alertgreen{keine Folgerung}}}
			\ex Überraschenderweise hat Peter Kinder. \jambox{\only<6->{\alertgreen{konventionelle Implikatur}}}
		\end{xlist} 
		
	\end{exe}

\end{itemize}
	
\end{frame}

%%%%%%%%%%%%%%%%%%%%%%%%%%%%%%%%%%

\begin{frame}

\begin{itemize}
	\item[21.] Bestimmen Sie jeweils eine semantische Implikation aus den folgenden Sätzen:
	
	\begin{exe}
		\exr{ex:26}
		
		\begin{xlist}
			\ex In einem Schuhkarton gibt es Platz für zwei Schuhe.
			\ex Saskia hat eine Schwedin geheiratet.
		\end{xlist}

	\end{exe}
	
\end{itemize}
	
	\only<2->{\alertgreen{(26a): $\vDash$ In einem Schuhkarton gibt es Platz für einen Schuh.}} ~\\
\medskip
	\only<3->{\alertgreen{(26b): $\vDash$ Saskia hat eine Nordeuropäerin geheiratet.}}
	

\end{frame}

%%%%%%%%%%%%%%%%%%%%%%%%%%%%%%%%%%

\begin{frame}
	
\begin{itemize}
		
	\item[22.] Bestimmen Sie jeweils eine Präsupposition aus den folgenden Sätzen:
	
	\begin{exe}
		\exr{ex:27}
		
		\begin{xlist}
			\ex Ich freue mich darüber, dass wir die Klausur bestanden haben.
			\ex Auch Maria ist schwanger.
			\ex Alle Geiseln wurden gerettet.
			\ex Sie mögen immer noch Syntax.
		\end{xlist}

	\end{exe}
		
\end{itemize}

	\only<2->{\alertgreen{(27a): \prspp Wir haben die Klausur bestanden.}} ~\\
\medskip
	\only<3->{\alertgreen{(27b): \prspp Mindestens eine weitere Entität neben Maria ist schwanger.}} ~\\
\medskip
	\only<4->{\alertgreen{(27c): \prspp Die Geiseln waren in Gefahr.}} ~\\
\medskip
	\only<5->{\alertgreen{(27d): \prspp Sie mochten bisher Syntax.}}
	
\end{frame}

%%%%%%%%%%%%%%%%%%%%%%%%%%%%%%%%%%

\begin{frame}
	
\begin{itemize}
	\item[23.] Geben Sie an, ob eine Maxime (scheinbar) verletzt oder befolgt wurde und um welche es sich handelt, um die angegebene Implikatur zu erhalten.
	
	\begin{exe}
		\exr{ex:28} Wir haben einige Personen entlassen.\\
		$+>$ Es wurden nicht alle entlassen.
	\end{exe}
	
	\begin{exe}
		\exr{ex:29} A: Wie war das Bewerbungsgespräch?\\
		B: Das Wetter ist ja super heute!\\
		$+>$ Es war furchtbar!
	\end{exe}
	
\end{itemize}

	\only<2->{\alertgreen{(\ref{ex:28}): Befolgung der Quantitätsmaxime}} ~\\
\medskip
	\only<3->{\alertgreen{(\ref{ex:29}): (scheinbare) Verletzung der Relevanzmaxime}}

\end{frame}

%%%%%%%%%%%%%%%%%%%%%%%%%%%%%%%%%%
}

%%%%%%%%%%%%%%%%%%%%%%%%%%%%%%%%%%
%%%%%%%%%%%%%%%%%%%%%%%%%%%%%%%%%%
