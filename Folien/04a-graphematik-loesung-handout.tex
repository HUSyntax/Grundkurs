%%%%%%%%%%%%%%%%%%%%%%%%%%%%%%%%%%%%%%%%%%%%%%
%% Compile: XeLaTeX BibTeX XeLaTeX XeLaTeX
%% Loesung-Handout: Antonio Machicao y Priemer
%% Course: GK Linguistik
%%%%%%%%%%%%%%%%%%%%%%%%%%%%%%%%%%%%%%%%%%%%%%

%\documentclass[a4paper,10pt, bibtotoc]{beamer}
\documentclass[10pt,handout]{beamer}

%%%%%%%%%%%%%%%%%%%%%%%%
%%     PACKAGES      
%%%%%%%%%%%%%%%%%%%%%%%%

%%%%%%%%%%%%%%%%%%%%%%%%
%%     PACKAGES       %%
%%%%%%%%%%%%%%%%%%%%%%%%



%\usepackage[utf8]{inputenc}
%\usepackage[vietnamese, english,ngerman]{babel}   % seems incompatible with german.sty
%\usepackage[T3,T1]{fontenc} breaks xelatex
\usepackage{lmodern}

\usepackage{amsmath}
\usepackage{amsfonts}
\usepackage{amssymb}
%% MnSymbol: Mathematische Klammern und Symbole (Inkompatibel mit ams-Packages!)
%% Bedeutungs- und Graphemklammern: $\lsem$ Tisch $\rsem$ $\langle TEXT \rangle$ $\llangle$ TEXT $\rrangle$ 
\usepackage{MnSymbol}
%% ulem: Strike out
\usepackage[normalem]{ulem}  

%% Special Spaces (s. Commands)
\usepackage{xspace}				
\usepackage{setspace}
%	\onehalfspacing

%% mdwlist: Special lists
\usepackage{mdwlist}	

\usepackage[noenc,safe]{tipa}

% maybe define \textipa to use \originalTeX to avoid problems with `"'.
%
%	\ex \textipa{\originalTeX [pa.pa."g\t{aI}]}

%

\usepackage{etex}		%For Forest bug

%
%\usepackage{jambox}
%


%\usepackage{forest-v105}
%\usepackage{modified-langsci-forest-setup}

\usepackage{xeCJK}
\setCJKmainfont{SimSun}


%\usepackage{natbib}
%\setcitestyle{notesep={:~}}




% for toggles
\usepackage{etex}



% Fraktur!
\usepackage{yfonts}

\usepackage{url}

% für UDOP
\usepackage{adjustbox}


%% huberlin: Style sheet
%\usepackage{huberlin}
\usepackage{hu-beamer-includes-pdflatex}
\huberlinlogon{0.86cm}


%% Last Packages
%\usepackage{hyperref}	%URLs
%\usepackage{gb4e}		%Linguistic examples

% sorry this was incompatible with gb4e and had to go.
%\usepackage{linguex-cgloss}	%Linguistic examples (patched version that works with jambox

\usepackage{multirow}  %Mehrere Zeilen in einer Tabelle
%\usepackage{array}
\usepackage{marginnote}	%Notizen




%%%%%%%%%%%%%%%%%%%%%%%%%%%%%%%%%%%%%%%%%%%%%%%%%%%%
%%%          Commands                            %%%
%%%%%%%%%%%%%%%%%%%%%%%%%%%%%%%%%%%%%%%%%%%%%%%%%%%%

%%%%%%%%%%%%%%%%%%%%%%%%%%%%%%%%
% German quotation marks:
\newcommand{\gqq}[1]{\glqq{}#1\grqq{}}		%double
\newcommand{\gq}[1]{\glq{}#1\grq{}}			%simple


%%%%%%%%%%%%%%%%%%%%%%%%%%%%%%%%
% Abbreviations in German
% package needed: xspace
% Short space in German abbreviations: \,	
\newcommand{\idR}{\mbox{i.\,d.\,R.}\xspace}
\newcommand{\su}{\mbox{s.\,u.}\xspace}
%\newcommand{\ua}{\mbox{u.\,a.}\xspace}       % in abbrev
%\newcommand{\zB}{\mbox{z.\,B.}\xspace}       % in abbrev
%\newcommand{\s}{s.~}
%not possibel: \dh --> d.\,h.


%%%%%%%%%%%%%%%%%%%%%%%%%%%%%%%%
%Abbreviations in English
\newcommand{\ao}{a.o.\ }	% among others
\newcommand{\cf}[1]{(cf.~#1)}	% confer = compare
\renewcommand{\ia}{i.a.}	% inter alia = among others
\newcommand{\ie}{i.e.~}	% id est = that is
\newcommand{\fe}{e.g.~}	% exempli gratia = for example
%not possible: \eg --> e.g.~
\newcommand{\vs}{vs.\ }	% versus
\newcommand{\wrt}{w.r.t.\ }	% with respect to


%%%%%%%%%%%%%%%%%%%%%%%%%%%%%%%%
% Dash:
\newcommand{\gs}[1]{--\,#1\,--}


%%%%%%%%%%%%%%%%%%%%%%%%%%%%%%%%
% Rightarrow with and without space
\def\ra{\ensuremath\rightarrow}			%without space
\def\ras{\ensuremath\rightarrow\ }		%with space


%%%%%%%%%%%%%%%%%%%%%%%%%%%%%%%%
%% X-bar notation

%% Notation with primes (not emphasized): \xbar{X}
\newcommand{\MyPxbar}[1]{#1$^{\prime}$}
\newcommand{\xxbar}[1]{#1$^{\prime\prime}$}
\newcommand{\xxxbar}[1]{#1$^{\prime\prime\prime}$}

%% Notation with primes (emphasized): \exbar{X}
\newcommand{\exbar}[1]{\emph{#1}$^{\prime}$}
\newcommand{\exxbar}[1]{\emph{#1}$^{\prime\prime}$}
\newcommand{\exxxbar}[1]{\emph{#1}$^{\prime\prime\prime}$}

% Notation with zero and max (not emphasized): \xbar{X}
\newcommand{\zerobar}[1]{#1$^{0}$}
\newcommand{\maxbar}[1]{#1$^{\textsc{max}}$}

% Notation with zero and max (emphasized): \xbar{X}
\newcommand{\ezerobar}[1]{\emph{#1}$^{0}$}
\newcommand{\emaxbar}[1]{\emph{#1}$^{\textsc{max}}$}

%% Notation with bars (already implemented in gb4e):
% \obar{X}, \ibar{X}, \iibar{X}, \mbar{X} %Problems with \mbar!
%
%% Without gb4e:
\newcommand{\overbar}[1]{\mkern 1.5mu\overline{\mkern-1.5mu#1\mkern-1.5mu}\mkern 1.5mu}
%
%% OR:
\newcommand{\MyPibar}[1]{$\overline{\textrm{#1}}$}
\newcommand{\MyPiibar}[1]{$\overline{\overline{\textrm{#1}}}$}
%% (emphasized):
\newcommand{\eibar}[1]{$\overline{#1}$}
\newcommand{\eiibar}[1]{\overline{$\overline{#1}}$}

%%%%%%%%%%%%%%%%%%%%%%%%%%%%%%%%
%% Subscript & Superscript: no italics
\newcommand{\MyPdown}[1]{$_{\textrm{#1}}$}
\newcommand{\MyPup}[1]{$^{\textrm{#1}}$}


%%%%%%%%%%%%%%%%%%%%%%%%%%%%%%%%
% Objekt language marking:
%\newcommand{\obj}[1]{\glqq{}#1\grqq{}}	%German double quotes
%\newcommand{\obj}[1]{``#1''}			%English double quotes
\newcommand{\MyPobj}[1]{\emph{#1}}		%Emphasising


%%%%%%%%%%%%%%%%%%%%%%%%%%%%%%%%
%% Semantic types (<e,t>), features, variables and graphemes in angled brackets 

%%% types and variables, in math mode: angled brackets + italics + no space
%\newcommand{\type}[1]{$<#1>$}

%%% OR more correctly: 
%%% types and variables, in math mode: chevrons! + italics + no space
\newcommand{\MyPtype}[1]{$\langle #1 \rangle$}

%%% features and graphemes, in math mode: chevrons! + italics + no space
\newcommand{\abe}[1]{$\langle #1 \rangle$}


%%% features and graphemes, in math mode: chevrons! + no italics + space
\newcommand{\ab}[1]{$\langle$#1$\rangle$}  %%same as \abu  
\newcommand{\abu}[1]{$\langle$#1$\rangle$} %%Umlaute

%%% Notizen
\renewcommand{\marginfont}{\singlespacing}
\renewcommand{\marginfont}{\footnotesize}
\renewcommand{\marginfont}{\color{black}}

\newcommand{\myp}[1]{%
	\marginnote{%
		\begin{spacing}{1}
			\vspace{-\baselineskip}%
			\color{red}\footnotesize#1
		\end{spacing}
	}
}
%%%%%%%%%%%%%%%%%%%%%%%%%%%%%%%%
%% Outputbox
\newcommand{\outputbox}[1]{\noindent\fbox{\parbox[t][][t]{0.98\linewidth}{#1}}\vspace{0.5em}}

%%%%%%%%%%%%%%%%%%%%%%%%%%%%%%%%
%% (Syntactic) Trees
% package needed: forest
%
%% Setting for simple trees
\forestset{
	MyP edges/.style={for tree={parent anchor=south, child anchor=north}}
}

%% this is taken from langsci-setup file
%% Setting for complex trees
%% \forestset{
%% 	sn edges/.style={for tree={parent anchor=south, child anchor=north,align=center}}, 
%% background tree/.style={for tree={text opacity=0.2,draw opacity=0.2,edge={draw opacity=0.2}}}
%% }

\newcommand\HideWd[1]{%
	\makebox[0pt]{#1}%
}


%%%%%%%%%%%%%%%%%%%%%%%%%%%%%%%%%%%%%%%%%%%%%%%%%%%%
%%%          Useful commands                     %%%
%%%%%%%%%%%%%%%%%%%%%%%%%%%%%%%%%%%%%%%%%%%%%%%%%%%%

%%%%%%%%%%%%%%%%%%%%%
%% FOR ITEMS:
%\begin{itemize}
%  \item<2-> from point 2
%  \item<3-> from point 3 
%  \item<4-> from point 4 
%\end{itemize}
%
% or: \onslide<2->
% or: \pause

%%%%%%%%%%%%%%%%%%%%%
%% VERTICAL SPACE:
% \vspace{.5cm}
% \vfill

%%%%%%%%%%%%%%%%%%%%%
% RED MARKING OF TEXT:
%\alert{bis spätestens Mittwoch, 18 Uhr}

%%%%%%%%%%%%%%%%%%%%%
%% RESCALE BIG TABLES:
%\scalebox{0.8}{
%For Big Tables
%}

%%%%%%%%%%%%%%%%%%%%%
%% BLOCKS:
%\begin{alertblock}{Title}
%Text
%\end{alertblock}
%
%\begin{block}{Title}
%Text
%\end{block}
%
%\begin{exampleblock}{Title}
%Text
%\end{exampleblock}


\newtoggle{uebung}
\newtoggle{loesung}
\newtoggle{toc}

% The toc is not needed on Handouts. Safe trees.
\mode<handout>{
\togglefalse{toc}
}

\newtoggle{hpsgvorlesung}\togglefalse{hpsgvorlesung}
\newtoggle{syntaxvorlesungen}\togglefalse{syntaxvorlesungen}

%\includecomment{psgbegriffe}
%\excludecomment{konstituentenprobleme}
%\includecomment{konstituentenprobleme-hinweis}

\newtoggle{konstituentenprobleme}\togglefalse{konstituentenprobleme}
\newtoggle{konstituentenprobleme-hinweis}\toggletrue{konstituentenprobleme-hinweis}

%\includecomment{einfsprachwiss-include}
%\excludecomment{einfsprachwiss-exclude}
\newtoggle{einfsprachwiss-include}\toggletrue{einfsprachwiss-include}
\newtoggle{einfsprachwiss-exclude}\togglefalse{einfsprachwiss-exclude}

\newtoggle{psgbegriffe}\toggletrue{psgbegriffe}

\newtoggle{gb-intro}\togglefalse{gb-intro}



%%%%%%%%%%%%%%%%%%%%%%%%%%%%%%%%%%%%%%%%%%%%%%%%%%%%
%%%             Preamble's End                   
%%%%%%%%%%%%%%%%%%%%%%%%%%%%%%%%%%%%%%%%%%%%%%%%%%%% 

\begin{document}
	
	
%%%% ue-loesung
%%%% true: Übung & Lösungen (slides) / false: nur Übung (handout)
%	\toggletrue{ue-loesung}

%%%% ha-loesung
%%%% true: Hausaufgabe & Lösungen (slides) / false: nur Hausaufgabe (handout)
%	\toggletrue{ha-loesung}

%%%% toc
%%%% true: TOC am Anfang von Slides / false: keine TOC am Anfang von Slides
\toggletrue{toc}

%%%% sectoc
%%%% true: TOC für Sections / false: keine TOC für Sections (StM handout)
%	\toggletrue{sectoc}

%%%% gliederung
%%%% true: Gliederung für Sections / false: keine Gliederung für Sections
%	\toggletrue{gliederung}


%%%%%%%%%%%%%%%%%%%%%%%%%%%%%%%%%%%%%%%%%%%%%%%%%%%%
%%%             Metadata                         
%%%%%%%%%%%%%%%%%%%%%%%%%%%%%%%%%%%%%%%%%%%%%%%%%%%%      

\title{Grundkurs Linguistik}

\subtitle{Lösungen -- Graphematik}

\author[A. Machicao y Priemer]{
	{\small Antonio Machicao y Priemer}
	\\
	{\footnotesize \url{http://www.linguistik.hu-berlin.de/staff/amyp}}
	%	\\
	%	\href{mailto:mapriema@hu-berlin.de}{mapriema@hu-berlin.de}}
}

\institute{Institut für deutsche Sprache und Linguistik}


% bitte lassen, sonst kann man nicht sehen, von wann die PDF-Datei ist.
%\date{ }

%\publishers{\textbf{6. linguistischer Methodenworkshop \\ Humboldt-Universität zu Berlin}}

%\hyphenation{nobreak}


%%%%%%%%%%%%%%%%%%%%%%%%%%%%%%%%%%%%%%%%%%%%%%%%%%%%
%%%             Preamble's End                  
%%%%%%%%%%%%%%%%%%%%%%%%%%%%%%%%%%%%%%%%%%%%%%%%%%%%      


%%%%%%%%%%%%%%%%%%%%%%%%%      
\huberlintitlepage[22pt]
\iftoggle{toc}{
	\frame{
		\begin{multicols}{2}
			\frametitle{Inhaltsverzeichnis}
			\tableofcontents
			%[pausesections]
			\columnbreak
			\textcolor{white}{
				\ea \label{ex:04aphilo}
				\ex\label{ex:04amutter}
				\ex\label{ex:04azimmer}
				\ex\label{ex:04anacht}
				\ex\label{ex:04avasenstück}
				\ex\label{ex:04aHA3}
				\ex\label{ex:04aHA4}
				\ex\label{ex:04aHA5}
				\z
			}
		\end{multicols}
	}
}


%%%%%%%%%%%%%%%%%%%%%%%%%%%%%%%%%%%
%%%%%%%%%%%%%%%%%%%%%%%%%%%%%%%%%%%
\section{Übungen}

%%%%%%%%%%%%%%%%%%%%%%%%%%%%%%%%%%
%% UE 1 - 04a Graphematik
%%%%%%%%%%%%%%%%%%%%%%%%%%%%%%%%%%

\begin{frame}
\frametitle{Übung -- Lösung}

\begin{itemize}
	\item Geben Sie 10 Wörter an, die phonographisch geschrieben werden.
	
		\begin{description}
			\item[\alertred{\textbf{Beispiele:}}] \alertred{Beurteilung, schön, Gabel, suchen, Macht, Lager, kurz, niesen, Zopf, Gewerkschaft}
		\end{description}
	
	\item Wie würden Sie die folgenden Wörter phonographisch schreiben?
		
	\begin{exe}
		\exr{ex:04aphilo}
	\settowidth\jamwidth{XXXXXXXXXXXXXXXXXXXXXXXXXXXXX}
		\begin{xlist}
			\ex Philosophie \loesung{2}{\ab{filosofie}}
			\ex Balkon \loesung{3}{\ab{balkong}}
			\ex Creme \loesung{4}{\ab{krem} oder \ab{kreme}}
			\ex Mutter \loesung{5}{\ab{muter}}
			\ex Streithahn \loesung{6}{\ab{schtreithan}}
		\end{xlist}
	\end{exe}
		

\end{itemize}

\end{frame}



%%%%%%%%%%%%%%%%%%%%%%%%%%%%%%%%%%
%% UE 2 - 04a Graphematik
%%%%%%%%%%%%%%%%%%%%%%%%%%%%%%%%%%

\begin{frame}
\frametitle{Übung -- Lösung}

\begin{itemize}	
	\item Versuchen Sie, graphematische Regularitäten und Prinzipien zu finden, die die Unterscheidung lang \vs kurz bei Vokalen anzeigen. Gibt es Ausnahmen?
	
	\ea\label{ex:mutter}
	
	\begin{multicols}{2}
		\ea Mutter
		\ex Mehl
		\ex See
		\ex Nase
		\ex dehnen
		\ex gehen
		\ex Bier
		\ex Moor
		\ex an
		\ex zum
		\ex Mann
		\ex man
		\ex Herbst
		\ex Laub
		\ex sehr
		\ex Bohrer
		\z
	\end{multicols}
	
	\z
	
	
	
\end{itemize}

\end{frame}



%%%%%%%%%%%%%%%%%%%%%%%%%%%%%%%%%%
%% UE 3 - 04a Graphematik
%%%%%%%%%%%%%%%%%%%%%%%%%%%%%%%%%%

\begin{frame}
\frametitle{Übung -- Lösung}

\begin{itemize}
	\item Warum schreibt man \ab{dehnen} mit \ab{h}, obwohl das erste \ab{e} in einer offenen Silbe steht und daher nach silbischen Prinzipien sowieso lang gesprochen werden müsste?
		\item[] \alertred{Morphemkonstanz, da bei Flexionsformen wie \ab{dehnst} geschlossene Silbe}
	\item Warum schreibt man \ab{mann} und \ab{ball}, obwohl nach silbischen Prinzipien die Geminate einen ambisyllabischen Konsonanten anzeigt?
		\item<2->[] \alertred{Morphemkonstanz, da Pluralform Silbengelenk hat}
	\end{itemize}


\end{frame}


%%%%%%%%%%%%%%%%%%%%%%%%%%
\begin{frame}
\frametitle{Übung -- Lösung}

\begin{itemize}
	\item Warum sind die Wörter \ab{(du) ziehst}, \ab{säubern} und \ab{(er) fällt} Beispiele für morphologisches Schreiben?
		\item[] \alertred{\ab{h} im Infinitiv silbentrennend, Singular-Flexionsformen sind jedoch einsilbig;}
		\item<2->[] \alertred{\ab{ä} zeigt die Verwandschaft zu \ab{sauber}: laut PGK schriebe man \textipa{[O\texttoptiebar{}I]} \ab{eu};}
		\item<3->[] \alertred{Konsonantenverdopplung wegen Silbengelenk im Infinitiv, Singular-Flexionsformen sind jedoch einsilbig,\\ \ab{ä} wegen \ab{a} im Infinitiv: \textipa{[E]} wäre nach PGK \ab{e}}
	\item Wie hätte eine Person, die \ab{Rad} und \ab{König} als Beispiele für das morphologische Prinzip anführt, \gqq{phonographisches Schreiben} verstanden?
		\item<4->[] \alertred{Verschriftlichung der \emph{phonetischen} (richtig wäre: \emph{phonologische}) Repräsentation}
\end{itemize}


\end{frame}



%%%%%%%%%%%%%%%%%%%%%%%%%%%%%%%%%%
%% UE 4 - 04a Graphematik
%%%%%%%%%%%%%%%%%%%%%%%%%%%%%%%%%%

\begin{frame}
\frametitle{Übung -- Lösung}

\begin{itemize}
	\item Welche graphematischen Prinzipien (abgesehen von der phonographischen Schreibung) erklären die Schreibung der folgenden Wörter?
	
\begin{exe}
	\exr{ex:04azimmer}
	\settowidth\jamwidth{XXXXXXXXXXXXXXXXXXXXXXXXXXXXXXXXXXX}
	\begin{xlist}
		\ex \ab{Zi\rotul{mm}er} \loesung{1}{silbisches Prinzip: Silbengelenk}
		\ex \ab{W\rotul<2->{a}ise} \loesung{2}{Differenzierung homophoner Formen: zu \ab{weise}}
		\ex \ab{We\rotul<3->{h}en} \loesung{3}{silbisches Prinzip: silbentrennend}
		\ex \ab{Ru\rotul<4->{h}m} \loesung{4}{silbisches Prinzip: Dehnungs-h}
		\ex \ab{\rotul<5->{S}paß} \loesung{5}{ästhetische Schreibung: kein \ab{schp}}
		\ex \ab{A\rotul<6->{llee}} \loesung{6}{silbisches Prinzip: Silbengelenk \ab{ll}, Gespanntheit \ab{ee}}
		%          \ex \ab{Gras}
	\end{xlist}
\end{exe}
	
	\item Welche graphematische Funktion erfüllt das \ab{h} in den folgenden Wörtern?
	
\begin{exe}
	\exr{ex:04anacht}
	\settowidth\jamwidth{XXXXXXXXXXXXXXXXXXXXXXXXXXXXXXXXXXX}
	\begin{xlist}
		\ex \ab{Nacht} \loesung{7}{Teil von Digraph \ab{ch}}
		\ex \ab{Hilfe} \loesung{8}{Phonem-Graphem-Korrespondenz zu \textipa{[h]}}
		\ex \ab{sehen} \loesung{9}{silbentrennend}
		\ex \ab{Mehl} \loesung{10}{Dehnungs-h}
	\end{xlist}
\end{exe}
	
    \end{itemize}

\end{frame}


%%%%%%%%%%%%%%%%%%%%%%%%%%%%%%%%%%%%%%%
\begin{frame}{Übung -- Lösung}

	\begin{itemize}
	\item Wie würden die folgenden Wörter in phonographischer Schreibung aussehen? Geben Sie zunächst eine phonologische Transkription an (Notation mit / ~ /) und schreiben Sie anschließend phonographisch (Notation in \ab{ ~ }).

\begin{exe}
	\exr{ex:04avasenstück}
	\begin{multicols}{3}
	\begin{xlist}
		\ex \ab{Handy} 
		
		\ex \ab{Vasenstück}
		
		\ex \ab{Wannenbad}
		
%		\exi{} 
		\exi{} \only<2->{\alertred{\textipa{/hEn.di/}}}
		\exi{} \only<4->{\alertred{\textipa{/va:.z@n.StYk/}}}
		\exi{} \only<6->{\alertred{\textipa{/va\.n@n.ba:d/}}}

%		\exi{} 
		\exi{} \only<3->{\alertred{\ab{hendi}}}		
		\exi{} \only<5->{\alertred{\ab{wasenschtük}}}		
		\exi{} \only<7->{\alertred{\ab{wanenbad}}}		
	\end{xlist}
	\end{multicols}
\end{exe}
	
	\end{itemize}

\end{frame}




%%%%%%%%%%%%%%%%%%%%%%%%%%%%%%%%%%%
%%%%%%%%%%%%%%%%%%%%%%%%%%%%%%%%%%%
\section{Hausaufgaben}

%%%%%%%%%%%%%%%%%%%%%%%%%%%%%%%%%%
%% HA 1 - 04a Graphematik
%%%%%%%%%%%%%%%%%%%%%%%%%%%%%%%%%%

\begin{frame}%[allowframebreaks]
\frametitle{Lösung der Hausaufgabe}

\begin{itemize}
	
	\item[1.] Kreuzen Sie die korrekten Aussagen an.
	
	\begin{itemize}
		\item[$\circ$] Die Orthographie ist eine linguistische Teildisziplin, die beschreibt wie man schreibt. Die Graphematik ist dagegen keine Teildisziplin der Linguistik, sondern eine \gqq{willkürliche} (normierende) Festlegung.
		
		\item[\alertred{$\checkmark$}] \textcolor{red}{Die Graphematik sollte intuitiv beherrschbar sein und das Lesen und Schreiben vereinfachen.}
		
		\item[$\circ$] Das Wort \ab{kalt} ist eine graphematisch \gqq{nackte} Silbe.
		
		\item[$\circ$] Es gibt im Deutschen eine eindeutige 1-zu-1-Korrespondenz zwischen Buchstaben und Lauten.
		
		\item[\alertred{$\checkmark$}] \textcolor{red}{Das Wort \ab{aufwändig} wird aufgrund des morphologischen Prinzips (auch Prinzip der Schemakonstanz, Stammprinzip oder Verwandtschaftsprinzip) mit \ab{ä} geschrieben (vgl. \ab{Aufwand}).}
	\end{itemize}
\end{itemize}
\end{frame}


%%%%%%%%%%%%%%%%%%%%%%%%%%%%%%%%%%	
\begin{frame}

\begin{itemize}
\item[2.] Ordnen Sie die graphematischen Prinzipien links den passenden Beispielen für die entsprechenden Prinzipien rechts zu.

NB: Beachten Sie bitte nicht die Großschreibung.

\vspace{.5cm}

\begin{minipage}{0.45\textwidth}
	\centering
	\begin{tabular}{|l|}
		\hline
		(A) Etymologische Schreibung\\
		\hline
		(B) Homonymievermeidung\\
		\hline
		(C) Morphologisches Prinzip\\
		\hline
		(D) Silbische Prinzip\\
		\hline
		(E) Phonographisches Prinzip\\
		\hline
	\end{tabular}
\end{minipage}
\hfill%
\begin{minipage}{0.45\textwidth}
	\centering
	\begin{tabular}{|p{0.075\textwidth}|l|}
		\hline
		\only<2->{\alertred{C}} & Bad, Bäder \\
		\hline
		\only<3->{\alertred{D}} & gehen \\
		\hline
		\only<4->{\alertred{A}} & Cello, *Tschello \\
		\hline
		\only<5->{\alertred{B}} & Wahl, Wal\\
		\hline
		\only<6->{\alertred{E}} & Flasche \\
		\hline
	\end{tabular}
\end{minipage}

\end{itemize}
\end{frame}


%%%%%%%%%%%%%%%%%%%%%%%%%%%%%%%%%%	
\begin{frame}%[allowframebreaks]
\begin{itemize}
\item[3.] Betrachten Sie die unten angegebenen Kontexte. Diskutieren Sie kurz anhand dieser Beispiele, ob es sich bei der Groß- und Kleinschreibung des markierten Buchstabens um unterschiedliche Grapheme handeln kann oder nicht.

\begin{exe}
	\exr{ex:04aHA3}
	\begin{xlist}
	\ex Dieser \underline{W}eg ist sehr steil.
	\ex \underline{W}ege, die ich nicht bewandert habe, gibt es viele.
	\ex Meine Schlüssel sind \underline{w}eg.
	\ex \gqq{\underline{W}eg!}, schrie sie mich an und knallte mir die Tür vor der Nase zu.
	\ex Geh \underline{w}eg!
	\end{xlist}
\end{exe}
\end{itemize}
\end{frame}


%%%%%%%%%%%%%%%%%%%%%%%%%%%%%%%%%%	
\begin{frame}%[allowframebreaks]


\begin{itemize}
%\pause
\item[\alertred{--}] \alertred{Graphem: Kleinste bedeutungsunterscheidende Einheit im schriftlichen System}

%\pause

\item[\alertred{--}] \alertred{\ab{Weg} und \ab{weg} kann als Minimalpaar angesehen werden, und \ab{W} und \ab{w} als unterschiedliche Grapheme, da sie bedeutungsunterscheidend sind (vgl.\ a und e). Es gibt darüber hinaus weitere Beispiele, die diese Tendenz zu belegen scheinen \ab{Reisen} \vs \ab{reisen}, \ab{Sie} \vs \ab{sie}, \ab{Gut} \vs \ab{gut}.}

%\pause

\item[\alertred{--}] \alertred{Andererseits kann die Großschreibung durch andere Prinzipien bedingt werden (\zB Satzanfang) und verliert somit den bedeutungsunterscheidenden Charakter (vgl.\ d und e).}

%\pause

\item[\alertred{--}] \alertred{Unter Berücksichtigung der gegebenen Beispiele könnte man zunächst vermuten, dass \ab{W} und \ab{w} unterschiedliche Grapheme  (vgl.\ Minimalpaare (a) und (c)). Die Groß- und  Kleinschreibung hat jedoch eine andere Funktion im Schriftsystem des Deutschen (\zB Markierung von Nomina und Satzanfängen) und wirkt sich somit nicht notwendigerweise bedeutungsunterscheidend aus.}
\end{itemize}

\end{frame}


%%%%%%%%%%%%%%%%%%%%%%%%%%%%%%%%%%		
\begin{frame}

\begin{itemize}
\item[4.] Erläutern Sie stichpunktartig, welche (graphematische) Funktionen der Buchstabe \ab{h} in den folgenden Kontexten annimmt:

\begin{exe}
\exr{ex:04aHA4}
\settowidth\jamwidth{XXXXXXXXXXXXXXXXXXXXXXXXXXXXXXXXXX}
\begin{xlist}
	\ex Ha\underline{h}n: \loesung{2}{Dehnungs-h}
	
	\ex nä\underline{h}en: \loesung{3}{Silbentrennendes \ab{h}}
	
	\ex bein\underline{h}alten: \loesung{4}{Korrespondenz zu Phonem \textipa{/h/}}
	
	\ex Gesc\underline{h}ichte: \loesung{5}{Teil eines Trigraphen \ab{sch}} \loesung{5}{(Nicht Teil eines Lauts, sondern eines Graphems!)}
	
	\ex Geschic\underline{h}te: \loesung{6}{Teil eines Digraphen \ab{ch}} \loesung{6}{(Nicht Teil eines Lauts, sondern eines Graphems!)}
	
	\ex Dip\underline{h}thong: \loesung{7}{Teil eines Fremddigraphen \ab{ph}}
	
	\ex Dipht\underline{h}ong: \loesung{8}{Teil eines Fremddigraphen \ab{th}}
\end{xlist}
\end{exe}

\end{itemize}

\end{frame}


%%%%%%%%%%%%%%%%%%%%%%%%%%%%%%%%%%		
\begin{frame}

\begin{itemize}
\item[5.] Geben Sie die \textbf{phonologische} Transkription, die \textbf{phonetische} Transkription und die \textbf{phonographische} Schreibung (nach der Phonem-Graphem-Korrespondenz) des folgenden Wortes an.

\begin{exe}
	\exr{ex:04aHA5} Abstellkammer
\end{exe}

\settowidth\jamwidth{XXXXXXXXXXXXXXXXXXXXXXXXXXXXXXXXX}
\item[] phonologisch: \loesung{2}{\textipa{/abStElkam@\textscr /}}

\item[] phonetisch: \loesung{3}{\textipa{[PapStElkam5]}}

\item[] phonographisch: \loesung{4}{\ab{abschtelkamer}}

\item[] \only<5->{\alertred{Hier erkennt man, dass es sich bei der phonographischen Trankskription um eine Phonem-Graphem-Korrespondenz (und nicht um eine Phon-Graphem-Korrespondenz) handelt.}}

\end{itemize}
\end{frame}

%% -*- coding:utf-8 -*-

%%%%%%%%%%%%%%%%%%%%%%%%%%%%%%%%%%%%%%%%%%%%%%%%%%%%%%%%%


\def\insertsectionhead{\refname}
\def\insertsubsectionhead{}

\huberlinjustbarfootline


\ifpdf
\else
\ifxetex
\else
\let\url=\burl
\fi
\fi
\begin{multicols}{2}
{\tiny
%\beamertemplatearticlebibitems

\bibliography{gkbib,bib-abbr,biblio}
\bibliographystyle{unified}
}
\end{multicols}





%% \section{Literatur}
%% \begin{frame}[allowframebreaks]
%% \frametitle{Literatur}
%% 	\footnotesize

%% \bibliographystyle{unified}

%% 	%German
%% %	\bibliographystyle{deChicagoMyP}

%% %	%English
%% %	\bibliographystyle{chicago} 

%% 	\bibliography{gkbib,bib-abbr,biblio}
	
%% \end{frame}



\end{document}