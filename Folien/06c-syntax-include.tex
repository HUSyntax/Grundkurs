%%%%%%%%%%%%%%%%%%%%%%%%%%%%%%%%%%%%%%%%%%%%%%%%%%%%
%%%             Metadata                         %%%
%%%%%%%%%%%%%%%%%%%%%%%%%%%%%%%%%%%%%%%%%%%%%%%%%%%%      


\title{Grundkurs Linguistik}

\subtitle{Syntax III: Topologisches Feldermodell, Satztypen \& Satzmodi}

\author[aMyP]{
	{\small Antonio Machicao y Priemer}
%	\\
%	{\footnotesize \url{http://www.linguistik.hu-berlin.de/staff/amyp}\\
%	\href{mailto:mapriema@hu-berlin.de}{mapriema@hu-berlin.de}}
}

\institute{Institut für deutsche Sprache und Linguistik}

%%%%%%%%%%%%%%%%%%%%%%%%%      
\date{ }
%\publishers{\textbf{6. linguistischer Methodenworkshop \\ Humboldt-Universität zu Berlin}}

%\hyphenation{nobreak}


%%\togglefalse{uebung}
%\toggletrue{uebung}
%\togglefalse{loesung}
%%\toggletrue{loesung}
%\togglefalse{toc}
%%\toggletrue{toc}

%%%%%%%%%%%%%%%%%%%%%%%%%%%%%%%%%%%%%%%%%%%%%%%%%%%%
%%%             Preamble's End                   %%%
%%%%%%%%%%%%%%%%%%%%%%%%%%%%%%%%%%%%%%%%%%%%%%%%%%%%      


%%%%%%%%%%%%%%%%%%%%%%%%%      
\huberlintitlepage
\iftoggle{toc}{
\frame{
\begin{multicols}{2}
	\frametitle{Inhaltsverzeichnis}\tableofcontents
	%[pausesections]
\end{multicols}
	}
	}


%%%%%%%%%%%%%%%%%%%%%%%%%%%%%%%%%%
%%%%%%%%%%%%%%%%%%%%%%%%%%%%%%%%%%
%%%%%LITERATURE:

%\nocite{Brandt&Co06a}
\nocite{Glueck&Roedel16a}
%\nocite{Grewendorf&Co91a} 
\nocite{Luedeling2009} 
%\nocite{Meibauer&Co07a}
%\nocite{MuellerS13f} 
%\nocite{MuellerS15b}
\nocite{Repp&Co15a} 
%\nocite{Stechow&Sternefeld88a}
\nocite{Woellstein10a}

\nocite{Altmann&Hofmann08a}
\nocite{Altmann93a}


%%%%%%%%%%%%%%%%%%%%%%%%%%%%%%%%%%
%%%%%%%%%%%%%%%%%%%%%%%%%%%%%%%%%%
\section{Das topologische Modell}
%\frame{
%\frametitle{~}
%	\tableofcontents[currentsection]
%}


%%%%%%%%%%%%%%%%%%%%%%%%%%%%%%%%%%
\begin{frame}
\frametitle{Das topologische Modell}

\begin{block}{Topologie (Satztopologie, Feldgliederung)}
Zusammenfassende Bezeichnung für das Studium der Wortstellung und der Satzgliedstellung, d.\,h.\ der Anordnung der entsprechenden Elemente im Raum (Geschriebenes) oder der Zeit (Gesprochenes). \citep[vgl.][]{Glueck16a, Staffeldt16a} 
\end{block}

\pause 

\begin{itemize}

	\item Zweck \ras Wortstellung im deutschen Matrixsatz abzubilden (\textbf{Topologie})
%	\item[]
	\item Hilfreich für die \textbf{Beschreibung} und den \textbf{Vergleich} von Satzstrukturen
%	\item[]
	\item Es erfasst alle möglichen deutschen Satztypen und macht sie \textbf{vergleichbar}.
%	\item[]
	\item Syntaktische Strukturen können \textbf{topologisch} erfasst werden, d.\,h.\ ihre Elemente werden in ihren Positionen und Abfolgen \textbf{beschrieben}.

\end{itemize}

\end{frame}


%%%%%%%%%%%%%%%%%%%%%%%%%%%%%%%%%%%
\begin{frame}
\frametitle{Das topologische Modell}

\begin{itemize}
	\item Die toplogische Satzstrukturbetrachtung hat eine lange Tradition.
	
	\ras Herling 1821; Erdmann 1886; \textbf{\citealt{Drach39a}}
	
	\item[]
	\item Die \textbf{Verfeinerung} des Modells wurde in den 1980ern vorgenommen. 
	
	\ras Reis 1980; \citealt{Hoehle85a}
	
	\item[]
	
	\item Topologisches Modell wird immer noch verwendet 
	
	\ras \citealt{Ramers06a}; Pafel 2009; \citealt{Woellstein10a} 


\end{itemize}

\end{frame}


%%%%%%%%%%%%%%%%%%%%%%%%%%%%%%%%%%
%%%%%%%%%%%%%%%%%%%%%%%%%%%%%%%%%%
\subsection{Dreifeldermodell}
%\frame{
%\frametitle{~}
%	\tableofcontents[currentsection]
%}


%%%%%%%%%%%%%%%%%%%%%%%%%%%%%%%%%%
\begin{frame}
\frametitle{Dreifeldermodell (Drach 1937)}

\begin{itemize}
	\item Drachs Dreifeldermodell war ursprünglich für die Erfassung von \textbf{Aussagesätzen mit Verbzweitstellung} (V2-Sätze) gedacht.
	\item[]
	\item \textbf{Vorfeld (VF):} Abschnitt vor dem finiten Verb
	\item \textbf{Mitte:} Position für das finite Verb
	\item \textbf{Nachfeld (NF):} Abschnitt nach dem finiten Verb
\end{itemize}

\pause

\begin{table}
\centering

\begin{tabular}{l|l|l}
\textbf{Vorfeld} & \textbf{Mitte} & \textbf{Nachfeld} \\ 
\hline 
Sophia & \alertred{schreibt} & ihre Dissertation. \\ 
\hline
\pause
Sophia & \alertred{hat}  & ihre Dissertation geschrieben. \\ 
\hline 
\pause
Ihre Dissertation geschrieben & \alertred{hat} & Sophia längst!\\ 
\end{tabular} 

\end{table}

\end{frame}


%%%%%%%%%%%%%%%%%%%%%%%%%%%%%%%%%%
\begin{frame}
\frametitle{Dreifeldermodell (Drach 1937)}

\begin{itemize}
	\item \textbf{Probleme:}
	
	\begin{itemize}
		\item Modell erfasst nicht den komplexen Bereich nach dem finiten Verb
		\item[]
		\item Modell nur für V2-Strukturen \ras zu beschränkt		
	\end{itemize}
\end{itemize}

\pause

\eal
\ex \alertred{Hat} [Sophia] [ihre Dissertation] \alertred{geschrieben}?
\ex \alertred{Schreib} endlich deine Diss!
\ex	\alertred{Ob} sie ihre Diss \alertred{schreibt}?
\zl


\end{frame}


%%%%%%%%%%%%%%%%%%%%%%%%%%%%%%%%%%
%%%%%%%%%%%%%%%%%%%%%%%%%%%%%%%%%%
\subsection{Uniformes Grundmodell}
%\frame{
%\frametitle{~}
%	\tableofcontents[currentsection]
%}


%%%%%%%%%%%%%%%%%%%%%%%%%%%%%%%%%
\begin{frame}
\frametitle{Uniformes Grundmodell (in unserem Kurs)}

\begin{itemize}
	\item Auch: Stellungsfeldermodell,  lineares Modell, Felderstrukturenmodell 
	\item 5-gliedriges Grundmodell
	\item Erfasst \textbf{mehr Daten} als das Dreifeldermodell (d.\,h. es ist beschreibungsadäquater) 
	
	\ras Alle Verbstellungs- bzw. Satztypen werden in einem \textbf{einheitlichen} Muster abgebildet
\end{itemize}

\end{frame}

%%%%%%%%%%%%%%%%%%%%%%%%%%%%%%%%%%%%%%%%%%%%
\begin{frame}

\begin{itemize}
	\item Satz in topologische Abschnitte eingeteilt:
	\begin{itemize}
		\item []
		\item \textbf{Vorfeld (VF):} Feld vor dem finiten Verb
		\item[]
		\item \textbf{Linke Satzklammer (LSK):} Finites Verb oder Subjunktion
		\item[]
		\item \textbf{Mittelfeld (MF):} 0--$x$ Konstituente(n)
		\item[]		
		\item \textbf{Rechte Satzklammer (RSK):} Verb(komplex)
		\item[]		
		\item \textbf{Nachfeld (NF):} Satzartige oder \gqq{schwere} Konstituenten
	\end{itemize}
\pause
\end{itemize}

\begin{table}
\centering
\scalebox{0.85}{
\begin{tabular}{l|l|l|l|l}
\textbf{VF} & \textbf{LSK} & \textbf{MF} & \textbf{RSK} & \textbf{NF} \\ 
\hline 
Gestern & ist & Nathalie früh nach Hause & gegangen, & weil sie müde war. \\ 
\end{tabular} 
}
\end{table}
	
 \end{frame}


%%%%%%%%%%%%%%%%%%%%%%%%%%%%%%%%%%
\begin{frame}

\begin{itemize}
	\item Abschnitte bzw. Satzeinheiten resultieren aus der Stellung der finiten und/oder infiniten Verbform ($\approx$ aus dem Verbkomplex)
\end{itemize}

\begin{table}
\centering
\scalebox{0.8}{
\begin{tabular}{l|l|l|l|l}
\textbf{VF} & \textbf{LSK} & \textbf{MF} & \textbf{RSK} & \textbf{NF} \\ 
\hline 
Nathalie & \alertred{ist} & zu Hause & \alertred{geblieben}, & weil sie krank ist. \\ 
\end{tabular} 
}
\end{table}

\end{frame}


%%%%%%%%%%%%%%%%%%%%%%%%%%%%%%%%%%
\begin{frame}

\begin{itemize}
	\item Analyse von \textbf{Haupt- und Nebensätzen} und \textbf{komplexen Satzstrukturen} möglich!
	\item[]
	\item Erfasst die \textbf{Verbklammer} (typisch für das Deutsche) und
	die \textbf{Komplementarität} zwischen \textbf{Verb und Complementizer} in der LSK
\end{itemize}

\begin{table}
\centering
\scalebox{0.8}{
\begin{tabular}{l|l|l|l|l}
\textbf{VF} & \textbf{LSK} & \textbf{MF} & \textbf{RSK} & \textbf{NF} \\ 
\hline 
Nathalie & \alertred{ist} & zu Hause & \alertgreen{geblieben}, & weil sie krank ist. \\ 
\hline
$\emptyset$ & \alertred{weil} & sie krank & \alertgreen{ist} & $\emptyset$. \\ 
\end{tabular} 
}
\end{table}

\end{frame}


%%%%%%%%%%%%%%%%%%%%%%%%%%%%%%%%%%%
%%%%%%%%%%%%%%%%%%%%%%%%%%%%%%%%%%
\subsection{Eigenschaften der Felder}
%\frame{
%\frametitle{~}
%	\tableofcontents[currentsection]
%}


%%%%%%%%%%%%%%%%%%%%%%%%%%%%%%%%%%
\begin{frame}
\frametitle{Eigenschaften der Felder}

\begin{itemize}
	\item \textbf{VF:}

\begin{table}
\centering
\scalebox{0.8}{
\begin{tabular}{l|l|l|l|l}
\textbf{VF} & \textbf{LSK} & \textbf{MF} & \textbf{RSK} & \textbf{NF} \\ 
\hline 
\alertred{Marlijn} & ist & zu Hause. &  &  \\
\hline 
\alertred{Die Frau, die hier arbeitet,} & ist & sehr fleißig. & & \\  
\alertred{obwohl die Heizung ausgeschaltet ist,} & & & & \\
\hline
\alertred{$\emptyset$} & ist & Marlijn zu Hause ? &  &  \\
\hline
\alertred{$\emptyset$} & weil & sie krank & ist. &  \\ 
\end{tabular} 
}

\end{table}

\pause

\begin{itemize}
	\item \textbf{Fakultativ} besetzt
	\item Platz für \textbf{eine} (beliebig komplexe) Konstituente 
	\item \textbf{Leer} bei 
	
	\begin{itemize}
		\item sog. V1-Sätzen (Entscheidungsfragen, Imperativsätzen, \dots ),
		
		\item Sätzen mit nebensatzeinleitender Konjunktion ($\approx$ Complementizer): \MyPobj{dass}, \MyPobj{ob}, \MyPobj{weil}, \dots 
	
	\end{itemize}	
\end{itemize}


\end{itemize}

\end{frame}


%%%%%%%%%%%%%%%%%%%%%%%%%%%%%%%%%%
\begin{frame}
\frametitle{Eigenschaften der Felder}

\begin{itemize}
	\item \textbf{LSK:}

\begin{table}
\centering
\scalebox{0.8}{
\begin{tabular}{l|l|l|l|l|l}
& \textbf{VF} & \textbf{LSK} & \textbf{MF} & \textbf{RSK} & \textbf{NF} \\ 
\hline 
& Petra & \alertred{macht} & einen guten Kaffee. &  &  \\
\hline 
& & \alertred{dass} & Petra einen guten Kaffee & macht & \\  
\hline
Ich weiß, & wer & \alertred{$\emptyset$} & sie & ist. &  \\
\hline
Die Dame, & die & \alertred{$\emptyset$} & hier & arbeitet &  \\ 
\end{tabular} 
}
\end{table}

\pause 

\begin{itemize}
	\item Entweder finites Verb oder Complementizer
	\item \textbf{Leer} bei 
	
	\begin{itemize}
		\item eingebetteten Konstituentenfragen, 
		\item Relativsätzen, 
		\item Infinitivsätzen, \dots
	\end{itemize}
\end{itemize}

\end{itemize}

	
\end{frame}


%%%%%%%%%%%%%%%%%%%%%%%%%%%%%%%%%%
\begin{frame}
\frametitle{Eigenschaften der Felder}

\begin{itemize}
	\item \textbf{MF:}

\begin{table}
\centering
\scalebox{0.75}{
\begin{tabular}{l|l|l|l|l}
\textbf{VF} & \textbf{LSK} & \textbf{MF} & \textbf{RSK} & \textbf{NF} \\ 
\hline 
Sophie & hat & \alertred{gut} & geschlafen. & \\  
\hline 
Sophie & soll & \alertred{[trotz des Bahnchaos']} & bekommen haben. &  \\
& & \alertred{[nach ihrem Gespräch]} &  &  \\
& & \alertred{[in einer anderen Stadt]} &  &  \\
& & \alertred{[einen guten Job]} &  &  \\
\hline 
Sophie & hat & \alertred{$\emptyset$} & geschlafen. & \\  
\hline
Sie & hat & \alertred{die Frau} & eingestellt, & \alertred{die am qualifiziertesten} \\
 &  &  & & \alertred{war}. \\
\end{tabular} 
}
\end{table}

\pause 

\begin{itemize}
	\item Platz für \textbf{beliebig viele} Konstituenten
	\item Fakultativ auch leer
	\item Durch RSK können Konstituenten des MFs aufgebrochen werden.
\end{itemize}


\end{itemize}

\end{frame}


%%%%%%%%%%%%%%%%%%%%%%%%%%%%%%%%%%
\begin{frame}
\frametitle{Eigenschaften der Felder}

\begin{itemize}
	\item \textbf{RSK:}

\begin{table}
\centering
\scalebox{0.8}{
\begin{tabular}{l|l|l|l|l}
\textbf{VF} & \textbf{LSK} & \textbf{MF} & \textbf{RSK} & \textbf{NF} \\ 
\hline 
Monika & mag &  & \alertred{unterrichten}. & \\  
\hline 
 & Ob & Monika & \alertred{unterrichtet}? &  \\
\hline
mit dem & & Monika zur Arbeit & \alertred{fährt} &  \\
\hline
& weil & Monika & \alertred{angerufen haben will} &  \\
\hline
Monika & hat &  & \alertred{angerufen}. &  \\
\hline 
Monika & ruft &  & \alertred{an}.&  \\
\end{tabular} 
}
\end{table}

\pause 

\begin{itemize}
	\item Infinite Verben 
	\item Finites Verb (falls nicht in LSK) \ras \zB in Neben- oder Relativsätzen
	\item Fakultativ auch leer
\end{itemize}

\end{itemize}

\end{frame}


%%%%%%%%%%%%%%%%%%%%%%%%%%%%%%%%%%
\begin{frame}
\frametitle{Eigenschaften der Felder}

\begin{itemize}
	\item \textbf{NF:}

\begin{table}
\centering
\scalebox{0.8}{
\begin{tabular}{l|l|l|l|l}
\textbf{VF} & \textbf{LSK} & \textbf{MF} & \textbf{RSK} & \textbf{NF} \\ 
\hline 
Maria & hat &  & gesehen, & \alertred{dass Peter schläft}.\\  
\hline 
Maria & hat &  & gesehen, & \alertred{als sie in die Küche kam}\\  
 &  &  & & \alertgreen{dass Peter schlief}.\\  
\hline
Sie & hat & \alertred{die Frau} & eingestellt, & \alertred{die am qualifiziertesten} \\
 &  &  & & \alertred{war}. \\
\hline
Du & hast & uns alle & begeistert & \alertred{mit deiner großartigen Präsentation}. \\
\end{tabular} 
}
\end{table}

\pause 

\begin{itemize}
	\item Kann eine oder mehrere Konstituenten enthalten
	\item Fakultativ leer
	\item Hauptsächlich besetzt bei Subjekt-, Objekt-, Adverbial- oder Relativsätzen (Extraposition)
	\item Fakultativ auch bei \gqq{schweren} Konstituenten (PPs)
\end{itemize}


\end{itemize}

\end{frame}


%%%%%%%%%%%%%%%%%%%%%%%%%%%%%%%%%%
\begin{frame}

\begin{itemize}
	\item Im VF, MF, NF kann ein Satz enthalten sein (oder mehrere), der selbst wieder nach dem Feldermodell analysiert werden kann. 


\begin{table}
\centering
\scalebox{0.8}{
\begin{tabular}{l|l|l|l|l}
\textbf{VF} & \textbf{LSK} & \textbf{MF} & \textbf{RSK} & \textbf{NF} \\ 
\hline 
Maria & hat &  & gesehen, & \alertred{dass Peter schläft}.\\
\hline
 & dass & Peter & schläft & \\  
\hline
\hline 
Maria & hat &  & gemerkt, & \alertred{dass Peter denkt},\\  
 &  &  & & \alertred{dass sie schläft}.\\ 
\hline
   & dass & Peter & denkt, & \alertgreen{dass sie schläft}\\ 
\hline
   & dass & sie & schläft &  \\ 
\hline   
\hline
\alertred{Dass Peter denkt,} & hat & Maria sehr schnell & gemerkt. & \\
\alertgreen{dass sie schläft,} &  &  & & \\
\hline
\hline
Maria & hat, & \alertred{obwohl sie geschlafen hat}, & erwischt. &  \\
& & die Einbrecher & & \\
\end{tabular} 
}
\end{table}

\end{itemize}

\end{frame}


%%%%%%%%%%%%%%%%%%%%%%%%%%%%%%%%%%%
%%%%%%%%%%%%%%%%%%%%%%%%%%%%%%%%%%
\subsection{Fazit}
%\frame{
%\frametitle{~}
%	\tableofcontents[currentsection]
%}


%%%%%%%%%%%%%%%%%%%%%%%%%%%%%%%%%%
\begin{frame}
\frametitle{Fazit}

\begin{itemize}
	\item Das topologische Feldermodell eröffnet Möglichkeiten zur \textbf{Beschreibung} von \textbf{strukturellen (linearen) Gesetzmäßigkeiten} im Satzbau (\zB Satzklammer) und von grammatischen Konzepten (\zB Satztyp)
	\item[]
	\item Für eine intensive Beschäftigung mit der deutschen Syntax ist das Uniforme Modell allein allerdings \textbf{nicht ausreichend}. Die Unterteilung ist zu grob. Es sind Erweiterungen nötig. 
	
	vgl. Dreifeldermodell \ras Uniformes Modell \ras Differenzmodell

\end{itemize}

\end{frame}

%%%%%%%%%%%%%%%%%%%%%%%%%%%%%%%%%%
%%%%%%%%%%%%%%%%%%%%%%%%%%%%%%%%%%
\subsection{Übung}
%\frame{
%\frametitle{~}
%	\tableofcontents[currentsection]
%}


%%%%%%%%%%%%%%%%%%%%%%%%%%%%%%%%%%

\iftoggle{uebung}{

\begin{frame}
\frametitle{Übung}

\begin{itemize}
	\item[1.] Ordnen Sie die folgenden \textbf{Matrixsätze} in das topologische Modell ein:


\eal
\ex Christiane schläft.
\ex Schläft Christiane?
\ex Ob Christiane schläft?
\ex Schlaf!
\ex Dass Christiane schläft, ist mir klar.
\ex Ich wusste, dass Christiane schläft.
\ex Er hat sich gedacht, dass Christiane wieder schläft.
\ex Weil wir es uns nicht vorstellen konnten, haben wir uns gewundert, dass Christiane schläft.
\ex Weil wir es uns nicht vorstellen konnten, haben wir uns gewundert, als wir gemerkt haben, dass Christiane schläft.
\zl

\end{itemize}

\end{frame}

}

%%%%%%%%%%%%%%%%%%%%%%%%%%%%%%%%%%%%%%%%%%%%%%%%%%%

%\iftoggle{loesung}{
	
\begin{frame}
\frametitle{Lösung}

\begin{table}
\centering
\scalebox{0.75}{
\begin{tabular}{p{3.5cm}|l|p{2.5cm}|l|p{3.5cm}}
\textbf{VF} & \textbf{LSK} & \textbf{MF} & \textbf{RSK} & \textbf{NF} \\ 
\hline 
Christiane & schläft. & & & \\
\hline
& Schläft & Christiane? & & \\
\hline
& Ob & Christiane & schläft? & \\
\hline
& Schlaf! & & & \\
\hline
Dass Christiane schläft, & ist & mir klar. & & \\
\hline
Ich & wusste, & dass Christiane schläft. & & \\
\hline
Er & hat & sich & gedacht, & dass Christiane wieder schläft. \\
\hline
Weil wir es uns nicht vorstellen konnten, & haben & wir uns & gewundert, & dass Christiane schläft. \\
\hline
Weil wir es uns nicht vorstellen konnten, & haben & wir uns & gewundert, & als wir gemerkt haben, dass Christiane schläft. \\
\end{tabular}
}
\end{table}

\end{frame}

%}

%%%%%%%%%%%%%%%%%%%%%%%%%%%%%%%%%%

%\iftoggle{uebung}{

\begin{frame}
\frametitle{Übung}

\begin{itemize}

	\item[2.] Ordnen Sie die folgenden \textbf{Sätze und ihre eingebetteten Nebensätze} in das topologische Modell ein:
\eal
\ex Dass Christiane schläft, ist mir klar.
\ex Ich wusste, dass Christiane schläft.
\ex Er hat sich gedacht, dass Christiane wieder schläft.
\ex Weil wir es uns nicht vorstellen konnten, haben wir uns gewundert, dass Christiane schläft.
\ex Weil wir es uns nicht vorstellen konnten, haben wir uns gewundert, als wir gemerkt haben, dass Christiane schläft.
\ex Die Frau, die die roten Schuhe trägt, schläft wieder.
\ex Als Sophia in der Flüchtlingsunterkunft geholfen hat, hat sie dort den jungen Mann getroffen, mit dem sie sich ein Büro teilt.  	
\zl
	
\end{itemize}

\end{frame}

%}

%%%%%%%%%%%%%%%%%%%%%%%%%%%%%%%%%%

%\iftoggle{loesung}{


\begin{frame}
\frametitle{Lösung}

\begin{table}
\centering
\scalebox{0.63}{
\begin{tabular}{p{5cm}|l|p{4cm}|p{1.5cm}|p{4cm}}
\textbf{VF} & \textbf{LSK} & \textbf{MF} & \textbf{RSK} & \textbf{NF} \\ 
\hline
Dass Christiane schläft, & ist & mir klar. & & \\
\hline
& Dass & Christiane & schläft & \\
\hline
Ich & wusste, & dass Christiane schläft. & & \\
\hline
& dass & Christiane & schläft. & \\
\hline
Er & hat & sich & gedacht, & dass Christiane wieder schläft. \\
\hline
& dass & Christiane wieder & schläft. & \\
\hline
Weil wir es uns nicht vorstellen konnten, & haben & wir uns & gewundert, & dass Christiane schläft.\\
\hline
& Weil & wir es uns nicht & vorstellen konnten & \\
\hline
& dass & Christiane & schläft. & \\
\hline
Die Frau, die die roten Schuhe trägt, & schläft & wieder. & & \\
\hline
Die Frau, die & & die roten Schuhe & trägt & \\
\hline
Als Sophia in der Flüchtlingsunterkunft geholfen hat, & hat & sie dort den jungen Mann & getroffen, & mit dem sie sich ein Büro teilt. \\
\hline
& Als & Sophia in der Flüchtlingsunterkunft & geholfen hat & \\
\hline
mit dem & & sie sich ein Büro & teilt. & \\
\end{tabular}
}
\end{table}

\end{frame}
% }

%%%%%%%%%%%%%%%%%%%%%%%%%%%%%%%%%%
\section{Satztypen \& Satzmodi}
%\frame{
%\frametitle{~}
%	\tableofcontents[currentsection]
%}


%%%%%%%%%%%%%%%%%%%%%%%%%%%%%%%%%%
\begin{frame}
\frametitle{Satztypen \& Satzmodi}

\begin{itemize}
	\item Sätze im Deutschen können verschiedene \textbf{Formen} annehmen:
	\begin{itemize}
		\item \textbf{Verberststellung (V1)}
		\eal 
		\ex \label{ex:Frage} \alertred{Schläft} Norbert?
		\ex \label{ex:Imperat} \alertred{Schlaf} endlich!
		\zl
		
		\item \textbf{Verbzweitstellung (V2)}
		\ea \label{ex:Dekl} Norbert \alertred{schläft} gerne nach dem Mittagessen.
		\z
		
		\item \textbf{Verbletztstellung (VL)}
		\ea \label{ObFrage} Ob Norbert \alertred{schnarcht}?
		\z
		
	\end{itemize}	

\pause

	\item Verschiedene Formen entsprechen unterschiedlichen \textbf{Funktionen:}
	\begin{itemize}
		\item etwas in der Welt als \textbf{wahr} zu postulieren (\ref{ex:Dekl}),
		\item Zweifel oder \textbf{Unwissen} auszudrücken (\ref{ex:Frage}), (\ref{ObFrage}),
		\item etwas auszudrücken, von dem man \textbf{will, dass es wahr ist} (/wird) (\ref{ex:Imperat}).		
	\end{itemize}
\end{itemize}

\end{frame}


%%%%%%%%%%%%%%%%%%%%%%%%%%%%%%%%%%
\begin{frame}
\frametitle{Satztypen \& Satzmodi}

\begin{block}{Satzmodus}
Klassifikation von komplexen Zeichen (Sätzen) \textbf{mit einer Form- und einer Funktionsseite}. Der Sprecher wählt also eine bestimmte Form aus, um eine bestimmte Funktion zu erfüllen. \citep[vgl.][]{Altmann93a}
\end{block}


\pause 


\begin{block}{Satztyp}
Bezeichnung für unterschiedliche \textbf{Formtypen} von Sätzen, insbesondere solcher Hauptsätze, die in Äußerungen divergierende Handlungszwecke erfüllen bzw.\ erfüllen können. \citep[vgl.][]{Fries16d, Fries16e}
\end{block}


\end{frame}


%%%%%%%%%%%%%%%%%%%%%%%%%%%%%%%%%%
\begin{frame}
\frametitle{Satztypen \& Satzmodi}

\begin{itemize}
	\item \textbf{Formtyp} (Satztyp) \ras Formseite
	\begin{itemize}
		\item morphologische, syntaktische, phonologische Merkmale
		\item satzförmige Struktur mit formellen Eigenschaften
		\item Wort-/ Verbstellung
		\item Verbmorphologie
		\item Subkategorisierung
		\item Kategoriale Füllung
		\item Intonation, \dots
	\end{itemize}

	\item \textbf{Funktionstyp} \ras Funktionsseite
	\begin{itemize}
		\item Bedeutung zum Ausdruck einer Proposition oder zur Ausführung einer sprachlichen Handlung vom spezifischen Satztyp: etwas assertieren / erfragen / befehlen / wünschen \dots\ 
	\end{itemize}
\end{itemize}

\end{frame}


%%%%%%%%%%%%%%%%%%%%%%%%%%%%%%%%%%
\begin{frame}
\frametitle{Satztypen \& Satzmodi}

Die wichtigsten Satzmodi des Deutschen ausgehend von (\ref{ex:TypModus}):

\ea\label{ex:TypModus} Proposition: \gqq{Uta ihr Auto verschenkt}
\z

\pause 

\begin{itemize}
	\item \textbf{Modus}: Deklarativ(-satz); Aussage(satz) 
	
	prototyp. \textbf{Satztyp}: Deklarativsatz
	\ea Uta verschenkt ihr Auto.
	\z

\pause 
	
	\item \textbf{Modus}: Interrogativ(-satz); Frage(satz)
	
	prototyp. \textbf{Satztyp}: Interrogativsatz
	
	\ea Verschenkt Uta ihr Auto?
	\ex Ob Uta ihr Auto verschenkt?
	\z
\end{itemize}

\end{frame}


%%%%%%%%%%%%%%%%%%%%%%%%%%%%%%%%%%
\begin{frame}
\frametitle{Satztypen \& Satzmodi}


\begin{itemize}
	\item \textbf{Modus}: Imperativ(-satz); Aufforderungs(satz) 
	
	prototyp. \textbf{Satztyp}: Imperativsatz
	\ea Verschenk dein Auto, Uta!
	\z

\pause 
	
	\item \textbf{Modus}: Exklamativ(-satz); Ausrufe(satz)
	
	prototyp. \textbf{Satztyp}: ?
	
	\ea Uta hat ihr Auto verschenkt!
	\z

\pause 
	
	\item \textbf{Modus}: Optativ(-satz); Wunsch(satz)
	
	prototyp. \textbf{Satztyp}: ?
	
	\ea Wenn Uta nur ihr Auto verschenken würde!
	\z
	
\end{itemize}

Exklamativ und Optativ gelten wegen \textbf{des fehlenden eigenen Satztypen im Deutschen} als marginal.


\end{frame}


%%%%%%%%%%%%%%%%%%%%%%%%%%%%%%%%%%
%%%%%%%%%%%%%%%%%%%%%%%%%%%%%%%%%%
\subsection{Deklarativ}
%\frame{
%\frametitle{~}
%	\tableofcontents[currentsection]
%}


%%%%%%%%%%%%%%%%%%%%%%%%%%%%%%%%%%
\begin{frame}
\frametitle{Deklarativ}

\begin{itemize}
	\item \textbf{Satztyp:} V2-Aussagesatz
	\begin{itemize}
		\item Subkategorisierung: Kein W-Fragewort
		\item Verbstellung: V2 (V in LSK)
		\item Verbmodus: Indikativ (oder Konjunktiv)
		\item Intonation: fallend
	\end{itemize}
	\item \textbf{Funktionstyp:}
	\begin{itemize}
		\item unmarkierter Satzmodus
		\item kann für unterschiedliche Sprechakte verwendet werden (Behauptung, Mitteilung, Vermutung, Aufforderung, \dots )
	\end{itemize}
	
	\ea Du machst heute deine Hausaufgaben.
	\z
	
\end{itemize}

\end{frame}


%%%%%%%%%%%%%%%%%%%%%%%%%%%%%%%%%%
%%%%%%%%%%%%%%%%%%%%%%%%%%%%%%%%%%
\subsection{Interrogativ: E-Interrogativ}
%\frame{
%\frametitle{~}
%	\tableofcontents[currentsection]
%}


%%%%%%%%%%%%%%%%%%%%%%%%%%%%%%%%%%
\begin{frame}
\frametitle{Interrogativ: E-Interrogativ}

\begin{itemize}
	\item Auch: Entscheidungsfrage(satz)
	\item \textbf{Satztyp:} V1-Fragesatz $|$ VL-Fragesatz + \MyPobj{ob}
	\begin{itemize}
		\item Subkategorisierung: Kein W-Fragewort $|$ \MyPobj{ob} in LSK
		\item Verbstellung: V1 (V in LSK), VF leer $|$  VL (V in RSK)
		\item Verbmodus: Indikativ (oder Konjunktiv)
		\item Intonation: steigend
	\end{itemize}
	\item \textbf{Funktionstyp:}
	\begin{itemize}
		\item relativ unmarkierter Satzmodus
		\item kann für unterschiedliche Sprechakte verwendet werden: Fragen, Bitten, Aufforderung, \dots
		\item Eine Antwort wird verlangt.
	\end{itemize}
	
	\eal
	\ex Machst du heute deine Hausaufgaben?
	\ex Ob du heute deine Hausaufgaben machst?
	\zl
	
\end{itemize}

\end{frame}


%%%%%%%%%%%%%%%%%%%%%%%%%%%%%%%%%%
%%%%%%%%%%%%%%%%%%%%%%%%%%%%%%%%%%
\subsection{Interrogativ: K-Interrogativ}
%\frame{
%\frametitle{~}
%	\tableofcontents[currentsection]
%}


%%%%%%%%%%%%%%%%%%%%%%%%%%%%%%%%%%
\begin{frame}
\frametitle{Interrogativ: K-Interrogativ}

\begin{itemize}
	\item Auch: Konstituentenfrage(satz)
	\item \textbf{Satztyp:} V2-Fragesatz
	\begin{itemize}
		\item Subkategorisierung: W-Fragewort im VF
		\item Verbstellung: V2 (V in LSK)
		\item Verbmodus: Indikativ (oder Konjunktiv)
		\item Intonation: steigend
	\end{itemize}
	\item \textbf{Funktionstyp:}
	\begin{itemize}
		\item Eine Antwort dem Fragewort entsprechend wird verlangt.
	\end{itemize}
	
	\eal
	\ex Was machst du heute?
	\ex Wer macht heute seine Hausaufgaben?
	\zl
	
\end{itemize}

\end{frame}


%%%%%%%%%%%%%%%%%%%%%%%%%%%%%%%%%%
%%%%%%%%%%%%%%%%%%%%%%%%%%%%%%%%%%
\subsection{Imperativ}
%\frame{
%\frametitle{~}
%	\tableofcontents[currentsection]
%}


%%%%%%%%%%%%%%%%%%%%%%%%%%%%%%%%%%
\begin{frame}
\frametitle{Imperativ}

\begin{itemize}
	\item \textbf{Satztyp:} V1-Imperativsatz (auch V2 möglich)
	\begin{itemize}
		\item Subkategorisierung: kein W-Fragewort, Subjekt in \textsc{2.sg} und \textsc{pl} wird getilgt
		\item Verbstellung: V1 oder V2 (V in LSK)
		\item Verbmodus: Imperativ
		\item Intonation: fallend
	\end{itemize}
	\item \textbf{Funktionstyp:}
	\begin{itemize}
		\item zum Ausdrücken von Aufforderungen, Bitten, Befehlen, Drohungen, \dots
	\end{itemize}
	
	\eal
	\ex Mach heute deine Hausaufgaben!
	\ex Machen Sie heute Ihre Hausaufgaben!
	\ex Jetzt macht doch eure Hausaufgaben!
	\zl
	
\end{itemize}

\end{frame}


%%%%%%%%%%%%%%%%%%%%%%%%%%%%%%%%%%
%%%%%%%%%%%%%%%%%%%%%%%%%%%%%%%%%%
\subsection{Exklamativ}
%\frame{
%\frametitle{~}
%	\tableofcontents[currentsection]
%}


%%%%%%%%%%%%%%%%%%%%%%%%%%%%%%%%%%
\begin{frame}
\frametitle{Exklamativ}

\begin{itemize}
	\item \textbf{Satztyp:} V1-Exklamativsatz (auch V2 möglich)
	\begin{itemize}
		\item Subkategorisierung: keine Negation, kann W-Wort enthalten, häufig Verwendung von Partikeln
		\item Verbstellung: V1, V2 oder VL
		\item Verbmodus: eher Indikativ
		\item Intonation: fallend
	\end{itemize}
	\item \textbf{Funktionstyp:}
	\begin{itemize}
		\item zum Ausdrücken von Überraschungen (nicht dialogisch!)
	\end{itemize}
	
	\eal
	\ex Hat er (aber auch) tolle Hausaufgaben abgegeben!
	\ex Er hat (aber auch) tolle Hausaufgaben abgegeben!
	\ex Was für tolle Hausaufgaben er abgegeben hat!
	\ex Was für tolle Hausaufgaben hat er abgegeben!
	\zl
		
\end{itemize}

\end{frame}


%%%%%%%%%%%%%%%%%%%%%%%%%%%%%%%%%%
%%%%%%%%%%%%%%%%%%%%%%%%%%%%%%%%%%
\subsection{Optativ}
%\frame{
%\frametitle{~}
%	\tableofcontents[currentsection]
%}


%%%%%%%%%%%%%%%%%%%%%%%%%%%%%%%%%%
\begin{frame}
\frametitle{Optativ}

\begin{itemize}
	\item \textbf{Satztyp:} V1-Optativsatz $|$ VL + \MyPobj{wenn}
	\begin{itemize}
		\item Subkategorisierung: kein W-Fragewort, häufig Verwendung von \MyPobj{nur} oder \MyPobj{doch} $|$ \MyPobj{wenn} + VL
		\item Verbstellung: V1 $|$ VL
		\item Verbmodus: Konjunktiv
		\item Intonation: fallend
	\end{itemize}
	\item \textbf{Funktionstyp:}
	\begin{itemize}
		\item zum Ausdrücken von irrealen Wünschen (nicht dialogisch!)
	\end{itemize}
	
	\eal
	\ex Hätte er (doch / nur) tolle Hausaufgaben abgegeben! 
	\ex Wenn er (doch / nur) tolle Hausaufgaben abgegeben hätte!
	\zl
		
\end{itemize}

\end{frame}


%%%%%%%%%%%%%%%%%%%%%%%%%%%%%%%%%%
%%%%%%%%%%%%%%%%%%%%%%%%%%%%%%%%%%
\section{Hausaufgabe}
%\frame{
%\frametitle{~}
%	\tableofcontents[currentsection]
%}
%%%%%%%%%%%%%%%%%%%%%%%%%%%%%%%%%%


\iftoggle{uebung}{

\begin{frame}
\frametitle{Hausaufgabe}

\begin{itemize}
	\item Bestimmen Sie den Satzmodus der folgenden Sätze, geben Sie dabei die Merkmale zur Bestimmung des Satztyps, sowie den möglichen Funktionstyp an. 
\end{itemize}


\ea \label{ex:Rechnung} Wir haben unsere Rechnungen bezahlt. 
%\ex \label{ex:Wagen} Er hätte einen Wagen kaufen können. 
\ex \label{ex:Folien} Hast du endlich die Folien fertig?
\z

\end{frame}

} 
%% END uebung true = Q
%% BEGIN uebung false = A
{

\begin{frame}
\frametitle{Hausaufgabe (Lösung)}

%\begin{itemize}
%	\item Bestimmen Sie den Satzmodus der folgenden Sätze, geben Sie dabei die Merkmale zur Bestimmung des Satztyps, sowie den möglichen Funktionstyp an. 
%\end{itemize}
\ea \label{ex:Rechnung2} Wir haben unsere Rechnungen bezahlt. 
\pause 
	\begin{itemize}
		\item \alertgreen{Satzmodus: Deklarativsatz}
		\item \alertgreen{Satztyp: V2"=Aussagesatz (Kein W"=Fragewort, Indikativ, Intonation: fallend)}
		\item \alertgreen{Funktionstyp: unmarkierte Mitteilung}
	\end{itemize}
%\ex \label{ex:Wagen} Er hätte einen Wagen kaufen können. 
%	
%	\alertgreen{Deklarativsatz (V2), unmarkierte Vermutung}	
\ex \label{ex:Folien2} Hast du endlich die Folien fertig?
\pause	
\begin{itemize}
		\item \alertgreen{Satzmodus: E-Interrogativsatz}
		\item \alertgreen{Satztyp: V1"=Fragesatz (Kein Fragewort, Indikativ, Intonation: steigend)}
		\item \alertgreen{Funktionstyp: auffordernde Frage (\ras Beeil dich!), Antwort wird verlangt}
\end{itemize}
\z

\end{frame}
}


%%%%%%%%%%%%%%%%%%%%%%%%%%%%%%%%%

\iftoggle{uebung}{

\begin{frame}
\frametitle{Hausaufgabe}

%\ea \label{ex:krank} Ob ich morgen noch krank bin? 
\ea \label{ex:Iss} Iss!
\ex \label{ex:Geld} Wenn ich nur Geld hätte!
%\ex \label{ex:spät} Kannst du mir sagen, wie spät es ist? 
\z

\end{frame}

} 
%% END uebung true = Q
%% BEGIN uebung false = A
{

\begin{frame}
\frametitle{Hausaufgabe (Lösung)} 

%\ea \label{ex:krank} Ob ich morgen noch krank bin? 
%
%\alertgreen{E-Interogativsatz (VL + ob), Frage, Antwort wird verlangt}

\ea \label{ex:Iss2} Iss!
\pause
\begin{itemize}
	\item \alertgreen{Satzmodus: Imperativsatz}
	\item \alertgreen{Satztyp: V1"=Imperativsatz (kein W"=Fragewort, Tilgung des Subjekts in 2.Sg., V1, Verbmodus: Imperativ, fallende Intonation)}
	\item \alertgreen{Funktionstyp: Aufforderung/Befehl}
\end{itemize}
\ex \label{ex:Geld2} Wenn ich nur Geld hätte!
\pause
\begin{itemize}
	\item \alertgreen{Satzmodus: Optativ}
	\item \alertgreen{Satztyp: VL + wenn, kein W"=Fragewort, Verwendung von \emph{nur}, Konjunktiv, fallende Intonation}
	\item \alertgreen{Funktionstyp: irrealer Wunsch}
\end{itemize}
%\ex \label{ex:spät} Kannst du mir sagen, wie spät es ist? 
%
%\alertgreen{E-Interrogativ (V1) + K-Interrogativ (VL), wenn die erste Teilantwort \gqq{ja} ist, dann folgt eine Antwort auf das Fragewort (\gqq{Ja, es ist \dots\ Uhr.})}
\z 

\end{frame}
}


%%%%%%%%%%%%%%%%%%%%%%%%%%%%%%%%%%
%
%\iftoggle{uebung}{
%	
%\begin{frame}
%\frametitle{Hausaufgabe}
%
%\ea \label{ex:Baum} Was für einen tollen Baum hat er gemalt? 
%\ex \label{ex:leise} Sie sind jetzt aber leise!
%\ex \label{ex:Verständnis} Wir bitten um Verständnis.
%
%\z 
%\end{frame}
%
%} 
%%% END uebung true = Q
%%% BEGIN uebung false = A
%{
%
%\begin{frame}
%\frametitle{Hausaufgabe (Lösung)} 
%
%\ea \label{ex:Baum} Was für einen tollen Baum hat er gemalt? 
%
%\alertgreen{K-Interrogativ, Verständnis-Nachfrage}
%
%\ex \label{ex:leise} Sie sind jetzt aber leise!
%
%\alertgreen{Imperativ (V2), Befehl/ Aufforderung}
%
%\ex \label{ex:Verständnis} Wir bitten um Verständnis.
%
%\alertgreen{Deklarativ (V2), Aufforderung}
%
%\z 
%
%\end{frame}
%}


%%%%%%%%%%%%%%%%%%%%%%%%%%%%%%%%%%

\iftoggle{uebung}{
	
\begin{frame}
\frametitle{Hausaufgabe}

%\ea \label{ex:Störung} Verzeihen Sie die Störung.
\ea \label{ex:geschlagen} Wen hast du geschlagen?
\ex \label{ex:gewonnen} Ich habe gewonnen!
%\ex \label{ex:Prüfung} Wenn ich doch die Prüfung bestehe, kaufe ich mir ein Auto.
\z 		

\end{frame}

} 
%% END uebung true = Q
%% BEGIN uebung false = A
{
	
\begin{frame}
\frametitle{Hausaufgabe (Lösung)} 

%\ea \label{ex:Störung} Verzeihen Sie die Störung.
%
%\alertgreen{Imperativ (V1), Bitte}

\ea \label{ex:geschlagen2} Wen hast du geschlagen?
\pause 
\begin{itemize}
	\item \alertgreen{Satzmodus: K"=Interrogativ}
	\item \alertgreen{Satztyp: V2"=Fragesatz, W"=Fragewort im VF, Indikativ, steigende Intonation}
	\item \alertgreen{Funktionstyp: Antwort auf Fragewort wird verlangt}
\end{itemize}


\ex \label{ex:gewonnen2} Ich habe gewonnen!

\pause
\begin{itemize}
	\item \alertgreen{Satzmodus: Exklamativ}
	\item \alertgreen{Satztyp: V2, keine Negation, Indikativ, fallende Intonation}
	\item \alertgreen{Funktionstyp: Ausdruck einer Überraschung}
\end{itemize}


%\ex \label{ex:Prüfung} Wenn ich doch die Prüfung bestehe, kaufe ich mir ein Auto.
%
%\alertgreen{Deklarativ (V2), an Bedingung geknüpfte Mitteilung}

\z 

\end{frame}

}


%%%%%%%%%%%%%%%%%%%%%%%%%%%%%%%%%%%	

\iftoggle{uebung}{
	
\begin{frame}
\frametitle{Hausaufgabe}

\begin{itemize}
	\item Geben Sie eine Analyse der folgenden Sätze (\textbf{inkl. Nebensätze}) nach dem topologischen Feldermodell.
\end{itemize}

	\ea \label{ex:Top1} Wenn ich die Prüfung bestehe, werde ich mir ein Buch von Chomsky kaufen.
	\ex \label{ex:Top2} Werde ich mir das Buch kaufen, wenn ich bestehe?
	\z 		
	
\end{frame}

} 
%% END uebung true = Q
%% BEGIN uebung false = A
{

\begin{frame}
\frametitle{Hausaufgabe (Lösung)} 

\begin{itemize}
	\item Geben Sie eine Analyse der folgenden Sätze (\textbf{inkl. Nebensätze}) nach dem topologischen Feldermodell.
\end{itemize}

\ea \label{ex:Top1} Wenn ich die Prüfung bestehe, werde ich mir ein Buch von Chomsky kaufen.
\ex \label{ex:Top2} Werde ich mir das Buch kaufen, wenn ich bestehe?
\z 		



\begin{table}
	\centering
	\scalebox{0.8}{
\alertgreen{\begin{tabular}{p{3.5cm}|l|p{4cm}|p{1.5cm}|p{3cm}}
		\textbf{VF} & \textbf{LSK} & \textbf{MF} & \textbf{RSK} & \textbf{NF} \\ 
		\hline
		Wenn ich die Prüfung bestehe, & werde & ich mir ein Buch von Chomsky & kaufen. &\\
		\hline
														&Wenn& ich die Prüfung 							& bestehe & \\
		\hline
		\hline 
														&Werde& ich mir das Buch						&kaufen,	& wenn ich bestehe? \\
		\hline 
														&wenn& ich												&bestehe	&\\
\end{tabular}}
	}
\end{table}


\end{frame}

}


%%%%%%%%%%%%%%%%%%%%%%%%%%%%%%%%%%%		