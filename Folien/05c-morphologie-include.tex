%%%%%%%%%%%%%%%%%%%%%%%%%%%%%%%%%%%%%%%%%%%%%%%%%%%%
%%%             Metadata                         %%%
%%%%%%%%%%%%%%%%%%%%%%%%%%%%%%%%%%%%%%%%%%%%%%%%%%%%      

\title{Grundkurs Linguistik}

\subtitle{Morphologie III: Wortbildung II \& Flexion}

\author[aMyP]{
	{\small Antonio Machicao y Priemer}
%	\\
%	{\footnotesize \url{http://www.linguistik.hu-berlin.de/staff/amyp}\\
%	\href{mailto:mapriema@hu-berlin.de}{mapriema@hu-berlin.de}}
}

\institute{Institut für deutsche Sprache und Linguistik}

%%%%%%%%%%%%%%%%%%%%%%%%%      
\date{ }
%\publishers{\textbf{6. linguistischer Methodenworkshop \\ Humboldt-Universität zu Berlin}}

%\hyphenation{nobreak}


%%%%%%%%%%%%%%%%%%%%%%%%%%%%%%%%%%%%%%%%%%%%%%%%%%%%
%%%             Preamble's End                   %%%
%%%%%%%%%%%%%%%%%%%%%%%%%%%%%%%%%%%%%%%%%%%%%%%%%%%%      


%%%%%%%%%%%%%%%%%%%%%%%%%      
\huberlintitlepage

\iftoggle{toc}{
\frame{
\begin{multicols}{2}
	\frametitle{Inhaltsverzeichnis}\tableofcontents
	%[pausesections]
\end{multicols}
	}
	}


%%%%%%%%%%%%%%%%%%%%%%%%%%%%%%%%%%%
%%%%%%%%%%%%%%%%%%%%%%%%%%%%%%%%%%

%\nocite


%%%%%%%%%%%%%%%%%%%%%%%%%%%%%%%%%%%
%%%%%%%%%%%%%%%%%%%%%%%%%%%%%%%%%%



%%%%%%%%%%%%%%%%%%%%%%%%%%%%%%%%%%
%%%%%%%%%%%%%%%%%%%%%%%%%%%%%%%%%%
\section{Derivation}

%\frame{
%\frametitle{~}
%	\tableofcontents[currentsection]
%}


%%%%%%%%%%%%%%%%%%%%%%%%%%%%%%%
%%%%%%%%%%%%%%%%%%%%%%%%%%%%%%%%%%%
%%%%%%%%%%%%%%%%%%%%%%%%%%%%%%%%%%%

\begin{frame}
\frametitle{Derivation}

\begin{itemize}
	\item Derivation:
	
	\begin{itemize}
		\item[]
		\item Bildung von Wörtern (auch Lemmata oder Lexemen) mittels Affixen.
		\item []
		\item Derivation und Flexion sind Affigierungen (formal).
		\item[]
		\item Flexion $\rightarrow$ neue Wort\emph{formen}
	\end{itemize}
	
\end{itemize}


\end{frame}


%%%%%%%%%%%%%%%%%%%%%%%%%%%%%%%%%%%
%%%%%%%%%%%%%%%%%%%%%%%%%%%%%%%%%%%

\subsection{Die Basis}
%\frame{
%\frametitle{~}
%	\tableofcontents[currentsection]
%}


%%%%%%%%%%%%%%%%%%%%%%%%%%%%%%%%%%%
\begin{frame}
\frametitle{Die Basis}

\begin{itemize}
	\item \textbf{Basis:} (Pl. Basen)
	
	\begin{itemize}
		\item etwas, woran etwas affigiert werden kann
		\item Ausgangsform der \textbf{Derivation}
	\end{itemize}
	
	\item \textbf{Konkatenation} eines wortfähigen Elements (meistens eines freien Morphems) mit einem Affix
	\item[]
	\item Die Basis kann:
	
	\begin{itemize}
		\item \textbf{morphologisch einfach} (eine Wurzel) sein:
		
		\ea ehrbar: [ehr] + [-bar]
		\z
		
		\item oder \textbf{morphologisch komplex} (eine Wurzel mit einem oder mehreren Affixen) sein:
		
		\ea Ehrbarkeit: [ [Ehr] + [-bar] ] + [-keit]
		\z
		
		\item oder ein \textbf{Kompositum}:
		
		\ea ehrgefühlsmäßig: [ [ehr] + [gefühl(s)] ] + [mäßig]
		\z
		
	\end{itemize}
	
\end{itemize}


\end{frame}



%%%%%%%%%%%%%%%%%%%%%%%%%%%%%%%%%%%

\begin{frame}
\frametitle{Die Basis}

\begin{itemize}
	\item Wortartenwechsel
	
	\begin{itemize}
		\item Verb \ras Adjektiv, Substantiv
		
		\ea verkaufen \ras verkäuflich, Verkäufer
		\z
		
		\item Adverb \ras Adjektiv
		
		\ea heute \ras heutig
		\z
		
		\item Adjektiv \ras Substantiv, Verb, Adverb
		
		\eal 
			\ex schön \ras Schönheit, beschönigen
			\ex klug \ras klugerweise
		\zl
		
		\item Substantiv \ras Adjektiv, Verb, Adverb

		\eal 
			\ex Arzt \ras ärztlich, verarzten
			\ex Nacht \ras nachts
		\zl
		
	\end{itemize}
	
\end{itemize}


\end{frame}



%%%%%%%%%%%%%%%%%%%%%%%%%%%%%%%%%%%

\begin{frame}
\frametitle{Die Basis}

\begin{itemize}
	\item Wie erkennt man die Wortart der Basis?
	
	\begin{itemize}
		\item Substantiv-Verb-Derivationen \ras Semantik des Substantivs:
		
		\eal 
			\ex Reifen \ras bereifen
			\ex abnehmen \ras Abnahme
		\zl
		
		\item Handelt es sich beim Substantiv um ein \textbf{Objekt} (im Sinne von Ding) o.\,ä. ist meist das Substantiv zugrunde liegend, handelt es sich aber um einen \textbf{Vorgang}, ist meist das Verb zugrunde liegend.
		\item[]
		\item Weiterer Hinweis: Betrachten Sie die \textbf{Affixe}, die zur Derivation benutzt werden. Diese sind nämlich in der Regel bezüglich der Wortart, mit der sie sich verbinden können, beschränkt.
		
	\end{itemize}
	
\end{itemize}


\end{frame}


%%%%%%%%%%%%%%%%%%%%%%%%%%%%%%%%%%%
%%%%%%%%%%%%%%%%%%%%%%%%%%%%%%%%%%%
\subsection{Das Suffix}
%\frame{
%\frametitle{~}           
%	\tableofcontents[currentsection]
%}
%%%%%%%%%%%%%%%%%%%%%%%%%%%%%%%%%%%

\begin{frame}
\frametitle{Das Suffix}

\begin{itemize}
	\item \textbf{Suffixe} bestimmen die kategoriale Zugehörigkeit des Derivats, Präfixe tun das im Allgemeinen nicht:


	\ea Glück (N) -- glücklich (Adj) –- Unglück (N)
	\z
	
	\ea heizen (V) –- Heizung (N) –- vorheizen (V)
	\z
	
	\ea hören (V) –- hörbar (Adj) –- verhören (V)
	\z
	
	\ea achten (V) –- achtbar (Adj) -– missachten (V)
	\z

	\item Suffixe \ras \textbf{Köpfe}

	\item Der Kopf  \ras \textbf{rechtsperipher} (wie bei der Komposition)
\end{itemize}

\end{frame}



%%%%%%%%%%%%%%%%%%%%%%%%%%%%%%%%%%%

\begin{frame}
\frametitle{Das Suffix}

\begin{itemize}
	\item Ausnahmen (Präfix als Kopf?):
	
	\begin{itemize}
		\item be-, ent-, er-, ver-, zer-, durch-, über-, um-, unter- können aus Substantiven, Adjektiven und/ oder Partikeln Verben ableiten:
		\item[]
		\item mit Basis Substantiv:
		
		\ea besohlen (*sohlen), entkernen (*kernen), ergaunern (*gaunern)
		\z
		
		\item mit Basis Adjektiv:
		
		\ea verlängern (*längern), überraschen (*raschen)
		\z
		
		\item mit Basis Partikel:
		
		\ea bejahen, verneinen
		\z
		
	\end{itemize}

	\item In diesen Fällen werden manchmal die \textbf{Präfixe als Köpfe} analysiert, wobei das \textbf{Kopf-rechts-Prinzip} verletzt wird.
	
\end{itemize}

\end{frame}



%%%%%%%%%%%%%%%%%%%%%%%%%%%%%%%%%%%

\begin{frame}
\frametitle{Das Suffix}

\begin{figure}
\centering

\begin{minipage}[c]{0.55\textwidth}

\begin{itemize}
	\item Im Allgemeinen gilt für die Suffigierung die folgende Wortstrukturregel:
	
	\begin{itemize}
		\item X \ras Y $X^{af}$
	\end{itemize}
	
	\item Das Suffix \emph{-lich} wird als Adjektiv bildendes Affix markiert.
	\item Diese Darstellung macht deutlich, dass das Affix sein Wortkategorienmerkmal an den Mutterknoten weitergibt (\textbf{Perkolation}) und damit die Kategorie des \textbf{Derivats} bestimmt.
\end{itemize}

\end{minipage}
\begin{minipage}[c]{0.35\textwidth}

\begin{forest}
sm edges,
[Adj
	[N
		[glück]]
	[Aaf
		[lich]]
]
\end{forest}

\end{minipage}

\end{figure}

\end{frame}



%%%%%%%%%%%%%%%%%%%%%%%%%%%%%%%%%%%

\begin{frame}
\frametitle{Das Suffix}

\begin{itemize}
	\item Suffixe lassen sich somit danach unterscheiden, welche Art von Stämmen sie bilden.
	
	\begin{itemize}
		\item Beispiele:
		
		\ea -ung, -heit/-keit, -er, -schaft bilden Substantivstämme
		\z
		
		\ea -bar, -lich, -haft, -ig bilden Adjektivstämme
		\z
		
		\ea -(e)l, -(is/ifiz)ier, -ig bilden Verbstämme (häkeln, schlängeln, stabilisieren, ängstigen)
		\z
		
	\end{itemize}
	
\end{itemize}


\end{frame}



%%%%%%%%%%%%%%%%%%%%%%%%%%%%%%%%%%%

\begin{frame}
\frametitle{Das Suffix}

\begin{itemize}
	\item Ebenfalls wie in der Komposition bringt der Kopf auch bei der Derivation die Eigenschaft Genus (und andere morphosyntaktische Eigenschaften) mit sich und vererbt sie an das Derivat:
	
\vspace{1em}
	
	\begin{itemize}
		\item \textbf{-ung}: fem Achtung
		\item \textbf{-keit}: fem Tapferkeit
		\item \textbf{-bold}: mask Witzbold
		\item \textbf{-chen}: neut Häuschen
		\item \textbf{-ling}: mask Sonderling
		\item \textbf{-tum}: neut Brauchtum (aber: mask Reichtum)
		\item \textbf{-ian}: mask Grobian
	\end{itemize}
	
\end{itemize}


\end{frame}


%%%%%%%%%%%%%%%%%%%%%%%%%%%%%%%%%%%
%%%%%%%%%%%%%%%%%%%%%%%%%%%%%%%%%%%
\section{Beschränkungen der Basis}
%\frame{
%\frametitle{~}
%	\tableofcontents[currentsection]
%}


%%%%%%%%%%%%%%%%%%%%%%%%%%%%%%%%%%%
\begin{frame}
\frametitle{Beschränkungen der Basis}

\begin{itemize}
	\item Der Kopf bestimmt auch, welche Komplemente er nimmt. (Ein Komplement ist ein \gqq{Ergänzungsstück})
	\item Dabei gibt es verschiedene Arten von Beschränkungen:

\vspace{1em}
	
	\begin{itemize}
		\item \textbf{Syntaktische Beschränkungen}
		
		\begin{itemize}
			\item \textit{-bar} verbindet sich mit Verben:
			
			\ea lesbar, essbar, erziehbar vs. *grünbar
			\z
			
			\item \textit{-heit/-keit} verbinden sich mit Adjektiven:		
			
			\ea Blödheit, Freiheit, Unachtsamkeit \vs *Lesheit, *Esskeit
			\z
			
		\end{itemize}
		
	\item \textbf{Argumentstrukturelle Beschränkungen}
	
		\begin{itemize}
			\item \textit{-bar} verbindet sich mit transitiven Verben:	
			
			\ea *schlafbar, *frierbar
			\z
			
		\end{itemize}
		
	\end{itemize}
	
\end{itemize}


\end{frame}



%%%%%%%%%%%%%%%%%%%%%%%%%%%%%%%%%%%

\begin{frame}
\frametitle{Beschränkungen der Basis}

\begin{itemize}
	\item[]
	
	\begin{itemize}
		\item \textbf{Phonologische Beschränkungen}
		
		\begin{itemize}
			\item -keit folgt ausschließlich auf unbetonte Silben: 				
			
			\ea 'Wachsamkeit \vs *'Freikeit, *Nettheit \vs Nettigkeit
			\z
			
			\item[](aber nicht nach -haft, -los, -en, -e: *Schadhaftkeit, *Rastloskeit, *Müdekeit)
			
			\item[]
			\item \textit{-heit} lässt betonte und unbetonte Silben zu: 
			
			\ea 'Freiheit, 'Schüchternheit’, 
			\z
			
			\item[] (außer -e, -bar, -ig, -isch, -lich, -mäßig, -sam, -haft, -los)
			\item[]
			\item \textit-ei verbindet sich mit Wörtern, deren letzte Silbe unbetont ist (ansonsten werden die Allomorphe -erei/-elei verwendet): 

			\ea Wüstenei \vs Rennerei, Liebelei
			\z
			
		\end{itemize}
		
	\end{itemize}

\end{itemize}

\end{frame}



%%%%%%%%%%%%%%%%%%%%%%%%%%%%%%%%%%%

\begin{frame}
\frametitle{Beschränkungen der Basis}

\begin{itemize}
	\item[]

	\begin{itemize}
		\item \textbf{Morphologische Beschränkungen}
	
		\begin{itemize}
			\item Ge-{\dots}-e verbindet sich nicht mit komplexen Verben:
		
			\ea Gerede, Gemeckere \vs *Geverkaufe, *Geentlasse (Aber: Herumgehupe)
			\z
			
			\item -lich verbindet sich nicht mit Abkürzungen:
		
			\ea sportlich \vs *SPDlich, *DGfSlich
			\z
			
			\item -heit folgt auf Partizipien (-keit nicht):
		 
			\ea Gelassenheit, Aufgeregtheit, Zurückgezogenheit
		 	\z
		 	
		\end{itemize}
	
	\end{itemize}

\end{itemize}

\end{frame}



%%%%%%%%%%%%%%%%%%%%%%%%%%%%%%%%%%%

\begin{frame}
\frametitle{Beschränkungen der Basis}

\begin{itemize}
	\item[]
	
	\begin{itemize}
		\item \textbf{Semantische / konzeptuelle Beschränkungen}
	
		\begin{itemize}
			\item -fach verbindet sich nur mit Zahlen und \gqq{Quantitätsausdrücken}:
		
			\ea zweifach, hundertfach, vielfach, mehrfach \vs *grünfach, *freifach
			\z
		
			\item \textit{Ge-{\dots}-e} verbindet sich nicht mit (stativen) Verben, die einen Zustand ausdrücken:
		
			\ea Gerenne \vs *Gewisse, *Gekenne
			\z
			
		\end{itemize}
		
	\item \textbf{Beschränkungen der Herkunft:}
	
	\begin{itemize}
		\item -bar verbindet sich mit nativen und sog. neoklassischen Basen, -abel hingegen nur mit letzteren: 
		
		\ea tanzbar, nachvollziehbar, annehmbar, akzeptierbar vs. akzeptabel, *annehmabel (akzeptieren: lat. Ursprung)
		\z
		
		\end{itemize}
		
	\end{itemize}
	
\end{itemize}


\end{frame}



%%%%%%%%%%%%%%%%%%%%%%%%%%%%%%%%%%%

\begin{frame}
\frametitle{Beschränkungen der Basis}

\begin{itemize}
\item[]

	\begin{itemize}
	
		\item \textbf{Pragmatische Beschränkungen:}
	
		\begin{itemize}
			\item -er bildet Agensnomina: 
		
			\ea Raucher, Lastwagenfahrer, Linkshänder
			\z
			
			\item -er bildet Nomina instrumenti: 
		
			\ea Korkenzieher, Aktenordner
			\z
			
			\item -er bildet Nomina acti: 
		
			\ea Rülpser, Treffer
			\z
			
		\end{itemize}
		
		\item Fazit: Beschränkungen können alle linguistischen Ebenen betreffen. Sowohl Basen als auch Derivationsaffixe müssen mit all diesen Eigenschaften im Lexikon gespeichert sein.
	
	\end{itemize}

\item \textbf{ÜB.2}	
\end{itemize}

\end{frame}



%%%%%%%%%%%%%%%%%%%%%%%%%%%%%%%%%%%
%%%%%%%%%%%%%%%%%%%%%%%%%%%%%%%%%%%
\section{Präfixe und Zirkumfixe}
%\frame{
%\frametitle{~}
%	\tableofcontents[currentsection]
%}


%%%%%%%%%%%%%%%%%%%%%%%%%%%%%%%%%%%

\begin{frame}
\frametitle{Präfixe und Zirkumfixe}

\begin{itemize}
	\item Präfixe sind keine Köpfe (Ausnahmen, s.u.), aber Zirkumfixe sind Köpfe!
	\item Präfixe und Zirkumfixe treten an eine Derivationsbasis heran, wobei auch meist beschränkt ist, welche Kategorie die Basis haben kann.
	
	\begin{itemize}
		\item Nominale und adjektivische Präfixe sind:
		
		\eal
			\ex erz-: Erzfeind, erzreaktionär
			\ex[]
			\ex ge-: Gebüsch
			\ex[]
			\ex miss-: Misserfolg, missgelaunt
			\ex[]
			\ex un-: Ungeduld, unsauber
			\ex[]
			\ex ur-: Urgestein, uralt
		\zl
		
	\end{itemize}

\end{itemize}

\end{frame}




%%%%%%%%%%%%%%%%%%%%%%%%%%%%%%%%%%%

\begin{frame}
\frametitle{Präfixe und Zikumfixe}

\begin{itemize}
	\item Verbale Präfixe haben wir schon weiter oben kennengelernt.
	\item[]
	\item Zirkumfixe:
	
	\eal 
		\ex ge\dots e: Gelache
		\ex[]
		\ex ge{\dots}ig: geräumig
		\ex[]
		\ex un{\dots}lich: unglaublich
		\ex[]
		\ex \textit{un{\dots}bar}: unnahbar, unkaputtbar, unplattbar
		\ex[]
		\ex un{\dots}sam: unwegsam
		\ex[]
		\ex be / ge / ent / zer{\dots}t: bejahrt, genarbt, entgeistert, zernarbt
	\zl
	
\end{itemize}


\end{frame}



%%%%%%%%%%%%%%%%%%%%%%%%%%%%%%%%%%%

\begin{frame}
\frametitle{Präfixe und Zirkumfixe}

\begin{itemize}
	\item Im Baum sehen Präfixe wie folgt aus:
	
\begin{forest}
sm edges,
[Adj
	[Aaf
		[erz]]
	[A
		[reaktionär]]
]
\end{forest}

\item Zirkumfixe stellen natürlich ein Problem für diese Darstellung dar.
\end{itemize}

\end{frame}



%%%%%%%%%%%%%%%%%%%%%%%%%%%%%%%%%%%

\begin{frame}
\frametitle{Bedeutung von Affixen}

\begin{itemize}
	\item Die Bedeutung der Affixe ist nicht immer semantisch eindeutig erfassbar \ras ambig
	\item Meist haben Affixe eher einen grammatischen als einen semantischen Signalwert. Aber wir finden auch \textbf{produktive} Muster/Reihen mit klarem Bedeutungsbeitrag:
	
	\ea sich verfahren, versprechen, verschreiben, verlaufen, verhören…\\
	\ras('etw. falsch machen')
	\z
	
	\ea Benzin verfahren, Tinte verschreiben, Geld verspielen\dots \\
\ras ('etw. verbrauchen')\\
Aber:\\
verkaufen ('etw. gegen Geld tauschen')\\
verärgern ('jemanden ärgerlich machen')\\
verarmen ('arm werden')\\
verhungern ('aus Mangel an Nahrung sterben')
	\z

\end{itemize}


\end{frame}



%%%%%%%%%%%%%%%%%%%%%%%%%%%%%%%%%%%

\begin{frame}
\frametitle{Bedeutung von Affixen}

\begin{itemize}
	\item Beispiele für Bedeutungen von Suffixen:

\vspace{1em}

	\ea -ung: Besichtigung\\
	\ras Geschehen als Kontinuum und Resultat
	\z
	
	\ea -erei: Besserwisserei\\
\ras iteratives, unerwünschtes Geschehen
	\z

	\ea -er: Seufzer, Ausrutscher\\
\ras Geschehen als Einzelakt
	\z

	\ea -er: Maler, Raucher\\ 
	\ras Handelnder (Aber: Aufkleber)
	\z

\end{itemize}


\end{frame}



%%%%%%%%%%%%%%%%%%%%%%%%%%%%%%%%%%%

\begin{frame}
\frametitle{Bedeutung von Affixen}

\begin{itemize}
	\item Manche Muster sind \textbf{produktiv} andere lediglich \textbf{aktiv}.
	\item \textbf{Produktiv} meint, dass nach diesem Muster (unbemerkt) neue Wörter gebildet werden, während \textbf{aktive} Muster erkannt werden können, aber nicht mehr (oder selten und dann stilistisch markiert) verwendet werden.
	\item Ist ein Muster nicht mehr aktiv, nimmt man die Form als \textbf{Simplex} wahr: Ursache, Mädchen
	
	\ea \textbf{produktives Muster:} -ung-Suffigierung, ver-Präfigierung, Diminutivbildung mit -chen, Nominalisierung mit -heit und -keit.
	\z
	
	\ea  \textbf{aktives Muster:} -sam bei Adjektiven (geruhsam, sittsam,
unterhaltsam)
	\z

	\item \textbf{ÜB.3}
	
\end{itemize}


\end{frame}


%%%%%%%%%%%%%%%%%%%%%%%%%%%%%%%%%%%

\begin{frame}
\frametitle{Bedeutung von Affixen: Spezialfälle}

\begin{itemize}
	\item Sind die unterstrichenen Einheiten in den folgenden Wörtern Kompositionsglieder oder Affixe? 
	
\vspace{1em}
	
	\eal 
	\ex Verkehrs\underline{wesen}
	\ex[]
	\ex Schul\underline{wesen}
	\ex[]
	\ex Laub\underline{werk}
	\ex[]
	\ex \underline{Haupt}bahnhof
	\ex[]
	\ex abgas\underline{arm}
	\zl
	
\end{itemize}

\end{frame}



%%%%%%%%%%%%%%%%%%%%%%%%%%%%%%%%%%%

\begin{frame}
\frametitle{Bedeutung von Affixen: Spezialfälle}

\begin{itemize}
	\item Für Kompositionsglieder spricht: Morpheme treten auch \textbf{frei} auf.
	\item[]
	\item Für Affixe spricht: Morpheme haben im komplexen Wort eine sehr viel abstraktere Bedeutung.
	\item[]
	\item Es gibt hier unterschiedliche Auffassungen und es wurden auch Kompromisse wie der folgende vorgeschlagen:
	
	\begin{itemize}
		\item Es handelt sich bei den unterstrichenen Elementen um \textbf{Affixoide} (Suffixoide, Präfixoide), sog. \gqq{Halbaffixe}, da sie einerseits \textbf{reihenbildend} sind und andererseits \textbf{eine bedeutungsverwandte, freie Form} neben sich haben.
		
		\ea Wesen, Werk, Haupt, arm
		\z
		
	\end{itemize}
	
\end{itemize}


\end{frame}


%%%%%%%%%%%%%%%%%%%%%%%%%%%%%%%%%%%
%%%%%%%%%%%%%%%%%%%%%%%%%%%%%%%%%%%
\section{Partikelverbbildung: Komposition vs. Derivation}
%\frame{
%\frametitle{~}
%	\tableofcontents[currentsection]
%}


%%%%%%%%%%%%%%%%%%%%%%%%%%%%%%%%%%%

\begin{frame}
\frametitle{Partikelverbbildung: Komposition \vs Derivation}

\begin{itemize}
	\item Verben können auch aus mehreren Morphemen zusammengesetzt sein \ras komplex 
	\item[]
	\item Suffigierung in der verbalen Wortbildung \ras sehr selten (anders als bei Nomina und Adjektiven)
	\item[]
	\item nativ (sehr selten, wahrscheinlich nicht produktiv) -el:
	
	\ea lächel, hüstel, fremdel, \dots
	\z

	\item[]
	\item neoklassisch: -ier (und damit -ifiz-ier, -is-ier):
	
	\ea probier, elektrifizier, alphabetisier, \dots 
	\z
	
\end{itemize}

\end{frame}



%%%%%%%%%%%%%%%%%%%%%%%%%%%%%%%%%%%

\begin{frame}
\frametitle{Partikelverbbildung: Komposition \dots Derivation}

\begin{itemize}
	\item Die produktiven Muster zur Bildung komplexer Verben verwenden eher Präfixe und Partikeln.
	\item[]
	\item Bei den Präfixverbbildungen (beraten, verkaufen) handelt es sich um \textbf{Derivation}, da die Präfixe nicht frei vorkommen.
	\item[]
	\item Die Partikeln dagegen kommen auch frei vor, was Grund zur Annahme gibt, den Wortbildungsprozess als \textbf{Komposition} zu betrachten:
	
	\ea teilnehmen, festmachen, aufstellen
	\z
	
\end{itemize}

\end{frame}



%%%%%%%%%%%%%%%%%%%%%%%%%%%%%%%%%%%

\begin{frame}
\frametitle{Partikelverbbildung: Komposition \vs Derivation}

\begin{itemize}
	\item Hier verbindet sich ein Verb mit einem Substantiv, einem Adjektiv und einer Präposition.
	\item[]
	\item \textbf{Partikelverbbildung} ist keine Komposition, sondern ein eigener Wortbildungsprozess, denn, wie in der folgenden Unterscheidung klar wird, sind Partikelverben syntaktisch und morphologisch trennbar, was für die herkömmlichen Komposita nicht gilt.
\end{itemize}


\end{frame}



%%%%%%%%%%%%%%%%%%%%%%%%%%%%%%%%%%%

\begin{frame}
\frametitle{Partikelverbbildung: Komposition \vs Derivation}

\begin{itemize}
	\item Partikelverben sind wie folgt von Präfixverben zu unterscheiden:
	
	\ea  teilnehmen \vs bereifen
	\z
	
	\begin{itemize}
		\item Betonung:
		
		\ea 'teilnehmen \vs be'reifen
		\z
		
		\item Syntaktische Trennbarkeit:
		
		\eal
		\ex Ich nehme an dem Kongress teil.
		\ex[]\vs
		\ex Ich bereife den Wagen.
		\zl
			
		\item Morphologische Trennbarkeit:
		
		\eal 
		\ex teilgenommen \vs bereift 
		\ex teilzunehmen \vs zu bereifen 
		\zl
		
	\end{itemize}
	
	\item \textbf{ÜB.4}
\end{itemize}

\end{frame}



%%%%%%%%%%%%%%%%%%%%%%%%%%%%%%%%%%%

\begin{frame}
\frametitle{Partikelverbbildung: Komposition \vs Derivation}

\begin{table}
\centering

\begin{tabular}{p{1.6cm}|p{3cm}|p{3cm}|p{3cm}}
 & \textbf{Simplex} & \textbf{Präfixverb} & \textbf{Partikelverb}\\
& \textit{\textbf{kaufen}} & \textit{\textbf{ver + kaufen}} & \textit{\textbf{auf + kaufen}}\\
\hline
\textbf{NS} & [\dots] dass Peter die Firma \textit{kauft}. & [\dots] dass Peter die Firma \textit{verkauft}. & [\dots] dass Peter die Firma \textit{aufkauft}.\\
\hline
\textbf{HS} & Peter \textit{kauft} die Firma. & Peter \textit{verkauft} die Firma. & Peter \textit{kauft} die Firma \textit{auf}. \\
\hline
\textbf{Inf. mit zu} & Peter denkt nicht daran, die Firma \textit{zu kaufen}. & Peter denkt nicht daran, die Firma \textit{zu verkaufen}. & Peter denkt nicht daran, die Firma \textit{aufzukaufen}.\\
\hline
\textbf{Part. II} & Peter hat die Firma gekauft. & Peter hat die Firma verkauft. & Peter hat die Firma aufgekauft. \\
\hline
\textbf{Betonung} & \textit{'kaufen} & \textit{ver'kaufen} & \textit{'aufkaufen}\\
\end{tabular}

\end{table}


\end{frame}


%%%%%%%%%%%%%%%%%%%%%%%%%%%%%%%%%%%
%%%%%%%%%%%%%%%%%%%%%%%%%%%%%%%%%%%
\section{Konversion}
%\frame{
%\frametitle{~}
%	\tableofcontents[currentsection]
%}
%

%%%%%%%%%%%%%%%%%%%%%%%%%%%%%%%%%%%

\begin{frame}
\frametitle{Konversion}

\begin{itemize}
	\item Konversion ist die Umkategorisierung eines Stamms (ohne Flexionselemente) ohne Hilfe von Derivationsaffixen.
	
	\eal 
	\ex schlaf(en) \ras Schlaf
	\ex find(en) \ras Fund (vgl. gefunden)
	\zl
	
	\item Bei (53b) werden verschiedene Stämme gebraucht. Man kann dies ebenfalls zur Konversion rechnen, da ja keine Affixe benutzt werden. Wir haben jedoch diesen Wortbildungsprozess \textbf{\gqq{implizite Derivation}} genannt, und somit gehört er in unserer Klassifikation zur Derivation.
	\item []
	\item In einigen Theorien wird Konversion überhaupt als Derivation behandelt. Man nimmt dabei an, dass es ein \textbf{Nullmorphem} gibt, das die Kopfeigenschaften besitzt, s.\,u.
	
\end{itemize}


\end{frame}



%%%%%%%%%%%%%%%%%%%%%%%%%%%%%%%%%%%

\begin{frame}
\frametitle{Konversion}

\begin{itemize}
	\item \textbf{Syntaktische Konversion}\\ auch \gqq{Transposition} genannt, von einigen Grammatikern nicht zur Wortbildung, sondern zur \textbf{Syntax} gerechnet. Sie ist die Umklassifizierung eines Stammes mit seinen Flexionselementen:
	
	\eal 
	\ex lauf (en) -- das Laufen 
	\ex gefallene -- der / die / das Gefallene
	\zl
	
	\item \textbf{Morphologische Konversion:}
	
	\eal 
	\ex lauf (en) -- der Lauf
	\ex Kleid -- kleid (en)
	\zl
	
\end{itemize}


\end{frame}



%%%%%%%%%%%%%%%%%%%%%%%%%%%%%%%%%%%

\begin{frame}
\frametitle{Konversion}

	\ex. grasen

\begin{figure}
\centering

\begin{minipage}{0.45\textwidth}

\begin{itemize}
	\item Wortbildungstruktur:
\end{itemize}

\begin{forest}
sm edges,
[V
	[V
		[N
			[\textcolor{red}{gras}]]]
	[Fl
		[\textcolor{red}{en}]]
]
\end{forest}

\end{minipage}
%
\begin{minipage}{.45\textwidth}

\begin{itemize}	
	\item Alternative:
\end{itemize}

\begin{forest}
sm edges,
[V
	[V
		[N
			[\textcolor{red}{gras}]]
		[V
			[$\emptyset$]]
	]
	[Fl
		[\textcolor{red}{en}]]
]
\end{forest}

\end{minipage}

\end{figure}

\begin{itemize}
	\item Achtung: -en ist eine Flexionsendung, kein Wortbildungsaffix, deswegen $\emptyset$
\end{itemize}

\end{frame}



%%%%%%%%%%%%%%%%%%%%%%%%%%%%%%%%%%%

\begin{frame}
\frametitle{Konversion}

\begin{itemize}
	\item Auch bei der Konversion gibt es wieder vielfältige Möglichkeiten der Wortartenüberführung:
	
	\begin{itemize}
		\item Substantivbildung aus: 
		
		\ea Adjektiv: fremd \ras Fremder, Fremde\\ blau \ras das Blau
		\z
			
		\ea Verb: lesen \ras  Lesen, \\schlafen \ras Schlaf 
		\z
		
		\ea (Partizip: angestellt \ras Angestellter\\ reisend \ras  Reisender)
		\z
			
		\ea Andere Wortarten: nein \ras  das Nein\\ kein Wenn \& Aber 
		\z
		
		\ea Syntaktische Fügung: das So-tun-als-ob
		\z
		
	\end{itemize}
	
\end{itemize}

\end{frame}



%%%%%%%%%%%%%%%%%%%%%%%%%%%%%%%%%%%

\begin{frame}
\frametitle{Konversion}

\begin{itemize}
	\item[]
	
	\begin{itemize}

		\item Verbbildung aus: 
		
		\ea Adjektiv: grün \ras grünen\\ rot \ras röten 
		\z
		
		\ea Substantiv: Öl \ras ölen 
		\z
		
		\item[]

		\item Adjektivbildung aus: 
		
		\ea Substantiv: ernst \ras der Ernst 
		\z
		
		\ea Verb (Partizip): reizend, ausgezeichnet \ras reizend, ausgezeichnet
		\z
		
		\item[]
	
	\end{itemize}

	\item \textbf{ÜB.4}
	
\end{itemize}

\end{frame}



%%%%%%%%%%%%%%%%%%%%%%%%%%%%%%%%%%%
%%%%%%%%%%%%%%%%%%%%%%%%%%%%%%%%%%%
\section{Rekursivität}
%\frame{
%\frametitle{~}
%	\tableofcontents[currentsection]
%}


%%%%%%%%%%%%%%%%%%%%%%%%%%%%%%%%%%%

\begin{frame}
\frametitle{Rekursivität}

\begin{itemize}
	\item Sowohl Derivation wie auch Komposition können mehrfach an einem Wort angewendet werden, wobei hier aber nicht mehrfach dieselbe Affigierung stattfinden kann (sonst müsste es möglich sein, mehrfach dasselbe Affix hintereinander zu benutzen).
	\item[]
	\item Andererseits gibt es auch Affixe, die das Wortende anzeigen:
	
	\begin{itemize}
		\item -keit ist ein solches Schlussaffix:
		
		\ea lehr(en) \ras Lehrer \ras lehrerhaft \ras Lehrerhaftigkeit
		\z
		
	\end{itemize}
	
	\item Die Rekursion bei der Derivation ist also nicht so grenzenlos wie bei der Komposition.
	
\end{itemize}


\end{frame}


%%%%%%%%%%%%%%%%%%%%%%%%%%%%%%%%%%%
%%%%%%%%%%%%%%%%%%%%%%%%%%%%%%%%%%
\section{Flexion}
%\frame{
%\frametitle{~}
%	\tableofcontents[currentsection]
%}


%%%%%%%%%%%%%%%%%%%%%%%%%%%%%%%%%%%

\begin{frame}
\frametitle{Flexion}

\begin{itemize}
	\item Flexion \ras Bildung von Wortformen aus Stämmen
	\item Sprachspezifisch und wortartspezifisch werden verschiedene morphosyntaktische Flexionskategorien markiert (per Flexionsmorphem oder z.\,B. per Stammabwandlung)
	\item[]
	\item \textbf{Synthetische} Wortform: Flexionsstamm + Flexionsaffix
	\item \textbf{Analytische} Wortformen: mehrere Elemente werden verwendet um eine Wortform abzubilden.
	
	\ea les + (en) \ras las (Stammabwandlung per Ablaut)
	\z
	
	\ea kauf + (en) \ras kauf + tet (Affigierung \ras  synthetische Form)
	\z
	
	\ea verwend + (en) \ras wird verwendet haben (analytische Form)
	\z
	
\end{itemize}


\end{frame}



%%%%%%%%%%%%%%%%%%%%%%%%%%%%%%%%%%%

\begin{frame}
\frametitle{Flexion}

\begin{itemize}
	\item Flexionsstämme können selbst morphologisch komplex sein (ver+wend(en)).
	\item Ein morphologisch nicht komplexer Stamm heißt \gqq{Wurzel} (kauf(en)).
	\item[] 
	\item Bei der Flexion wird unterschieden zwischen:
	
	\begin{itemize}
		\item der \textbf{Deklination} von Nomina und anderer nominaler Kategorien (wozu auch Adjektive, Pronomina und Artikel gehören)
		\item[]
		\item der \textbf{Konjugation} von Verben 
		\item[]
		\item Ob die \textbf{Komparation} von Adjektiven -- mit den Kategorien Positiv, Komparativ und Superlativ -- zur Flexion oder zur Wortbildung gehört, ist umstritten.
	\end{itemize}
	
\end{itemize}

\end{frame}



%%%%%%%%%%%%%%%%%%%%%%%%%%%%%%%%%%%

\begin{frame}
\frametitle{Flexion}

\begin{itemize}
	\item Die nicht-flektierbaren Wortarten sind:
	
	\begin{itemize}
		\item[]
		\item Adverbien
		\item[]
		\item Partikeln
		\item[]
		\item Präpositionen
		\item[]
		\item Konjunktionen
	\end{itemize}
\end{itemize}


\end{frame}



%%%%%%%%%%%%%%%%%%%%%%%%%%%%%%%%%%%
%%%%%%%%%%%%%%%%%%%%%%%%%%%%%%%%%%%

\subsection{Deklination}
%\frame{
%\frametitle{~}
%	\tableofcontents[currentsection]
%}


%%%%%%%%%%%%%%%%%%%%%%%%%%%%%%%%%%%

\begin{frame}
\frametitle{Deklination}

\begin{itemize}
	\item Deklination umfasst die Bildung von Wortformen bei nominalen Kategorien.
	\item[]
	\item \textbf{Substantive} deklinieren nach:
	
	\begin{itemize}
		\item[]
		\item \textbf{Numerus:} Singular, Plural
		\item[]
		\item \textbf{Kasus:} Nominativ, Genitiv, Dativ, Akkusativ
		\item[]
		\item Substantive haben ein \textbf{inhärentes Genus}, d.\,h. sie flektieren nicht nach Genus. Die \textbf{Stärke} bei Nomina ist auch inhärent.
	\end{itemize}
	
\end{itemize}

\end{frame}



%%%%%%%%%%%%%%%%%%%%%%%%%%%%%%%%%%%

\begin{frame}
\frametitle{Deklination}

\begin{itemize}
	\item Beispiel:
	
	\begin{itemize}
		\item \textbf{Stark:} Maskulina und Neutra mit Nullendung im Nominativ und s-Genitiv
		
		\ea Tisch, Fenster
		\z
		
		\item \textbf{Schwach:} Maskulina au\ss{}er im Nominativ stets mit -(e)n
		
		\ea Held, Nachbar
		\z
		
		\item \textbf{Gemischt:} Maskulina und Neutra stark im Singular, schwach im Plural
		
		\ea Staat, Ende
		\z
		
		\item \textbf{Unveränderliche} Feminina: endungslos im Singular und mit konsequenter Markierung im Plural
		
		\ea Frau, Hand, Katze, Nadel
		\z
		
	\end{itemize}
\end{itemize}


\end{frame}



%%%%%%%%%%%%%%%%%%%%%%%%%%%%%%%%%%%

\begin{frame}
\frametitle{Deklination}

\begin{itemize}
	\item \textbf{Paradigma:}\\
	Die Gesamtheit der Flexionsformen eines Wortes (egal welcher Wortart) bilden sein Flexionsparadigma.
\end{itemize}

\begin{table}
\centering

\begin{tabular}{p{1.8cm}|p{1.8cm}|p{1.8cm}|p{1.8cm}|p{1.8cm}}
& \textbf{Nom} & \textbf{Akk} & \textbf{Dat} & \textbf{Gen}\\
\hline
\textbf{Singular} & Tisch & Tisch & Tisch(e) & Tisches\\
\hline
\textbf{Plural} & Tische & Tische & Tischen & Tische\\

\end{tabular}

\end{table}

\end{frame}



%%%%%%%%%%%%%%%%%%%%%%%%%%%%%%%%%%%

\begin{frame}
\frametitle{Deklination}

\begin{itemize}
	\item \textbf{Adjektive} deklinieren nach:
	
	\begin{itemize}
		\item[]
		\item \textbf{Numerus:} Singular, Plural
		\item []
		\item \textbf{Kasus:} Nominativ, Genitiv, Dativ, Akkusativ
		\item[]
		\item \textbf{Genus:} Maskulinum, Femininum, Neutrum
		\item[]
		\item \textbf{Stärke:} stark, schwach, gemischt (ob starke oder schwache Flexionsendungen beim Adjektiv verwendet werden, hängt vom Artikel ab)
		\item[]
		\item \textbf{Grad:} positiv (schön), komparativ (schöner), superlativ (am schönsten); ob Adjektive nach Grad flektieren, ist umstritten.
	\end{itemize}
	
\end{itemize}


\end{frame}



%%%%%%%%%%%%%%%%%%%%%%%%%%%%%%%%%%%

\begin{frame}
\frametitle{Deklination}

\begin{itemize}
	\item Beispiel Stärke
	
	\begin{itemize}
		\item \textbf{Stark:} ohne Artikel
		
		\ea schön\underline{es} Wetter, schön\underline{er} Tag, schön\underline{e} Frau
		\z
		
		\item \textbf{Schwach:} nach bestimmten Artikeln oder einer entsprechend deklinierten Einheit
		
		\ea das gut\underline{e} Kind, dieser schön\underline{e} Tag, jede schön\underline{e} Frau
		\z
		
		\item \textbf{Gemischt:} nach unbestimmten Artikeln oder einer entsprechend deklinierten Einheit
		
		\ea ein gut\underline{es} Kind, ein schön\underline{er} Tag, keine schön\underline{e} Frau
		\z
		
	\end{itemize}
	
\end{itemize}

\end{frame}



%%%%%%%%%%%%%%%%%%%%%%%%%%%%%%%%%%%

\begin{frame}
\frametitle{Deklination}

\begin{itemize}
	\item Im Deutschen wirken zur Flexionsanzeige Artikel, Adjektiv und Substantiv zusammen (= Wortgruppenflexion), da Artikel und Adjektiv mit dem Nomen in Numerus, Kasus und Genus \textbf{kongruieren} müssen, d.h. sie müssen die gleichen Numerus-, Kasus-, und Genusmerkmale aufweisen.
	
	\eal 
	\ex ein schöner Hund -- schöne Hunde -- des schönen Hundes
	\ex ein schlaues Buch -- schlaue Bücher -- des schlauen Buches
	\zl
	
	\item U.\,U. wird nur an einem Element Kasus und Numerus der gesamten Phrase ersichtlich:
	
	\eal 
	\ex Der dicke Balken muss ersetzt werden.
	\ex Der Architekt ordnete den Ersatz der dicken Balken an.
	\zl
	
\end{itemize}


\end{frame}


%%%%%%%%%%%%%%%%%%%%%%%%%%%%%%%%%%%
%%%%%%%%%%%%%%%%%%%%%%%%%%%%%%%%%%%
\subsection{Konjugation}
%\frame{
%\frametitle{~}
%	\tableofcontents[currentsection]
%}


%%%%%%%%%%%%%%%%%%%%%%%%%%%%%%%%%%%

\begin{frame}
\frametitle{Konjugation}

\begin{itemize}
	\item Bei der \textbf{Verbflexion} spricht man von \textbf{Konjugation}.
	\item[]
	\item Zunächst unterscheidet man zwischen \textbf{finiten und infiniten} Verbformen.
	\item[]
	\item \textbf{Infinite} Verbformen sind unveränderlich, d.h. egal in welchem Kontext sie stehen, sehen sie immer gleich aus.
	\item[]
	\item Dazu gehören: Infinitiv, Partizip I, Partizip II
	
	\ea essen, essend, gegessen
	\z
	
\end{itemize}


\end{frame}



%%%%%%%%%%%%%%%%%%%%%%%%%%%%%%%%%%%

\begin{frame}
\frametitle{Konjugation}

\begin{itemize}
	\item Finite Verbformen sind veränderlich. Sie verändern ihre Form nach: 
	
	\begin{itemize}
		\item[]
		\item \textbf{Person:} 1.,2.,3.
		\item[]
		\item \textbf{Numerus:} Singular, Plural
		\item[]
		\item \textbf{Modus:} Indikativ, Konjunktiv, Imperativ
		\item[]
		\item \textbf{Tempus:} Präsens, Präteritum, Perfekt, Plusquamperfekt, Futur I/ II
		\item[]
		\item \textbf{Genus verbi:} Aktiv, Passiv
	\end{itemize}
	
\end{itemize}


\end{frame}



%%%%%%%%%%%%%%%%%%%%%%%%%%%%%%%%%%%

\begin{frame}
\frametitle{Konjugation}

\begin{itemize}
	\item Beispiel Stärke:
	
	\begin{itemize}
		\item[]
		\item \textbf{Starke} Konjugation: Vokalwechsel, Ablaut
		
		\ea essen, aß, gegessen/ rufen, rief, gerufen
		\z
		
		\item \textbf{Schwache} Konjugation: immer mit -te im Präteritum, im mit -t im Partizip Perfekt
		
		\ea kaufen, kaufte, gekauft/ arbeiten, arbeitete, gearbeitet
		\z
		
		\item \textbf{Gemischte} Konjugation: Vokalwechsel, immer mit -te im Präteritum, immer mit -t im Partizip Perfekt
		
		\ea wissen, wusste, gewusst/ kennen, kannte, gekannt
		\z
		
	\end{itemize}
	
\end{itemize}


\end{frame}



%%%%%%%%%%%%%%%%%%%%%%%%%%%%%%%%%%

\begin{frame}
\frametitle{Konjugation}

\begin{itemize}
	\item Es gibt verschiedene Mittel, die Flexion bei Verben anzuzeigen. 
	
	\begin{itemize}
	\item[]
	\item \textbf{Flexionsaffixe} (\textit{nehm -- nehmt}) 
	\item[]
	\item \textbf{Ablautbildung} (fahr -- fuhr) mit anschließender \textbf{Umlautbildung} (führest) 
	\item[]
	\item Änderungen am \textbf{Konsonanten} im Stamm (\textit{bringen -- gebracht})
	\item[]
	\item \textbf{analytische} Mittel (Kombination mehrerer Wörter: ist abgeholt) 
	\item[]
	\item Oft werden diese Mittel miteinander kombiniert\\
	(s. gebracht: Zirkumfix \ab{ge-{\dots}-t} + Ablautbildung \ab{ie \ras a} + Konsonantenänderung \\ab{ng \ras ch})
	\end{itemize}

\end{itemize}


\end{frame}



%%%%%%%%%%%%%%%%%%%%%%%%%%%%%%%%%%%

\begin{frame}
\frametitle{Konjugation}

\begin{itemize}
	\item Von \textbf{Suppletion} spricht man, wenn bei bestimmten grammatischen Merkmalen ein völlig anderer Stamm benutzt wird:
	
\vspace{1em}
	
	\ea sein -- bist -- war
	\z
	
	\ea gut -- besser -- am besten
	\z
	
\vspace{6em}	
	
	\item \textbf{ÜB 5, 6 \& 7}
	
\end{itemize}


\end{frame}



%%%%%%%%%%%%%%%%%%%%%%%%%%%%%%%%%%%
%%%%%%%%%%%%%%%%%%%%%%%%%%%%%%%%%%%
%\section{X}
%%\frame{
%%\frametitle{~}
%%	\tableofcontents[currentsection]
%%}
%
%
%%%%%%%%%%%%%%%%%%%%%%%%%%%%%%%%%%%
%\begin{frame}
%\frametitle{Y}
%
%\begin{itemize}
%	\item 
%\end{itemize}
%
%
%\end{frame}




