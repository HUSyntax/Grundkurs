%%%%%%%%%%%%%%%%%%%%%%%%%%%%%%%%%%%%%%%%%%%%%%%%
%% Compile the master file!
%% 		Include: Antonio Machicao y Priemer
%% 		Course: GK Linguistik
%%%%%%%%%%%%%%%%%%%%%%%%%%%%%%%%%%%%%%%%%%%%%%%%


%%%%%%%%%%%%%%%%%%%%%%%%%%%%%%%%%%%%%%%%%%%%%%%%%%%%
%%%             Metadata                         
%%%%%%%%%%%%%%%%%%%%%%%%%%%%%%%%%%%%%%%%%%%%%%%%%%%%      

\title{Grundkurs Linguistik}

\subtitle{Morphologie III: Wortbildung II}

\author[A. Machicao y Priemer]{
	{\small Antonio Machicao y Priemer}
	\\
	{\footnotesize \url{http://www.linguistik.hu-berlin.de/staff/amyp}}
	%	\\
	%	{\small\href{mailto:mapriema@hu-berlin.de}{mapriema@hu-berlin.de}}
}

\institute{Institut für deutsche Sprache und Linguistik}

% bitte lassen, sonst kann man nicht sehen, von wann die PDF-Datei ist.
%\date{ }

%\publishers{\textbf{6. linguistischer Methodenworkshop \\ Humboldt-Universität zu Berlin}}

%\hyphenation{nobreak}


%%%%%%%%%%%%%%%%%%%%%%%%%%%%%%%%%%%%%%%%%%%%%%%%%%%%
%%%             Preamble's End                  
%%%%%%%%%%%%%%%%%%%%%%%%%%%%%%%%%%%%%%%%%%%%%%%%%%%%      


%%%%%%%%%%%%%%%%%%%%%%%%%      
\huberlintitlepage[22pt]
\iftoggle{toc}{
	\frame{
		\begin{multicols}{2}
			\frametitle{Inhaltsverzeichnis}\tableofcontents
			%[pausesections]
		\end{multicols}
	}
}


%%%%%%%%%%%%%%%%%%%%%%%%%%%%%%%%%%
%%%%%%%%%%%%%%%%%%%%%%%%%%%%%%%%%%
%%%%%LITERATURE:

%% Allgemein
\nocite{Glueck&Roedel16a}
\nocite{Schierholz&Co18}
\nocite{Luedeling2009a}
\nocite{Meibauer&Co07a} 
\nocite{Repp&Co15a} 

%%% Sprache & Sprachwissenschaft
%\nocite{Fries16c} %Adäquatheit
%\nocite{Fries16a} %Grammatikalität
%\nocite{Fries&MyP16c} %GG
%\nocite{Fries&MyP16b} %Akzeptabilität
%\nocite{Fries&MyP16d} %Kompetenz vs. Performanz

%%% Phonetik & Phonologie
%\nocite{Altmann&Co07a}
%\nocite{DudenAussprache00a}
%\nocite{Hall00a} 
%\nocite{Kohler99a}
%\nocite{Krech&Co09a}
%\nocite{Pompino95a}
%\nocite{Ramers08a}
%\nocite{Ramers&Vater92a}
%\nocite{Rues&Co07a}
%\nocite{WieseR96a}
%\nocite{WieseR11a}

%%% Graphematik
%\nocite{Altmann&Co07a}
%\nocite{Duerscheid04a}
%\nocite{Eisenberg00a}
%\nocite{Fuhrhop08a}
%\nocite{Fuhrhop09a}
%\nocite{Fuhrhop&Co13a}

%% Morphologie
%\nocite{Bauer00a} %Word
\nocite{Eisenberg00a}
\nocite{Fleischer00a} %Wortbildungsprozesse
\nocite{Fleischer&Barz12a} %Einführung Morphologie
\nocite{Fries&MyP16j} %Kopf
%\nocite{Fuhrhop96a} %Fugenelemente
%\nocite{Fuhrhop00a} %Fugenelemente
\nocite{Grewendorf&Co91a} %Betonung bei Komposita
\nocite{Haspelmath2002a}
\nocite{MyP18b} %Kopf
\nocite{Olsen86a} %Morphologie des Deutschen
%\nocite{Olsen14a} %Coordinative Structures
%\nocite{Plungian00a} %Morphologie im Sprachsystem
%\nocite{Salmon00a} %Term Morphology
%\nocite{Wegener03b} %Fugenelemente
%\nocite{Wurzel00a} %Gegenstand Morphologie
%\nocite{Wurzel00b} %Wort


%%%%%%%%%%%%%%%%%%%%%%%%%%%%%%%%%%%
%%%%%%%%%%%%%%%%%%%%%%%%%%%%%%%%%%%
\section{Morphologie III}
%%%%%%%%%%%%%%%%%%%%%%%%%%%%%%%%%%%

\begin{frame}
\frametitle{Begleitlektüre}

%\begin{itemize}
%	\item Derivation und andere Wortbildungsarten:

	\begin{itemize}
		\item \textbf{obligatorisch:}
		\item[] \citet[46--51]{Abramowski2016a}
		\item \textbf{optional:}
		\item[] \citet[Kap. 2, S. 48--66]{Meibauer&Co07a}
		\item[] \citet[Kap.~7]{Luedeling2009a}
		
	\end{itemize}

%\end{itemize}

\end{frame}


%%%%%%%%%%%%%%%%%%%%%%%%%%%%%%%%%%
%%%%%%%%%%%%%%%%%%%%%%%%%%%%%%%%%%
\subsection{Derivation}

\iftoggle{sectoc}{
	\frame{
		%		\begin{multicols}{2}
		\frametitle{~}
		\tableofcontents[currentsubsection,subsubsectionstyle=hide]
		%		\end{multicols}
	}
}

%%%%%%%%%%%%%%%%%%%%%%%%%%%%%%%%%%

\begin{frame}
\frametitle{Derivation}


	\begin{itemize}
		\item Wortbildungsprozess mittels \textbf{Konkatenation}, bei dem ein wortfähiges Element (freies Morphem oder komplexe Basis) mit einem \textbf{Affix} verbunden wird

		\item Derivation und Flexion sind -- formal gesehen -- \textbf{Affigierungen}.
		
		\item Ergebnis der Derivation: \textbf{Derivat}

		\item 	Derivation $\rightarrow$ neue \emph{Wörter} (auch Lexeme oder Lemmata)\\
		Flexion $\rightarrow$ neue Wort\emph{formen}
	
	\end{itemize}
	
\end{frame}


%%%%%%%%%%%%%%%%%%%%%%%%%%%%%%%%%%%
%%%%%%%%%%%%%%%%%%%%%%%%%%%%%%%%%%%

\subsubsection{Die Basis}
%\frame{
%\frametitle{~}
%	\tableofcontents[currentsection]
%}


%%%%%%%%%%%%%%%%%%%%%%%%%%%%%%%%%%%
\begin{frame}
\frametitle{Die Basis}

\begin{itemize}
	\item \textbf{Basis} (Pl. Basen):
	
	\begin{itemize}
		\item etwas, woran etwas affigiert werden kann
		\item Ausgangsform der \textbf{Derivation}
	\end{itemize}
	
	\item Die Basis kann:
	
	\begin{itemize}
		\item \textbf{morphologisch einfach} (eine Wurzel) sein:
		
		\ea trinkbar $\leftarrow$  [trink]$+$[-bar]
		\z
		
		\item oder \textbf{morphologisch komplex} sein:

		\settowidth\jamwidth{[Basis: Wurzel+Affix]X} 		
		
		
		\ea Trinkbarkeit $\leftarrow$  \alertred{[[trink]}$+$\alertred{[-bar]]}$+$[-keit] \jambox{[Basis: Wurzel$+$Affix]}

		\ex trinkwasserartig $\leftarrow$  \alertred{[[trink]}$+$\alertred{[wasser]]}$+$[-artig]  \jambox{[Basis: Kompositum]}
		\z
		
	\end{itemize}
	
\end{itemize}


\end{frame}



%%%%%%%%%%%%%%%%%%%%%%%%%%%%%%%%%%%
\begin{frame}
\frametitle{Basis: Wortartenwechsel}

	\begin{itemize}
		\item Verb \ras Adjektiv, Substantiv
		
		\ea verkauf(-en) \ras \alertred{verkäuf}$+$-lich, \alertred{Verkäuf}$+$-er
		\z
		
		\item Adverb \ras Adjektiv
		
		\ea heute \ras \alertred{heut}$+$-ig
		\z
		
		\item Adjektiv \ras Substantiv, Verb, Adverb
		
		\eal 
			\ex schön \ras \alertred{Schön}$+$-heit, be-$+$\alertred{schön}$+$-igen
			\ex klug \ras \alertred{klug}$+$-erweise
		\zl
		
		\item Substantiv \ras Adjektiv, Verb, Adverb

		\eal 
			\ex Arzt \ras \alertred{ärzt}$+$-lich, ver-$+$\alertred{arzt}(-en)
			\ex Nacht \ras \alertred{nacht}$+$-s
		\zl
		
	\end{itemize}
	
\end{frame}


%%%%%%%%%%%%%%%%%%%%%%%%%%%%%%%%%%%
\begin{frame}
\frametitle{Basis: Wortart}

\begin{itemize}
	\item Wie erkennt man die Wortart der Basis?
	
%	\begin{itemize}
		\item Substantiv oder Verb als Basis der Derivation:
		
		\eal 
			\ex lauf(-en) \ras (der) Lauf
%			\ex Reifen \ras bereifen
			\ex abnehm(-en) \ras Abnahme
		\zl


		\item \textbf{Formeller Hinweis:} Betrachten Sie die \textbf{Affixe}, die für die Derivation benutzt werden. Diese sind nämlich in der Regel bezüglich der Wortart, mit der sie sich verbinden können, \textbf{beschränkt}.
		
\medskip
				
		\item \textbf{Semantischer Hinweis:} Handelt es sich beim Substantiv um ein \textbf{Objekt} (im Sinne von Ding) o.\,ä. ist \emph{meist} das Substantiv zugrunde liegend, handelt es sich aber um einen \textbf{Vorgang}, ist \emph{meist} das Verb zugrunde liegend.
		
%	\end{itemize}
%	
\end{itemize}

\end{frame}


%%%%%%%%%%%%%%%%%%%%%%%%%%%%%%%%%%%
%%%%%%%%%%%%%%%%%%%%%%%%%%%%%%%%%%%
\subsubsection{Suffixe und Präfixe}
%\frame{
%\frametitle{~}           
%	\tableofcontents[currentsection]
%}
%%%%%%%%%%%%%%%%%%%%%%%%%%%%%%%%%%%

\begin{frame}
\frametitle{Suffixe und Präfixe}

\begin{itemize}
	\item \alertred{Suffixe} bestimmen die kategoriale Zugehörigkeit des Derivats.
	
	\item \alertblue{Präfixe} bestimmen \idR die kategoriale Zugehörigkeit des Derivats nicht.
	
	\ea 
	\ea {[}\MyPdown{N}Glück] -- [\MyPdown{\alertred{A}}glück$+$\alertred{-lich}], [\MyPdown{N}\alertblue{Un-}$+$glück]
	%	\z
	
	\ex {[}\MyPdown{V}setz(-en)] -- [\MyPdown{\alertred{N}}Setz$+$\alertred{-ung}], [\MyPdown{V}\alertblue{über-}$+$setz(-en)]
	%	\z
	
	\ex {[}\MyPdown{V}hör(-en)] -- [\MyPdown{\alertred{A}}hör$+$\alertred{-bar}], [\MyPdown{V}\alertblue{ver-}$+$hör(-en)]
	%	\z
	
	\ex {[}\MyPdown{V}acht(-en)] -- [\MyPdown{\alertred{A}}acht$+$\alertred{-bar}], [\MyPdown{V}\alertblue{miss-}$+$acht(-en)]
	\z
	\z 
	
	\medskip 
	
	\item Suffix \ras \textbf{Kopf}
	
	\item Präfix \ras \textbf{kein Kopf}
	
	\item Kopf  \ras \textbf{rechtsperipher} (wie bei der Komposition)
\end{itemize}
\end{frame}


%%%%%%%%%%%%%%%%%%%%%%%%%%%%%%%%%%%%
%\begin{frame}
%\frametitle{Präfix als Kopf?}
%
%\begin{itemize}
%	\item In einigen Fällen \emph{scheinen} Präfixe Verben aus Substantiven, Adjektiven oder Partikeln abzuleiten.
%
%	\settowidth\jamwidth{[Substantivbasis]X} 			
%	\ea be-, ent-, er-, ver-, zer-, durch-, über-, um-, unter- 
%	\z 
%
%		
%		\ea be-$+$sohlen (*sohlen), ent-$+$kernen (*kernen) \jambox{[Substantivbasis]}
%
%		\ex ver-$+$längern (*längern), über-$+$raschen (*raschen) \jambox{[Adjektivbasis]}
%	
%		\ex be-$+$jahen (*jahen), ver-$+$neinen (*neinen) \jambox{[Partikelbasis]}
%		\z
%		
%\end{itemize}
%		
%\pause
%\medskip
%
%\begin{minipage}[c]{0.6\textwidth}	
%	\begin{itemize}
%		\item In solchen Fällen werden \textbf{Präfixe} \emph{manchmal} \textbf{als Köpfe} analysiert.
%		
%		\item Diese Art von Analyse verletzt das \textbf{Rechtsköpfigkeitsprinzip}.
%	\end{itemize}
%\end{minipage}
%%%
%\begin{minipage}[c]{0.35\textwidth}
%	\centering 
%	\small
%	\begin{forest}
%		[V
%		[\alertblue{V}, name=A1
%		[\alertblue{V\MyPup{af}}, name=A0 
%		[be-]
%		]
%		[\alertred{N}, name=
%		[sohle]
%		]
%		]
%		[Fl [-n]]	
%		]
%		{\draw[->,dashed] (A0) to[out=west,in=west] (A1);
%		}
%	\end{forest}
%	
%\end{minipage}
%
%\end{frame}
%
%
%%%%%%%%%%%%%%%%%%%%%%%%%%%%%%%%%%%
\begin{frame}
\frametitle{Präfixe}

\begin{itemize}
	\item Einige Präfixe sind sowohl als \textbf{nominale} als auch als \textbf{adjektivische Präfixe} \textbf{kategorisiert}.
	
	\ea
	\ea erz-: [\alertred{\MyPdown{N\MyPup{af}}Erz-}]$+$feind, [\alertred{\MyPdown{A\MyPup{af}}erz-}]$+$katholisch
	%			\ex[]
	%		\ex ge-: \alertred{Ge-}$+$büsch
	%			\ex[]
	\ex miss-: [\alertred{\MyPdown{N\MyPup{af}}\alertred{Miss-}}]$+$erfolg, [\alertred{\MyPdown{A\MyPup{af}}miss-}]$+$mutig
	%			\ex[]
	\ex un-: [\alertred{\MyPdown{N\MyPup{af}}\alertred{Un-}}]$+$geduld, [\alertred{\MyPdown{A\MyPup{af}}un-}]$+$sauber
	%			\ex[]
	\ex ur-: [\alertred{\MyPdown{N\MyPup{af}}\alertred{Ur-}}]$+$gestein, [\alertred{\MyPdown{A\MyPup{af}}ur-}]$+$alt
	\z 
	\z
\end{itemize}


%%
\begin{minipage}[c]{0.49\textwidth}
	\centering 			
	\begin{forest}
		[N, name=A1
		[\alertred{N\MyPup{af}}, name=V1 
		[Un-]
		]
		[\alertblue{N}, name=A0
		[mensch]
		]
		]
		{\draw[->,dashed] (A0) to[out=east,in=east] (A1);
		}
	\end{forest}
	
\end{minipage}
%%
\begin{minipage}[c]{0.49\textwidth}
	\centering 			
	\begin{forest}
		[A, name=A1
		[\alertred{A\MyPup{af}}, name=V1 
		[un-]
		]
		[\alertblue{A}, name=A0
		[schön]
		]
		]
		{\draw[->,dashed] (A0) to[out=east,in=east] (A1);
		}
	\end{forest}
	
\end{minipage}

\end{frame}


%%%%%%%%%%%%%%%%%%%%%%%%%%%%%%%%%%
%%%%%%%%%%%%%%%%%%%%%%%%%%%%%%%%%%
\subsubsection{Zirkumfixe}
%\frame{
%%\begin{multicols}{2}
%\frametitle{~}
%	\tableofcontents[currentsection, hideallsubsections]
%%\end{multicols}
%}
%%%%%%%%%%%%%%%%%%%%%%%%%%%%%%%%%%
\begin{frame}
\frametitle{Zirkumfixe}

\begin{itemize}
	\item Zirkumfixe sind \textbf{Köpfe}, da sie (\emph{auch}) auf der rechten Seite des Derivats stehen.
	

\settowidth\jamwidth{XXXXXXXXXXXXXXXXXXXXXXXXXXXXXtXXXXX}	
	\ea
	\ea ge- \dots\ -e: \jambox{[\alertred{\MyPdown{N\MyPup{zf}}Ge-}]$+$\alertblue{\MyPdown{V}}lach$+$[\alertred{\MyPdown{N\MyPup{zf}}-e}]}
	
	\ex ge- \dots\ -ig: \jambox{[\alertred{\MyPdown{A\MyPup{zf}}ge-}]$+$\alertblue{\MyPdown{N}}räum$+$[\alertred{\MyPdown{A\MyPup{zf}}-ig}]}
	
	\ex un- \dots\ -lich: \jambox{[\alertred{\MyPdown{A\MyPup{zf}}un-}]$+$\alertblue{\MyPdown{V}}glaub$+$[\alertred{\MyPdown{A\MyPup{zf}}-lich}]}
	
	\ex un- \dots\ -bar: \jambox{[\alertred{\MyPdown{A\MyPup{zf}}un-}]$+$\alertblue{\MyPdown{A}}platt$+$[\alertred{\MyPdown{A\MyPup{zf}}-bar}]}
	%un-kaputt-bar, un-nah-bar
	
	\ex un- \dots\ -sam: 
	\jambox{[\alertred{\MyPdown{A\MyPup{zf}}un-}]$+$\alertblue{\MyPdown{N}}weg$+$[\alertred{\MyPdown{A\MyPup{zf}}-sam}]}
	
	\ex be- \dots\ -t: 
	\jambox{[\alertred{\MyPdown{A\MyPup{zf}}be-}]$+$\alertblue{\MyPdown{N}}jahr$+$[\alertred{\MyPdown{A\MyPup{zf}}-t}]}
	
	\ex ent- \dots\ -t: 
	\jambox{[\alertred{\MyPdown{A\MyPup{zf}}ent-}]$+$\alertblue{\MyPdown{N}}geister$+$[\alertred{\MyPdown{A\MyPup{zf}}-t}]}
	
	\ex zer- \dots\ -t: 
	\jambox{[\alertred{\MyPdown{A\MyPup{zf}}zer-}]$+$\alertblue{\MyPdown{N}}narb$+$[\alertred{\MyPdown{A\MyPup{zf}}-t}]}
	\z 
	\z
\end{itemize}

\end{frame}


%%%%%%%%%%%%%%%%%%%%%%%%%%%%%%%%%%%
\begin{frame}
\frametitle{Zirkumfixe}

\begin{itemize}

\item Zirkumfixe sind \textbf{Köpfe}, da sie (\emph{auch}) auf der rechten Seite des Derivats stehen.

\item Zirkumfixe stellen ein Problem für Strukturierung dar. \ras \textbf{trinäre Struktur}
\end{itemize}

%%
\begin{minipage}[c]{0.49\textwidth}
\centering 			
\begin{forest}
	[N, name=A1
	[\alertred{A\MyPup{zf}}, %name=V1 
	[un-]
	]
	[\alertblue{A}, name=A0
	[kaputt]
	]
	[\alertred{A\MyPup{zf}}, name=V1 
	[-bar]
	]
	]
	{\draw[->,dashed] (V1) to[out=east,in=east] (A1);
	}
\end{forest}

\end{minipage}
%%
\begin{minipage}[c]{0.49\textwidth}
\centering 			
\begin{forest}
	[A, name=A1
	[\alertred{A\MyPup{af}}, name=A2 
	[un-]
	]
	[\alertblue{V}, name=A0
	[glaub]
	]
	[\alertred{A\MyPup{zf}}, name=V1 
	[-lich]
	]
	]
	{\draw[->,dashed] (V1) to[out=east,in=east] (A1);
	}
\end{forest}

\end{minipage}

\ea 
\gll un- glaub -lich $\neq$ [ un- $+$ [\MyPdown{\alertred{A}} glaub {} $+$ -lich ]]\\
{} {} {}  $\neq$ [[\MyPdown{\alertred{V}} un- $+$ {} glaub ] $+$ -lich ]\\
\z 

\end{frame}	

%%%%%%%%%%%%%%%%%%%%%%%%%%%%%%%%%%
%%%%%%%%%%%%%%%%%%%%%%%%%%%%%%%%%%
\subsubsection{Struktur}
%\frame{
%%\begin{multicols}{2}
%\frametitle{~}
%	\tableofcontents[currentsection, hideallsubsections]
%%\end{multicols}
%}
%%%%%%%%%%%%%%%%%%%%%%%%%%%%%%%%%%

\begin{frame}
\frametitle{Struktur}

\begin{minipage}[c]{0.98\textwidth}

\begin{minipage}[c]{0.83\textwidth}

%	\begin{itemize}
%		\item Wortstrukturregel für \textbf{Suffigierung}:

\ea Wortstrukturregel für \textbf{Suffigierung}: \alertred{X} \ras Y \alertred{X}\MyPup{af}
\z 

\begin{itemize}
	\item Das Suffix \emph{-heit} ist ein \textbf{nomenbildendes Suffix}.
	
	\item Das Suffix agiert als \textbf{Kopf}: Es \textbf{projiziert} seine \textbf{morphosyntaktischen Eigenschaften} an den Mutterknoten, \\
	\dash es bestimmt die Eigenschaften des Derivats.
	
\end{itemize}
%	\end{itemize}

\end{minipage}
%%
\begin{minipage}[c]{0.15\textwidth}
\centering 
\begin{forest}
	[\alertred{N}, name=A1
	[\alertblue{A}, name=V1 
	[frei]
	]
	[\alertred{N}\MyPup{af}, name=A0
	[-heit]
	]
	]
	{\draw[->,dashed] (A0) to[out=east,in=east] (A1);
	}
\end{forest}

\end{minipage}

\end{minipage}
%%
\pause

\vspace{.5cm}
%%
\begin{minipage}[c]{0.98\textwidth}

\begin{minipage}[c]{0.83\textwidth}

%	\begin{itemize}

%		\item Wortstrukturregel für \textbf{Präfigierung}:

\ea Wortstrukturregel für \textbf{Präfigierung}: X \ras \alertblue{X}\MyPup{af} \alertblue{X}
\z 

\begin{itemize}
	\item Das Präfix \emph{ver-} ist ein \textbf{Verbaffix}, \dash es verbindet sich\\
	\textbf{nur mit Verben}.
	
	\item Das Präfix agiert \textbf{nicht} als \textbf{Kopf}. Die  \textbf{morphosyntaktischen Eigenschaften} der Basis werden an den Mutterknoten \textbf{projiziert}.
	
\end{itemize}		

%	\end{itemize}

\end{minipage}
%%
%%
\begin{minipage}[c]{0.15\textwidth}
\centering 			
\begin{forest}
	[V, name=A1
	[\alertblue{V}\MyPup{af}, name=V1 
	[ver-]
	]
	[\alertblue{V}, name=A0
	[schreib]
	]
	]
	{\draw[->,dashed] (A0) to[out=east,in=east] (A1);
	}
\end{forest}

\end{minipage}

\end{minipage}

\end{frame}


%%%%%%%%%%%%%%%%%%%%%%%%%%%%%%%%%%

\begin{frame}
\frametitle{Struktur}

\ea Wortstrukturregel für \textbf{Zirkumfigierung}: \alertred{X} \ras \alertred{X}\MyPup{zf} Y \alertred{X}\MyPup{zf}
\z 

\begin{itemize}
\item Das Zirkumfix \emph{Ge- \dots\ -e} ist ein \textbf{nomenbildendes Suffix}.

\item Das Zirkumfix agiert als \textbf{Kopf}: Es \textbf{projiziert} seine \textbf{morphosyntaktischen Eigenschaften} an den Mutterknoten, \\
\dash es bestimmt die Eigenschaften des Derivats.

\end{itemize}

\begin{figure}
\centering 
\begin{forest}
[\alertred{N}, name=A1
[\alertred{N}\MyPup{zf}, name=A2
[Ge-]
]
[\alertblue{V}, name=V1 
[kreisch]
]
[\alertred{N}\MyPup{zf}, name=A0
[-e]
]
]
{\draw[->,dashed] (A0) to[out=east,in=east] (A1);
}
\end{forest}

\end{figure}

\end{frame}


%%%%%%%%%%%%%%%%%%%%%%%%%%%%%%%%%%%
\begin{frame}
\frametitle{Projektion von Eigenschaften}

\begin{itemize}
\item Suffixe und Zirkumfixe (qua Köpfigkeit) bestimmen die \textbf{Kategorie des Derivats}, d.\,h. die Art von Stämmen, die sie bilden.

\ea 
\ea \textbf{substantivbildende} Suffixe: -ung, -heit/-keit, -er, -schaft, \dots

\ex Schreib\alertred{ung}, Frei\alertred{heit}, Rauch\alertred{er}, Hörer\alertred{schaft}
\z 
\z

\ea 
\ea \textbf{adjektivbildende} Suffixe: -bar, -lich, -haft, -ig, \dots 
\ex hör\alertred{bar}, röt\alertred{lich}, lach\alertred{haft}, mut\alertred{ig}
\z 
\z

\ea 
\ea \textbf{verbbildende} Suffixe: -(e)l, -(is/ifiz)ier, -ig, \dots
\ex häk\alertred{el}(n), schläng\alertred{el}(n), stabil\alertred{isier}(en), ängst\alertred{ig}(en)
\z 
\z

\end{itemize}
\end{frame}


%%%%%%%%%%%%%%%%%%%%%%%%%%%%%%%%%%%
\begin{frame}
\frametitle{Projektion von Eigenschaften}

\begin{itemize}

\item Nominale Suffixe und Zirkumfixe bestimmen das \textbf{Genus des Derivats}.

%\vspace{1em}
\begin{multicols}{2}

\begin{exe}
\ex \textbf{-ung}:  Acht $+$ -ung\MyPdown{f}
\ex \textbf{-keit}: Tapfer $+$ -keit\MyPdown{f}
\ex \textbf{-chen}: Häus $+$ -chen\MyPdown{n}
\ex \textbf{-tum}: Brauch $+$ -tum\MyPdown{n}\\
(aber: Reichtum\MyPdown{\alertred{m}})
\end{exe}

\columnbreak

\begin{exe}
\ex \textbf{-ian}: Grob $+$ -ian\MyPdown{m}
\ex \textbf{-ling}: Lehr $+$ -ling\MyPdown{m}
\ex \textbf{-bold}: Witz $+$ -bold\MyPdown{m}
\ex \textbf{ge- -e}: Ge- $+$ heul $+$ -e\MyPdown{n}

~ % wegen Spaltenteilung
\end{exe}

\end{multicols}

\end{itemize}

\end{frame}


%%%%%%%%%%%%%%%%%%%%%%%%%%%%%%%%%%%
\begin{frame}
\frametitle{Rekursivität}

\begin{itemize}
\item Derivation und Komposition sind im Prinzip \textbf{rekursive Wortbildungsprozesse} (d.\,h. sie können mehrfach angewendet werden). 

\item \textbf{Dieselbe Affigierung} kann jedoch in seltenen Fällen erfolgen\\
(vgl.\ (\ref{ex:5bDerRek1}) und (\ref{ex:5bDerRek2})).

\ea 
\ea\label{ex:5bDerRek1} \alertred{Ur-}$+$großmutter, \alertred{Ur-}$+$urgroßmutter, \alertred{Ur-}$+$ururgroßmutter

\ex\label{ex:5bDerRek2} trink$+$\alertred{-bar}, *trinkbar$+$\alertred{-bar}, *trinkbarbar$+$\alertred{-bar}
\z 
\z 

\pause 

\item Einige Affixe (\textbf{Schlussaffixe}) zeigen an, dass \textbf{keine weiteren Suffixe} angeschlossen werden können (vgl.\ \ab{-igkeit} in (\ref{ex:5bEndAff})):

\ea\label{ex:5bEndAff} lehr(-en) \ras Lehr$+$\alertred{-er} \ras lehrer$+$\alertred{-haft} \ras Lehrerhaft$+$\alertred{-igkeit}
\z

\item Rekursion bei der Derivation ist also \textbf{beschränkter} als bei der Komposition.

\end{itemize}


\end{frame}


%%%%%%%%%%%%%%%%%%%%%%%%%%%%%%%%%%
%%%%%%%%%%%%%%%%%%%%%%%%%%%%%%%%%%
\subsubsection{Beschränkungen von Affixen}
%\frame{
%%\begin{multicols}{2}
%\frametitle{~}
%	\tableofcontents[currentsection, hideallsubsections]
%%\end{multicols}
%}
%%%%%%%%%%%%%%%%%%%%%%%%%%%%%%%%%%

\begin{frame}
\frametitle{Beschränkungen von Affixen}

\begin{itemize}
\item Der Kopf bestimmt (bzw. \emph{beschränkt}), \textbf{mit welchen Elementen} er sich verbinden kann.
\item Dabei gibt es verschiedene Arten von Beschränkungen.

\medskip 	

\item \textbf{Syntaktische Beschränkungen}

\begin{itemize}
\item \textit{-bar} verbindet sich mit \textbf{Verben}:

\ea \alertred{les}bar, \alertred{ess}bar, \alertred{erzieh}bar vs. *\alertred{grün}bar
\z

\item \textit{-heit/-keit} verbinden sich mit \textbf{Adjektiven}:		

\ea \alertred{Blöd}heit, \alertred{Frei}heit, \alertred{Unachtsam}keit \vs *\alertred{Les}heit, *\alertred{Ess}keit
\z

\end{itemize}

\item \textbf{Argumentstrukturelle Beschränkungen}

\begin{itemize}
\item \textit{-bar} verbindet sich mit \textbf{transitiven Verben}:	

\ea *\alertred{schlaf}bar, *\alertred{frier}bar
\z

\end{itemize}

\end{itemize}

\end{frame}


%%%%%%%%%%%%%%%%%%%%%%%%%%%%%%%%%%%
\begin{frame}
\frametitle{Beschränkungen von Affixen}

\begin{itemize}
\item \textbf{Phonologische Beschränkungen}

\begin{itemize}
\item \emph{-keit} folgt ausschließlich auf \textbf{unbetonte} Silben: 				

\ea \alertred{\textipa{\textprimstress}Wach}\alertblue{sam}$+$-keit \vs *\alertred{\textipa{\textprimstress}Frei}$+$-keit, *\alertred{\textipa{\textprimstress}Nett}$+$-heit \vs \alertred{\textipa{\textprimstress}Net}\alertblue{tig}$+$-keit
\z

%		\item[]
(aber nicht nach \emph{-haft}, \emph{-los}, \emph{-en}, \emph{-e}: *Schad\alertblue{haft}$+$-keit, *Rast\alertblue{los}$+$-keit, *Mü\alertblue{de}$+$-keit)

\pause 
\medskip 

\item \textit{-heit} lässt \textbf{betonte und unbetonte} Silben zu: 

\ea \alertred{\textipa{\textprimstress}Frei}heit, \alertred{\textipa{\textprimstress}Schüch}\alertblue{tern}heit, 
\z

%		\item[] 
(außer \emph{-e}, \emph{-bar}, \emph{-ig}, \emph{-isch}, \emph{-lich}, \emph{-mäßig}, \emph{-sam}, \emph{-haft}, \emph{-los})

\pause 
\medskip 

\item \emph{-ei} verbindet sich mit Wörtern, deren letzte Silbe \textbf{unbetont} ist (ansonsten werden die Allomorphe \emph{-erei}/\emph{-elei} verwendet): 

\ea Wüs\alertblue{te(n)}$+$-ei \vs \alertred{Renn}$+$-rei, \alertred{Lieb}$+$-elei
\z

\end{itemize}

\end{itemize}

\end{frame}


%%%%%%%%%%%%%%%%%%%%%%%%%%%%%%%%%%%
\begin{frame}
\frametitle{Beschränkungen von Affixen}


\begin{itemize}
\item \textbf{Morphologische Beschränkungen}

\begin{itemize}
\item \emph{Ge-}{\dots}\emph{-e} verbindet sich \textbf{nicht mit komplexen Verben}:

\ea Ge-\alertred{red}-e, Ge-\alertred{mecker}-e \vs *Ge-\alertred{verkauf}-e, *Ge-\alertred{entlass}-e\\
(Aber: \alertblue{Herum}-ge-\alertred{hup}-e)
\z

\pause

\item \emph{-lich} verbindet sich \textbf{nicht mit Abkürzungen}:

\ea \alertred{sport}$+$-lich \vs *\alertred{SPD}$+$-lich, *\alertred{DGfS}$+$-lich
\z

\pause 

\item \emph{-heit} folgt auf \textbf{Partizipien} (\emph{-keit} nicht):

\ea \alertred{Gelassen}$+$-heit, \alertred{Aufgeregt}$+$-heit, \alertred{Zurückgezogen}$+$-heit
\z

\end{itemize}

\end{itemize}

\end{frame}


%%%%%%%%%%%%%%%%%%%%%%%%%%%%%%%%%%%
\begin{frame}
\frametitle{Beschränkungen von Affixen}

\begin{itemize}
\item \textbf{Semantisch-konzeptuelle Beschränkungen}

\begin{itemize}
\item \emph{-fach} verbindet sich nur mit \textbf{Zahlen} und \textbf{Quantitätsausdrücken}:

\ea \alertred{zwei}$+$-fach, \alertred{hundert}$+$-fach, \alertred{viel}$+$-fach, \alertred{mehr}$+$-fach \vs *\alertred{grün}$+$-fach, *\alertred{frei}$+$-fach
\z

\pause

\item \textit{Ge-{\dots}-e} verbindet sich \textbf{nicht mit stativen Verben} (d.\,h. Verben, die einen Zustand ausdrücken):

\ea Ge-\alertred{renn}-e \vs *Ge-\alertred{wiss}-e, *Ge-\alertred{kenn}-e
\z

\end{itemize}

\end{itemize}

\end{frame}


%%%%%%%%%%%%%%%%%%%%%%%%%%%%%%%%%%%
\begin{frame}
\frametitle{Beschränkungen von Affixen}

\begin{itemize}

\item \textbf{Beschränkungen der Herkunft}

\begin{itemize}
\item \emph{-bar} verbindet sich mit \textbf{nativen} und sog. \textbf{neoklassischen} Basen, \emph{-abel} hingegen \textbf{nur mit neoklassischen}: 

\ea 
\ea \alertred{tanz}$+$-bar, \alertred{nachvollzieh}$+$-bar, \alertred{annehm}$+$-bar, \alertred{akzeptier}$+$-bar 
\ex \alertred{akzept}$+$-abel, *\alertred{annehm}$+$-abel (akzeptieren: lat.\ Ursprung)
\z 
\z

\end{itemize}

\pause 

\item \textbf{Fazit:} Beschränkungen können alle linguistischen Ebenen betreffen. 

Diese Information muss im \textbf{Lexikoneintrag} sowohl von \textbf{Basen} als auch von \textbf{Derivationsaffixen} gespeichert sein.	
\end{itemize}

\end{frame}


%%%%%%%%%%%%%%%%%%%%%%%%%%%%%%%%%%
%%%%%%%%%%%%%%%%%%%%%%%%%%%%%%%%%%
\subsubsection{Bedeutung von Affixen}
%\frame{
%%\begin{multicols}{2}
%\frametitle{~}
%	\tableofcontents[currentsection, hideallsubsections]
%%\end{multicols}
%}
%%%%%%%%%%%%%%%%%%%%%%%%%%%%%%%%%%

\begin{frame}
\frametitle{Bedeutung von Affixen}

\begin{itemize}
\item Die Bedeutung der Affixe ist \textbf{nicht immer semantisch eindeutig erfassbar}. Sie sind \textbf{ambig} \size{\citep[vgl.][]{Fries&MyP16k}}.

\item Meist haben Affixe eher eine \textbf{grammatische} als eine lexikalisch-semantische Bedeutung. 

\item Es lassen sich jedoch auch \textbf{produktive} Muster/Reihen mit klarem Funktions/Bedeutungsbeitrag:

\ea 
\ea sich \alertred{ver}fahren, sich \alertred{ver}schreiben, %sich versprechen, sich verlaufen, sich verhören 
\dots\ \ras \gq{$x$ falsch machen}
%	\z

\ex Benzin \alertred{ver}fahren, Tinte \alertred{ver}schreiben, %Geld verspielen
\dots\ \ras \gq{$x$ verbrauchen}\\

\ex Aber:\\
\alertred{ver}kaufen \ras \gq{etw. gegen Geld tauschen}\\
%	verärgern \ras \gq{jemanden ärgerlich machen}\\ % Inchoativ
%	verarmen \ras \gq{arm werden}\\ % Inchoativ
\alertred{ver}hungern \ras \gq{aus Mangel an Nahrung sterben}
\z
\z 
\end{itemize}

\end{frame}


%%%%%%%%%%%%%%%%%%%%%%%%%%%%%%%%%%
\begin{frame}
\frametitle{Bedeutung von Affixen}

\begin{itemize}
\item Viele Suffixe sind \textbf{ambig}. Die Untersuchung ihrer \textbf{Bedeutungsvarianten} ist Untersuchungsgegenstand der Morphologie-Semantik-Schnittstelle.

\ea \emph{-ung:} Ereignis, Zustand oder Objekt \size{\citep[vgl.][]{Doelling15a}}
\ea Die [Absperr\alertred{ung} der Straße]\MyPdown{Ereignis} wurde behindert.
\ex Die [Absperr\alertred{ung} der Straße]\MyPdown{Zustand} wurde aufgehoben.
\ex Die [Absperr\alertred{ung} der Straße]\MyPdown{Objekt} wurde abgebaut.
\z 

\ex \emph{-erei:} iteratives, unerwünschtes Geschehen
\ea Besserwiss\alertred{erei}, Heul\alertred{erei}
\z

\ex \emph{-er}: Agens, Instrument, Geschehen als Einzelakt\\
\hfill \size{\citep[vgl.][]{Alexiadou&Co10c}}
\ea Lehr\alertred{er}, Mal\alertred{er}, Rauch\alertred{er}
\ex Öffn\alertred{er}	
\ex Seufz\alertred{er}, Ausrutsch\alertred{er}, Treff\alertred{er} %(aber: Aufkleber)
\z 
\z 
\end{itemize}

\end{frame}


%%%%%%%%%%%%%%%%%%%%%%%%%%%%%%%%%%%%
%\begin{frame}
%\frametitle{Exkurs: Spezialfälle der Affixbedeutung}
%
%
%Sind die rot markierten Einheiten in den folgenden Wörtern Kompositionsglieder oder Affixe? 
%
%\eal 
%\ex Verkehrs\alertred{-wesen}, Schul\alertred{-wesen}
%
%\ex essens\alertred{-technisch}
%
%%	\ex Laub\alertred{-werk}
%
%\ex \alertred{Haupt-}bahnhof
%
%\ex abgas\alertred{-arm}
%\zl
%
%\pause
%
%\begin{itemize}
%\item Pro \textbf{Kompositionsglied}: Morpheme treten auch \textbf{frei} auf.
%
%\item Pro \textbf{Affix}: viel abstraktere Bedeutung als frei vorkommende Form
%\end{itemize}
%
%\begin{block}{Halbaffixe/Affixoide (Suffixoide, Präfixoide)}
%Wortbildungselemente, die von frei vorkommenden Stämmen \textbf{grammatikalisiert} werden und dabei ihre \textbf{Bedeutung erweitern} bzw. \textbf{abstrakter} machen. Sie sind wie Affixe \textbf{reihenbildend}. In der gebundenen Version sind sie mit der frei vorkommenden Form \textbf{nicht bedeutungsgleich}.
%\end{block}
%
%\end{frame}
%
%
%%%%%%%%%%%%%%%%%%%%%%%%%%%%%%%%%%
%%%%%%%%%%%%%%%%%%%%%%%%%%%%%%%%%%
\subsubsection{Produktive und aktive Muster}
%\frame{
%%\begin{multicols}{2}
%\frametitle{~}
%	\tableofcontents[currentsection, hideallsubsections]
%%\end{multicols}
%}
%%%%%%%%%%%%%%%%%%%%%%%%%%%%%%%%%%

\begin{frame}
\frametitle{Produktive und aktive Muster}

\begin{itemize}
\item Manche Muster sind \textbf{produktiv} andere lediglich \textbf{aktiv}.
\end{itemize}

\begin{block}{produktive Wortbildungsregel}
Eine Wortbildungsregel (oder ein Wortbildungsmuster) ist produktiv, wenn mit ihr \textbf{häufig} \textbf{Neubildungen} vorgenommen werden können. Produktivität ist \textbf{graduell}, \zB sind \emph{-ung} und \emph{-er}-Suffigierungen hochproduktiv (\ref{ex:5cProd1}), während \emph{-eur}-Suffigierungen weniger produktiv sind und weitere Beschränkungen an die Basis stellen (\ref{ex:5cProd2}). 
\hfill \size{(\vgl \citealp{Eins16f}; \citealp[45ff]{Meibauer&Co07a})}
\end{block}

\ea 
\ea \label{ex:5cProd1} Absperr$+$\alertred{-ung}, Les$+$\alertred{-ung}
\ex \label{ex:5cProd2} Fris$+$\alertred{-eur}, Mont$+$\alertred{-eur} %Herkunftsbeschränkung
%	\ex \label{ex:5cProd3}
\z 
\z 

\end{frame}


%%%%%%%%%%%%%%%%%%%%%%%%%%%%%%%%%%
\begin{frame}
\frametitle{Produktive und aktive Muster}

\begin{block}{aktives Wortbildungsmuster}
Ein Wortbildungmuster ist \textbf{aktiv}, wenn \textbf{keine neuen Wörter} nach dem Muster gebildet werden können (oder selten und dann stilistisch markiert), aber das Muster \textbf{erkannt} werden kann (\ref{ex:5cProd4}), d.\,h. das Muster ist \textbf{transparent}.
\end{block}

\ea \label{ex:5cProd4} lieb$+$\alertred{-sam}, unterhalt$+$\alertred{-sam}
\z 

\begin{itemize}
\item Ist ein Muster \textbf{nicht mehr aktiv} und \textbf{nicht mehr transparent}, nimmt man die Form als \textbf{Simplex} wahr.

\ea Ur-$+$sache, Mäd$+$-chen
\z 
\end{itemize}

%\textbf{ÜB.3}	
\end{frame}


%%%%%%%%%%%%%%%%%%%%%%%%%%%%%%%%%%
%%%%%%%%%%%%%%%%%%%%%%%%%%%%%%%%%%
\subsection{Partikelverbbildung}

\iftoggle{sectoc}{
	\frame{
		%		\begin{multicols}{2}
		\frametitle{~}
		\tableofcontents[currentsubsection,subsubsectionstyle=hide]
		%		\end{multicols}
	}
}

%%%%%%%%%%%%%%%%%%%%%%%%%%%%%%%%%%

\begin{frame}
\frametitle{Partikelverbbildung}

\begin{itemize}
\item Komposition oder Derivation?

\ea 
\ea teil$+$nehm(-en)
\ex an$+$mach(-en)
\ex über$+$setz(-en) 
\z 
\z 
\pause 

\item Verben können auch aus \textbf{mehreren Morphemen} zusammengesetzt sein (komplex)

\item \textbf{Derivationssuffigierung} ist in der verbalen Wortbildung eher selten\\
(anders als bei Nomina und Adjektiven)

\begin{itemize}
\item nativ (selten, nicht produktiv): \ab{-el}

\ea läch$+$\alertred{-el}, hüst$+$\alertred{-el}, fremd$+$\alertred{-el}, \dots
\z

\item neoklassisch: \ab{-ier} (Allomorphe: \ab{-ifizier}, \ab{-isier})

\ea prob$+$\alertred{-ier}, elektr$+$\alertred{-ifizier}, alphabet$+$\alertred{-isier}, \dots 
\z
\end{itemize}
\end{itemize}

\end{frame}


%%%%%%%%%%%%%%%%%%%%%%%%%%%%%%%%%%
\begin{frame}
\frametitle{Partikelverbbildung}

\begin{itemize}
\item Die \textbf{produktiven Muster} zur Bildung komplexer Verben verwenden eher \textbf{Präfixe} (\ref{ex:5cPraV}) und \textbf{Partikeln} (\ref{ex:5cPartV}).

\ea 
\ea\label{ex:5cPraV} \alertred{be-}rat(-en), \alertred{ver-}kauf(-en)
\ex\label{ex:5cPartV} \alertred{teil-}nehm(-en), \alertred{fest}mach(-en), \alertred{auf}stell(-en)
\z 
\z 

\item Bei den Präfixverbbildungen handelt es sich um \textbf{Derivation}, da die \textbf{Präfixe} \textbf{gebundene} Affixe sind.


\item Die \textbf{Partikeln} dagegen kommen auch \textbf{frei} vor

\medskip

Partikel $+$ Verb \ras \textbf{Komposition}?

\item In (\ref{ex:5cPartV}): Nomen$+$Verb, Adjektiv$+$Verb, Präposition$+$Verb

\end{itemize}

\end{frame}


%%%%%%%%%%%%%%%%%%%%%%%%%%%%%%%%%%

\begin{frame}{Partikelverbbildung}
Wortbildungsprozess, bei dem -- anders als bei der Derivation -- zwei \textbf{frei vorkommende Formen} (Morpheme oder komplexere Stämme) verbunden werden. Partikelverben sind \textbf{morphologisch} und \textbf{syntaktisch trennbar} -- anders als bei der Komposition \size{\citep[vgl.][]{Eins16g,Eisenberg00a}}. 
\end{frame}



%%%%%%%%%%%%%%%%%%%%%%%%%%%%%%%%%%
%%%%%%%%%%%%%%%%%%%%%%%%%%%%%%%%%%
\subsubsection{Partikelverb \vs Präfixverb}
%\frame{
%	\frametitle{~}
%	%	\begin{multicols}{2}
%	\tableofcontents[currentsection, hideallsubsections]
%	%	\end{multicols}
%}
%%%%%%%%%%%%%%%%%%%%%%%%%%%%%%%%%%

\begin{frame}
\frametitle{Partikelverb \vs Präfixverb}

Partikelverben sind wie folgt von Präfixverben zu unterscheiden:

\ea  
\ea \alertred{an}schreiben 
\ex \alertred{be}schreiben
\z 
\z	
\begin{enumerate}
\item Betonung:

\ea \textipa{\textprimstress}\alertred{an}.schrei.ben \vs be.\textipa{\textprimstress}\alertred{schrei}.ben
\z

\item syntaktische Trennbarkeit:

\eal
\ex Ich \alertred{schreibe} dich \alertred{an}.
\ex Ich \alertred{beschreibe} dich.
\zl

\item morphologische Trennbarkeit:

\eal 
\ex Ich habe dich \alertred{an}ge\alertred{schrieb}en. 
\ex Ich habe dich \alertred{beschrieb}en.
\zl

%	\item \textbf{ÜB.4}	
\end{enumerate}

\end{frame}


%%%%%%%%%%%%%%%%%%%%%%%%%%%%%%%%%%
%%%%%%%%%%%%%%%%%%%%%%%%%%%%%%%%%%
\subsubsection{Simplex-, Präfix- und  Partikelverb}
%\frame{
%	\frametitle{~}
%	%	\begin{multicols}{2}
%	\tableofcontents[currentsection, hideallsubsections]
%	%	\end{multicols}
%}
%%%%%%%%%%%%%%%%%%%%%%%%%%%%%%%%%%

\begin{frame}
\frametitle{Simplex-, Präfix- und  Partikelverb}

\begin{table}
\centering
\scalebox{.93}{
\begin{tabular}{l|p{3cm}|p{3cm}|p{3cm}}
& \textbf{Simplex} & \textbf{Präfixverb} & \textbf{Partikelverb}\\
& \textit{\textbf{kaufen}} & \textit{\textbf{ver $+$ kaufen}} & \textit{\textbf{auf $+$ kaufen}}\\
\hline
\textbf{NS} & [\dots] dass Peter die Firma \alertred{kauft}. & [\dots] dass Peter die Firma \alertred{verkauft}. & [\dots] dass Peter die Firma \alertred{aufkauft}.\\
\hline
\textbf{HS} & Peter \alertred{kauft} die Firma. & Peter \alertred{verkauft} die Firma. & Peter \alertred{kauft} die Firma \alertred{auf}. \\
\hline
\textbf{Inf. mit zu} & Peter denkt nicht daran, die Firma \alertblue{zu}~\alertred{kaufen}. & Peter denkt nicht daran, die Firma \alertblue{zu}~\alertred{verkaufen}. & Peter denkt nicht daran, die Firma \alertred{auf}\alertblue{zu}\alertred{kaufen}.\\
\hline
\textbf{Part. II} & Peter hat die Firma \alertblue{ge}\alertred{kauf}\alertblue{t}. & Peter hat die Firma \alertred{verkauf}\alertblue{t}. & Peter hat die Firma \alertred{auf}\alertblue{ge}\alertred{kauf}\alertblue{t}. \\
\hline
\textbf{Betonung} & \textipa{\textprimstress}\alertred{kauf}.en & ver.\textipa{\textprimstress}\alertred{kauf}.en & \textipa{\textprimstress}\alertred{auf}.kauf.en\\
\end{tabular}
}

\end{table}

\end{frame}



%%%%%%%%%%%%%%%%%%%%%%%%%%%%%%%%%%
%%%%%%%%%%%%%%%%%%%%%%%%%%%%%%%%%%
\subsection{Konversion}

\iftoggle{sectoc}{
	\frame{
		%		\begin{multicols}{2}
		\frametitle{~}
		\tableofcontents[currentsubsection,subsubsectionstyle=hide]
		%		\end{multicols}
	}
}

%%%%%%%%%%%%%%%%%%%%%%%%%%%%%%%%%%

\begin{frame}
\frametitle{Konversion}

\begin{itemize}
\item Konversion ist die \textbf{Umkategorisierung} eines Elements \textbf{ohne} Zuhilfenahme von \textbf{Derivationsaffixen} \size{\citep{Eins16h,Eisenberg00a}}.

\item Konversion ist ein \textbf{nicht-konkatenativer Wortbildungsprozess}.

\eal 
\ex \alertred{\MyPdown{V}}schlaf(-en) \ras (der) \alertred{\MyPdown{N}}Schlaf, (des) Schlaf\alertblue{s}
\ex \alertred{\MyPdown{V}}schlaf(-en) \ras (das) \alertred{\MyPdown{N}}Schlafen, (des) Schlafen\alertblue{s}
%	\ex find(-en) \ras (der) Fund (vgl. gefunden)
\zl

\end{itemize}

\end{frame}


%%%%%%%%%%%%%%%%%%%%%%%%%%%%%%%%%%%
\begin{frame}
\frametitle{verschiedene Möglichkeiten der Umkategorisierung}

%Verschiedene Möglichkeiten der Umkategorisierung:

\medskip 

\begin{minipage}[t][][t]{.47\textwidth}

\textbf{Substantivbildung} aus

\ea \textbf{Adjektiv}\\
\alertred{\MyPdown{A}}blau \ras (das) \alertred{\MyPdown{N}}Blau\\
\alertred{\MyPdown{A}}fremd \ras \alertred{\MyPdown{N}}Fremder, \alertred{\MyPdown{N}}Fremde

\ex \textbf{Verb}\\
\alertred{\MyPdown{V}}schlaf(en) \ras \alertred{\MyPdown{N}}Schlaf \\
\alertred{\MyPdown{V}}les(en) \ras  \alertred{\MyPdown{N}}Lesen

\ex \textbf{Verb/Partizip}\\
\alertred{\MyPdown{V}}angestellt \ras \alertred{\MyPdown{N}}Angestellter\\ 
\alertred{\MyPdown{V}}reisend \ras   \alertred{\MyPdown{N}}Reisender

\ex \textbf{Partikel}\\
\alertred{\MyPdown{Part}}nein \ras  (das)  \alertred{\MyPdown{N}}Nein

%\ex \textbf{syntaktischer Fügung}\\
%ohne Wenn \& Aber\\
%das So-tun-als-ob
\z

\end{minipage}
%%
\pause 
\hfill 
%%
\begin{minipage}[t][][t]{.51\textwidth}
\textbf{Verbbildung} aus 

\ea \textbf{Adjektiv}\\
\alertred{\MyPdown{A}}grün \ras  \alertred{\MyPdown{V}}grün(en)\\ 
\alertred{\MyPdown{A}}rot \ras  \alertred{\MyPdown{V}}röt(en)

\ex \textbf{Substantiv}\\
\alertred{\MyPdown{N}}Öl \ras  \alertred{\MyPdown{V}}öl(en)
\z

\pause 
\medskip 

\textbf{Adjektivbildung} aus

\ea \textbf{Substantiv}\\
(der) \alertred{\MyPdown{N}}Ernst \ras  \alertred{\MyPdown{N}}ernst(e)

\ex \textbf{Verb/Partizip}\\
\alertred{\MyPdown{V}}reizend \ras  \alertred{\MyPdown{A}}reizend(e) \\
\alertred{\MyPdown{V}}ausgezeichnet~\ra~\alertred{\MyPdown{A}}ausgezeichnet(e)
\z

%\item \textbf{ÜB.4}

\end{minipage}

\end{frame}


%%%%%%%%%%%%%%%%%%%%%%%%%%%%%%%%%%
%%%%%%%%%%%%%%%%%%%%%%%%%%%%%%%%%%
\subsubsection{Untertypen der Konversion}
%\frame{
%	\frametitle{~}
%	%	\begin{multicols}{2}
%	\tableofcontents[currentsection, hideallsubsections]
%	%	\end{multicols}
%}
%%%%%%%%%%%%%%%%%%%%%%%%%%%%%%%%%%

\begin{frame}
\frametitle{Untertypen der Konversion}


\begin{block}{Morphologische Konversion}
Umkategorisierung eines \textbf{Stammes ohne Flexionselemente}
\end{block}

\eal 
\ex \alertred{\MyPdown{V}}\alertblue{lauf}(-en) \ras (der) \alertred{\MyPdown{N}}\alertblue{Lauf}
\ex \alertred{\MyPdown{N}}\alertblue{Kleid} \ras \alertred{\MyPdown{V}}\alertblue{kleid}(-en)
\zl

\pause 

\begin{block}{Syntaktische Konversion}
Umkategorisierung eines \textbf{Stammes mit Flexionselementen}. In einigen Ansätzen wird die syntaktische Konversion als \textbf{syntaktisches} (und nicht als morphologisches) \textbf{Phänomen} analysiert.
\end{block}

\eal 
\ex \alertred{\MyPdown{V}}lauf(\alertred{-en}) \ras  (das) \alertred{\MyPdown{V}}Lauf\alertblue{en} 
\ex \alertred{\MyPdown{A}}arbeitslos(\alertred{-e})/(\alertred{-er}) \ras (der) \alertred{\MyPdown{N}}Arbeitslos\alertblue{e}, (ein) \alertred{\MyPdown{N}}Arbeitslos\alertblue{er}
%	\ex gefallen(-e) \ras der/die/das Gefallene, ein Gefallener
\zl

\end{frame}


%%%%%%%%%%%%%%%%%%%%%%%%%%%%%%%%%%
\begin{frame}
\frametitle{Konversion \vs implizite Derivation}

\begin{itemize}
\item Bei der Konversion werden \textbf{keine Affixe} für die Umkategorisierung verwendet.

\item Bei der \textbf{impliziten Derivation} (\ref{ex:5cImplDer}) handelt es ich um die Umkategorisierung eines Stammes mittels \textbf{Ablaut} (\zB\ \textipa{[I]} \ras \textipa{[U]}), aber ohne Affixe (wie in der \textbf{expliziten Derivation} (\ref{ex:5cExplDer})).

\ea \label{ex:5cImplDer}
\ea \alertred{\MyPdown{V}}find(-en) \ras \alertred{\MyPdown{N}}F\alertblue{u}nd
\ex \alertred{\MyPdown{V}}greif(-en) \ras \alertred{\MyPdown{N}}Gr\alertblue{i}ff
\z 
\ex \label{ex:5cExplDer} \alertred{\MyPdown{V}}trink(-en) \ras \alertred{\MyPdown{A}}trink$+$bar	
\z 

\item Die Behandlung von impliziten Derivationen als Konversionen ist jedoch \textbf{umstritten} (aufgrund der \emph{Veränderung} im Stamm). Bei der \textbf{Konversion im strengeren Sinne} soll es \textbf{keinerlei Veränderung} geben.

\item Die \textbf{implizite Derivation} ist im Deutschen \textbf{nicht mehr produktiv}.
\end{itemize}

\end{frame}


%%%%%%%%%%%%%%%%%%%%%%%%%%%%%%%%%%
%%%%%%%%%%%%%%%%%%%%%%%%%%%%%%%%%%
\subsubsection{Struktur der Konversion}
%\frame{
%	\frametitle{~}
%	%	\begin{multicols}{2}
%	\tableofcontents[currentsection, hideallsubsections]
%	%	\end{multicols}
%}
%%%%%%%%%%%%%%%%%%%%%%%%%%%%%%%%%%

\begin{frame}
\frametitle{Struktur der Konversion}

\begin{itemize}
\item Konversionen werden als \textbf{unäre Projektionen} analysiert (vgl.\ Abb. \ref{fig:5cKonUnary}). 

\item Unäre Projektionen sind für das \textbf{Köpfigkeitsprinzip} problematisch.

Beachten Sie, dass das \textbf{Flexionssuffix kein morphologischer Kopf} ist!

\end{itemize}

\begin{minipage}{.49\textwidth}

\begin{figure}	
\centering
\scalebox{.7}{
\begin{forest}
sn edges,
[V
[\alertred{V}, name=Verb
[\alertred{N}, name=Noun
[gras]
]
]
[Fl [-en]]
]{
\draw[->,dashed] (Noun) to[out=west,in=west] (Verb);
}
\end{forest}}
\caption{unäre Struktur}
\label{fig:5cKonUnary}
\end{figure}

\end{minipage}
%
\hfill %
\pause %
%
\begin{minipage}{.49\textwidth}

\begin{figure}	
\centering
\scalebox{.7}{
\begin{forest}
sn edges,
[V
[\alertred{V}, name=Verb
[N, name=Noun
[gras]
]
[\alertred{V\MyPup{af}}, name=VerbAf
[$\emptyset$]
]
]
[Fl [-en]]
]{
\draw[->,dashed] (VerbAf) to[out=east,in=east] (Verb);
}
\end{forest}}
\caption{Konversion als Derivation}
\label{fig:5cKonDer}
\end{figure}

\end{minipage}	

\begin{itemize}
\item<2-> \textbf{Alternativ} wird \textbf{Konversion} \textbf{als explizite Ableitung} analysiert (s. rechts), bei der ein \textbf{Nullmorphem}, d.\,h. ein phonetisch leeres Element ($\emptyset$), als Kopf der Derivation angenommen wird (aus anderen Gründen problematisch). 

\item<2-> \textbf{In diesem Kurs} behandeln wir Konversionen als \textbf{unäre Projektionen}.
\end{itemize}

\end{frame}


%%%%%%%%%%%%%%%%%%%%%%%%%%%%%%%%%%%
%%%%%%%%%%%%%%%%%%%%%%%%%%%%%%%%%%%
%\subsection{Exkurs: Andere Wortbildungsarten}
%
%\iftoggle{sectoc}{
%	\frame{
%		%		\begin{multicols}{2}
%		\frametitle{~}
%		\tableofcontents[currentsubsection,subsubsectionstyle=hide]
%		%		\end{multicols}
%	}
%}
%
%%%%%%%%%%%%%%%%%%%%%%%%%%%%%%%%%%%
%
%\begin{frame}
%\frametitle{Exkurs: Andere Wortbildungsarten}
%
%
%\textbf{Rückbildung} (Reanalyse): häufig zur Bildung von Verben verwendete Umdrehung einer Wortbildungsregel
%
%\ea staubsaugen:\\
%\alertred{\MyPdown{V}}saug- \ras \alertred{\MyPdown{N}}Saug $+$ -er \ras \alertred{\MyPdown{N}}Staub $+$ sauger \ras \alertred{\MyPdown{V}}staubsaug(-en)
%\z
%
%\begin{itemize}
%
%\item Verb als Ergebnis der Regel tritt häufig nur in \textbf{finaler Satzposition} (keine V2-Stellung), und hat \idR \textbf{kein vollständiges Paradigma}.
%
%
%\ea 
%\ea[]{bergsteigen, schleichwerben, farbkopieren, bausparen, notlanden}
%%				mähdreschen
%\ex[]{Sie mussten wieder \alertred{schleichwerben}.}
%\ex[]{Sie haben lange \alertred{baugespart}.}
%\ex[*]{Sie \alertred{werben} wieder \alertred{schleich}.}
%\z 
%\z
%
%\item seltener auch bei der Bildung von Nomina oder Adjektiven
%
%\ea sympath \ras sympath $+$ -isch \ras un- $+$ sympathisch \ras Unsympath
%\z
%
%\end{itemize}
%
%%\end{itemize}
%
%\end{frame}
%
%
%%%%%%%%%%%%%%%%%%%%%%%%%%%%%%%%%%%%
%\begin{frame}
%\frametitle{Exkurs: Andere Wortbildungsarten}
%
%\textbf{Phrasenkomposition} (auch Zusammenrückung): recht produktive Zusammenrückung \textbf{syntaktischer Phrasen}, wobei Wortfolge und Flexionsmarkierungen beibehalten werden. Einige sind bereits lexikalisiert (\ref{ex:5SonstWBA}).
%
%\settowidth\jamwidth{[Bsp. in (XX) aus Pafel 2017I]} 	
%\ea \label{ex:5SonstWBA} Möchte$+$gern, in$+$folge, wasser$+$triefend
%
%\ex \label{ex:5PKomp0}
%\ea (das) Das-haben-wir-immer-schon-so-gemacht
%
%%		\ex Kaufe-Ihr-Auto-Kärtchen
%\ex Zwischen-den-Zeilen \gq{Ihr könnt mich mal}-Attitüde
%\ex \gq{Ich werde-dich-ewig-lieben}-Briefchen
%\ex Lauf-dich-gesund-Bewegung 		
%\jambox{[Bsp. in (\ref{ex:5PKomp0}) aus \size{\citealp{Pafel17a}}]}
%\z
%\z
%
%
%%\medskip
%\pause 
%
%
%\textbf{Zusammenbildung:} Dreigliedrige Konkatenation von Elementen. \textbf{Weder die ersten} beiden \textbf{noch die letzten} beiden Glieder kommen \textbf{frei} vor. Sie werden manchmal als \textbf{Derivation} mit einem \textbf{nicht lexikalischen ersten Teil} analysiert.
%
%\ea Schrift$+$stell$+$er, Alt$+$sprach$+$ler
%\ex 
%\ea[?]{ {[}\MyPdown{V} schriftstell-{]} $+$ {[}\MyPdown{N\MyPup{af}} -er{]} }
%\ex[?]{ {[}\MyPdown{N} Schrift{]} $+$ {[}\MyPdown{N} steller{]}}
%\z 
%\z
%
%%\item[] \textbf{ÜB.1}
%
%\end{frame}
%
%
%%%%%%%%%%%%%%%%%%%%%%%%%%%%%%%%%%%
%\begin{frame}
%
%
%\textbf{Kontamination} (Wortverschmelzung, -kreuzung, Amalgamierung):
%
%Verschmelzung zweier Wörter, so dass Wortmaterial aus einem der Originalwörter (oder aus beiden) getilgt wird.
%
%\ea 
%\ea Info\alertred{rmation} $+$ \alertred{Enter}tainment \ras Info$+$tainment
%\ex Bio\alertred{logisch} $+$ \alertred{Jo}ghurt \ras Bio$+$ghurt 
%\ex Mainz $+$ \alertred{einz}igartig \ras mainz$+$igartig
%\ex Eur\alertred{opa} $+$ Asien \ras Eur$+$asien
%\z 
%\z
%
%
%\medskip
%\pause 
%
%
%\textbf{Kurzwortbildung}:
%
%\settowidth\jamwidth{I[phonetisch ungebunden: \textbf{Abkürzung}]} 
%\ea ARD, EU, CIA
%\jambox{[phonetisch ungebunden: \textbf{Abkürzung}]} 
%\z
%
%\ea DAX, PIN, UFO
%\jambox{[phonetisch gebunden: \textbf{Akronym}]} 
%\z
%
%
%\medskip
%
%
%\textbf{Sonstige Kurzwörter:} Wortmaterial am Wortanfang oder -ende wird getilgt.
%
%\ea Kripo, Bus, Auto, bi, öko, Schum\alertred{i}, Alk\alertred{i}, Schok\alertred{i}
%\z
%
%\end{frame}
%
%
%%%%%%%%%%%%%%%%%%%%%%%%%%%%%%%%%%%
%\begin{frame}
%
%
%
%\textbf{Analogie:} Bildung eines neuen Wortes durch Ersetzung eines Morphems eines komplexen Wortes durch ein anderes
%
%\ea e-\alertred{card} (von \emph{e-mail}), \alertred{slow} food (von \emph{fast food})
%\z
%
%
%\medskip
%\pause 
%
%
%\textbf{Reduplikation:}
%
%\settowidth\jamwidth{XX[komplette Dopplung]} 
%\ea Bla$+$bla, Wau$+$wau \jambox{[komplette Dopplung]}
%
%\ex L\alertred{ari}$+$f\alertred{ari}, H\alertred{okus}$+$p\alertred{okus} \jambox{[Reimdopplung]}
%
%\ex W\alertred{i}rr$+$w\alertred{a}rr, W\alertred{i}schi$+$w\alertred{a}schi, S\alertred{i}ng$+$s\alertred{a}ng  \jambox{[Ablautdopplung]}
%\z
%
%\end{frame}
%
%
%%%%%%%%%%%%%%%%%%%%%%%%%%%%%%%%%%%
%\begin{frame}
%
%
%\textbf{Generifizierung:} Ausweitung auf Gattungsbezeichnung
%
%\ea Tempo (für \gq{Taschentücher}), Edding (für \gq{Permanentmarker}),\\Zewa (für \gq{Küchenpapier}), TippEx (für \gq{Korrekturflüssigkeit})
%\z
%
%
%\medskip
%\pause 
%
%
%\textbf{Wortschöpfung:}
%
%\ea Vileda (\emph{wie Leder}), Iglo (von \emph{Iglu}), Haribo (\emph{Hans Riegel Bonn})
%\z
%
%
%\medskip
%\pause 
%
%
%\textbf{Fremdwortbildung:} Bildung von Wörter nach dem Muster einer Fremdsprache. Diese Wörter gibt es in der \textbf{Ursprungssprache} nicht oder nicht mit dieser Bedeutung (vgl. (\ref{ex:M5bFWB1})). Prozess ist auch produktiv auch mit sog. \textbf{Konfixen} (vgl. (\ref{ex:M5bFWB2}))
%
%\ea\label{ex:M5bFWB1} Handy, Wellness, Beamer
%
%\ex\label{ex:M5bFWB2} \alertred{Thermo}$+$hose, Schok$+$\alertred{-aholic}
%\z
%
%\end{frame}
%
%
%%%%%%%%%%%%%%%%%%%%%%%%%%%%%%%%%%%
%%%%%%%%%%%%%%%%%%%%%%%%%%%%%%%%%%%
%todo @ee HA komplett nach 05d verschieben
\subsection{Hausaufgabe}
%
%\iftoggle{sectoc}{
%	\frame{
%		%		\begin{multicols}{2}
%		\frametitle{~}
%		\tableofcontents[currentsubsection,subsubsectionstyle=hide]
%		%		\end{multicols}
%	}
%}
%
%%%%%%%%%%%%%%%%%%%%%%%%%%%%%%%%%%%

%% @ee	Muss irgendiwe aufgeteilt werden auf 05c und 05d
%%			Aufg. 9 muss auf jeden Fall nach 05d

\begin{frame}
\frametitle{Hausaufgabe}


\begin{enumerate}
	\item Kreuzen Sie die korrekten Aussagen an: %\hfill(0,5 Punkte pro Aussage)\\
	
	\begin{itemize}
		\item[$\circ$] Die Graphemkette \emph{abarbeiten} ist ein einzelnes phonologisches Wort im Deutschen.
		\item[$\circ$] \emph{Morphologieeinführungsbuch} ist ein orthographisch-graphemisches Wort des Deutschen, sowie \emph{introductory morphology book} ein orthographisch-graphemisches Wort des Englischen ist.
		\item[$\circ$] Ein Morphem ist die kleinste bedeutungsunterscheidende Einheit in einem bestimmten Sprachsystem.
		\item[$\circ$] \ab{Brot} und \ab{Bröt} sind Allomorphe eines einzelnen Morphems.
	\end{itemize}
	
	\item Erklären Sie das Prinzip der Rechtsköpfigkeit in der Morphologie des Deutschen. Verwenden Sie bei Ihrer Erklärung die unten angegebenen Beispiele. %\hfill(4 Punkte)\\
	
	\eal\label{ex:05cHA2}
	\ex\label{ex:05cHA2a} lichtblau, Blaulicht
	\ex\label{ex:05cHA2b} die Fotowelt, das Weltfoto
	\ex\label{ex:05cHA2c} der Bücherrücken/die Bücherrücken, das Rückenbuch/die Rückenbücher
	\zl
\end{enumerate}

\end{frame}


%%%%%%%%%%%%%%%%%%%%%%%%%%%%%%%%%%%
\begin{frame}
\frametitle{Hausaufgabe}

\begin{itemize}
\item[3.] Geben Sie Argumente für oder gegen die Behandlung von \emph{ver-} in den folgenden Wörtern als Morphem an. Wenn es sich um ein Morphem handelt, ist das immer das gleiche Morphem? %(4 Punkte)

\eal\label{ex:05cHA3}
\ex\label{ex:05cHA3a} \emph{Ver}zweiflung
\ex\label{ex:05cHA3b} \emph{Ver}s
\ex\label{ex:05cHA3c} \emph{ver}kaufen
\ex\label{ex:05cHA3d} \emph{ver}schreiben
\ex\label{ex:05cHA3e} \emph{ver}fahren
\zl

\end{itemize}
\end{frame}


%%%%%%%%%%%%%%%%%%%%%%%%%%%%%%%%%%%
\begin{frame}
\frametitle{Hausaufgabe}

\begin{itemize}
\item[4.] Ordnen Sie die Wortbildungsprozesse links den passenden Beispielen rechts zu (dazu müssen Sie nur den entsprechenden Buchstaben neben das passende Beispiel schreiben). %(0,5 Punkte pro Aussage)
%\end{itemize}

\begin{table}[h!]
\begin{minipage}{0.4\linewidth}
\centering
\begin{tabular}{l|p{0.1\textwidth}|}
	Determinativkompositum & (A)\\
	\hline
	Konversion & (B)\\
	\hline
	Zirkumfigierung (Derivation) & (C)\\
	\hline
	Rektionskompositum & (D)\\
	\hline
	Possessivkompositum & (E)\\
\end{tabular}

\end{minipage}\hfill%
\begin{minipage}{0.4\linewidth}
\centering
\begin{tabular}{|p{0.1\textwidth}|r}
	& \emph{Gerede} \\
	\hline
	& \emph{Milchgesicht}\\
	\hline
	& \emph{Lauf} \\
	\hline
	& \emph{Kettenraucher}  \\
	\hline
	& \emph{Klausurbesprechung}  \\
\end{tabular}
\end{minipage}
\end{table}

		
\item [5.] Geben Sie für die folgende Wortform die Flexionskategorien an, nach denen sie flektiert ist.\\
%\hfill(3 Punkte)\\
\ea\label{ex:05cHA5}
bestehe
\z

\end{itemize}

\end{frame}


%%%%%%%%%%%%%%%%%%%%%%%%%%%%%%%%%%%
\begin{frame}
\frametitle{Hausaufgabe}

\begin{itemize}
\item[6.] Warum sind die Wörter unter (\ref{ex:05cHA6a}) grammatisch und die unter (\ref{ex:05cHA6b}) ungrammatisch? %(4 Punkte)
\eal\label{ex:05cHA6}
\ex\label{ex:05cHA6a} kaufbar, trinkbar
\ex\label{ex:05cHA6b} *fensterbar, *helfbar, *schönbar
\zl

\item [7.] Sind die folgenden Verben Präfixverben oder Partikelverben? Begründen Sie Ihre Entscheidungen. %(3 Punkte)

\eal\label{ex:05cHA7}
\ex auskennen
\ex erkennen
\ex aberkennen
\zl

\item [8.] Geben Sie für das folgende Wort eine morphologische Konstituentenstruktur (inklusive Konstituentenkategorien (N, N\textsuperscript{af}, V, V\textsuperscript{af}, \dots)) an, und bestimmen Sie für jeden Knoten den Wortbildungstyp. %(6,5 Punkte)

\ea\label{ex:05cHA8}
Wahlkampfberaterinnen
\z

\end{itemize}

\end{frame}


%%%%%%%%%%%%%%%%%%%%%%%%%%%%%%%%%%%
\begin{frame}
\frametitle{Hausaufgabe}

\begin{itemize}

\item [9.] Paraphrasieren Sie das folgende komplexe Wort so, dass es der angegebenen Struktur entspricht (auch wenn Sie selbst eine andere Struktur plausibler finden sollten). %(2 Punkte)

\begin{forest}sn edges,
[N
[N[N[Reserve]]
[N[V[lehr]][N\textsuperscript{af}[-er]]]]
[N[zimmer]]
]
\end{forest}

\end{itemize}

\end{frame}


%%%%%%%%%%%%%%%%%%%%%%%%%%%%%%%%%
%%%%%%%%%%%%%%%%%%%%%%%%%%%%%%%%%
\iftoggle{ha-loesung}{
	%%%%%%%%%%%%%%%%%%%%%%%%%%%%%%%%%%
%% HA 1 - 05c Morphologie
%%%%%%%%%%%%%%%%%%%%%%%%%%%%%%%%%%

\begin{frame}{Hausaufgabe -- Lösung}

\begin{enumerate}
	\item Kreuzen Sie die korrekten Aussagen an %\hfill(0,5 Punkte pro Aussage)\\

\begin{itemize}
	\item[$\circ$] Die Graphemkette abarbeiten ist ein einzelnes phonologisches Wort im Deutschen.
	\item[$\circ$] \emph{Morphologieeinführungsbuch} ist ein orthographisch-graphemisches Wort des Deutschen, sowie \emph{introductory morphology book} ein orthographisch-graphemisches Wort des Englischen ist.
	\item[$\circ$] Ein Morphem ist die kleinste bedeutungsunterscheidende Einheit in einem bestimmten Sprachsystem.
	\alertred{
			\item[\alertred{$\checkmark$}] \ab{Brot} und \ab{Bröt} sind Allomorphe eines einzelnen Morphems.
		}
\end{itemize}

\end{enumerate}
\end{frame}


%%%%%%%%%%%%%%%%%%%%%%%%%%%%%%%%%%
\begin{frame}{Hausaufgabe -- Lösung}

\begin{enumerate}
	\item[2.] Erklären Sie das Prinzip der Rechtsköpfigkeit in der Morphologie des Deutschen. Verwenden Sie bei Ihrer Erklärung die unten angegebenen Beispiele. %\hfill(4 Punkte)\\

\begin{exe}
	\exr{ex:05cHA2}
	\begin{xlist}
		\ex lichtblau, Blaulicht
		\ex die Fotowelt, das Weltfoto
		\ex	die Bücherrücken, die Rückenbücher
	\end{xlist}
\end{exe}

\pause

\alertred{Der Kopf eines Wortes ist immer rechtsperipher. Er bestimmt die morphosyntaktischen Eigenschaften eines Wortes sowie viele semantische Aspekte, \zB}

	\begin{itemize}
		\item[\alertred{--}] \alertred{(\ref{ex:05cHA2a}): Wortart (A \vs N)}
		\item[\alertred{--}] \alertred{(\ref{ex:05cHA2b}): Genus (f \vs n)}
		\item[\alertred{--}] \alertred{(\ref{ex:05cHA2c}): Pluralflexion (endungslos \vs \emph{-er})}
		\item[\alertred{--}] \alertred{Bei Determinativkomposita bildet das Kompositum eine Unterart des Kopfes, bspw. geht es in (\ref{ex:05cHA2c}) im ersten Fall um einen bestimmen Blauton und im zweiten um eine bestimmte Art von Licht.}
	\end{itemize}

\end{enumerate}
\end{frame}


%%%%%%%%%%%%%%%%%%%%%%%%%%%%%%%%%

\begin{frame}{Hausaufgabe -- Lösung}

\begin{enumerate}
\item[3.] Geben Sie Argumente für oder gegen die Behandlung von \emph{ver-} in den folgenden Wörtern als Morphem an. Wenn es sich um ein Morphem handelt, ist das immer das gleiche Morphem?%\\
%\hfill(4 Punkte)\\

\begin{exe}
	\exr{ex:05cHA3}
	\begin{xlist}
		\ex \emph{Ver}zweiflung
		\ex \emph{Ver}s
		\ex \emph{ver}kaufen
		\ex \emph{ver}schreiben
		\ex \emph{ver}fahren
	\end{xlist}
\end{exe}

\pause

\alertred{Morphem: Kleinste bedeutungstragende Einheit im Sprachsystem.}

\begin{itemize}
	\item[\alertred{--}] \alertred{\emph{ver-} in (\ref{ex:05cHA3b}) ist kein Morphem, sondern Bestandteil des Stammes.}
	\item[\alertred{--}] \alertred{\emph{ver-} in (\ref{ex:05cHA3d}) und (\ref{ex:05cHA3e}) ist ein Morphem mit der Bedeutung \gq{X falsch machen}.}
	\item[\alertred{--}] \alertred{\emph{ver-} in (\ref{ex:05cHA3a}) und (\ref{ex:05cHA3c}) sind auch Morpheme, aber zwei andere Morpheme, weil sie jeweils abweichende Bedeutungen tragen:}
		\begin{itemize}
			\item[] \alertred{\emph{ver-} in (\ref{ex:05cHA3c}) kehrt die Bedeutung von X um.}
			\item[] \alertred{\emph{ver-} in (\ref{ex:05cHA3a}) trägt eine intensivierende(?) Bedeutung.}
		\end{itemize}
\end{itemize}

\end{enumerate}
\end{frame}


%%%%%%%%%%%%%%%%%%%%%%%%%%%%%%%%%
\begin{frame}{Hausaufgabe -- Lösung}

\begin{enumerate}
\item[4.] Ordnen Sie die Wortbildungsprozesse links den passenden Beispielen rechts zu (dazu müssen Sie nur den entsprechenden Buchstaben neben das passende Beispiel schreiben). %\\
%\hfill(0,5 Punkte pro Aussage)\\

\begin{table}[h!]
	\begin{minipage}{0.4\linewidth}
		\centering
		\begin{tabular}{l|p{0.1\textwidth}|}
			Determinativkompositum & (A)\\
			\hline
			Konversion & (B)\\
			\hline
			Zirkumfigierung (Derivation) & (C)\\
			\hline
			Rektionskompositum & (D)\\
			\hline
			Possessivkompositum & (E)\\
		\end{tabular}
	
\end{minipage}\hfill%
\begin{minipage}{0.4\linewidth}
\centering
		\begin{tabular}{|p{0.1\textwidth}|r}
			\only<2->{\alertred{C}} & \emph{Gerede} \\
			\hline
			\only<3->{\alertred{E}} & \emph{Milchgesicht}\\
			\hline
			\only<4->{\alertred{B}} & \emph{Lauf} \\
			\hline
			\only<5->{\alertred{A}} & \emph{Kettenraucher}  \\
			\hline
			\only<6->{\alertred{D}} & \emph{Klausurbesprechung}  \\
		\end{tabular}
	\end{minipage}
\end{table}


\item[5.] Geben Sie für die folgende Wortform die Flexionskategorien an, nach denen sie flektiert ist. %\\
%\hfill(3 Punkte)\\
\end{enumerate}

\begin{columns}
\column[t]{.28\textwidth}
	\begin{exe}
		\exr{ex:05cHA5} bestehe
	\end{exe}

\column[t]{.6\textwidth}
\visible<7->{%
\textcolor{red}{%
1. / Sg. / Präsens / Indikativ / Aktiv\\
1. / Sg. / Präsens / Konjunktiv I / Aktiv\\
3. / Sg. / Präsens / Konjunktiv I / Aktiv\\
2. / Sg. / Präsens / Imperativ / Aktiv
}
}
\end{columns}



\end{frame}


%%%%%%%%%%%%%%%%%%%%%%%%%%%%%%%

\begin{frame}{Hausaufgabe -- Lösung}

\begin{enumerate}
	\item[6.] Warum sind die Wörter unter (\ref{ex:05cHA6a}) grammatisch und die unter (\ref{ex:05cHA6b}) ungrammatisch? %(4 Punkte)
	
	\begin{exe}
		\exr{ex:05cHA6}
		\begin{xlist}
			\ex kaufbar, trinkbar
			\ex *fensterbar, *helfbar, *schönbar
		\end{xlist}
	\end{exe}
	
\pause

\alertred{Das Suffix \emph{-bar} hat die folgenden Beschränkungen bzgl. der Basis X, mit der es sich verbindet:}
		\begin{itemize}
			\item[\alertred{--}] \alertred{X muss ein Verb sein (nicht Nomen oder Adjektiv)}
			\item[\alertred{--}] \alertred{X muss transitiv sein (nicht wie \emph{helfen})}
		\end{itemize}

\end{enumerate}
\end{frame}


%%%%%%%%%%%%%%%%%%%%%%%%%%%%%%%

\begin{frame}{Hausaufgabe -- Lösung}

\begin{enumerate}
\item[7.] Sind die folgenden Verben Präfixverben oder Partikelverben? Begründen Sie Ihre Entscheidungen. %\hfill(3 Punkte)\\

\begin{exe}
	\exr{ex:05cHA7}
	\begin{xlist}
		\ex auskennen
		\ex erkennen
		\ex aberkennen
	\end{xlist}
\end{exe}

\pause

\alertred{Partikelverben sind:}
\begin{itemize}
	\item[\alertred{--}] \alertred{morphologisch trennbar (\emph{aus-ge-kannt}, \emph{ab-zu-erkennen}). }
	\item[\alertred{--}] \alertred{syntaktisch trennbar (\emph{Peter kennt sich aus.}, \emph{Die Frau erkennt die Urkunde ab.}). }
	\item[\alertred{--}] \alertred{betont (\emph{\textprimstress auskennen} und \emph{\textprimstress aberkennen}).}
\end{itemize}
		
\pause
\medskip
		
\alertred{Präfixverben sind:}
\begin{itemize}
	\item[\alertred{--}] \alertred{weder morphologisch noch syntaktisch trennbar (*\emph{ergekannt}, *\emph{Sie kannte ihn er.}). }
	\item[\alertred{--}] \alertred{nicht betont (\emph{er}\textprimstress \emph{kennen}). }
\end{itemize}

\pause

\alertred{\emph{aberkennen} ist ein Partikelverb, welches aus einem Präfixverb und einer Partikel besteht (ab$+$erkennen).}

\end{enumerate}
\end{frame}


%%%%%%%%%%%%%%%%%%%%%%%%%%%%%%%%%

\begin{frame}{Hausaufgabe -- Lösung}

\begin{enumerate}
\item[8.] Geben Sie für das folgende Wort eine morphologische Konstituentenstruktur (inklusive Konstituentenkategorien (N, N\textsuperscript{af}, V, V\textsuperscript{af}, \dots)) an, und bestimmen Sie für jeden Knoten den Wortbildungstyp. %\hfill(6,5 Punkte)\\

\begin{exe}
	\exr{ex:05cHA8} Wahlkampfberaterinnen
\end{exe}

\end{enumerate}

\vspace{-.25cm}

\begin{figure}
\centering

\scalebox{.6}{

\textcolor{red}{
\begin{forest} MyP edges,
	[N, name=N1
	[N, name=N2
	[N, name=N3
	[N, name=N4
	[N, name=N6 [V[wahl/wähl]]]
	[N, name=N7 [V[kampf/kämpf]]]]
	[N, name=N5[V, name=V1	[V\textsubscript{af}[be-]]
	[V[rat]]]
	[N\textsuperscript{af}[-er]]]]
	[N\textsuperscript{af}[-in]]]
	[Fl[-nen]]]	
{
	\draw[<-, red] (N1.west)--++(-12em,0pt)
	node[anchor=east,align=center]{Flexion (KEIN Wortbildungsporzess)};
	\draw[<-, red] (N2.west)--++(-14em,0pt)
	node[anchor=east,align=center]{Derivation (Movierung)};
	\draw[<-, red] (N3.west)--++(-9.5em,0pt)
	node[anchor=east,align=center]{Determinativkompositum};
	\draw[<-, red] (N4.west)--++(-4em,0pt)
	node[anchor=east,align=center]{Determinativkompositum};
	\draw[<-, red] (N5.west)--++(-3em,0pt)
	node[anchor=east,align=center]{Derivation};
	\draw[<-, red] (N6.west)--++(-3em,0pt)
	node[anchor=east,align=center]{Implizite Derivation};
	\draw[<-, red] (N7.east)--++(2.5em,0pt)--++(0em,-18ex)%--++(2em,0pt)
	node[anchor=north,align=center]{Implizite Derivation};
	\draw[<-, red] (V1.east)--++(1.5em,0pt)--++(0em,-14ex)--++(2em,0pt)
	node[anchor=west,align=center]{Derivation};
}
\end{forest}	
%\begin{itemize}
%	\item[]N1: Flexion (KEIN Wortbildungsporzess)
%	\item[]N2: Derivation (Movierung)
%	\item[]N3: Determinativkompositum
%	\item[]N4: Determinativkompositum
%	\item[]N5: Derivation
%	\item[]N6: Implizite Derivation
%	\item[]N7: Implizite Derivation
%	\item[]V1: Derivation
%\end{itemize}
}
}

\end{figure}
\end{frame}


%%%%%%%%%%%%%%%%%%%%%%%%%%%%%%%%%%%

\begin{frame}{Hausaufgabe -- Lösung}

\begin{enumerate}
\item[9.] Paraphrasieren Sie das folgende komplexe Wort so, dass es der angegebenen Struktur entspricht (auch wenn Sie selbst eine andere Struktur plausibler finden sollten). %\\
%\hfill(2 Punkte)\\


\begin{forest}sn edges,
	[N
	[N[N[Reserve]]
	[N[V[lehr]][N\textsuperscript{af}[-er]]]]
	[N[zimmer]]
	]
\end{forest}

\pause

\textcolor{red}{Ein Zimmer für Reservelehrer}

\end{enumerate}
\end{frame}


}

%%%%%%%%%%%%%%%%%%%%%%%%%%%%%%%%%%%
%%%%%%%%%%%%%%%%%%%%%%%%%%%%%%%%%%%
%\section{X}
%%\frame{
%%\frametitle{~}
%%	\tableofcontents[currentsection]
%%}
%
%
%%%%%%%%%%%%%%%%%%%%%%%%%%%%%%%%%%%
%\begin{frame}
%\frametitle{Y}
%
%\begin{itemize}
%	\item 
%\end{itemize}
%
%
%\end{frame}




