
% \documentclass[c]{beamer}


% \usepackage{ngerman}
% %\usepackage[german,english]{babel}
% \usepackage[utf8x]{inputenc}
% \usepackage[T1]{fontenc}

% %\usepackage{url}
% \usepackage{multicol}
% %\usepackage{multirow}
% \usepackage{graphicx}
% \usepackage{felix-mystyle}
% \usepackage{textcomp}
% \usepackage{epsf}
% %\usepackage{tree-dvips}
% %\usepackage{cgloss4e}
% \usepackage{felix-gb4e-slides}
% %\usepackage{gb4e+}

% \input{fuberlinbeamerincludes.tex}
% \usepackage[sectionbib]{natbib}
% \setlength{\bibsep}{1mm}


% \let\bibsection=~
% \let\bibfont=\footnotesize

%  \beamersetleftmargin{3mm}
%  \beamersetrightmargin{3mm}

% \setlength\leftmargini{1em}
% \setlength\leftmarginii{1em}
% \setlength\leftmarginiii{1em}
% \setbeamersize{text margin left=1em,text margin right=1em}

% \let\citew=\citealp
% \bibpunct[, ]{(}{)}{;}{a}{}{,}
% \newcommand{\newblock}{}



% \newcommand{\ipa}[1]{\textipa{#1}}



% \parindent0pt
% %\avmoptions{center}



% \selectlanguage{german}
% % \treelinewidth=.6pt
% % \arrowwidth=4pt
% % \arrowlength=4pt

% % \arrowhead{3pt}{4pt}{1pt}


% \newcommand{\tsem}[1]{\textlbrackdbl #1\textrbrackdbl$^{\textrm{\tiny{t}}}$}




% \usetheme{FUBerlin}
% %\logo{\includegraphics[height=.9cm]{logo.eps}}
% %\titlegraphic{\includegraphics[height=3cm]{cd.eps}}

% \setbeamertemplate{navigation symbols}{}






% \title[Pragmatik]{Pragmatik}
% \author[F.\ Bildhauer]{Felix Bildhauer}
% \institute[FU Berlin]{Freie Universität Berlin}
% \date{\today}


% \begin{document}

% \frame{\titlepage\thispagestyle{empty}}

% \begin{frame}
%   \tableofcontents
% \end{frame}

\toggletrue{uebung}

\section{Pragmatik}

\author{Felix Bildhauer}

\begin{frame}{Pragmatik: Material}

Bester kurzer Überblick (knapp 90 Seiten Text):
\begin{itemize}
\item Yule, George.\ 1996.\ Pragmatics.\ Oxford: Oxford University Press.
\end{itemize}


  
\end{frame}


\subsection{Allgemeines}

\subsubsection{Abgrenzung}

\begin{frame}{Rekapitulation: Ebenen linguistischer Analyse}

  \begin{itemize}
  \item Phonetik/Phonologie\\
Welche Eigenschaften haben Laute und Töne einer Sprache, welchen Regeln unterliegen sie, und welche dieser Eigenschaften dienen in einer Sprache dazu, Bedeutungen zu unterscheiden?
  \item<2-> Morphologie\\
   Welche Lautkombinationen haben eine Bedeutung und\\ nach welchen Regeln lassen sich diese zu Wörtern zusammensetzen?
  \item<3-> Syntax\\
 Nach welchen Regeln lassen sich Wörter zu Satzteilen und Satzteile zu ganzen Sätzen zusammenfügen?
  \item<4-> Semantik\\
 Welche Bedeutung haben Wörter bzw.\ Morpheme und nach welchen Regeln lässt sich die Bedeutung von Wörtern,
 Satzteilen und Sätzen aus der Bedeutung der Einzelteile (Morpheme, Wörter, andere Satzteile) erschließen?
  \end{itemize}
\end{frame}

\begin{frame}{Pragmatik}
  \begin{itemize}
  \item Pragmatik: Nicht klar umrissen. Annäherung: Pragmatik untersucht den Gebrauch von Sprache
    und die Rolle bestimmter außersprachlicher Faktoren.\\
        Auf den anderen Analyseebenen werden Konzepte wie \emph{Sprecher}, \emph{Hörer}, \emph{kommunikative Absicht}, \emph{Kontext} oder \emph{Weltwissen} normalerweise nicht berücksichtigt.
  \item<2-> Die Abgrenzung zur Semantik ist manchmal schwierig und\\
            verschiedene Forscher haben verschiedene Antworten darauf gegeben.
  \item<3-> Die Abgrenzung zu verschiedenen "`Bindestrichlinguistiken"' ist nicht immer klar umrissen (insbesondere Psycholinguistik, kognitive Linguistik, Soziolinguistik)
  \end{itemize}

\end{frame}


\begin{frame}{Woher kommt linguistische Pragmatik?}
  
Die linguistische Pragmatik hat sich u.\,a.\ aus folgenden Disziplinen heraus entwickelt:

\begin{itemize}
\item \alert{Logik}\\
  Man interessiert sich für den Wahrheitswert von Aussagen.\\
  Wie lässt sich Aussagen ein Wahrheitswert zuordnen,\\
  die z.\,B.\ deiktische Ausdrücke enthalten oder einen Wunsch ausdrücken?
\item<2-> \alert{Philosophie}\\
  Durch Sprechen verändert sich die Welt.\\
  Wie lässt sich Sprechen als Handeln beschreiben?
\item<3-> \alert{Linguistik}\\
  Es gibt mehrere Aspekte von Bedeutung. Einige davon sind veränderlich:\\
  Wie entstehen sie?\\
 Wie werden Merkmale des Kontextes sprachlich kodiert?
\end{itemize}

\end{frame}




\begin{frame}{Satz vs.\ Äußerung}

  \begin{itemize}
  \item In der Pragmatik ist oft die Rede von \textit{Äußerung} (statt \textit{Satz}).\\
  Wo liegt der Unterschied?
  \end{itemize}

\pause
\scalebox{.8}{ 
\begin{tabular}{p{.55\linewidth}|p{.55\linewidth}}
{\bf Satz} & {\bf Äußerung}\\
\hline
\visible<2->{abstrakt}  & \visible<2->{konkret (Ereignis, von einem Sprecher in einer Situation hervorgebracht)}\\
% \hline
\visible<3->{Einheit der Grammatik} & \visible<3->{Einheit des Diskurses}\\
% \hline
\visible<4->{Bedeutung abhängig von Einzelteilen und Struktur} & \visible<4->{Bedeutung abhängig von Einzelteilen,\newline Struktur und Kommunikationssituation}\\
% \hline
\visible<5->{Bewertung nach formalen Kriterien:\newline grammatisch oder ungrammatisch} & \visible<5->{Bewertung nach pragmatischen Kriterien:\newline adäquat oder inadäquat}\\
% \hline

\end{tabular}
}
\end{frame}




\begin{frame}{Satz vs.\ Äußerung (2)}
  
 Wie ist das Verhältnis von Satz und Äußerung?

\begin{itemize}[<+->]
\item Die Gleichung \alert{Äußerung = Satz + Kontext} ist etwas irreführend.
\item Einer Äußerung muss nicht unbedingt die grammatische Kategorie \textit{Satz} zugrunde liegen.
\end{itemize}

\visible<2->{
\begin{exe}
  \ex Weg da!
  \ex Schluß!
\end{exe}}

\visible<3->{
\begin{itemize}
\item  Auch ein ganzes Buch kann man als eine Äußerung auf"|fassen.
\end{itemize}
}
\end{frame}





\begin{frame}{Kommunikationssituation}
  
Welche materiellen Faktoren charakterisieren eine Kommunikationssituation?


\begin{itemize}
\item \alert{Sender}: Sprecher/in, Schreiber/in etc.
\item \alert{Empfänger}: Hörer/in, Leser/in etc.
\item \alert{Umfeld:} das "`Wann und Wo"'
\end{itemize}
\end{frame}



\begin{frame}{Kommunikationssituation (2)}

Welche nicht-materiellen Faktoren charakterisieren eine Kommunikationssituation?

\begin{itemize}
\item \alert{Pragmatische Informationen}:\\
  Das, was die Teilnehmer zu einem gegebenen Zeitpunkt wissen, glauben, vermuten, einschließlich Hypothesen darüber, was die anderen Teilnehmer wissen, glauben und vermuten.\pause
  \begin{itemize}
    \item \alert{generell}: Wissen über die Welt
    \item \alert{situational}: Wissen darüber, was man während der Interaktion wahrnimmt
    \item \alert{kontextuell}: Wissen, das sich aus dem zuvor Gesagten ableitet
  \end{itemize}\pause
\item \alert{Soziale Beziehungen}\pause
\item \alert{Absichten der Teilnehmer}

\end{itemize}


\end{frame}





\subsubsection{Überblick: Forschungsgegenstände}


\begin{frame}{Kernbereiche I: Deixis}

\begin{exe}
\ex  Der \alert{hier} ist größer als der \alert{da}.
\ex  \alert{Du} hast braune Augen und \alert{ich} habe blaue.
\ex  \alert{Heute} schneit es.
\end{exe}

\pause

\begin{itemize}
\item Manche Äußerungen enthalten deiktische Ausdrücke.\pause
\item räumliche-, zeitliche-, Personendeixis\pause
\item Deiktische Ausdrücke können nur mit Bezug auf die Äußerungssituation interpretiert werden.\pause
\item Die Bedeutung (und Wahrheit) der gesamten Äußerung hängt von der Interpretation der deiktischen Ausdrücke ab.

\bigskip

\item Gegensatz dazu: Anaphorik = Verweis auf Elemente im Text:
\ea
Ein Mann$_i$ kam herein. Er$_i$ hatte \ldots
\z

\end{itemize}


\end{frame}


 \begin{frame}{Kernbereiche II: Präsupposition}

\begin{exe}     
        \ex Der König von Deutschland hat einen weißen Bart.\\

          \visible<2->{$\to$ \alert{Es gibt einen König von Deutschland.}}

        \ex Otto bedauert, das Geheimnis verraten zu haben.\\

          \visible<2->{$\to$ \alert{Otto hat das Geheimnis verraten.}}

\end{exe}



\begin{itemize}
\item Manche Äußerungen können nur sinnvoll interpretiert werden,\\
      wenn bestimmte andere Sachverhalte wahr sind.\pause
\item Solche Sachverhalte nennt man \alert{Präsuppositionen}. \pause
\item Präsuppositionen können durch verschiedene sprachliche Ausdrücke hervorgerufen werden (bestimmte Verben, Determinierer, bestimmte Nebensätze etc.)
\end{itemize}

\end{frame}




\frame[shrink=10]{
\frametitle{Kernbereiche III: Implikatur}

\begin{exe}
\ex Einige Abgeordnete haben gegen das Gesetz gestimmt.\\

  \visible<2->{$\to$ \alert{Nicht alle Abgeordneten haben gegen das Gesetz gestimmt.}}
\ex Kommst Du heute abend mit ins Kino? -- Meine Oma ist zu Besuch!\\

    \visible<3->{$\to$ \alert{\ldots{} und deswegen bleibe ich zuhause und gehe nicht mit ins Kino.}}

% \ex Otto mag Birnen und Anna mag Äpfel.
% \begin{xlist}
% \ex  Bedeutet logisch das gleiche wie \textit{Anna mag Äpfel und Otto mag Birnen.}
% \end{xlist}
\ex Otto zog sich an und ging aus dem Haus.\\
    \visible<4->{$\to$ \alert{Otto zog sich erst an und ging dann aus dem Haus.}}\\
    \visible<4->{(`und' bedeutet hier `und dann')}

 \end{exe}

 \begin{itemize}
 \item Manchmal wird mehr mitverstanden/zu verstehen gegeben,
       als eigentlich gesagt wird.
 \visible<5->{\item Bestimmte Fälle solcher mitverstandenen Bedeutung nennt man Implikatur.}
 \end{itemize}
}




\begin{frame}{Kernbereiche IV: Sprechakte}

\begin{exe}
\ex 
  \begin{xlist}
    \ex  Du kannst ruhig mein Rad nehmen.
    \ex  Gib mir mal bitte die Zeitung rüber!
  \end{xlist}
\visible<2->{\ex
\begin{xlist}
    \ex Ich entschuldige mich für mein dummes Verhalten.
    \ex Ich verspreche dir, dass ich morgen komme.
\end{xlist}}
\visible<3->{\ex
\begin{xlist}
  \ex Hiermit taufe ich den Dampfer auf den Namen Titanic.
  \ex Ich erkläre euch hiermit zu Mann und Frau.
\end{xlist}}

% \ex
% \begin{xlist}
%     \ex Hiermit beleidige ich dich!
%     \ex Ich verspreche dir, dass die Erde morgen immer noch rund ist.
% \end{xlist}
\end{exe}

  \begin{itemize}[<+->]
  \item Manche Äußerungen lassen sich besser als geglückt/mißglückt analysieren,\\
        nicht als wahr/falsch.
  \item<2-> Viele Handlungen werden konventionell durch Sprechen durchgeführt.
  \item<3-> Manche Handlungen können überhaupt nicht anders durchgeführt werden (abhängig von der Kultur)
  %\item<3-> Man kann also  handeln, indem man spricht, und die Handlung geht über das Produzieren von Sprache hinaus. 
  %\item Damit eine Sprechhandlung glückt müssen (je nach Handlung) bestimmte zusätzliche Bedingungen erfüllt sein.
  \end{itemize}


\end{frame}




\begin{frame}{Kernbereiche V: Höflichkeit}


\begin{exe}
\visible<6->{\ex Was du sagst ist falsch.}
 %  \begin{xlist}
%     \ex<4-> \alert{Könnte} es sein, dass das falsch ist\alert{?}
%   \end{xlist}
\visible<7->{ \ex Mach das Fenster zu! }
%   \begin{xlist}
%     \ex<4-> \alert{Würde es dir etwas ausmachen}, das Fenster zu schließen\alert{?}
%     \ex<4-> \alert{Wir frieren doch beide}, \alert{könntest} du \alert{vielleicht} das Fenster schließen\alert{?}
%   \end{xlist}
\end{exe}


%      \begin{itemize}
%   \item \emph{Würde es dir etwas ausmachen, noch ein Stück Torte zu nehmen?}
%   \item \emph{Nimm dir noch ein Stück Torte!}
% \emph{Würde es dir etwas ausmachen, mir ein bißchen Geld zu leihen?}
%      \end{itemize}
 

  \begin{itemize}[<+->]
  \item Annahme: Menschen wollen
    \begin{itemize}
    \item dass andere Menschen mögen, wie man ist und was man tut.
    \item dass die eigene Handlungsfreiheit nicht eingeschränkt wird.
    \end{itemize}
 \item Sprecher respektieren gegenseitig dieses Bedürfnis.
  \item Manche Sprechakte sind aber geeignet 
    \begin{itemize}
    \item Eigenschaften und Handlungen des Adressaten zu kritisieren.
    \item die Handlungsfreiheit des Adressaten einzuschränken.
    \end{itemize}
 \end{itemize}
\end{frame}


\begin{frame}{Kernbereiche V: Höflichkeit (2)}

\begin{exe}
  \ex Was du sagst ist falsch.
  \begin{xlist}
   \ex \alert{Könnte} es sein, dass das falsch ist\alert{?}
  \end{xlist}
  \ex Mach das Fenster zu! 
  \begin{xlist}
    \ex \alert{Würde es dir etwas ausmachen}, das Fenster zu schließen\alert{?}
    \ex \alert{Wir frieren doch beide}, kannst du \alert{vielleicht} das Fenster zumachen\alert{?}
  \end{xlist}
\end{exe}


\begin{itemize}[<+->]
\item Bei potentiell bedrohlichen Sprechakten gibt es verschiedene Strategien,\\
      die Bedrohung abzufedern.
   \item Je nach Sprache/Kultur gelten unterschiedliche Sprechakte als bedrohlich.
   \item Je nach Sprache/Kultur gibt es Präferenzen für bestimmte Abfederungsstrategien. 
\end{itemize}

\end{frame}




\subsection{Zwei Forschungsbereiche der Pragmatik im Detail} 

\subsubsection{Sprechakte}


\begin{frame}{Handeln durch Sprechen}

\cite{Austin1962}: \textit{How to do things with words}.

\begin{itemize}[<+->]
\item Austin (und \citew{Searle1969}): Nicht alle Äußerungen beschreiben die Welt.\\
      Für viele Äußerungen ist wahr/falsch kein adäquates Kriterium.
\item Viele Arten von Äußerungen dienen dazu, eine Handlung durchzuführen,\\
       die über das bloße Sprechen der Wörter hinausgeht.
\item Beispiele: \emph{loben}, \emph{beleidigen}, \emph{sich beschweren}, \emph{auf"|fordern}, \emph{verbieten}, \emph{drohen}, \emph{versprechen} etc.
\item Mit solchen Äußerungen führen Sprecher \alert{Sprechakte} aus.
\end{itemize}
\end{frame}




\begin{frame}{Lokution, Illokution, Perlokution}
  
Analyse eines Sprechakts in drei Komponenten:

\begin{enumerate}
\item \alert{lokutionärer Akt}: bestimmte Laute mit einer bestimmten Bedeutung und einer bestimmten
  Referenz produzieren\\
lat. \emph{locutio} `das Reden, Redensart'; zu \emph{loqui} `reden', `sprechen'
\pause
\item \alert{illokutionärer Akt}: Die Lokution zu einem bestimmten Zweck verwenden.\\
 Das, was man tut, \emph{indem} man etwas sagt.\pause
\item \alert{perlokutionärer Akt}: Einen Effekt im Hörer auslösen.\\
von lateinisch \emph{per} `durch' und \emph{locutio} `das Sprechen'
\end{enumerate}

Im Gegensatz zu Illokutionen, die das Ergebnis einer Sprachhandlung sind und damit zeitlich mit deren Vollzug zusammenfallen, sind Perlokutionen Folgen einer Sprachhandlung, die sich an den Vollzug anschließen.

% \begin{enumerate}
% \item Lokution: Der bloße Akt des Äußern von Wörtern/Sätzen. 
% \item Illokution: Die Durchführung der damit beabsichtigten Handlung.
% \item Perlokution: Das erreichen eines bestimmten Effekts im Hörer/in der Welt.
% \end{enumerate}

\end{frame}




\begin{frame}{Lokution, Illokution, Perlokution (2)}
\begin{exe}
\ex
\begin{xlist}
\ex Otto sagte zu Anna: "`Hier, nimm eine Mohnschnecke!"'\\
      (\alert{Lokution})
\ex Otto bot Anna noch eine Mohnschnecke an.\\
       (\alert{Illokution})
\ex Otto brachte Anna dazu, eine Mohnschnecke zu nehmen.
      (\alert{Perlokution})
\end{xlist}
\end{exe}
\pause

\begin{exe}
\ex
\begin{xlist}
\ex Anna sagte zu Otto: "`Wehe Du erzählst mir das Ende des Films!"'%\\
    %  (Lokution)
\ex Anna warnte Otto davor, ihr das Ende des Films zu erzählen.%\\
     % (Illokution)
\ex Anna hielt Otto davon ab, ihr das Ende des Films zu erzählen.%\\
      % (Perlokution)
\end{xlist}
\end{exe}

\end{frame}





\begin{frame}{Illokutionäre Rolle}
  \begin{itemize}[<+->]
  \item \alert{Illokutionäre Rolle} einer Äußerung:\\ die Handlung, die mit der Äußerung durchgeführt werden soll.
  \item Der Sprecher verlässt sich darauf, dass der Adressat die illokutionäre Rolle der Äußerung erkennt/richtig interpretiert.
  \item Zur Interpretation der illokutionären Rolle müssen außersprachliche Faktoren berücksichtigt werden.
  \item Dieselbe Äußerung kann verschiedene illokutionäre Rollen haben:
\visible<4->{

  \begin{exe}
    \ex Ich komme gleich wieder.
    \begin{itemize}
     \item \alert{Versprechen}
     \item \alert{Entschuldigung}
     \item \alert{Drohung}
     \item \ldots
\end{itemize}
  \end{exe}}
  \end{itemize}


\end{frame}





\begin{frame}{Woran erkennt man die illokutionäre Rolle?}
 
  \begin{quote}
% \scalebox{.8}{
    -- ``Don't you think you'd be safer down on the ground?'' Alice went on, not with any idea of making another riddle, but simply in her good-natured anxiety for the queer creature. ``That wall is so very narrow!''\\
 -- ``What tremendously easy riddles you ask!'' Humpty Dumpty growled out. ``Of course I don't think so!''
  \end{quote}

 {\small Lewis Carroll, \textit{Through the Looking-Glass, and What Alice Found There} (1871)}


\end{frame}


\begin{frame}{Woran erkennt man die illokutionäre Rolle? (2)}

Oft wird die illokutionäre Rolle durch sprachliche Mittel angezeigt\\ (oder zumindest eingeschränkt).
\pause

\begin{itemize}
\item am offensichtlichsten durch \alert{performative Verben}:
\end{itemize}

\begin{exe}
  \ex Ich \alert{bitte} dich, jetzt zu gehen.
  \ex Ich \alert{verspreche} dir, morgen zu kommen.
  \ex Ich \alert{entschuldige} mich für mein schlechtes Benehmen.
\end{exe}
\pause


\begin{itemize}
\item Außerdem: Konstituentenstellung, Intonation, Adverbien u.\,a.
\end{itemize}
\pause

\begin{exe}
  \ex Kommst du mit? (V1)
  \ex Du kommst nicht mit? (Intonation)
  \ex Du kommst jetzt mal bitte mit. (ADV)
\end{exe}

\end{frame}


\begin{frame}{Woran erkennt man die illokutionäre Rolle? (3)}
  
\begin{itemize}
\item Manchmal wird die illokutionäre Rolle auch weniger direkt oder gar nicht mit sprachlichen Mitteln angezeigt. \pause
\item Es gibt keine 1:1 Entsprechung zwischen Satztyp und illokutionärer Rolle. \pause
\item Beispiel: Ein V1-Satz (Fragesatz) kann zum Fragen verwendet werden,\\
      aber auch, um etwas vorzuschlagen.\pause
\item Normalerweise kann man trotzdem die illukutionäre Rolle erkennen,\\
      aber siehe Humpty Dumpty.
\end{itemize}


\end{frame}

\begin{frame}{Klassifikation von Sprechakten}

Es gibt verschiedene Vorschläge zur Klassifikation.\\
 Eine der bekanntesten ist die von \cite{Searle1979}:

\begin{figure}
\begin{tabular}{p{.2\textwidth}|p{.7\textwidth}}
  \alert{Typ}   & \alert{Funktion}\\
\hline
  deklarativ    &  institutionelle Akte (verurteilen, taufen)\\
  repräsentativ & drücken aus, was jemand glaubt/denkt\\
  expressiv     & drücken aus, was jemand fühlt\\
  direktiv      & drücken aus, was jemand will (bitten, anordnen)\\
  kommissiv     & jmd.\ verpflichtet sich, etw. zu tun \\
                & (versprechen, anbieten)\\
\end{tabular}
\end{figure}

\end{frame}





\begin{frame}{Gelingensbedingungen für Sprechakte}
 \begin{quote} 
-- "`Nimm dir etwas Wein!"' sagte der Märzhase einladend. Alice spähte über den Tisch, konnte aber nur Tee entdecken.\\
-- "`Ich sehe keinen Wein!"' sagte sie.\\
-- "`Ist auch keiner da!"' antwortete der Märzhase.\\
-- "`Dann ist es unhöflich von dir, mir welchen anzubieten!"' versetzte Alice ärgerlich.\\
\end{quote}

{\small Lewis Carroll, \textit{Alice Im Wunderland} (1864)}
\end{frame}



\begin{frame}{Gelingensbedingungen  für Sprechakte (2)}

Damit ein Sprechakt "`gelingt"', müssen je nach Typ des Sprechakts bestimmte Rahmenbedingungen
erfüllt sein. 

\cite{Austin1962}, \cite{Searle1969,Searle1979} und andere:\\
Gelingensbedingungen für verschiedene Typen von Sprechakten

Beispiel "`versprechen"':

\begin{enumerate}[<+->]
\item S sagt, dass er eine Handlung ausführen wird.
\item S hat vor, diese Handlung auszuführen.
\item S ist sicher, dass er diese Handlung ausführen kann.
\item S glaubt, dass er diese Handlung nicht ohnehin ausführen würde.
\item S glaubt, dass H will, dass S die Handlung ausführt.
\item S will sich mit der Äußerung A verpflichten. 
\end{enumerate}
\end{frame}



\subsubsection{Kooperationsprinzip}



\begin{frame}{Implikaturen}

 Oft wird mehr "`gemeint"' (oder "`zu verstehen gegeben"'), als tatsächlich "`gesagt"' wird. Wie lässt sich diese zusätzliche Bedeutung erklären?
  

 \begin{exe}
  \ex Otto hat ein paar von den Weihnachtsplätzchen gegessen.\\

       $\to$ \alert{Aber nicht \textit{alle}.}

\pause

\ex Der Kirchenchor gab eine Reihe von Tönen von sich,\\
    die Bachs Weihnachtsoratorium entsprachen.\pause

$\to$ \alert{Aber \textit{singen} konnte man es nicht nennen.}

\pause
\ex Otto: Hast Du in letzter Zeit Harald gesehen?\\
    Anna: Mit Idioten gebe ich mich nicht ab.\pause


       $\to$ \alert{Harald ist ein Idiot.}\\
       $\to$ \alert{Nein, ich habe ihn in letzter Zeit nicht gesehen.}
 \end{exe}



\end{frame}








\begin{frame}{Kooperationsprinzip}

\cite{Grice1975}: \textit{Logic and Conversation}


\begin{itemize}
\item Bei Tätigkeiten, die ein gemeinsames Ziel haben,\\
      verhalten sich Menschen normalerweise kooperativ.
\pause
\item Wenn man mit einer Handlung etwas beiträgt, dann soll dieser Beitrag angemessen sein für das Erreichen des gemeinsamen Ziels.
\pause
\item Sprachliche Interaktion ist eine Tätigkeit, \\
      bei der Sprecher auf diese Weise zusammenarbeiten.
\end{itemize}

\pause
Das Kooperationsprinzip:

\begin{quote}
  Make your conversational contribution such as is required, at the stage at which it occurs, by the accepted purpose or direction of the talk exchange in which you are engaged.
\end{quote}
  
\end{frame}



\begin{frame}{Kooperationsprinzip (2)}

Warum sollte man annehmen, dass es ein Kooperationsprinzip gibt?
 \begin{itemize}
     \item Grice: Man kann beobachten, dass Menschen sich so verhalten.
  \end{itemize}
\pause

Warum verhalten sich Menschen so?
  \begin{itemize}
    \item Grice: Weil es vernünftig ist, sich so zu benehmen.
\end{itemize}
\end{frame}



\begin{frame}{Vier Konversationsmaximen}

Das globale Kooperationsprinzip lässt sich weiter verfeinern:

\begin{enumerate}
\item Maxime der \alert{Quantität}
    \begin{enumerate}
      \item Gib so viel Information, wie nötig ist.
      \item Gib nicht mehr Information, als nötig ist.
     \end{enumerate}\pause

\pause
\item Maxime der \alert{Qualität}: Dein Beitrag soll wahr sein.
  \begin{enumerate}
  \item Behaupte nichts, wovon du glaubst, dass es nicht stimmt.
  \item Behaupte nichts, wofür du keine ausreichende Evidenz hast.
  \end{enumerate}\pause

\pause
\item Maxime der \alert{Relevanz}: Was du sagst soll relevant sein.\pause

\pause
\item Maxime der \alert{Art und Weise}: Drücke dich klar aus.
  \begin{enumerate}
  \item Vermeide Unklarheiten.
  \item Vermeide Mehrdeutigkeiten.
  \item Fasse dich kurz (keine unnötige Weitschweifigkeit).
  \item Rede geordnet.
  \end{enumerate}
\end{enumerate}
\end{frame}


\begin{frame}{Implikaturen}

Wie können die Konversationsmaximen Implikaturen\\ ("`zusätzliche Bedeutungen"') erklären?

\begin{itemize}
\item Grice: Konversationelle Implikaturen entstehen dadurch,\\
      dass Sprecher voneinander annehmen, dass sie die Maximen\\
     (oder zumindest das Kooperationsprinzip als ganzes) beachten.
\end{itemize}\pause

Implikaturen können dann auf zwei verschiedene Arten zustande kommen:

\begin{enumerate}
\item Der Sprecher hält sich an die Maximen.\pause
\item Der Sprecher verletzt eine oder mehrere der Maximen,\\ und zwar ganz offensichtlich.\pause
\end{enumerate}

Dazu einige Beispiele.

\end{frame}



\begin{frame}{Quantität, nicht verletzt}

 \begin{exe}
  \ex Otto hat ein paar von den Keksen gegessen. (\alert{Aber nicht \textit{alle}.})\label{keks}
  \ex Anna hat drei Kinder. (\alert{Anna hat genau drei Kinder.})\label{kind}
\end{exe}


\begin{itemize}
\item Genau genommen ist (\ref{keks}) auch wahr, wenn Otto alle Kekse gegessen hat, und  (\ref{kind}) ist auch wahr, wenn Anna vier Kinder hat.\pause
\item Aber: Wenn sich der Sprecher an die Quantitätsmaxime hält,\\ dann gibt er so viel Information, wie nötig.\pause
\item Wenn Otto alle Kekse gegessen hätte,\\ dann würde der Sprecher mit (\ref{keks}) die Quantitätsmaxime verletzen.\pause
\item Wenn es keinen guten Grund gibt anzunehmen,\\ dass der Sprecher die Quantitätsmaxime absichtlich verletzt,\\ erhält man die Implikatur "`nicht alle"'.
\end{itemize}


\end{frame}



 \begin{frame}{Art und Weise, nicht verletzt}

 \begin{exe}
   \ex Otto zog sich an und ging aus dem Haus. (\alert{und = und danach})
 \end{exe}


   \begin{itemize}
   \item  Wenn es keinen guten Grund gibt, das Gegenteil anzunehmen,\\ geht der Hörer davon aus,\\
          dass der Sprecher die Maxime der Art und Weise beachtet.\pause
   \item Wenn jemand "`geordnet"' redet, kann man normalerweise annehmen,\\
         dass er die Dinge in der Reihenfolge erzählt, in der sie passiert sind.\pause
   \item Dadurch erhält man die Implikatur "`und danach"'.
   \end{itemize}


\end{frame}



\begin{frame}{Qualität, verletzt}
  

\begin{exe}
  \ex Anna hat tausend und ein Kind. (\alert{Anna hat viele Kinder.})\label{tausend}
\end{exe}


\begin{itemize}
\item Hier ist die Qualitätsmaxime so offensichtlich verletzt,\\
      dass es der Hörer merken muss.\pause
\item Wenn der Sprecher nur die Qualitätsmaxime verletzt,\\
      aber sich an alle anderen Maximen hält, kann der Hörer davon ausgehen,\\ dass der Sprecher generell kooperiert.\pause
\item Um beides in Einklang zu bringen, muss der Hörer eine geeignete  Interpretation für (\mex{0}) finden.
\end{itemize}


\end{frame}



\begin{frame}{Art und Weise, verletzt}

\begin{exe}
\ex Der Chor gab eine Reihe von Tönen von sich, die Bachs Weihnachtsoratorium entsprachen.\\
       \alert{Aber \textit{singen} konnte man es nicht nennen.}
\end{exe}

\begin{itemize}
\item Warum nicht einfach "`sang das Weihnachtsoratorium"'?\pause
\item Hier ist die Maxime der Art und Weise so offensichtlich verletzt,\\ dass es der Hörer merken muss.\pause
\item Wenn der Sprecher sich an alle anderen Maximen hält,\\ kann der Hörer davon ausgehen, dass der Sprecher generell kooperiert.\pause
\item Um beides in Einklang zu bringen, muss der Hörer eine geeignete  Interpretation für (\mex{0}) finden.\pause
\item Eine nahe liegende Interpretation ist: \textit{singen} konnte man es nicht nennen.
\end{itemize}
\end{frame}

%%%%%%%%%%%%%%%%%%%%%%%%%%%%%%%%%%%%%%%%%%%
\subsection{Übungen}
%%%%%%%%%%%%%%%%%%%%%%%%%%%%%%%%%%%%%%%%%%%
\iftoggle{uebung}{
	
	\begin{frame}[shrink=5]
	\frametitle{Übungen}
	
	\begin{enumerate}
		\item Gegeben sei der Satz unter (\ref{ex:Prag42}):
		\ea \label{ex:Prag42}
		Einige der US-amerikanischen Beamten wissen, wer Richard erdrosselt hat.
		\z
		\item [] Geben Sie bei jedem der Sätze unter (\ref{ex:Prag43})--(\ref{ex:Prag46}) an, ob es sich um eine Implikatur, oder ob es sich um eine Präsupposition zu (1) handelt. Schreiben Sie die richtige Antwort hinter den jeweiligen Satz. Wenn es sich um eine Präsupposition handelt, testen Sie dies anhand eines der Präsuppositionstests. \\
		\textbf{NB:} Vorsicht, zuweilen wird keine der Relationen wiedergegeben!
		\ea\label{ex:Prag43} Es existieren US-amerikanische Beamte.
		\ex\label{ex:Prag44} Richard war ein Semantiker.
		\ex\label{ex:Prag45} Nicht alle US"=amerikanischen Beamten wissen, wer den Mord begangen hat.
		\ex\label{ex:Prag46} Richard wurde erdrosselt.
		\z	
	\end{enumerate}
	
\end{frame}

%%%%%%%%%%%%%%%%%%%%%%%%%%%%%%%%%%%%%%%%%%%

\begin{frame}
\begin{enumerate}
	\item[2.] Bestimmen und kennzeichnen Sie zwei deiktische Ausdrücke im Satz (\ref{ex:Prag47}). Geben Sie zudem eine Anapher mit ihrem Antezendens an.
	\ea\label{ex:Prag47} Angelika hat gestern erwähnt, dass Irene sich dort mit den Formeln amüsiert hat.
	\z 
	\item[3.] Kreuzen Sie für Satz (\ref{ex:Prag48}) alle Sätze in der unten stehenden Liste an,\\
                  die (konversationelle) Implikaturen dieses Satzes darstellen.
	\ea \label{ex:Prag48} Gottfried hat einige Nachbarn beleidigt.
	\z 
	\begin{itemize}
		\item[$\circ$] Gottfried hat einen Nachbarn.
		\item[$\circ$] Gottfried hat nicht alle Nachbarn beleidigt.
		\item[$\circ$] Gottfried ist ein unbeliebter Mensch.
		\item[$\circ$] Gottfried hat etwas Unhöfliches gesagt.
		\item[$\circ$] Gottfried hat einige Nachbarn nicht beleidigt.
	\end{itemize}
\end{enumerate}
\end{frame}

%%%%%%%%%%%%%%%%%%%%%%%%%%%%%%%%%%%%%%%

\iftoggle{ue-loesung}{
	
	%%%%%%%%%%%%%%%%%%%%%%%%%%%%%%%%%%
%% UE 2 - 08 Pragmatik
%%%%%%%%%%%%%%%%%%%%%%%%%%%%%%%%%%

\begin{frame}
\frametitle{Übung -- Lösung}
	
	\begin{enumerate}
		\item Gegeben sei der Satz unter (\ref{ex:08ue1}):
		
	\begin{exe}
		\exr{ex:08ue1} Einige der US-amerikanischen Beamten wissen, wer Richard erdrosselt hat.
	\end{exe}

		\item [] Geben Sie bei jedem der Sätze unter (\ref{ex:08ue2})--(\ref{ex:08ue5}) an, ob es sich um eine Implikatur, oder ob es sich um eine Präsupposition zu (\ref{ex:08ue1}) handelt. Schreiben Sie die richtige Antwort hinter den jeweiligen Satz. Wenn es sich um eine Präsupposition handelt, testen Sie dies anhand eines der Präsuppositionstests. \\
		\textbf{NB:} Vorsicht, zuweilen wird keine der Relationen wiedergegeben!
	\begin{exe}
		\exr{ex:08ue2} Es existieren US-amerikanische Beamte. \textcolor{red}{\ras Präsupposition}
		\exr{ex:08ue3} Richard war ein Semantiker. \textcolor{red}{\ras weder noch}
		\exr{ex:08ue4} Nicht alle US"=amerikanischen Beamten wissen, wer den Mord begangen hat. \textcolor{red}{\ras Implikatur}
		\exr{ex:08ue5} Richard wurde erdrosselt. \textcolor{red}{\ras Präsupposition}
	\end{exe}

	\end{enumerate}
	
\end{frame}

%%%%%%%%%%%%%%%%%%%%%%%%%%%%%%%%%%%%%%%%%%%

\begin{frame}
	\frametitle{Übung -- Lösung}
	
	\begin{enumerate}
		\item[2.] Bestimmen und kennzeichnen Sie zwei deiktische Ausdrücke im Satz (\ref{ex:08ue6}). Geben Sie zudem eine Anapher mit ihrem Antezendens an.
	
	\begin{exe}
		\exr{ex:08ue6} Angelika hat gestern erwähnt, dass Irene sich dort mit den Formeln amüsiert hat.
	\end{exe}
		
		\begin{itemize}
			\item[] \textcolor{red}{Ausdruck: \emph{gestern}, Art: Temporaldeixis}
			\item[] \textcolor{red}{Ausdruck: \emph{dort}, Art: Lokaldeixis}
			\item[] \textcolor{red}{Anapher: \emph{sich}, Antezedens: \emph{Irene}}
			\item[] \textcolor{red}{Auch möglich: \emph{den Formeln} \ras Objektdeixis}
		\end{itemize}
		
	\end{enumerate}
	
\end{frame}

%%%%%%%%%%%%%%%%%%%%%%%%%%%%%%%%%%%%%%%%%%%%

\begin{frame}
	\frametitle{Übung -- Lösung}
	
	\begin{enumerate}
		\item[3.] Kreuzen Sie für Satz (\ref{ex:08ue7}) alle Sätze in der unten stehenden Liste an, die (konversationelle) Implikaturen dieses Satzes darstellen.
		
	\begin{exe}
		\exr{ex:08ue7} Gottfried hat einige Nachbarn beleidigt.
	\end{exe}

		\begin{itemize}
			\item[$\circ$] Gottfried hat einen Nachbarn.
			\item[\textcolor{red}{$\checkmark$}] \textcolor{red}{Gottfried hat nicht alle Nachbarn beleidigt.}
			\item[$\circ$] Gottfried ist ein unbeliebter Mensch.
			\item[\textcolor{red}{$\checkmark$}] \textcolor{red}{Gottfried hat etwas Unhöfliches gesagt.}
			\item[\textcolor{red}{$\checkmark$}] \textcolor{red}{Gottfried hat einige Nachbarn nicht beleidigt.}
		\end{itemize}
		
	\end{enumerate}


\end{frame}
	
}

%%%%%%%%%%%%%%%%%%%%%%%%%%%%%%%%%%%%%%%


}



\nocite{Levinson1983}






