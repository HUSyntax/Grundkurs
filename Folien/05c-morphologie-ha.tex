%% -*- coding:utf-8 -*-
\begin{frame}
	\frametitle{Hausaufgaben}

	
\begin{enumerate}
	 \item Kreuzen Sie die korrekten Aussagen an\hfill(0,5 Punkte pro Aussage)\\

\begin{itemize}
	\item[$\circ$] Die Graphemkette \emph{abarbeiten} ist ein einzelnes phonologisches Wort im Deutschen.
	\item[$\circ$] \emph{Morphologieeinführungsbuch} ist ein orthographisch-graphemisches Wort des Deutschen, sowie \emph{introductory morphology book} ein orthographisch-graphemisches Wort des Englischen ist.
	\item[$\circ$] Ein Morphem ist die kleinste bedeutungsunterscheidende Einheit in einem bestimmten Sprachsystem.
	\item[$\circ$] \ab{Brot} und \ab{Bröt} sind Allomorphe eines einzelnen Morphems.
\end{itemize}

	\item Erklären Sie das Prinzip der Rechtsköpfigkeit in der Morphologie des Deutschen. Verwenden Sie bei Ihrer Erklärung die unten angegebenen Beispiele.\hfill(4 Punkte)\\

\begin{itemize}
	\item[i.] lichtblau, Blaulicht
	\item[ii.] die Fotowelt, das Weltfoto
	\item[iii.] die Bücherrücken, die Rückenbücher
\end{itemize}
\end{enumerate}

\end{frame}



\begin{frame}
\frametitle{Hausaufgaben}
\begin{itemize}
	\item[3.] Geben Sie Argumente für oder gegen die Behandlung von \emph{ver-} in den folgenden Wörtern als Morphem an. Wenn es sich um ein Morphem handelt, ist das immer das gleiche Morphem? (4 Punkte)

	\begin{itemize}
		\item[i.] \emph{Ver}zweiflung
		\item[ii.] \emph{Ver}s
		\item[iii.] \emph{ver}kaufen
		\item[iv.] \emph{ver}schreiben
		\item[v.] \emph{ver}fahren
	\end{itemize}

\end{itemize}
\end{frame}



\begin{frame}
	\frametitle{Hausaufgaben}
	
\begin{itemize}
\item[4.] Ordnen Sie die Wortbildungsprozesse links den passenden Beispielen rechts zu (dazu müssen Sie nur den entsprechenden Buchstaben neben das passende Beispiel schreiben). (0,5 Punkte pro Aussage)
\end{itemize}

\begin{table}[h!]
	\begin{minipage}{0.4\linewidth}
		\centering
		\begin{tabular}{l|p{0.1\textwidth}|}
			Determinativkompositum & (A)\\
			\hline
			Konversion & (B)\\
			\hline
			Zirkumfigierung (Derivation) & (C)\\
			\hline
			Rektionskompositum & (D)\\
			\hline
			Possessivkompositum & (E)\\
		\end{tabular}
		
	\end{minipage}\hfill%
	\begin{minipage}{0.4\linewidth}
		\centering
		\begin{tabular}{|p{0.1\textwidth}|r}
			& \emph{Gerede} \\
			\hline
			& \emph{Milchgesicht}\\
			\hline
			& \emph{Lauf} \\
			\hline
			& \emph{Kettenraucher}  \\
			\hline
			& \emph{Klausurbesprechung}  \\
		\end{tabular}
	\end{minipage}
\end{table}
\end{frame}



\begin{frame}
	\frametitle{Hausaufgaben}
\begin{itemize}
\item[5.] Warum sind die Wörter unter (i.) grammatisch und die unter (ii.) ungrammatisch?(4 Punkte)
	\begin{itemize}
		\item[i.] kaufbar, trinkbar
		\item[ii.] *fensterbar, *helfbar, *schönbar
	\end{itemize}

      \item [6.] Sind die folgenden Verben Präfixverben oder Partikelverben? Begründen Sie Ihre Entscheidungen. (3 Punkte)

	\begin{itemize}
		\item[i.] auskennen
		\item[ii.] erkennen
		\item[iii.] aberkennen
	\end{itemize}

\item [7.] Geben Sie für das folgende Wort eine morphologische Konstituentenstruktur (inklusive Konstituentenkategorien (N, N\textsuperscript{af}, V, V\textsuperscript{af}, \dots)) an, und bestimmen Sie für jeden Knoten den Wortbildungstyp. (6,5 Punkte)

	\begin{itemize}
		\item[i.] Wahlkampfberaterinnen
	\end{itemize}

\end{itemize}

\end{frame}



\begin{frame}
	\frametitle{Hausaufgaben}
	
\begin{itemize}

\item [8.] Paraphrasieren Sie das folgende komplexe Wort so, dass es der angegebenen Struktur entspricht (auch wenn Sie selbst eine andere Struktur plausibler finden sollten). (2 Punkte)

\begin{forest}sn edges,
	[N
	[N[N[Reserve]]
	[N[V[lehr]][N\textsuperscript{af}[-er]]]]
	[N[zimmer]]
	]
\end{forest}

\end{itemize}

\end{frame}


\begin{frame}
	\frametitle{Hausaufgaben}
	
\begin{itemize}

\item [9.] Geben Sie für die folgende Wortform die Flexionskategorien an, nach denen sie flektiert ist.\\
\hfill(3 Punkte)\\
	\begin{itemize}
		\item[i.] bestehe
	\end{itemize}

\end{itemize}

\end{frame}
