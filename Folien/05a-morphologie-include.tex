%%%%%%%%%%%%%%%%%%%%%%%%%%%%%%%%%%%%%%%%%%%%%%%%
%% Compile the master file!
%% 		Include: Antonio Machicao y Priemer
%% 		Course: GK Linguistik
%%%%%%%%%%%%%%%%%%%%%%%%%%%%%%%%%%%%%%%%%%%%%%%%


%%%%%%%%%%%%%%%%%%%%%%%%%%%%%%%%%%%%%%%%%%%%%%%%%%%%
%%%             Metadata                         
%%%%%%%%%%%%%%%%%%%%%%%%%%%%%%%%%%%%%%%%%%%%%%%%%%%%      

\title{Grundkurs Linguistik}

\subtitle{Morphologie I: Einführung \& Begriffe}

\author[A. Machicao y Priemer, S. Müller, A. Lüdeling]{
	{\small Antonio Machicao y Priemer}
	\\
	{\footnotesize \url{http://www.linguistik.hu-berlin.de/staff/amyp}}
	%	\\
	%	{\small\href{mailto:mapriema@hu-berlin.de}{mapriema@hu-berlin.de}}
}

\institute{Institut für deutsche Sprache und Linguistik}

% bitte lassen, sonst kann man nicht sehen, von wann die PDF-Datei ist.
%\date{ }

%\publishers{\textbf{6. linguistischer Methodenworkshop \\ Humboldt-Universität zu Berlin}}

%\hyphenation{nobreak}


%%%%%%%%%%%%%%%%%%%%%%%%%%%%%%%%%%%%%%%%%%%%%%%%%%%%
%%%             Preamble's End                  
%%%%%%%%%%%%%%%%%%%%%%%%%%%%%%%%%%%%%%%%%%%%%%%%%%%%      


%%%%%%%%%%%%%%%%%%%%%%%%%
\huberlintitlepage[22pt]
\iftoggle{toc}{
	\frame{
		\begin{multicols}{2}
			\frametitle{Inhaltsverzeichnis}
			\tableofcontents
			%[pausesections]
		\end{multicols}
	}
}

%%%%%%%%%%%%%%%%%%%%%%%%%%%%%%%%%%
%%%%%%%%%%%%%%%%%%%%%%%%%%%%%%%%%%
%%%%%LITERATURE:

%% Allgemein
\nocite{Glueck&Roedel16a}
\nocite{Schierholz&Co18}
\nocite{Luedeling2009a}
\nocite{Meibauer&Co07a} 
\nocite{Repp&Co15a} 

%%% Sprache & Sprachwissenschaft
%\nocite{Fries16c} %Adäquatheit
%\nocite{Fries16a} %Grammatikalität
%\nocite{Fries&MyP16c} %GG
%\nocite{Fries&MyP16b} %Akzeptabilität
%\nocite{Fries&MyP16d} %Kompetenz vs. Performanz

%%% Phonetik & Phonologie
%\nocite{Altmann&Co07a}
%\nocite{DudenAussprache00a}
%\nocite{Hall00a} 
%\nocite{Kohler99a}
%\nocite{Krech&Co09a}
%\nocite{Pompino95a}
%\nocite{Ramers08a}
%\nocite{Ramers&Vater92a}
%\nocite{Rues&Co07a}
%\nocite{WieseR96a}
%\nocite{WieseR11a}

%%% Graphematik
%\nocite{Altmann&Co07a}
%\nocite{Duerscheid04a}
%\nocite{Eisenberg00a}
%\nocite{Fuhrhop08a}
%\nocite{Fuhrhop09a}
%\nocite{Fuhrhop&Co13a}

%% Morphologie
\nocite{Bauer00a} %Word
\nocite{Eisenberg00a}
\nocite{Fleischer00a} %Wortbildungsprozesse
\nocite{Fleischer&Barz12a} %Einführung Morphologie
\nocite{Grewendorf&Co91a} %Betonung bei Komposita
\nocite{Haspelmath2002a}
\nocite{Plungian00a} %Morphologie im Sprachsystem
\nocite{Salmon00a} %Term Morphology
\nocite{Wurzel00a} %Gegenstand Morphologie
\nocite{Wurzel00b} %Wort



%%%%%%%%%%%%%%%%%%%%%%%%%%%%%%%%%%%
%%%%%%%%%%%%%%%%%%%%%%%%%%%%%%%%%%%
\section{Morphologie I}
%%%%%%%%%%%%%%%%%%%%%%%%%%%%%%%%%%%

\begin{frame}
\frametitle{Begleitlektüre}

\begin{itemize}
	\item \citet[35--40]{Abramowski2016} 
	\item Lüdeling
\end{itemize}

\end{frame}


%%%%%%%%%%%%%%%%%%%%%%%%%%%%%%%%%%
%%%%%%%%%%%%%%%%%%%%%%%%%%%%%%%%%%
\subsection{Einführung}
\iftoggle{toc}{
\frame{
\begin{multicols}{2}
\frametitle{~}
	\tableofcontents[currentsection]
\end{multicols}
}
}


%%%%%%%%%%%%%%%%%%%%%%%%%%%%%%%%%%
\begin{frame}
\frametitle{Einführung}

\begin{itemize}
	\item Morphologie: \textbf{Formenlehre} in der Biologie \citep[vgl.][]{Salmon00a, Wurzel00a}\\
	(griech. \emph{morphe}: \gq{Form, Gestalt}; \emph{logos} \gq{Sinn, Lehre}

	\item Goethe (1796): Bezeichnung der \textbf{Lehre von Form und Struktur lebender Organismen}.
	
	\item August Schleicher (1859): Übernahme in die Sprachwissenschaft zur Bezeichnung von \textbf{Wörtern als Untersuchungsgegenstand}

\end{itemize}

\pause 

\begin{block}{Morphologie}
	Linguistische Disziplin, die sich mit der \textbf{Struktur} und dem \textbf{Aufbau} von \textbf{Wörtern} und mit \textbf{Theorien} von komplexen Wörtern (Produktivität, Schnittstellen zu Phonologie, Syntax, Semantik) befasst.
\end{block}

\settowidth\jamwidth{(Luedeling, 2009)} 
\ea Brunnenkressesüppchens \jambox{\citep{Luedeling2009a}}

\pause 

{[[[[Brunnen $+$ kresse] $+$ süpp] $+$ -chen] $+$ s]}
\z 

\end{frame}


%%%%%%%%%%%%%%%%%%%%%%%%%%%%%%%%%%
\begin{frame}
\frametitle{Unterteilung der Morphologie}

\begin{itemize}
	\item Morphologie unterteilt sich in:
	
	\begin{itemize}
		\item \textbf{Wort}bildung: Ableitung und Zusammensetzung lexikalischer Wörter (Lemmata)
		
		\ea {[[[Brunnen $+$ kresse] $+$ süpp] $+$ -chen]}		
		\z 
		
		\item \textbf{Wortformen}bildung (\textbf{Flexion}): grammatische Wortformveränderungen
		
		\ea {[Brunnenkressesüppchen] $+$ s}
		\z 
	\end{itemize}
\end{itemize}

\begin{figure}	
\centering
\scalebox{0.6}{
\begin{forest} %% how to scale forest?
%sm edges,
	[\textbf{Morphologie}
		[\alertblue{Flexion}
			[\alertred{Konjugation}] 
			[\alertred{Deklination}]]
		[\alertblue{Wortbildung}
			[\alertred{Komposition}
				[Determinativ]
				[Kopulativ]
				[\dots]]
			[\alertred{Derivation}
				[Suffigierung]
				[Präfigierung]
				[\dots]]
			[\alertred{Konversion}
				[morphologisch]
				[syntaktisch]]
			[\alertred{Rückbildung}]
			[\dots]]]										
\end{forest}}
\end{figure}


\end{frame}


%%%%%%%%%%%%%%%%%%%%%%%%%%%%%%%%%%
%%%%%%%%%%%%%%%%%%%%%%%%%%%%%%%%%%
\subsection{Wortbegriff}
\iftoggle{toc}{
\frame{
\begin{multicols}{2}
\frametitle{~}
	\tableofcontents[currentsection]
\end{multicols}
}
}

%%%%%%%%%%%%%%%%%%%%%%%%%%%%%%%%%%
\begin{frame}
\frametitle{Wortbegriff}

\begin{block}{Wort}
\textbf{Intuitiv} vorgegebener und \textbf{umgangssprachlich} verwendeter
Begriff für \textbf{sprachliche Grundeinheiten}. Seine Definition ist
\textbf{uneinheitlich} und \textbf{kontrovers}. \citep[vgl.][]{Bussmann2002a, Glueck&Roedel16a}
\end{block}

\pause 

\begin{itemize}
	\item Wörter werden auf verschiedenen Ebenen unterschiedlich definiert.
	
	\begin{itemize}
		\item phonetisch-phonologisches Wort
		\item orthographisch-graphemisches Wort
		\item morphologisches Wort
		\item flektivisches Wort
		\item leixikalisch-semantisches Wort
		\item syntaktisches Wort
	\end{itemize}

\item Je nach Ebene gibt es eine \textbf{unterschiedliche Menge} von \gqq{Wörtern}.
\end{itemize}

\end{frame}


%%%%%%%%%%%%%%%%%%%%%%%%%%%%%%%%%%
%%%%%%%%%%%%%%%%%%%%%%%%%%%%%%%%%%
\subsubsection{Phonetisch-phonologisches Wort}
%\frame{
%\frametitle{~}
%	\tableofcontents[currentsection]
%}


%%%%%%%%%%%%%%%%%%%%%%%%%%%%%%%%%%
\begin{frame}
\frametitle{Phonetisch-phonologisches Wort}

\begin{itemize}
	\item kleinsten durch \textbf{Wortakzent} und \textbf{Grenzsignale} (Pause, Knacklaut) theoretisch isolierbare Lautsegmente
	\item Es stimmt nicht immer mit dem orthographisch-graphemischen Wort überein.

\pause 
	
	\item Viele \textbf{phonologische Prozesse} haben das phonologische Wort als Domäne:
	
	\begin{itemize}
		\item Die \textbf{Silbifizierung} erfolgt nur innerhalb des phonologischen Wortes.
		
		\ea kindlich \vs kindisch \ab{-lich} ist ein phonolog. Wort, aber \ab{-isch} nicht.
		\z 

\pause 

		\item \textbf{Assimilationsprozesse} sind nur innerhalb des phonolog. Wortes obligatorisch.
		
		\ea ungern \vs Bearbeitung \ab{un-} ist ein phonolog. Wort
		\z 

\pause 
		
		\item \textbf{Vokalharmonie} (\zB im Türkischen) erfolgt nur innerhalb eines phonolog. Wortes.
	\end{itemize}
\end{itemize}

\end{frame}


%%%%%%%%%%%%%%%%%%%%%%%%%%%%%%%%%%
%%%%%%%%%%%%%%%%%%%%%%%%%%%%%%%%%%
\subsubsection{Orthographisch-graphemisches Wort}
%\frame{
%\frametitle{~}
%	\tableofcontents[currentsection]
%}


%%%%%%%%%%%%%%%%%%%%%%%%%%%%%%%%%%
\begin{frame}
\frametitle{Orthographisch-graphemisches Wort}

\begin{itemize}
	\item Buchstabensequenz zwischen zwei \textbf{Leerzeichen} (Spatien) oder zwischen einem \textbf{Leerzeichen} und einem \textbf{Satzzeichen}

\pause 

	\item Es enthält selbst \textbf{kein Leerzeichen}.
	
	\ea Hör auf! \vs Aufhören!
	\ex New York \vs Berlin
	\z 

	\item Definition ist \textbf{sprachspezifisch}:
	
	\ea Sommerschule \vs summer school
	\ex Morphologieeinführungsbuch \vs introductory morphology book
	\z 
	
\pause 

	\item Seit der letzten großen Rechtschreibreform gibt es im Deutschen \textbf{weniger orth.-
graph. Wörter} (obwohl der \textbf{Wortstatus} dieser Buchstabensequenzen sich nicht verändert hat!)
	
		\ea \ab{radfahren} \ras \ab{Rad fahren} 
		\z 
		
\end{itemize}

\end{frame}


%%%%%%%%%%%%%%%%%%%%%%%%%%%%%%%%%%
\begin{frame}
\frametitle{Orthographisch-graphemisches Wort}

\begin{itemize}
	\item Definition gilt nur für Sprachen mit \textbf{alphabetischem Schriftsystem}.

	\ea 
%\begin{CJK*}{UTF8}{gbsn} % ist jetzt gefixt (SuSE 11.1)
近年来,``应用语言学''作为语言学的一个分支,在国内外都得到了较大的发展,但对于``什么是应用语言学'',
``应用语言学包括哪些研究领域'' 等最基本的问题,学者们却始终没有一个统一的看法。对于一门发展中的、涉及内容广泛的学科而言这是正常的,但长期下去,又会对学科的发展产生不利影响。
%\end{CJK*}
	\z 

\pause 

\begin{itemize}
	\item Chinesische Wörter können aus einem oder mehreren Symbolen bestehen.
	
	\item Texte werden von oben nach unten geschrieben.
	
	\item Auf Computern von links nach rechts.
	
	\item Es gibt \textbf{keine Leerzeichen} zwischen Wörtern.
\end{itemize}

\pause

	\item Es gibt \textbf{Sprachen ohne Schriftsystem}, \dash ohne orth.-graph. Wörter.
\end{itemize}

\end{frame}


%%%%%%%%%%%%%%%%%%%%%%%%%%%%%%%%%%
%%%%%%%%%%%%%%%%%%%%%%%%%%%%%%%%%%
\subsubsection{Morphologisches Wort}
%\frame{
%\frametitle{~}
%	\tableofcontents[currentsection]
%}


%%%%%%%%%%%%%%%%%%%%%%%%%%%%%%%%%%
\begin{frame}
\frametitle{Morphologisches Wort}

\begin{itemize}
	\item \textbf{strukturell stabile} (und \textbf{nicht trennbare}) Grundeinheit eines grammatischen \textbf{Paradigmas} (auch \textbf{lexikalisches Wort} oder \textbf{Lexem} genannt)

	\ea 
		\ea \alertred{schreib}(-en): schreibe, schreibst, schrieb, geschrieben, \ldots 
		\ex \alertred{Tisch}: Tisches, Tische, Tischen
		\z 
	\z 

\pause 
	
	\item Sie können \textbf{morphologisch einfach} oder \textbf{komplex} sein.
	
	\ea Tisch, Tischbein, Hals-Nasen-Ohren-Arzt
	\z 

\pause 
	
	\item Komplexe morph. Wörter sind durch spezifische \textbf{Regeln der Wortbildung} zu beschreiben.
	
	\ea Tischbein $=$ Tisch $+$ Bein (Komposition)
	\z 
\end{itemize}

\end{frame}


%%%%%%%%%%%%%%%%%%%%%%%%%%%%%%%%%%
\begin{frame}
\frametitle{Morphologisches Wort}

\begin{itemize}
	\item \textbf{Nichttrennbarkeitskriterium} ist problematisch für Partikelverben:
	
	\ea umfahr(-en), mitkomm(-en), anruf(-en)
	\z 
	
	\item \textbf{Paradigmenbildung} ist problematisch für nicht-flektierbare Wortarten (Präpositionen, Partikeln, Subjuktionen, \ldots)
	
	\ea auf, ja, wohl, dass
	\z 
	
	\item Das morphologische Wort (\textbf{Lexem}) gilt als Basiseinheit des Lexikons.
	
	\item \textbf{Lexeme} sind die lexikalischen Einheiten der Sprache.
\end{itemize}

\end{frame}


%%%%%%%%%%%%%%%%%%%%%%%%%%%%%%%%%%
\begin{frame}
\frametitle{Morphologisches \vs flektivisches Wort}

\begin{itemize}
	\item Das \textbf{morphologische Wort} sollte von dem \textbf{flektivischen Wort} (Wortform) unterschieden werden.
	
	\item Das \textbf{morphologische Wort} ist die Grundeinheit eines Paradigmas.
\end{itemize}

	\begin{block}{Paradigma}
		alle vorkommenden Wortformen eines Lexems
	\end{block}
	
%	\item 
	Die \textbf{flektivischen Wörter} sind die \textbf{verschiedenen Realisierungen} eines \textbf{morphologischen Wortes}. Sie sind \textbf{hinsichtlich grammatischer Kategorien} wie
	Tempus, Person, Numerus, Kasus, \dots\ spezifiziert.
	
	\ea 
	\ea flektivische Wörter von \emph{Bank} (\gq{Geldinstitut}): Bank, Banken
	\ex flektivische Wörter von \emph{Bank} (\gq{Sitzgelegenheit}): Bank, Bänke, Bänken
	\ex flektivische Wörter von \emph{verkauf}(\emph{-en}): verkaufe, verkaufte, verkaufest, \dots
	\z    
	\z

\end{frame}


%%%%%%%%%%%%%%%%%%%%%%%%%%%%%%%%%%
\begin{frame}
\frametitle{Morphologisches \vs flektivisches Wort}

\begin{itemize}
	\item Die morphologischen Wörter (\textbf{Lexeme}) sind die lexikalischen Einheiten der Sprache.
	
	\item Um auf Lexeme zu referieren verwendet man häufig eine \textbf{Zitierform}.
\end{itemize}
	
	\begin{block}{Zitierform (Lemma)}
		\textbf{konventionell festgelegte} Form eines Paradigmas, die stellvertretend für das gesamte Paradigma steht.
		
		Im Deutschen bei \textbf{Nomina} \ras Nominativ Singular
		
		Im Deutschen bei \textbf{Verben} \ras Infinitiv
		
	\end{block} 

\begin{itemize}
	\item Zitierform von Verben ist eine \textbf{komplexe Wortform} (\zB \emph{verlieben}):
	
	\ea
	\gll verlieb- {} -en\\
	{\footnotesize morph. Wort (Imperativform)} $+$ {\footnotesize gebundenes Morphem}\\
	
	\z 
\end{itemize}
\end{frame}


%%%%%%%%%%%%%%%%%%%%%%%%%%%%%%%%%
%%%%%%%%%%%%%%%%%%%%%%%%%%%%%%%%%
\subsubsection{Syntaktisches Wort}
%\frame{
%\frametitle{~}
%	\tableofcontents[currentsection]
%}


%%%%%%%%%%%%%%%%%%%%%%%%%%%%%%%%%%
\begin{frame}
\frametitle{Syntaktisches Wort}

\begin{itemize}
	\item \textbf{kleinste} \textbf{verschiebbare} und \textbf{ersetzbare} Einheit eines \textbf{Satzes}
	(Problem: Artikel, manche Präpositionen, Partikelverben)
	
	\ea
		\ea[]{Wir bauen \alertred{Häuser}.}
		\ex[]{\alertred{Häuser} bauen wir.}
		\ex[*]{\alertred{Ein} bauten wir \alertred{Haus}.}
		\z 
	\ex 
		\ea[]{weil die Schule gestern \alertred{angefangen} hat.}
		\ex[]{\alertred{Angefangen} hat die Schule noch nicht.}
		\ex[]{Die Schule \alertred{fing} gestern \alertred{an}.}
		\z 
	\z
	
%	\item Auch definiert als die kleinste Einheit, die alleine als Satz
%	vorkommen kann.
%	
%	\eal
%	\ex Heißt es \gqq{ein} oder \gqq{eine} Hund?
%	\ex \gqq{Ein}
%	\zl

\end{itemize}

\end{frame}


%%%%%%%%%%%%%%%%%%%%%%%%%%%%%%%%%%
%%%%%%%%%%%%%%%%%%%%%%%%%%%%%%%%%%
\subsubsection{Lexikalisch-semantisches Wort}
%\frame{
%\frametitle{~}
%	\tableofcontents[currentsection]
%}


%%%%%%%%%%%%%%%%%%%%%%%%%%%%%%%%%%
\begin{frame}
\frametitle{Lexikalisch-semantisches Wort}

\begin{itemize}
	\item \textbf{kleinste} im Lexikon kodifizierte Einheit, der eine \textbf{Bedeutung} (vgl.\ (\ref{ex:LexWort1})) oder eine syntaktische/pragmatische Funktion (vgl.\ (\ref{ex:LexWort2})) zugeordnet werden kann.
	
	\ea 
		\ea\label{ex:LexWort1} Tisch, Tischbein
		\ex\label{ex:LexWort2} der, ja, dass, und 
		\z 
	\z 

	\item Problem: unikale Elemente
	
	\ea 
		\ea \alertred{klipp} und klar
		\ex auf \alertred{Anhieb}
		\z 
	\z 
	
%	\item Problem: \emph{der US-amerikanische Präsident}

\end{itemize}

\end{frame}


%%%%%%%%%%%%%%%%%%%%%%%%%%%%%%%%%%
%%%%%%%%%%%%%%%%%%%%%%%%%%%%%%%%%%
\subsubsection{Wort: Hauptkriterien}
%\frame{
%\frametitle{~}
%	\tableofcontents[currentsection]
%}


%%%%%%%%%%%%%%%%%%%%%%%%%%%%%%%%%%
\begin{frame}
\frametitle{Wort: Hauptkriterien}

\begin{itemize}
	\item akustische und semantische Identität,
	\item morphologische Stabilität und
	\item syntaktische Mobilität
	\item[]
	\item Jede Wortdefinition liefert eine \textbf{unterschiedliche Menge} von potentiellen \gqq{Wörtern}. 
	
	\item Jede linguistische Subdisziplin arbeitet mit ihrer eigenen Definition. 
	
	I.\,d.\,R.\ wird der Wortbegriff vermieden.

\end{itemize}

\end{frame}


%%%%%%%%%%%%%%%%%%%%%%%%%%%%%%%%%%
%%%%%%%%%%%%%%%%%%%%%%%%%%%%%%%%%%
\subsection{Morphologische Grundbegriffe}
\iftoggle{toc}{
\frame{
\begin{multicols}{2}
\frametitle{~}
	\tableofcontents[currentsection]
\end{multicols}
}
}

%%%%%%%%%%%%%%%%%%%%%%%%%%%%%%%%%%
%%%%%%%%%%%%%%%%%%%%%%%%%%%%%%%%%%
\subsubsection{Morph, Morphem, Allomorph}
%\frame{
%\frametitle{Morphologische Grundbegriffe}
%	\tableofcontents[currentsection]
%}


%%%%%%%%%%%%%%%%%%%%%%%%%%%%%%%%%%
\begin{frame}
\frametitle{Morph, Morphem, Allomorph}

\begin{block}{Morphem (strukturalistische Definition)}
	\textbf{kleinste} \textbf{bedeutungstragende} Einheit in einem Sprachsystem
\end{block}

\pause

\begin{block}{Morphem \citep[38]{Wurzel84x}}
	Ein Morphem ist die \textbf{kleinste}, in ihren \textbf{verschiedenen Vorkommen} als formal \textbf{einheitlich identifizierbare Folge von Segmenten}, der (wenigstens) eine als einheitlich identifizierbare \textbf{außerphonologische Eigenschaft} zugeordnet ist.
\end{block}

\end{frame}


%%%%%%%%%%%%%%%%%%%%%%%%%%%%%%%%%%
\begin{frame}
\frametitle{Morph, Morphem, Allomorph}

\begin{itemize}
	\item \textbf{außerphonologische Eigenschaften}: \alertred{grammatische} (\zB Kasus, Numerus) und/oder \alertblue{lexikalische} Bedeutung
	
	\ea
		\ea
%		\gll Tisches $=$ Tisch $+$ es\\  
%		{} {} \gq{\alertblue{Tisch}} {} \gq{\textsc{\alertred{gen.sg}}}\\
%		
%		\ex	
		\gll Haustüren $=$ Haus $+$ tür $+$ en \\
		{} {} \gq{\alertblue{Haus}} {} \gq{\alertblue{Tür}} {} \gq{\textsc{\alertred{pl}}} \\
		
		\ex	
		\gll (ihr) essen $=$ ess $+$ t \\
		{} {} {} \gq{\alertblue{ess}} {} \gq{\textsc{\alertred{2.pl}}}\\
		
		\z
	\z 

\pause 
	
	\item \textbf{verschiedene Vorkommen}: unterschiedliche Realisierungen (\textbf{Morphe}) können dieselbe Funktion/Bedeutung tragen.
	
	\ea\label{ex:Allomorph} Tür$+$\alertred{en}, Kind$+$\alertred{er}, Schal$+$\alertred{s}
	\z

\pause 
	
	\item \textbf{Allomorphe:} Varianten eines Morphems, mit derselben Bedeutung/Kategorie
	
	\begin{itemize}
		\item \{\emph{-en}, \emph{-er}, \emph{-s}\} in (\ref{ex:Allomorph}) tragen die Bedeutung \gq{\textsc{pl}}. Sie sind unterschiedliche \textbf{Morphe}, sie sind \textbf{Allomorphe} zu einem \textbf{Morphem} (abstrakte Einheit).
	\end{itemize}

\end{itemize}
\end{frame}


%%%%%%%%%%%%%%%%%%%%%%%%%%%%%%%%%%
\begin{frame}
\frametitle{Gründe für Allomorphie}

\begin{itemize}
	\item \textbf{phonologisch bedingt} 
	
	\ea 
	
	\ea \textsc{land}: \{\textipa{[lan\alertred{d}]}, \textipa{[lan\alertred{t}]}\}

	Auslautverhärtung in \ab{Landes} vs. \ab{Land}
	
	\ex \textsc{infinitivendung}: \{\textipa{[n]}, \textipa{[\alertred{@}n]}\}
	
	Schwatilgung nach Liquiden: \ab{segeln} vs.\ \ab{formen}, \ab{turnen}
	\z
	\z 

\pause 

	\item \textbf{morphologisch bedingt}
	\ea \textsc{haus}: \{\textipa{[h\alertred{\t{aU}}s]}, \textipa{[h\alertred{\t{O}I}s]}\}
	
	Neutra mit \emph{-er}-Plural und umlautfähigem Stammvokal erhalten \textbf{immer} einen Umlaut (s.\ \ab{Häuser}, \ab{Fässer}, \ab{Bücher}, \ab{Hörner}).
	\z 

\pause 

	\item \textbf{lexikalisch bedingt}
	
	\ea \textsc{kuss}: \{\textipa{[k\alertred{U}s]}, \textipa{[k\alertred{Y}s]}\}
	
	Maskulina mit \emph{-e}-Plural erhalten \textbf{manchmal} einen Umlaut (\ab{Küsse} aber \ab{Tage}). Allomorphe müssen im Lexikon gespeichert werden.
	\z 
	
%	\item Häufig verwendet man den Begriff \textbf{morphologisch bedingte Allomorphie} auch für die \textbf{lexikalisch bedingte Allomorphie}.
\end{itemize}

\end{frame}


%%%%%%%%%%%%%%%%%%%%%%%%%%%%%%%%%%
\begin{frame}
\frametitle{Minimalpaare}

\begin{itemize}
	\item Morpheme (sowie Phoneme) findet man mithilfe von (morphologischen) \textbf{Minimalpaaren}:
\end{itemize}

\begin{table}
\begin{tabular}{l | l}
lach + t  & träum + t \\
lach + st & träum + st \\
lach + en & träum + en \\
lach + te & träum + te \\
\end{tabular}
\end{table}

\end{frame}


%%%%%%%%%%%%%%%%%%%%%%%%%%%%%%%%%%
%%%%%%%%%%%%%%%%%%%%%%%%%%%%%%%%%%
\subsection{Kriterien der Morphemklassifikation}
\frame{
\frametitle{~}
	\tableofcontents[currentsection]
}


%%%%%%%%%%%%%%%%%%%%%%%%%%%%%%%%%%
\begin{frame}
\frametitle{Kriterien der Morphemklassifikation}

\begin{itemize}
	\item Morpheme lassen sich hinsichtlich verschiedener Kriterien klassifizieren:
	
	\begin{itemize}
	 \item Verhältnis Form und Bedeutung
	 \item Art der Bedeutung
	 \item Distribution und Selbstständigkeit
	\end{itemize}
\end{itemize}


\end{frame}



%%%%%%%%%%%%%%%%%%%%%%%%%%%%%%%%%%
\begin{frame}
\frametitle{Form \& Bedeutung}

\begin{itemize}
	\item \textbf{strukturalistischer Idealfall:} eine Form mit einer Bedeutung
	
	\ea	Jakob ist der schön\alertred{st}e.
	
	Form: \emph{-st} \\
	Bedeutung: Superlativ
	\z

\pause 
	
	\item \textbf{Portmanteau-Morphem:}  eine Form mit Komplex mehrerer Bedeutungen
	
	\ea Luise \alertred{gab} Jakob ein Buch.
	
	Form: \emph{gab} \\
	Bedeutung: \gq{geb-} $+$ \gq{\textsc{3.sg.prät.ind.akt}}
	\z
	
	\item Die \textbf{Verschmelzung zweier Morpheme} wird manchmal auch Portmanteau-Morphem genannt: \ab{zum}, \ab{am}, \ab{im}

\pause 
	
	\item \textbf{diskontinuierliches Morphem:} zwei Formen mit einer Bedeutung 
	
	\ea Luise hat das Buch \alertred{ge}klau\alertred{t}.
	
	Form: \emph{ge-} $+$ \emph{-t} \\
	Bedeutung: \gq{Partizip II} \\

	\z	

\end{itemize}

\end{frame}


%%%%%%%%%%%%%%%%%%%%%%%%%%%%%%%%%%
\begin{frame}
\frametitle{Bedeutungsart}


\ea Jakob \alertred{denk}t, \alertblue{dass} er \alertblue{den} \alertred{Nachbar}\alertblue{n} \alertred{treff}\alertblue{en} wird.
\z

\begin{itemize}
	\item \alertred{lexikalische} Morpheme:
	\begin{itemize}
		\item Sie bezeichnen \textbf{Außersprachliches} (Objekte, Sachverhalte). 
		
		\item Inhalt ist Gegenstand semantischer/lexikologischer Analyse.
		
		\item Ihre Klasse ist erweiterbar (\textbf{offene Klasse}).
	
	\end{itemize}

\pause 
		
	\item \alertblue{grammatische} Morpheme:
	
	\begin{itemize}
		\item Sie kodieren \textbf{grammatische Information}, bzw.\ dienen der Realisierung grammatischer Beziehungen im Satz.
		
		\item Ihre Klasse ist nicht mit den herkömmlichen Wortbildungsprozessen erweiterbar (\textbf{geschlossene Klasse})
	
%		Umstritten: Wortbildungsmorpheme wie \emph{-lich}, \emph{-heit}; sog. Funktionswörter wie Präpositionen, Konjunktionen, etc.
	\end{itemize}
\end{itemize}

\end{frame}


%%%%%%%%%%%%%%%%%%%%%%%%%%%%%%%%%%
\begin{frame}
\frametitle{Distribution/Selbstständigkeit}

\ea \alertblue{Jakob} denk\alertred{t}, \alertblue{dass} er \alertblue{den} \alertblue{Nachbar}\alertred{n} \alertblue{treff}\alertred{en} wird.
\z

\begin{itemize}

	\item \alertblue{freie} Morpheme: 
	
	\begin{itemize}
		\item Sie kommen \gqq{\textbf{frei}} vor (sie sind \gqq{wortfähig}).
		
		\item Sie können sowohl \textbf{lexikalische} als auch \textbf{grammatische} Bedeutung tragen.
	\end{itemize}
	
\pause
	
	\item \alertred{gebundene} Morpheme (Affixe): 
	
	\begin{itemize}
		\item Sie sind an andere Morpheme gebunden.
		
		\item Sie treten \textbf{nicht selbstständig} auf (sie sind nicht \gqq{wortfähig}).
		
		\item Umstritten: Einordnung bestimmter lexikalischer Morpheme, wie \emph{geb-}, \emph{weiger-}, da sie nicht frei vorkommen (meist dient die Wortform des Imperativs als Kriterium).
		
\pause

		\item Sonderfall: \textbf{unikales Morph(em)} (\emph{cranberry morph})
	
	\end{itemize}
	
	
\end{itemize}
\end{frame}


%%%%%%%%%%%%%%%%%%%%%%%%%%%%%%%%%%
\begin{frame}
\frametitle{Unikales Morph(em)}

\begin{itemize}
	\item Sonderfall des gebundenen Morphems
	
	\item auch: \emph{cranberry morph}
	
	
	\ea \alertred{Brom}beere, \alertred{Him}beere, \alertred{Schorn}stein, ver\alertred{geu}den, Tausend\alertred{sassa}
	\z


	\item Sie kommen nur in einer einzigen Kombination vor (nicht \textbf{produktiv})

	\item Bedeutung synchron nicht mehr erschließbar (nicht \textbf{transparent})

	\item Ihre Bedeutung ist auf eine \textbf{distinktive Funktion} beschränkt.
	
	\ea \alertred{Brom}beere \vs \alertred{Him}beere
	\z 

\end{itemize}

\end{frame}


%%%%%%%%%%%%%%%%%%%%%%%%%%%%%%%%%%
%%%%%%%%%%%%%%%%%%%%%%%%%%%%%%%%%%
\subsection{Morphologische Einheiten}
\frame{
\frametitle{~}
	\tableofcontents[currentsection]
}

%%%%%%%%%%%%%%%%%%%%%%%%%%%%%%%%%%
\begin{frame}
\frametitle{Wurzel \& Stamm}

\begin{itemize}
	\item \textbf{Wurzel} (auch Wurzelmorphem, Basismorphem): 
	
	\begin{itemize}
		\item Unterste, atomare Basis komplexer Wörter
		\item hinsichtlich \textbf{Wortbildung} und \textbf{Flexion} nicht mehr zerlegbar
		\item oft, aber nicht immer frei
		
		\ea \alertred{Ehr}-e, ver-\alertred{ehr}-st, \alertred{ess}-bar, \alertred{Ess}-\alertred{tisch}				
		\z 
	\end{itemize}


	\item \textbf{Stamm:}
	
	\begin{itemize}
		\item Ausgangsform der \textbf{Flexion} und der \textbf{Komposition}
		\item Ein Stamm kann \textbf{morphologisch einfach} (Wurzel) oder \textbf{komplex} sein.
	
		\ea 
			\ea \alertred{sag}-st, \alertred{Tisch}-es
			\ex \alertred{be-lächel}-st, \alertred{Tisch-bein}-es
			\z 
		\z 
	\end{itemize}
\end{itemize}

\end{frame}


%%%%%%%%%%%%%%%%%%%%%%%%%%%%%%%%%%
\begin{frame}
\frametitle{Basis \& Derivat}

\begin{itemize}
	\item \textbf{Basis:} (Pl. Basen)
	
	\begin{itemize}
		\item Ausgangsform der \textbf{Derivation} (Wortbildung durch Affigierung)
		\item Eine Basis kann \textbf{morphologisch einfach} (Wurzel) oder \textbf{komplex} sein.
		
		\ea \alertred{freund}-lich, un-\alertred{freund-lich}, \alertred{Un-freund-lich}-keit
		\z 

	\end{itemize}

	\item \textbf{Derivat:} Resultat der \textbf{Derivation}
	
	\ea 
		\ea \alertred{freund-lich} $=$ freund $+$ -lich
		\ex \alertred{unfreundlich} $=$ un- $+$ freundlich
		\z 
	\z 

\end{itemize}

\end{frame}


%%%%%%%%%%%%%%%%%%%%%%%%%%%%%%%%%%
\begin{frame}
\frametitle{Affix \& Konfix}

\begin{itemize}
	\item \textbf{Affix:}
	
	\begin{itemize}
		\item \textbf{nicht frei vorkommende} (gebundene) Morpheme
		
		\item nach der \textbf{Stellung} zum Stamm / zur Basis:
		
		\settowidth\jamwidth{[ TTagalog (Philippinen)]} 
		\ea
			\ea \textbf{Präfix:} \alertred{un}-schön, \alertred{ver}-teilen
		
			\ex \textbf{Suffix:} teil-\alertred{bar}, Bäck-\alertred{er}
		
			\ex \textbf{Zirkumfix:} \alertred{ge}-sag-\alertred{t}, \alertred{Ge}-red-\alertred{e}
		
			\ex \textbf{Infix:} 
			
			 v\u{o}h \gq{wissen} \ras v\alertred{an}\u{o}h \gq{weise} \jambox{[Chrau (Vietnam)]}
			 
			sulat \gq{schreiben} \ras su\alertred{mu}lat \gq{schrieb} \jambox{[Tagalog (Philippinen)]}
		
			\z 
		\z 

\pause 
		
		\item nach der \textbf{morphologischen Funktion}: \\
		\textbf{Derivationsaffixe} (s. (\ref{ex:DerAff})) \vs \textbf{Flexionsaffixe} (s.\ (\ref{ex:FlexAff})):
		
		\ea 
		\ea\label{ex:DerAff} -ig, -lich, -keit, ver-, be-, ent-, un-, \dots
		\ex\label{ex:FlexAff} \emph{-st} in \ab{kommst}, \emph{-en} in \ab{Betten}, \emph{-er} in \ab{kleiner}, \dots
		\z
		\z 
	\end{itemize}
\end{itemize}
\end{frame}


%%%%%%%%%%%%%%%%%%%%%%%%%%%%%%%%%%
\begin{frame}
\frametitle{Affix \& Konfix}

\begin{itemize}
\item \textbf{Konfix}: 

\begin{itemize}
	\item \textbf{nicht frei vorkommende} Morpheme (ähnlich wie Affixe)
	\item Aber sie lassen sich zu einem selbständigen Wort kombinieren (wie normale Wurzeln/ Stämme).
	
	\ea
		\ea[]{\alertred{Bio}-\alertred{log}e}
		\ex[]{\alertred{Soft}-ie}
		\ex[*]{un-keit, ver-lich}
		\z 
	\z 
	
	\item Ihre Bedeutung ist \textbf{stärker lexikalisch} als bei Affixen.
	
	\item Sie können sowohl als \textbf{Präfixe} als auch als \textbf{Suffixe} fungieren.
	
		\ea 
			\ea {kino-\alertred{phil} \vs \alertred{Phil}-anthrop}
			\ex {\alertred{Un-}Mensch \vs *Mensch\alertred{-un}}
			\ex {Soft-getränk}
			\z 
		\z 
	\end{itemize}
\end{itemize}

\end{frame}


%%%%%%%%%%%%%%%%%%%%%%%%%%%%%%%%%%
\begin{frame}
\frametitle{Simplex vs.\ komplexe Lexeme}

\begin{itemize}
	\item \textbf{Simplex:} (Pl. Simplizia)
	
	\begin{itemize}
		\item nicht zusammengesetztes oder abgeleitetes Lexem (\textbf{nicht komplex})
		\item Sie können als Basis für Neubildungen dienen.
		
		\ea Tisch, geb(-en), rot
		\z 
		
		\item Wenn Derivationsaffixe oder Stämme/Wurzeln \textbf{nicht mehr aktiv} (auch \textbf{nicht mehr produktiv}) sind, nimmt man die komplexe Form als Simplex wahr.

		\ea Ursache, Mädchen, freilich
		\z 
	\end{itemize}

\pause 

	\item \textbf{komplexe Lexeme:}
	
	\begin{itemize}
		\item durch Anwendung eines \textbf{Wortbildungsprozesses} erzeugte \textbf{komplexe Form}
		
		\ea un-freundlich (Derivation), Tisch-bein (Komposition)
		\z 
	\end{itemize}
\end{itemize}

\end{frame}