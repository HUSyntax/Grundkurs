%%%%%%%%%%%%%%%%%%%%%%%%%%%%%%%%%%%%%%%%%%%%%%%%%%%%
%%%             Metadata                         %%%
%%%%%%%%%%%%%%%%%%%%%%%%%%%%%%%%%%%%%%%%%%%%%%%%%%%%      

\title{Grundkurs Linguistik}

\subtitle{Morphologie I: Einführung \& Begriffe}

\author[aMyP]{
	{\small Antonio Machicao y Priemer}
%	\\
%	{\footnotesize \url{http://www.linguistik.hu-berlin.de/staff/amyp}\\
%	\href{mailto:mapriema@hu-berlin.de}{mapriema@hu-berlin.de}}
}

\institute{Institut für deutsche Sprache und Linguistik}

%%%%%%%%%%%%%%%%%%%%%%%%%      
\date{ }
%\publishers{\textbf{6. linguistischer Methodenworkshop \\ Humboldt-Universität zu Berlin}}

%\hyphenation{nobreak}


%%%%%%%%%%%%%%%%%%%%%%%%%%%%%%%%%%%%%%%%%%%%%%%%%%%%
%%%             Preamble's End                   %%%
%%%%%%%%%%%%%%%%%%%%%%%%%%%%%%%%%%%%%%%%%%%%%%%%%%%%      


\huberlintitlepage[22pt]
\iftoggle{toc}{
\frame{
\begin{multicols}{2}
	\frametitle{Inhaltsverzeichnis}\tableofcontents
	%[pausesections]
\end{multicols}
	}
	}


%%%%%%%%%%%%%%%%%%%%%%%%%%%%%%%%%%%
%%%%%%%%%%%%%%%%%%%%%%%%%%%%%%%%%%

%\nocite


%%%%%%%%%%%%%%%%%%%%%%%%%%%%%%%%%%%
%%%%%%%%%%%%%%%%%%%%%%%%%%%%%%%%%%



%%%%%%%%%%%%%%%%%%%%%%%%%%%%%%%%%%
%%%%%%%%%%%%%%%%%%%%%%%%%%%%%%%%%%
\section{Einführung}
\iftoggle{toc}{
\frame{
\begin{multicols}{2}
\frametitle{~}
	\tableofcontents[currentsection]
\end{multicols}
}
}


%%%%%%%%%%%%%%%%%%%%%%%%%%%%%%%%%%
\begin{frame}
\frametitle{Einführung}

\begin{itemize}
	\item Morphologie = \textbf{Formenlehre} \\
	(griech. \emph{morphe}: \gq{Form, Gestalt}, \emph{logos} \gq{Sinn, Lehre}
	\item[]
	\item Goethe (1796): Bezeichnung der \textbf{Lehre von Form und Struktur lebender Organismen}.
	\item[]
	\item August Schleicher (19. Jh.): Übernahme in die Sprachwissenschaft zur Bezeichnung von \textbf{Wörtern als Untersuchungsgegenstand}.
	
	\begin{itemize}
		\item \ras \bf{Struktur und Aufbau von Wörtern} und \bf{Theorien von komplexen Wörtern} (Produktivität, Schnittstellen zu Phonologie, Syntax, Semantik).
	\end{itemize}
\end{itemize}


\end{frame}


%%%%%%%%%%%%%%%%%%%%%%%%%%%%%%%%%%
\begin{frame}
\frametitle{Einführung}

\begin{itemize}
	\item Morphologie unterteilt sich in:
	
	\begin{itemize}
		\item \textbf{Wortformen}bildung (Flexion) \ras grammatische Wortformveränderungen
		\item[]
		\item \textbf{Wort}bildung \ras Ableitung und Zusammensetzung lexikalischer Wörter
	\end{itemize}
\end{itemize}

\begin{figure}	
\centering
\scalebox{0.55}{
\begin{forest} %% how to scale forest?
sn edges,
	[Morphologie
		[Flexion
			[Konjugation] 
			[Deklination]]
		[Wortbildung
			[Komposition
				[Determinativ]
				[Kopulativ]
				[\dots]]
			[Derivation
				[Suffigierung]
				[Präfigierung]
				[\dots]]
			[Konversion
				[morphologisch]
				[syntaktisch]]
			[Rückbildung]
			[\dots]]]										
\end{forest}}
\end{figure}


\end{frame}


%%%%%%%%%%%%%%%%%%%%%%%%%%%%%%%%%%%
%%%%%%%%%%%%%%%%%%%%%%%%%%%%%%%%%%
\section{Wort}
\iftoggle{toc}{
\frame{
\begin{multicols}{2}
\frametitle{~}
	\tableofcontents[currentsection]
\end{multicols}
}
}

%%%%%%%%%%%%%%%%%%%%%%%%%%%%%%%%%%
\begin{frame}
\frametitle{Wort}

\begin{itemize}
	\item \textbf{Intuitiv} vorgegebener und \textbf{umgangssprachlich} verwendeter
Begriff für \textbf{sprachliche Grundeinheiten}. Seine Definition ist
\textbf{uneinheitlich} und \textbf{kontrovers}.
\end{itemize}


\end{frame}


%%%%%%%%%%%%%%%%%%%%%%%%%%%%%%%%%%
%%%%%%%%%%%%%%%%%%%%%%%%%%%%%%%%%%
\subsection{Wort: phonetisch-phonologisch}
%\frame{
%\frametitle{~}
%	\tableofcontents[currentsection]
%}


%%%%%%%%%%%%%%%%%%%%%%%%%%%%%%%%%%
\begin{frame}
\frametitle{Wort: phonetisch-phonologisch}

\begin{itemize}
	\item die kleinsten durch Wortakzent und Grenzsignale (Pause, Knacklaut) theoretisch isolierbare Lautsegmente
	\item[]
	\item Es stimmt nicht immer mit dem graphemischen Wort überein.
	\item[]
	\item Viele phonologische Prozesse haben das phonologische Wort
als Domäne:
	
	\begin{itemize}
		\item[]
		\item Die Silbifizierung erfolgt nur innerhalb des phonologischen
Wortes.
		\item Assimilationsprozesse sind nur innerhalb des phonologischen
Wortes obligatorisch.
	\end{itemize}
\end{itemize}


\end{frame}


%%%%%%%%%%%%%%%%%%%%%%%%%%%%%%%%%%
%%%%%%%%%%%%%%%%%%%%%%%%%%%%%%%%%%
\subsection{Wort: orthographisch-graphemisch}
%\frame{
%\frametitle{~}
%	\tableofcontents[currentsection]
%}


%%%%%%%%%%%%%%%%%%%%%%%%%%%%%%%%%%
\begin{frame}
\frametitle{Wort: orthographisch-graphemisch}

\begin{itemize}
	\item Buchstabensequenz zwischen zwei Leerzeichen (Spatien) oder
zwischen einem Leerzeichen und einem Satzzeichen
	\item[]
	\item Es enthält selbst kein Leerzeichen
	\item[]
	\item Definition gilt nur für Sprachen mit alphabetischem Schriftsystem!
	\item[]
	\item Sprachspezifisch:
	
	\begin{itemize}
		\item \emph{Morphologieeinführungsbuch} vs. \emph{introductory morphology book}
	\end{itemize}
	
	\item Seit der letzen Rechtschreibreform gibt es im Deutschen weniger
graphemische Wörter:
	
	\begin{itemize}
		\item \emph{heilig sprechen}, \emph{Rad fahren}
	\end{itemize}
\end{itemize}


\end{frame}


%%%%%%%%%%%%%%%%%%%%%%%%%%%%%%%%%%
%%%%%%%%%%%%%%%%%%%%%%%%%%%%%%%%%%
\subsection{Wort: morphologisch}
%\frame{
%\frametitle{~}
%	\tableofcontents[currentsection]
%}


%%%%%%%%%%%%%%%%%%%%%%%%%%%%%%%%%%
\begin{frame}
\frametitle{Wort: morphologisch}

\begin{itemize}
	\item auch \textbf{lexikalisches} Wort oder \textbf{Lexem} genannt
	\item[]
	\item Grundeinheit von einem grammatischen Paradigma
	\item[]
	\item strukturell stabil und nicht trennbar
	\item[]
	\item durch spezifische Regeln der Wortbildung zu beschreiben
	\item[]
	\item im Lexikon kodifiziert (Basiseinheit des Lexikons)
\end{itemize}


\end{frame}


%%%%%%%%%%%%%%%%%%%%%%%%%%%%%%%%%%
\begin{frame}
\frametitle{Wort: morphologisch}

\begin{itemize}
	\item kommt in verschiedenen grammatischen Wortformen vor \ras \\ 
	\textbf{flektivische Wörter} (Wortformen)
	\item[]
	\item \textbf{flektivische Wörter}: hinsichtlich grammatischer Kategorien wie
Tempus, Person, Numerus, Kasus, etc \dots spezifiziert:
	
	\begin{itemize}
		\item Bank (Geldinstitut) \ras flektivische Wörter: Bank, Banken
		\item Bank (Sitzgelegenheit) \ras flektivische Wörter: Bank, Bänke, Bänken
	\end{itemize}
	
	\item[]
	\item Ein \textbf{Paradigma} sind alle vorkommenden Wortformen eines
Lexems.
\end{itemize}


\end{frame}


%%%%%%%%%%%%%%%%%%%%%%%%%%%%%%%%%%
\begin{frame}
\frametitle{Wort: morphologisch}

\begin{itemize}
	\item \textbf{Zitierform} (Lemma)
	\item[]
	\item konventionell festgelegte Form eines Paradigmas
	\item stellvertretend für das gesamte Paradigma
	\item im Deutschen bei Nomina \ras Nominativ Singular
	\item im Deutschen bei Verben \ras Infinitiv
	\item Zitierform von Verben enthält
	
	\begin{itemize}
		\item[]
		\item freies Morphem (Imperativform) + gebundenes Morphem (-\emph{en})
		\item \emph{lach} oder \emph{schlaf} + -\emph{en}
	\end{itemize}
\end{itemize}


\end{frame}


%%%%%%%%%%%%%%%%%%%%%%%%%%%%%%%%%
%%%%%%%%%%%%%%%%%%%%%%%%%%%%%%%%%
\subsection{Wort: syntaktisch}
%\frame{
%\frametitle{~}
%	\tableofcontents[currentsection]
%}


%%%%%%%%%%%%%%%%%%%%%%%%%%%%%%%%%%
\begin{frame}
\frametitle{Wort: syntaktisch}

\begin{itemize}
	\item die kleinste verschiebbare und ersetzbare Einheit eines Satzes
(Problem: Artikel, manche Präpositionen)
	
	  \eal
          \ex[]{
            Wir bauten Häuser.
            }
	    \ex[]{
              Häuser bauten wir.
            }
	  \ex[*]{
            Ein bauten wir Haus.
          }
          \zl

	\item Auch definiert als die kleinste Einheit, die alleine als Satz
vorkommen kann.

\eal
\ex Heißt es \gqq{ein} oder \gqq{eine} Hund?
\ex \gqq{Ein}
\zl
		 
\end{itemize}


\end{frame}


%%%%%%%%%%%%%%%%%%%%%%%%%%%%%%%%%%
%%%%%%%%%%%%%%%%%%%%%%%%%%%%%%%%%%
\subsection{Wort: lexikalisch-semantisch}
%\frame{
%\frametitle{~}
%	\tableofcontents[currentsection]
%}


%%%%%%%%%%%%%%%%%%%%%%%%%%%%%%%%%%
\begin{frame}
\frametitle{Wort: lexikalisch-semantisch}

\begin{itemize}
	\item \textbf{die kleinste Einheit},
	
	\begin{itemize}
		\item[]
		\item der eine Bedeutung zugeordnet werden kann (\emph{Tisch}) oder
		\item[]		
		\item die eine syntaktische/pragmatische Funktion hat (\emph{der}, \emph{ja})
		\item[]
		\item Problem: \emph{der US-amerikanische Präsident}
	\end{itemize}
\end{itemize}


\end{frame}


%%%%%%%%%%%%%%%%%%%%%%%%%%%%%%%%%%
%%%%%%%%%%%%%%%%%%%%%%%%%%%%%%%%%%
\subsection{Wort: Hauptkriterien}
%\frame{
%\frametitle{~}
%	\tableofcontents[currentsection]
%}


%%%%%%%%%%%%%%%%%%%%%%%%%%%%%%%%%%
\begin{frame}
\frametitle{Wort: Hauptkriterien}

\begin{itemize}
	\item akustische und semantische Identität,
	\item morphologische Stabilität und
	\item syntaktische Mobilität
	\item[]
	\item Jede unterschiedliche Wortdefinition liefert eine unterschiedliche Menge von \gqq{Wörtern}, mit denen in den verschiedenen Teilgebieten der Linguistik gearbeitet wird.
	
	\begin{itemize}
		\item Morphologie \ras \gqq{morphologische und flektivische Wörter}
	\end{itemize}
\end{itemize}


\end{frame}


%%%%%%%%%%%%%%%%%%%%%%%%%%%%%%%%%%
%%%%%%%%%%%%%%%%%%%%%%%%%%%%%%%%%%
\section{Morphologische Grundbegriffe}
\iftoggle{toc}{
\frame{
\begin{multicols}{2}
\frametitle{~}
	\tableofcontents[currentsection]
\end{multicols}
}
}
%%%%%%%%%%%%%%%%%%%%%%%%%%%%%%%%%%
%%%%%%%%%%%%%%%%%%%%%%%%%%%%%%%%%%
\subsection{Morph, Morphem, Allomorph}
%\frame{
%\frametitle{Morphologische Grundbegriffe}
%	\tableofcontents[currentsection]
%}


%%%%%%%%%%%%%%%%%%%%%%%%%%%%%%%%%%
\begin{frame}
\frametitle{Morph, Morphem, Allomorph}

\begin{itemize}
	\item \textbf{Morphem}:

	\begin{itemize}
		\item[]
		\item Strukturalistische Definition: \\
		kleinste bedeutungstragende Einheit
		\item[]
		\item Wurzel 1984: \\
		Ein Morphem ist die kleinste, in ihren \textbf{verschiedenen Vorkommen} als formal \textbf{einheitlich identifizierbare Folge von Segmenten}, der (wenigstens) eine als einheitlich identifizierbare \textbf{außerphonologische Eigenschaft} zugeordnet ist.
	\end{itemize}
\end{itemize}


\end{frame}



%%%%%%%%%%%%%%%%%%%%%%%%%%%%%%%%%%
\begin{frame}
\frametitle{Morph, Morphem, Allomorph}

\begin{itemize}
	\item \textbf{Morphem}:
	
	\begin{itemize}
		\item[]
		\item Außerphonologische Eigenschaften: grammatische (\zB Kasus, Numerus) und/ oder lexikalische Bedeutung
		
		  \eal
                  \ex	Tisches = Tisch + 
                  es = Bed. \gq{TISCH} + Bed./Kat. \gq{GEN.SG}
		  \ex	Haustüren = Haus + tür + en = Bed. \gq{HAUS} + Bed. \gq{TÜR} + Bed./Kat. \gq{PL} 
		  \ex	(sie) essen = ess + en = Bed. \gq{ESS} +
                  Bed./Kat. \gq{3.P.PL}
                  \zl
			 
	\end{itemize}
\end{itemize}


\end{frame}



%%%%%%%%%%%%%%%%%%%%%%%%%%%%%%%%%%
\begin{frame}
\frametitle{Morph, Morphem, Allomorph}

\begin{itemize}
	\item \textbf{Morphem vs. Morph vs. Allomorph:}
	
	\begin{itemize}
		\item[]
		\item Verschiedene Vorkommen: Unterschiedliche Formen (\textbf{Morphe}) können dieselbe Funktion/Bedeutung tragen.
		
		  \ea
                  Tür + en, Kind + er, Schal + s
           \z
		
		\item \textbf{Allomorphe} \ras Varianten eines Morphems, die dieselbe Bedeutung/Kategorie tragen
		
		\begin{itemize}
			\item[]
			\item \{\emph{-en}, \emph{-er}, \emph{-s}\} tragen eine einzelne Bedeutung \gq{PLURAL}; sie sind unterschiedliche \textbf{Morphe} und alle \textbf{Allomorphe} zu einem \textbf{Morphem} (abstrakte Einheit).
		\end{itemize}
	\end{itemize}
\end{itemize}


\end{frame}



%%%%%%%%%%%%%%%%%%%%%%%%%%%%%%%%%%
\begin{frame}
\frametitle{Morph, Morphem, Allomorph}

\begin{itemize}
	\item \textbf{phonologisch bedingte Allomorphie:}
	
	\begin{itemize}
		\item[]
		\item Ein Morphem kann verschiedene Allomorphe aufgrund phonologischer Regularitäten haben:
		
		\begin{itemize}
			\item[]
			\item Allomorphe \textipa{[land]} und \textipa{[lant]} \\
			durch Auslautverhärtung in Landes vs. Land
			\item[]
			\item Allomorphe \textipa{[n]} und \textipa{[@n]} für Infinitiv: \\
			durch Schwaeinsetzung: segeln vs. formen, turnen
		\end{itemize}
	\end{itemize}
\end{itemize}


\end{frame}



%%%%%%%%%%%%%%%%%%%%%%%%%%%%%%%%%%
\begin{frame}
\frametitle{Morph, Morphem, Allomorph}

\begin{itemize}
	\item \textbf{morphologisch bedingte Allomorphie:}
	
	\begin{itemize}
		\item Allomorphe \textipa{[haUs]} und \textipa{[hOIs]} in Haus vs. Häuschen, häuslich
		\item Regel: Neutra mit \emph{-er}-Plural und umlautfähigem Stammvokal erhalten immer einen Umlaut (Fässer, Bücher, Hörner).
	\end{itemize}
	
	\item \textbf{lexikalisch bedingte Allomorphie:}
	
	\begin{itemize}
		\item Allomorphe \textipa{[kUs]} und \textipa{[kYs]} in Kuss vs. Küsse (auch: Küsschen) im Lexikon festgelegt: Maskulina mit der Pluralendung \emph{-e} erhalten manchmal einen Umlaut und manchmal nicht (Tage)
	\end{itemize}
	
	\item[]
	\item Häufig verwendet man den Begriff \textbf{morphologisch bedingte Allomorphie} auch für die \textbf{lexikalisch bedingte Allomorphie}.
\end{itemize}


\end{frame}



%%%%%%%%%%%%%%%%%%%%%%%%%%%%%%%%%%
\begin{frame}
\frametitle{Morph, Morphem, Allomorph}

\begin{itemize}
	\item Morpheme (sowie Phoneme) findet man mithilfe von \textbf{Minimalpaaren}:
\end{itemize}

\begin{table}
\begin{tabular}{l | l}
lach + t  & träum + t \\
lach + st & träum + st \\
lach + en & träum + en \\
lach + te & träum + te \\
\end{tabular}
\end{table}

\end{frame}


%%%%%%%%%%%%%%%%%%%%%%%%%%%%%%%%%%
%%%%%%%%%%%%%%%%%%%%%%%%%%%%%%%%%%
\subsection{Morphemklassifikation}
%\frame{
%\frametitle{~}
%	\tableofcontents[currentsection]
%}


%%%%%%%%%%%%%%%%%%%%%%%%%%%%%%%%%%
\begin{frame}
\frametitle{Morphemklassifikation}

\begin{itemize}
	\item \textbf{Morpheme lassen sich hinsichtlich verschiedener Kriterien klassifizieren:}
	
	\begin{itemize}
	 \item[]
	 \item Verhältnis Form und Bedeutung
	 \item[]
	 \item Art der Bedeutung
	 \item[]
	 \item Distribution und Selbstständigkeit
	\end{itemize}
\end{itemize}


\end{frame}



%%%%%%%%%%%%%%%%%%%%%%%%%%%%%%%%%%
\begin{frame}
\frametitle{Form \& Bedeutung}

\begin{itemize}
	\item \textbf{Wodurch unterscheiden sich die unterstrichenen Morpheme?}
		
	  \only<1->{
            \ea
            Helga ist die schön\underline{st}e.
          \z
          }
	
	\only<2>{eine Form - eine Bedeutung: \\
		Form: \emph{-st} \\
		gramm. Funktion: Superlativ\\
		\textbf{= strukturalistischer Idealfall}}
	
	\only<1->{\ea
          Karl \underline{gab} Ilse die Hauptrolle.
          \z}
	
	\only<3>{eine Form - Komplex mehrerer Bedeutungen \\
		Form: \emph{gab} \\
		Bedeutung: \gq{GEBEN} + \gq{3.P.SG.PRÄT.IND.AKTIV} \\
		\textbf{= Portmanteau-Morphem} \\
		Die Verschmelzung zweier Morpheme wird manchmal auch Portmanteau-Morphem genannt: \ab{zum, am, im}}
	
	\only<1->{\ea Paul hat Ilse wirklich \underline{ge}lieb\underline{t}!
        \z}	
		
	\only<4>{zwei Formen - eine Bedeutung (gramm. Funktion) \\
		Form: \emph{ge-} + \emph{-t} \\
		gramm. Funktion: \gq{Partizip II} \\
		\textbf{= diskontinuierliches Morphem}}
	
\end{itemize}


\end{frame}


%%%%%%%%%%%%%%%%%%%%%%%%%%%%%%%%%%
\begin{frame}
\frametitle{Bedeutungsart}

\begin{itemize}
	\item \textbf{Wodurch unterscheiden sich die unterstrichenen Morpheme?}
	
		\only<1->{\ea Paul \underline{geht} mit Lisa ins \underline{Kino}.\z}
				
		\only<2>{Morpheme bezeichnen Außersprachliches (Objekte, Sachverhalte). \\
		Inhalt ist Gegenstand semantischer/lexikologischer Analyse. \\
		Ihre Klasse ist erweiterbar. \\
		\textbf{= lexikalische Morpheme (offene Klasse)}}		
		
		\only<1->{\ea
                  Karl spiel\underline{t} in der Küche den Held\underline{en}, \underline{dass} es
                  einen graust.
                \z}
		
		\only<3>{Morpheme kodieren grammatische Information, dienen der Realisierung grammatischer Beziehungen im Satz \\
		\textbf{= grammatische Morpheme (geschlossene Klasse)} \\
		Umstritten: Wortbildungsmorpheme wie \emph{-lich}, \emph{-heit}; sog. Funktionswörter wie Präpositionen, Konjunktionen, etc.}
		
	
\end{itemize}


\end{frame}


%%%%%%%%%%%%%%%%%%%%%%%%%%%%%%%%%%
\begin{frame}
\frametitle{Distribution/Selbstständigkeit}

\begin{itemize}
	\item \textbf{Wodurch unterscheiden sich die unterstrichenen Morpheme?}

	  \only<1->{\ea
            \underline{Und} Paul sieht \underline{rot}, \underline{weil} Lisa \underline{sehr}
            \underline{schnell} \underline{mit} Peter verschwand.
          \z}
		
		\only<2>{Morpheme kommen frei vor; können sowohl lexikalische als auch grammatische Bedeutung haben \\
		\textbf{= freie Morpheme}}
		
		\only<1->{\ea
                  Sprachwissen\underline{schaft} kann auch sehr
                  \underline{un}übersicht\underline{lich} sein.}
                \z
		
		\only<3>{Morpheme sind an andere Morpheme gebunden; treten nicht selbstständig auf (sie sind nicht \gqq{wortfähig}) \\
		\textbf{= gebundene Morpheme} \\
		Umstritten: die Einordnung bestimmter lexikalischer Morpheme, wie \emph{geb-}, \emph{weiger-}, wenn sie nicht frei vorkommen (meist dient die Wortform des Imperativs als Kriterium).}

	
\end{itemize}


\end{frame}


%%%%%%%%%%%%%%%%%%%%%%%%%%%%%%%%%%
\begin{frame}
\frametitle{Distribution/Selbstständigkeit}

\begin{itemize}
	\item Sonderfall des gebundenen Morphems: \textbf{Unikales Morph(em)} (\emph{cranberry morph})
	
	\begin{itemize}
		\item[]
		  \ea
                  \underline{Brom}beere, \underline{Him}beere, \underline{Schorn}stein,
                  ver\underline{geu}den, Tausend\underline{sassa}
                  \z
		\item[]
		\item nur in einer einzigen Kombination
		\item[]
		\item keine produktiven Morpheme
		\item[]
		\item Bedeutung synchron nicht mehr erschließbar
		\item[]
		\item Bedeutung auf distinktive Funktion beschränkt
	\end{itemize}
\end{itemize}


\end{frame}


%%%%%%%%%%%%%%%%%%%%%%%%%%%%%%%%%%
%%%%%%%%%%%%%%%%%%%%%%%%%%%%%%%%%%
\subsection{Wurzel, Stamm, Basis, Simplex}
%\frame{
%\frametitle{~}
%	\tableofcontents[currentsection]
%}


%%%%%%%%%%%%%%%%%%%%%%%%%%%%%%%%%%
\begin{frame}
\frametitle{Wurzel, Stamm, Basis, Simplex}

\begin{itemize}
	\item \textbf{Wurzel:} (Wurzelmorphem, Basismorphem)
	
	\begin{itemize}
		\item[]
		\item Unterste, atomare Basis komplexer Wörter
		\item[]
		\item hinsichtlich \textbf{Wortbildung und Flexion} nicht mehr zerlegbar
		\item[]
		\item oft, aber nicht immer frei
		
		\begin{itemize}
			\item[]			
			\item Wurzel \emph{ehr}: \emph{Ehr-e, Ehr-gefühl, ehr-bar}
			\item Wurzel \emph{ess}: \emph{ess-en, ess-bar}
		\end{itemize}
	\end{itemize}
\end{itemize}


\end{frame}



%%%%%%%%%%%%%%%%%%%%%%%%%%%%%%%%%
\begin{frame}
\frametitle{Wurzel, Stamm, Basis, Simplex}

\begin{itemize}
	\item \textbf{Stamm:}
	
	\begin{itemize}
		\item[]
		\item Ausgangsform der \textbf{Flexion}
		\item[]
		\item kann Wurzel oder komplexe morphologische Einheit sein
		
		\begin{itemize}
			\item[]
			\item \ab{sag} + \emph{-st}
			\item \abu{be-lächel} + \emph{-st}
		\end{itemize}
	\end{itemize}
\end{itemize}

\end{frame}



%%%%%%%%%%%%%%%%%%%%%%%%%%%%%%%%%%
\begin{frame}
\frametitle{Wurzel, Stamm, Basis, Simplex}

\begin{itemize}
	\item \textbf{Basis:} (Pl. Basen)
	
	\begin{itemize}
		\item etwas, woran etwas affigiert werden kann
		\item Ausgangsformen der \textbf{Derivation}
		\item kann selber auch komplex sein
		
		\begin{itemize}
			\item (Basis) \emph{Les} + (Suffix) \emph{ung}
			\item (Präfix) \emph{un} + (Basis) \emph{freundlich}
			\item (Basis) \emph{freund} + (Suffix) \emph{lich}
		\end{itemize}
	\end{itemize}
	\item[]
	\item \textbf{Derivat:} Resultat der \textbf{Derivation}
	
	\begin{itemize}
		\item[]
		\begin{itemize}
			\item Lesung
			\item unfreundlich
			\item freundlich
		\end{itemize}
	\end{itemize}
\end{itemize}


\end{frame}



%%%%%%%%%%%%%%%%%%%%%%%%%%%%%%%%%%
\begin{frame}
\frametitle{Wurzel, Stamm, Basis, Simplex}

\begin{itemize}
	\item \textbf{Simplex}: (Pl. Simplizia)
	
	\begin{itemize}
		\item[]
		\item nicht zusammengesetztes oder abgeleitetes Lexem
		\item kann als Basis für Neubildungen dienen.
		
		\begin{itemize}
			\item[]
			\item geben
			\item in angeben, vergeblich, Zugabe
		\end{itemize}
		
	\end{itemize}
		
		\item[]
		\item Wenn Derivationsaffixe oder Stämme/Wurzeln nicht mehr aktiv (auch nicht mehr produktiv) sind, nimmt man die Form als Simplex wahr.
		
	\begin{itemize}
		\item[]
		\begin{itemize}
			\item Ursache, Mädchen, freilich
		\end{itemize}
	\end{itemize}
\end{itemize}


\end{frame}


%%%%%%%%%%%%%%%%%%%%%%%%%%%%%%%%%%
%%%%%%%%%%%%%%%%%%%%%%%%%%%%%%%%%%
\subsection{Affix \& Konfix}
%\frame{
%\frametitle{~}
%	\tableofcontents[currentsection]
%}


%%%%%%%%%%%%%%%%%%%%%%%%%%%%%%%%%%
\begin{frame}
\frametitle{Affix \& Konfix}

\begin{itemize}
	\item \textbf{Affixe}
	
	\begin{itemize}
		\item \textbf{nicht frei vorkommende} \emph{Wort}bildungs- oder \emph{Wortform}bildungselemente
		\item Nach ihrer \textbf{Stellung zum Stamm/Basis} unterscheidet man:
		
		\begin{itemize}
			\item Präfix: \\
			\underline{un}-schön, \underline{ver}-teilen
			\item Suffix: \\
			teil-\underline{bar}, Bäck-\underline{er}
			\item Zirkumfix: \\
			\underline{ge}-sag-\underline{t}, \underline{Ge}-red-\underline{e}
			\item Infix: \\
			Chrau (Vietnam): v\u{o}h \gq{wissen} \ras v\underline{an}\u{o}h \gq{weise} \\
			Tagalog (Philippinen): sulat \gq{schreiben} \ras su\underline{mu}lat \gq{schrieb}
		\end{itemize}
	\end{itemize}
\end{itemize}


\end{frame}


%%%%%%%%%%%%%%%%%%%%%%%%%%%%%%%%%%
\begin{frame}
\frametitle{Affix \& Konfix}

\begin{itemize}
	\item \textbf{Affixe}
	
	\begin{itemize}
		\item[]
		\item Nach ihrer morphologischen Funktion unterscheidet man:
		
		\begin{itemize}
			\item[]
			\item Derivationsaffixe (\emph{Wort}bildungsaffixe): \\
			\emph{-ig, -lich, -keit; ver-, be-, ent-, un-, \dots}
			\item[]
			\item Flexionsaffixe (\emph{Wortformen}bildungsaffixe): \\
			\emph{-st} (kommst), \emph{-(e)n} (gehen, Betten), \emph{-er} (Kinder, kleiner), \dots
		\end{itemize}
	\end{itemize}
\end{itemize}


\end{frame}



%%%%%%%%%%%%%%%%%%%%%%%%%%%%%%%%%%
\begin{frame}
\frametitle{Affix \& Konfix}

\begin{itemize}
	\item \textbf{Konfixe}
	
	\begin{itemize}
		\item nicht frei vorkommende Elemente (ähnlich wie Affixe)
		\item Sie lassen sich zu einem selbständigen Wort kombinieren (wie normale Wurzeln/ Stämme)
		
		\begin{itemize}
			\item \underline{Bio}-loge
			\item Soft-ie
		\end{itemize}
		
		\item[]
		\item stärker lexikalische Grundbedeutung als Affixe, können jedoch als Präfixe oder Suffixe fungieren
		
		\begin{itemize}
			\item kino-phil
			\item Phil-anthrop
			\item Soft-getränk
		\end{itemize}
	\end{itemize}
\end{itemize}


\end{frame}


