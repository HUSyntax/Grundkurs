%%%%%%%%%%%%%%%%%%%%%%%%%%%%%%%%%%%%%%%%%%%%%%%%
%% Compile the master file!
%% 		Slides: Stefan Müller
%% 		Course: GK Linguistik
%%%%%%%%%%%%%%%%%%%%%%%%%%%%%%%%%%%%%%%%%%%%%%%%

\subsection{Flexion}

\author{Stefan Müller (Anke Lüdeling)}

\frame[shrink=10]{
\frametitle{Flexion}

\begin{itemize}
\item Wortbildung beschäftigt sich mit Bildung neuer Lexeme.
\pause
\item Wortformen eines Lexems werden in verschiedenen Kontexten benötigt:
\begin{tabular}{@{}l@{~}l@{~}l@{~}l@{~}l@{}}
Klaus           & schmiert   & ein & belegtes & Brot.\\
Klaus           & schmierte  & ein & belegtes & Brot.\\
Klaus und Karin & schmierten & die & belegten & Brote.\\
Du              & schmierst  &     & belegte  & Brote.\\
\end{tabular}
\pause
\item Formen von \emph{schmieren} unterscheiden sich in Person, Numerus bzw.\ Tempus.
\pause
\item Formen von \emph{belegt} unterscheiden sich in Numerus und Stärke.
\pause
\item Formen von \emph{Brot} unterscheiden sich im Numerus.
\pause
\item Der Bereich, der sich mit diesen Variationen beschäftigt, heißt \alert{Flexion}.
\pause
\item Wie bei Derivation werden bei der Flexion Stämme mit einem oder mehreren Affixen kombiniert.
\pause
\item Art der Affixe hängt von Wortart ab.

\end{itemize}


}

\subsubsection{Wortarten}

\frame{
\frametitle{Wortarten}

\begin{itemize}
\item Wortarten sind Klassen von Wörtern mit ähnlichem Eigenschaften.
\pause
\item Klassische Wortarten (2. Jh. v. Chr.): Nomen, Verb, Partizip, Artikel, Pronomen, Präposition,
Adverb, Konjunktion
\pause
\item Beispiel für Definition:
\begin{quote}
Das Nomen ist ein kasusbildender Satzteil, welcher ein Ding, z.B. Stein, oder eine Handlung,
z.B. Erziehung, bezeichnet [\ldots].\\
Das Nomen hat fünf verschiedene Begleiterscheinungen:\\
Geschlecht, Art, Form, Zahl und Kasus.
\end{quote}
\pause
\item Vermischung verschiedener Kriterien aus Syntax, Semantik und Morphologie.

\pause
\item Unterscheidung zwischen flektierbaren und nichtflektierbaren Wortarten.
\end{itemize}

}


\frame{
\frametitle{Flektierbare und unflektierbare Wortarten}

\vfill
\begin{tabular}{|l|l|l|l|l|l|}\hline
\multicolumn{3}{|l|}{flektierbare Wortarten} & \multicolumn{3}{l|}{unflektierbare Wortarten}\\\hline
Name     & Abk. & Beispiele & Name         & Abk. & Beispiele\\\hline
Nomen    & N         & Tisch      & Präposition  & P         & auf, neben\\
         &           & Haus,Suppe &              &           & während\\\hline
%
Verb     & V         & koch, ess & Adverb       & Adv       & oft \\
         &           & schlaf    &              &           & gestern\\ \hline
%
Adjektiv & Adj       & schnell   & Konjunktion  & C         & dass,\\
         &           & blau      &              &           & weil,\\
         &           &           &              &           & und, oder\\\hline
%
Artikel  & D         & der, ein  & Interjektion & Int       & tja, pst, Hurra!\\\hline 
         &           &           & Partikel     & Part      & auf, an (mit Verb)\\
         &           &           &              &           & nur (drei Tage)\\\hline
\end{tabular}

\vfill

}

\subsubsubsection{Nicht flektierbare Wortarten}

\frame{
\frametitle{Nicht flektierbare Wortarten: Präpositionen}

\begin{itemize}


\item Nicht flektierbare können wir anhand ihrer syntaktischen Umgebung unterscheiden:


\alert{Präpositionen} werden mit einer Nominalgruppe kombiniert und bestimmen deren Kasus.
\eal
\ex \alert{auf} dem Sofa
\ex \alert{während} des Treffens
\zl

\pause
Präpositionalgruppen können sich auf Verben oder Nomina beziehen:
\eal
\ex die Zeitung auf dem Sofa
\ex Er schläft auf dem Sofa.
\zl

\end{itemize}

}

\frame{
\frametitle{Nicht flektierbare Wortarten: Konjunktionen}

\begin{itemize}
\item \alert{Konjunktionen} verbinden Teilsätze miteinander (\mex{1}a) oder ordnen Teilsätze einem Verb
unter (\mex{1}b):
\eal
\ex Er kommt später, \alert{weil} er noch arbeiten muss.
\ex Er glaubt, \alert{dass} er es noch schafft.
\zl

\pause
\item Auch in sogenannten Koordinationen kommen Konjunktionen vor:
\eal
\ex Er kennt \alert{und} liebt diese Schallplatte.
\ex Die Musik \alert{und} der Text ist von Frank Zappa.
\zl



\end{itemize}

}

\frame{
\frametitle{Nicht flektierbare Wortarten: Adverbien}

\alert{Adverbien} haben mehrere Funktionen.

\begin{itemize}
\item Sie modifizieren Verben (daher der Name):
\eal
\ex Max lacht \alert{oft}.
\ex Er kam \alert{gestern}.
\zl

\pause
\item Aber auch die Modifikation von Adjektiven ist möglich:
\eal
\ex das oft gelesene Buch
\ex das gestern gekaufte Buch
\zl

\pause
\item Vorsicht: Viele Adjektive können adverbial verwendet werden:
\ea
Er hat das Buch \alert{schnell} gelesen.
\z

\end{itemize}


}


\frame{
\frametitle{Nicht flektierbare Wortarten: Partikeln}

\begin{itemize}
\item Der Duden \citeyearpar{Duden2005} unterscheidet zwischen Adverbien und Partikeln.
\item \alert{Partikeln} sind wie Adverbien nicht flektierbar,\\
im Gegensatz zu Adverbien aber nicht voranstellbar:
\eal
\ex[]{
Max lacht oft.
}
\ex[]{
Oft lacht Max. (Adverb)
}
\zl
\eal
\ex[]{
Max hat sogar gelacht.
}
\ex[*]{
Sogar hat Max gelacht. (Partikel)
}
\zl
\end{itemize}

}

\frame{
\frametitle{Nicht flektierbare Wortarten: Interjektionen}

\begin{itemize}
\item  Interjektionen sind satzwertige Ausdrücke:

\begin{itemize}
\item Interjektionen im Gespräch:
\ea
Ja! Jawohl! Nein! Doch! Bitte! Danke! Servus!
              Adieu! Tschüs! Halt! Stopp! Marsch! Pst! He! Hallo!
\z

\pause
\item Interjektionen als Ausdruck von Empfindungen: 
\ea Hurra! Juchhe! Heißa! Ei!
              Bravo! Pfui! Ach! Oh! O weh! Ah! Hahaha! Potz! Hu! Hui! Iiiiii! Ätsch! Aha!
              Hm! Brrr!
\z
\pause
\item Tier- und Geräuschnachahmungen:  
\ea
Muh! Miau! Wauwau! Quak! Kikeriki!
              Knacks! Trara! Kling, klang! Piff, paff! Klipp, klapp! Plumps! Blabla!
\z
\end{itemize}
\end{itemize}

}




\subsubsubsection{Flektierbare Wortarten}

%\subsubsubsubsection{Nomina und Artikel}

\frame{
\frametitle{Nomina}


\begin{itemize}[<+->]
\item Deutsche Nomina haben ein Genus (maskulin, feminin, neutrum).
\item Es gibt keine Beziehung zwischen Bedeutung und Genus (außer bei Personenbezeichnungen).
\item Genus ändert sich nicht in Abhängigkeit vom syntaktischen Kontext.\\
      Bezeichnung: \alert{inheränte Flexionskategorie}.
\item Abhängig vom Kontext Flexion nach Numerus (singular, Plural) und Kasus (Nominativ, Genitiv,
Dativ, Akkusativ).

\medskip


\begin{tabular}{|l|l|l|l|l|l|l|}\hline
          & \multicolumn{3}{l|}{Singular} & \multicolumn{3}{l|}{Plural}\\\hline
Nominativ & Tisch   & Suppe & Haus   & Tische  & Suppen & Häuser\\\hline
Genitiv   & Tisches & Suppe & Hauses & Tische  & Suppen & Häuser\\\hline
Dativ     & Tisch   & Suppe & Haus   & Tischen & Suppen & Häusern\\\hline
Akkusativ & Tisch   & Suppe & Haus   & Tische  & Suppen & Häuser\\\hline
\end{tabular}
\end{itemize}

}

\author{Stefan Müller}

\frame[shrink]{
\frametitle{Pronomina und Artikelwörter}


\begin{itemize}
\item Pronomina und Artikelwörter bilden eine Restkategorie.
\pause
\item Der Begriff \emph{Pronomen} kommt aus der Grammatik des Latein und steht
      traditionell sowohl für Artikel als auch für Wörter, die ganze Nominalgruppen ersetzen.
\pause
\item Das war sinnvoll, denn die Formen waren identisch.

Sie haben sich aber historisch auseinanderentwickelt.

\pause
\item Statt \emph{Pronomen} im obigen Sinn verwenden Grammatiken die stärker differenzierenden Begriffe \alert{Stellvertreter} und \alert{Begleiter}.

\pause
\item \alert{Artikel}/\alert{Determinator}: Element, das mit Nomen bzw.\ Adjektiven eine Nominalgruppe bildet
\pause
\item \alert{Pronomen}: Element, das für eine Nominalgruppe steht.

Zu den Pronomina werden auch die sogenannten Pronominaladverbien gezählt (\emph{darüber}, \emph{damit}, \ldots).

Diese stehen für Präpositionalgruppen (\emph{über dem Tisch}).
\end{itemize}

}

% \frame{
% \frametitle{Unterschied: definiter Artikel und Demonstrativpronomen}


% Die Formen des definiten Artikels und des Demonstrativpronomens sind meistens identisch,
% jedoch nicht immer:

% \medskip

% \begin{tabular}{|l|l|l|}\hline
% Kasus & Definiter Artikel & Demonstrativpronomen \\\hline
% Nom & der Mann   & der\\
% Gen & \alert{des} Mannes & \alert{dessen}\\
% Dat & dem Mann   & dem\\
% Akk & den Mann   & den\\\hline
% Nom & die Manner & die\\
% Gen & \alert{der} Männer & \alert{derer}\\
% Dat & \alert{den} Männern & \alert{denen}\\
% Akk & die Männer  & die\\\hline
% \end{tabular}


% }


\frame[shrink]{
\frametitle{Artikel/Determinator}

\begin{itemize}
\item Artikel stehen vor Nomina (oder Adjektiven) und bestimmen Definitheit:
\eal
\ex das/dieses/jenes Haus
\ex ein/kein Haus
\ex einige/mehrere Häuser
\zl

\pause
\item Klassisch: \alert{definiter Artikel} = \emph{der}, \emph{die}, \emph{das}
      \alert{indefiniter Artikel} = \emph{ein}

Duden-Grammatik nennt \emph{etwas}, \emph{nichts}, \emph{einige} \alert{indefinite Artikelwörter}
\eal
\ex etwas Farbe
\ex nichts Süßes
\ex einige Minuten
\ex alle Leute
\ex irgendwelche Kollegen
\zl

\end{itemize}

}

\author{Stefan Müller (Anke Lüdeling)}

\frame{
\frametitle{Artikel/Determinator: Flexionskategorien}

\begin{itemize}
\item Artikel haben dieselben Flexionskategorien wie Nomina.


\medskip


\begin{tabular}{|l|l|l|l|l|}\hline
          & \multicolumn{3}{l|}{Singular} & Plural\\\hline
Nominativ & der & die & das & die\\\hline
Genitiv   & des & der & des & der\\\hline
Dativ     & dem & der & dem & den\\\hline
Akkusativ & den & die & das & die\\\hline
\end{tabular}
\end{itemize}

}

\frame{
\frametitle{Synkretismus}

\begin{itemize}
\item Die Pluralformen sind für alle drei Genera identisch:

\medskip

\begin{tabular}{|l|l|l|l|l|}\hline
          & \multicolumn{3}{l|}{Singular} & Plural\\\hline
Nominativ & der & die & das & die\\\hline
Genitiv   & des & der & des & der\\\hline
Dativ     & dem & der & dem & den\\\hline
Akkusativ & den & die & das & die\\\hline
\end{tabular}

~

\pause
\item Auch im nominalen Paradigma fallen viele Formen zusammen.

Diesen Zusammenfall von Formen nennt man \alert{Synkretismus}.

\pause
\item Kasus lässt sich nicht eindeutig von der Form ablesen.

Kombination der Information von Artikel und Nomen hilft mitunter:
\eal
\ex der Tisch
\ex dem Tisch
\zl
\end{itemize}

}

\author{Stefan Müller}

\frame{
\frametitle{Synkretismus und Sexismus}

\begin{itemize}
\item Das hilft aber bei femininen Nomina nicht:
\ea
die Tochter (Nominativ oder Akkusativ)
\z
\pause

In Beispielen werden deshalb oft maskuline Nomina verwendet.

Kein Sexismus, sondern Vermeidung von Mehrdeutigkeit.

\pause
\item Meist hilft der Kontext, die Abfolge der Nominalgruppen im Satz oder die Prosodie:
\eal
\ex Den Vater liebt die Tochter nicht. Die Mutter liebt die Tochter.
\ex Die Mutter liebt den Sohn nicht. Die Mutter liebt die Tochter.
\zl

\end{itemize}

}


%{Pronomina}
\frame{
\frametitle{Pronomina -- I}

\begin{itemize}
\item \alert{Personalpronomen} (persönliche Fürwörter):\\
\emph{ich}, \emph{du}, \emph{er}, \emph{sie}, \emph{es}, \emph{wir}, \emph{ihr}, \emph{sie}

\pause
\item \alert{Possessivpronomen} (besitzanzeigende Fürwörter):\\
\emph{mein}, \emph{dein}, \emph{sein}, \emph{unser}, \emph{euer}, \emph{ihr}

\pause
\item \alert{Reflexivpronomen} (rückbezügliche Fürwörter):\\
\emph{mich}, \emph{dich}, \emph{sich}, \emph{uns}, \emph{euch}

\ea
Ich erhole \alert{mich}.
\z

\pause
\medskip
Reflexiv gebrauchtes Personalpronomen: auch Dativformen

\eal
\ex Ich wasche \alert{mich}.
\ex Ich wasche \alert{mir} den Rücken.
\zl

\pause
\alert{Reziprokpronomen} (wechselseitige Fürwörter): \emph{einander}

\end{itemize}


}


\frame{
\frametitle{Pronomina -- II}

\begin{itemize}

\item \alert{Demonstrativpronomen} (hinweisende Fürwörter):\\

    \emph{der}, \emph{dieser}, \emph{jener}, \emph{derjenige}, \emph{derselbe},\\
    \emph{die}, \emph{diese}, \emph{jene}, \emph{diejenige}, \emph{dieselbe},\\
    \emph{das}, \emph{dieses}, \emph{jenes}, \emph{dasjenige}, \emph{dasselbe}

\pause
\item \alert{Relativpronomen} (bezügliche Fürwörter):\\
\emph{der}, \emph{die}, \emph{das}, \emph{welcher}, \emph{welches}, \emph{welche},\\

\emph{wer}, \emph{was} (in freien Relativsätzen)

\pause
\item \alert{Interrogativpronomen} (fragende Fürwörter):\\
\emph{wer}, \emph{was}, \emph{welcher}

\pause
Frageadverbien auch hier einordnen? \emph{wofür}, \emph{womit}

\pause
\item \alert{Indefinitpronomen} (unbestimmte Fürwörter):\\
\emph{jemand}, \emph{alle}, \emph{einer}, \emph{keiner}, \emph{mancher}, \emph{man}, \emph{wer}, \emph{etwas}, \ldots

\end{itemize}


}

\author{Stefan Müller (Anke Lüdeling)}

\frame{
\frametitle{Adjektive: Flexionsklasse}

\begin{itemize}
\item Adjektive modifizieren Nomina (\mex{1}a) o.\ werden prädikativ verwendet (\mex{1}b):
\eal
\ex das rote Haus
\ex Das Haus ist rot.
\zl

\pause
\item Wie bei Nomina nach Kasus, Genus, Numerus unterschieden.
\pause
\item Zusätzlich Flexionsklasse: stark, schwach, gemischt:
\eal
\ex leckerer Auflauf, leckere Aufläufe\\ (ohne Artikel = stark)
\pause
\ex der leckere Auflauf, die leckeren Aufläufe\\ (definit = schwach)
\pause
\ex ein leckerer Auflauf, einige leckere Aufläufe\\ (ein/kein = gemischt)
\zl
\end{itemize}

}


\frame{
\frametitle{Adjektive: Grad}

\begin{itemize}
\item Flexion nach Grad:
\begin{itemize}
\item Positiv: \emph{lecker}
\item Komparativ: \emph{leckerer}
\item Superlativ: \emph{am leckersten}
\end{itemize}
\pause
\item Das ganze Paradigma unter \url{http://www.canoo.net/}.
\end{itemize}


}

\frame{
\frametitle{Verben}

\begin{itemize}
\item Verben unterteilen sich in Vollverben, Hilfsverben (Auxiliare) und Modalverben.
\pause
\item Vollverben teilen sich in schwache (regelmäßige) und starke (unregelmäßige) auf.\\
      Stark vs.\ schwach unterscheidet sich von den Klassen bei Adjektiven.
\pause
\item Vollverben und Hilfsverben flektieren nach Person, Numerus, Tempus, Modus und Genus Verbi.
\pause
\item Person und Nummerus sind für den syntaktischen Kontext wichtig (Kongruenz):

\medskip

~
\begin{tabular}{@{}lll@{}}
           & Singular  & Plural\\
1.\ Person & ich lache & wir lachen\\
2.\ Person & du lachst & ihr lacht\\
3.\ Person & er/sie/es lacht & sie lachen\\
\end{tabular}
\end{itemize}

}


\frame{
\frametitle{Verben: Tempus}

\begin{itemize}
\item Tempus, Modus und Genus Verbi fügen semantische Information hinzu.

\pause
\item Vereinfacht: Tempus sagt etwas darüber aus, wann die Handlung stattfindet.
\ea
Er lachte / lacht / wird lachen.
\z
\pause
\item Allerdings kann Präsens auch in Sätzen benutzt werden,\\die die Vergangenheit oder Zukunft beschreiben:
\eal
\ex Napoleon wird 1769 in Ajaccio auf der Insel Korsika geboren.
\ex Kommt er gestern in die Küche
\ex Ich bringe den Müll morgen runter.
\zl

\pause
\item Es gibt morphologisch einfache Formen und zusammengesetzte mit Hilfsverb + Partizip/Infinitiv.

\end{itemize}

}



\frame{
\frametitle{Flexionsparadigma: schwaches Verb, Aktiv, Indikativ}

\oneline{\begin{tabular}{|l|l|l|l|l|l|l|}\hline
Person \& & Präsens & Präteri- & Perfekt & Plusquam- & Futur I & Futur II\\
Numerus   &         & tum      &         & perfekt   &         &         \\\hline
%
1.\ Sg    & koche   & kochte   & habe    & hatte     & werde   & werde \\
          &         &          & gekocht & gekocht   & kochen  & gekocht haben\\\hline
%
2.\ Sg    & kochst  & kochtest & hast    & hattest   & wirst   & wirst \\
          &         &          & gekocht & gekocht   & kochen  & gekocht haben\\\hline
%
3.\ Sg    & kocht   & kochte   & hat     & hatte     & wird    & wird\\
          &         &          & gekocht & gekocht   & kochen  & gekocht haben\\\hline
%
1.\ Pl    & kochen  & kochten  & haben   & hatten    & werden & werden\\
          &         &          & gekocht & gekocht   & kochen & gekocht haben\\\hline
%
2.\ Pl    & kocht   & kochtet  & habt    & hattet    & werdet & werdet\\
          &         &          & gekocht & gekocht   & kochen & gekocht haben\\\hline
%
3.\ Pl    & kochen  & kochten  & haben   & hatten    & werden & werden\\
          &         &          & gekocht & gekocht   & kochen & gekocht haben\\\hline
\end{tabular}}


}

\frame{
\frametitlefit{Flexionsschema: schwache Verben, Präsens, Indikativ, Aktiv}


\vfill

\hfill
\begin{tabular}{|l|l|l|}\hline
Person \& & \multicolumn{2}{c|}{Präsens}\\
Numerus   &  \multicolumn{2}{c|}{} \\\hline
%           
1.\ Sg    &        & \suffix{e}\\\cline{1-1}\cline{3-3}
%           
2.\ Sg    &        & \suffix{st}\\\cline{1-1}\cline{3-3}
%           
3.\ Sg    &        & \suffix{t}\\\cline{1-1}\cline{3-3}
%           
1.\ Pl    & Stamm  & \suffix{en}\\\cline{1-1}\cline{3-3}
%           
2.\ Pl    &        & \suffix{t}\\\cline{1-1}\cline{3-3}
%           
3.\ Pl    &        & \suffix{en}\\\hline
\end{tabular}\hfill\hfill\mbox{}

\vfill

}

\frame{
\frametitle{Modus}


\begin{itemize}
\item Verbmodus: Indikativ, Konjunktiv I, Konjunktiv II
\pause
\item Bedeutung unscharf, kann aber wie folgt umrissen werden:
\begin{itemize}
\item Indikativ teilt Faktum mit
\ea
Max schläft. (Ich habe es selbst gesehen.)
\z
\pause
\item Konjunktiv I: Man hat von etwas gehört.
\ea
Barbara sagt, Max schlafe. (Ich glaube Barbara.)
\z
\pause
\item Konjunktiv II: Man hat von etwas gehört und zweifelt es an.
\ea
Barbara sagt, Max schliefe. (Ich glaube Barbara nicht.)
\z
\end{itemize}
\end{itemize}


}


\frame{
\frametitle{Genus Verbi}




\begin{itemize}
\item Genus Verbi: Aktiv und Passiv

\eal
\ex Er schlägt den Weltmeister.
\ex Der Weltmeister wird geschlagen.
\zl
\pause
\item Passiv = Unterdrückung des Subjekts (Agens im weiteren Sinne)
\pause
\item Die häufigste Form des Passivs wird mit dem Hilfsverb \emph{werden} gebildet.

\end{itemize}

}

\frame{
\frametitle{Andere Verbformen}


\eal
\ex geben (Infinitiv)
\pause
\ex gebend (Partizip Präsens)
\pause
\ex gegeben (Partizip Perfekt)
\pause
\ex gib (Imperativ Singular)
\pause
\ex gebt (Imperativ Plural)
\zl


}

\frame{
\frametitle{Modalverben und \emph{wissen}}


\begin{itemize}
\item Modalverben (\emph{dürfen}, \emph{können}, \emph{mögen}, \emph{müssen}, \emph{sollen}, \emph{wollen}),\\
      und damit gebildete Präfix- oder Partikelverben (\emph{bedürfen}, \emph{durchmüssen})\\
      und das Verb \emph{wissen} verhalten sich etwas anders.
\item Im Präsens verwenden sie die Präteritumsendungen der starken Verben.

~

\vfill


\hfill
\begin{tabular}{|l|l|l|l|l|}\hline
Person \& & \multicolumn{2}{c|}{Präteritum starke Verben} & \multicolumn{2}{c|}{Präsens Modalverben}\\
Numerus   &  \multicolumn{2}{c|}{} & \multicolumn{2}{c|}{} \\\hline
%           
1.\ Sg    &                  & $\varnothing$    && $\varnothing$    \\\cline{1-1}\cline{3-3}\cline{5-5}
%                                                                                    
2.\ Sg    & Präteritumsstam  & \suffix{st}      & Stamm & \suffix{st}      \\\cline{1-1}\cline{3-3}\cline{5-5}
%                                                                                    
3.\ Sg    & \emph{kam}       & $\varnothing$    & \emph{darf}/\emph{dürf}& $\varnothing$    \\\cline{1-1}\cline{3-3}\cline{5-5}
%                                                                                    
1.\ Pl    & \emph{schlief}   & \suffix{en}      & \emph{will}/\emph{woll} & \suffix{en}      \\\cline{1-1}\cline{3-3}\cline{5-5}
%                                                                                    
2.\ Pl    &                  & \suffix{t}       && \suffix{t}       \\\cline{1-1}\cline{3-3}\cline{5-5}
%                                                                                    
3.\ Pl    &                  & \suffix{en}      && \suffix{en}      \\\hline
\end{tabular}\hfill\hfill\mbox{}

\vfill

\end{itemize}

}

\subsubsubsection{Überblick}

\author{Stefan Müller (Peter Gallmann)}

\frame{
\frametitle{Überblick über die Wortarten (Peter Gallmann/Duden)}

\vfill

% used to work without tabular, I do not know why it needs tabular here. Maybe order of loading of packages
\centerfit{%
\begin{forest}
word tier, for tree={fit=rectangle}
[Wortart
       [flektierbar
          [nach Tempus [Verb] ]
          [nach Kasus 
            [festes Genus [Noun] ]
            [veränderbares Genus 
               [nicht komparierbar [\begin{tabular}{@{}c@{}}Artikelwort\\Pronomen\end{tabular} ] ]
               [komparierbar [Adjektiv] ] ] ] ]
       [nicht flektierbar [\begin{tabular}{@{}c@{}}Adverb\\Konjunktion\\Präposition\\Interjektion\end{tabular}] ] ]
\end{forest}
}

\vfill

}

\author{Stefan Müller (Anke Lüdeling)}

\subsubsection{Form und Funktion}

\frame{
\frametitle{Form und Funktion: Portmanteau-Morpheme}

\begin{itemize}
\item Wortbildung: Jedes Morphem hat eine Funktion/Bedeutungsbeitrag:
\eal
\ex Haus+tür
\ex Stör+ung
\zl

\pause
\item Flexion: Mitunter fallen mehrere Funktionen zusammen:
\eal
\ex ich lache -- lachte
\ex er lacht -- lachte
\zl
Steht das \suffix{t} für Präteritum, wie (\mex{0}a) nahelegt?

\pause
Steht das \suffix{e} für Präteritum, wie (\mex{0}b) nahelegt?

\pause
\item \suffix{te} ist ein kombiniertes Affix,\\
      das sowohl Tempus- als auch Kongruenzinformation enthält.

Solche Morpheme werden \alert{Portmanteau-Morpheme} oder \alert{Schachtelmorpheme} genannt.

\end{itemize}

}

\frame{
\frametitle{Form und Funktion: mehrfache Exponenten}

\begin{itemize}
\item Bei Portmanteau-Morphemen werden mehrere Funktionen von einem Morphem wahrgenommen.
\pause
\item Aber es gibt auch Fälle, in denen eine Funktion sich an mehreren Stellen manifestiert.

Beispiel: bestimmte Nomina im Deutschen, die mit Suffix und Umlautung den Plural bilden:
\ea
Mann -- Männer
\z
\end{itemize}

}



\frame{
\frametitle{Inhärente Flexion, regierte Flexion und Kongruenz}

\begin{itemize}
\item Flexion hilft bei der Bestimmung der Zusammengehörigkeit und Funktion von Elementen im Satz.
\item Können Flexionsinformation unterteilen in 
\begin{itemize}
\item inhärente Flexion, 
\item kontextabhängige Flexion,
\item regierte Flexion und 
\item Kongruenz

\end{itemize}

\end{itemize}

}

\frame{
\frametitle{Inhärente und kontextuelle Kategorien}

\begin{itemize}
\item Inhärente Flexionskategorien, \zb Genus bei Nomina oder Definitheit bei Artikeln: Diese
Informationen gehören zum Lexem, sie ändern sich nie. 

Sie können aber durchaus Auswirkungen auf andere Elemente in ihrer Umgebung haben.

\pause
\item Kontextuelle Kategorien, \zb Modus oder Tempus bei Verben. Solche Kategorien sind nicht durch
die Syntax vorgegeben, sondern durch das Informationsziel. 

\end{itemize}



}

\frame{
\frametitle{Regierte Flexion}

Regierte Kategorien, \zb Kasus bei nominalen Konstituenten in einer präpositionalen
Konstituente. 
\ea
in einem Korb
\z

Regierendes Element (\emph{in}) verlangt Dativ, steht aber nicht selbst im Dativ.

Durch Rektion wird Abhängigkeit aufgezeigt:\\
Alles, was von der Präposition abhängt, muss im Dativ stehen.
}

\frame{
\frametitle{Kongruenz}

Kongruenz: Ein Element stimmt mit anderen Elementen in seiner Umgebung in einem oder mehreren
Merkmalen überein. 

\eal
\ex Max lacht.
\ex Max und Friederike lachen.
\zl
\eal
\ex ein gutes Ergebnis
\ex das gute Ergebnis
\ex des guten Ergebnisses
\zl

}

\subsection{Übung: Derivation, Komposition, Flexion}

\frame{
\frametitle{Übung}

Analysieren Sie:
\eal
\ex Vorlesungsankündigung
\ex Straßenbahnhaltestelle
\ex Kinderschlafsack
\ex Kinderschreibtische
\zl

}

\iftoggle{ue-loesung}{
	%%%%%%%%%%%%%%%%%%%%%%%%%%%%%%%%%%
%% UE 1 - 05b Morphologie Stefan
%%%%%%%%%%%%%%%%%%%%%%%%%%%%%%%%%%

\frame{
\frametitle{Lösung: Vorlesungsankündigung}

\centerline{
\begin{forest}
sm edges
[N
  [N
    [N 
      [V 
        [Part [vor]]
        [V [les]]]
      [N-Aff [ung-s]]]
    [N [V [Part [an]]
          [V [kündig]]]
       [N-Aff [ung]]]]
  [Flex [$\varnothing$]]]
\end{forest}
}

\emph{kündigen}: mhd. für `mitteilen, künden'

}


\frame{
\frametitle{Lösung: Straßenbahnhaltestelle}

\centerline{%
\begin{forest}
sm edges
[N 
  [N
    [N
      [N [Straße-n]]
      [N [bahn]]]
    [N [V [halt-e]]
       [N [stelle]]]]
  [Flex [$\varnothing$]]]
\end{forest}}

}

\frame{
\frametitle{Lösung: Kinderschlafsack}

\centerline{%
\begin{forest}
sm edges
[N
  [N [N [kind-er]]
     [N
       [V [schlaf]]
       [N [sack]]]]
  [Flex [$\varnothing$]]]
\end{forest}
}

}

\frame{
\frametitle{Lösung: Kinderschreibtische}

\centerline{%
\begin{forest}
sm edges
[N
  [N [N [kind-er]]
     [N
       [V [schreib]]
       [N [tisch]]]]
  [Flex [e]]]
\end{forest}
}

}



}


%%%%%%%%%%%%%%%%%%%%%%%%%%%%%%%%%%%%%%%%%%%
%%%%%%%%%%%%%%%%%%%%%%%%%%%%%%%%%%%%%%%%%%%
