%%%%%%%%%%%%%%%%%%%%%%%%%%%%%%%%%%%%%%%%%%%%%%
%% Compile: XeLaTeX BibTeX XeLaTeX XeLaTeX
%% Loesung-Handout: Antonio Machicao y Priemer
%% Course: GK Linguistik
%%%%%%%%%%%%%%%%%%%%%%%%%%%%%%%%%%%%%%%%%%%%%%

%\documentclass[a4paper,10pt, bibtotoc]{beamer}
\documentclass[10pt,handout]{beamer}

%%%%%%%%%%%%%%%%%%%%%%%%
%%     PACKAGES      
%%%%%%%%%%%%%%%%%%%%%%%%

%%%%%%%%%%%%%%%%%%%%%%%%
%%     PACKAGES       %%
%%%%%%%%%%%%%%%%%%%%%%%%



%\usepackage[utf8]{inputenc}
%\usepackage[vietnamese, english,ngerman]{babel}   % seems incompatible with german.sty
%\usepackage[T3,T1]{fontenc} breaks xelatex
\usepackage{lmodern}

\usepackage{amsmath}
\usepackage{amsfonts}
\usepackage{amssymb}
%% MnSymbol: Mathematische Klammern und Symbole (Inkompatibel mit ams-Packages!)
%% Bedeutungs- und Graphemklammern: $\lsem$ Tisch $\rsem$ $\langle TEXT \rangle$ $\llangle$ TEXT $\rrangle$ 
\usepackage{MnSymbol}
%% ulem: Strike out
\usepackage[normalem]{ulem}  

%% Special Spaces (s. Commands)
\usepackage{xspace}				
\usepackage{setspace}
%	\onehalfspacing

%% mdwlist: Special lists
\usepackage{mdwlist}	

\usepackage[noenc,safe]{tipa}

% maybe define \textipa to use \originalTeX to avoid problems with `"'.
%
%	\ex \textipa{\originalTeX [pa.pa."g\t{aI}]}

%

\usepackage{etex}		%For Forest bug

%
%\usepackage{jambox}
%


%\usepackage{forest-v105}
%\usepackage{modified-langsci-forest-setup}

\usepackage{xeCJK}
\setCJKmainfont{SimSun}


%\usepackage{natbib}
%\setcitestyle{notesep={:~}}




% for toggles
\usepackage{etex}



% Fraktur!
\usepackage{yfonts}

\usepackage{url}

% für UDOP
\usepackage{adjustbox}


%% huberlin: Style sheet
%\usepackage{huberlin}
\usepackage{hu-beamer-includes-pdflatex}
\huberlinlogon{0.86cm}


%% Last Packages
%\usepackage{hyperref}	%URLs
%\usepackage{gb4e}		%Linguistic examples

% sorry this was incompatible with gb4e and had to go.
%\usepackage{linguex-cgloss}	%Linguistic examples (patched version that works with jambox

\usepackage{multirow}  %Mehrere Zeilen in einer Tabelle
%\usepackage{array}
\usepackage{marginnote}	%Notizen




%%%%%%%%%%%%%%%%%%%%%%%%%%%%%%%%%%%%%%%%%%%%%%%%%%%%
%%%          Commands                            %%%
%%%%%%%%%%%%%%%%%%%%%%%%%%%%%%%%%%%%%%%%%%%%%%%%%%%%

%%%%%%%%%%%%%%%%%%%%%%%%%%%%%%%%
% German quotation marks:
\newcommand{\gqq}[1]{\glqq{}#1\grqq{}}		%double
\newcommand{\gq}[1]{\glq{}#1\grq{}}			%simple


%%%%%%%%%%%%%%%%%%%%%%%%%%%%%%%%
% Abbreviations in German
% package needed: xspace
% Short space in German abbreviations: \,	
\newcommand{\idR}{\mbox{i.\,d.\,R.}\xspace}
\newcommand{\su}{\mbox{s.\,u.}\xspace}
%\newcommand{\ua}{\mbox{u.\,a.}\xspace}       % in abbrev
%\newcommand{\zB}{\mbox{z.\,B.}\xspace}       % in abbrev
%\newcommand{\s}{s.~}
%not possibel: \dh --> d.\,h.


%%%%%%%%%%%%%%%%%%%%%%%%%%%%%%%%
%Abbreviations in English
\newcommand{\ao}{a.o.\ }	% among others
\newcommand{\cf}[1]{(cf.~#1)}	% confer = compare
\renewcommand{\ia}{i.a.}	% inter alia = among others
\newcommand{\ie}{i.e.~}	% id est = that is
\newcommand{\fe}{e.g.~}	% exempli gratia = for example
%not possible: \eg --> e.g.~
\newcommand{\vs}{vs.\ }	% versus
\newcommand{\wrt}{w.r.t.\ }	% with respect to


%%%%%%%%%%%%%%%%%%%%%%%%%%%%%%%%
% Dash:
\newcommand{\gs}[1]{--\,#1\,--}


%%%%%%%%%%%%%%%%%%%%%%%%%%%%%%%%
% Rightarrow with and without space
\def\ra{\ensuremath\rightarrow}			%without space
\def\ras{\ensuremath\rightarrow\ }		%with space


%%%%%%%%%%%%%%%%%%%%%%%%%%%%%%%%
%% X-bar notation

%% Notation with primes (not emphasized): \xbar{X}
\newcommand{\MyPxbar}[1]{#1$^{\prime}$}
\newcommand{\xxbar}[1]{#1$^{\prime\prime}$}
\newcommand{\xxxbar}[1]{#1$^{\prime\prime\prime}$}

%% Notation with primes (emphasized): \exbar{X}
\newcommand{\exbar}[1]{\emph{#1}$^{\prime}$}
\newcommand{\exxbar}[1]{\emph{#1}$^{\prime\prime}$}
\newcommand{\exxxbar}[1]{\emph{#1}$^{\prime\prime\prime}$}

% Notation with zero and max (not emphasized): \xbar{X}
\newcommand{\zerobar}[1]{#1$^{0}$}
\newcommand{\maxbar}[1]{#1$^{\textsc{max}}$}

% Notation with zero and max (emphasized): \xbar{X}
\newcommand{\ezerobar}[1]{\emph{#1}$^{0}$}
\newcommand{\emaxbar}[1]{\emph{#1}$^{\textsc{max}}$}

%% Notation with bars (already implemented in gb4e):
% \obar{X}, \ibar{X}, \iibar{X}, \mbar{X} %Problems with \mbar!
%
%% Without gb4e:
\newcommand{\overbar}[1]{\mkern 1.5mu\overline{\mkern-1.5mu#1\mkern-1.5mu}\mkern 1.5mu}
%
%% OR:
\newcommand{\MyPibar}[1]{$\overline{\textrm{#1}}$}
\newcommand{\MyPiibar}[1]{$\overline{\overline{\textrm{#1}}}$}
%% (emphasized):
\newcommand{\eibar}[1]{$\overline{#1}$}
\newcommand{\eiibar}[1]{\overline{$\overline{#1}}$}

%%%%%%%%%%%%%%%%%%%%%%%%%%%%%%%%
%% Subscript & Superscript: no italics
\newcommand{\MyPdown}[1]{$_{\textrm{#1}}$}
\newcommand{\MyPup}[1]{$^{\textrm{#1}}$}


%%%%%%%%%%%%%%%%%%%%%%%%%%%%%%%%
% Objekt language marking:
%\newcommand{\obj}[1]{\glqq{}#1\grqq{}}	%German double quotes
%\newcommand{\obj}[1]{``#1''}			%English double quotes
\newcommand{\MyPobj}[1]{\emph{#1}}		%Emphasising


%%%%%%%%%%%%%%%%%%%%%%%%%%%%%%%%
%% Semantic types (<e,t>), features, variables and graphemes in angled brackets 

%%% types and variables, in math mode: angled brackets + italics + no space
%\newcommand{\type}[1]{$<#1>$}

%%% OR more correctly: 
%%% types and variables, in math mode: chevrons! + italics + no space
\newcommand{\MyPtype}[1]{$\langle #1 \rangle$}

%%% features and graphemes, in math mode: chevrons! + italics + no space
\newcommand{\abe}[1]{$\langle #1 \rangle$}


%%% features and graphemes, in math mode: chevrons! + no italics + space
\newcommand{\ab}[1]{$\langle$#1$\rangle$}  %%same as \abu  
\newcommand{\abu}[1]{$\langle$#1$\rangle$} %%Umlaute

%%% Notizen
\renewcommand{\marginfont}{\singlespacing}
\renewcommand{\marginfont}{\footnotesize}
\renewcommand{\marginfont}{\color{black}}

\newcommand{\myp}[1]{%
	\marginnote{%
		\begin{spacing}{1}
			\vspace{-\baselineskip}%
			\color{red}\footnotesize#1
		\end{spacing}
	}
}
%%%%%%%%%%%%%%%%%%%%%%%%%%%%%%%%
%% Outputbox
\newcommand{\outputbox}[1]{\noindent\fbox{\parbox[t][][t]{0.98\linewidth}{#1}}\vspace{0.5em}}

%%%%%%%%%%%%%%%%%%%%%%%%%%%%%%%%
%% (Syntactic) Trees
% package needed: forest
%
%% Setting for simple trees
\forestset{
	MyP edges/.style={for tree={parent anchor=south, child anchor=north}}
}

%% this is taken from langsci-setup file
%% Setting for complex trees
%% \forestset{
%% 	sn edges/.style={for tree={parent anchor=south, child anchor=north,align=center}}, 
%% background tree/.style={for tree={text opacity=0.2,draw opacity=0.2,edge={draw opacity=0.2}}}
%% }

\newcommand\HideWd[1]{%
	\makebox[0pt]{#1}%
}


%%%%%%%%%%%%%%%%%%%%%%%%%%%%%%%%%%%%%%%%%%%%%%%%%%%%
%%%          Useful commands                     %%%
%%%%%%%%%%%%%%%%%%%%%%%%%%%%%%%%%%%%%%%%%%%%%%%%%%%%

%%%%%%%%%%%%%%%%%%%%%
%% FOR ITEMS:
%\begin{itemize}
%  \item<2-> from point 2
%  \item<3-> from point 3 
%  \item<4-> from point 4 
%\end{itemize}
%
% or: \onslide<2->
% or: \pause

%%%%%%%%%%%%%%%%%%%%%
%% VERTICAL SPACE:
% \vspace{.5cm}
% \vfill

%%%%%%%%%%%%%%%%%%%%%
% RED MARKING OF TEXT:
%\alert{bis spätestens Mittwoch, 18 Uhr}

%%%%%%%%%%%%%%%%%%%%%
%% RESCALE BIG TABLES:
%\scalebox{0.8}{
%For Big Tables
%}

%%%%%%%%%%%%%%%%%%%%%
%% BLOCKS:
%\begin{alertblock}{Title}
%Text
%\end{alertblock}
%
%\begin{block}{Title}
%Text
%\end{block}
%
%\begin{exampleblock}{Title}
%Text
%\end{exampleblock}


\newtoggle{uebung}
\newtoggle{loesung}
\newtoggle{toc}

% The toc is not needed on Handouts. Safe trees.
\mode<handout>{
\togglefalse{toc}
}

\newtoggle{hpsgvorlesung}\togglefalse{hpsgvorlesung}
\newtoggle{syntaxvorlesungen}\togglefalse{syntaxvorlesungen}

%\includecomment{psgbegriffe}
%\excludecomment{konstituentenprobleme}
%\includecomment{konstituentenprobleme-hinweis}

\newtoggle{konstituentenprobleme}\togglefalse{konstituentenprobleme}
\newtoggle{konstituentenprobleme-hinweis}\toggletrue{konstituentenprobleme-hinweis}

%\includecomment{einfsprachwiss-include}
%\excludecomment{einfsprachwiss-exclude}
\newtoggle{einfsprachwiss-include}\toggletrue{einfsprachwiss-include}
\newtoggle{einfsprachwiss-exclude}\togglefalse{einfsprachwiss-exclude}

\newtoggle{psgbegriffe}\toggletrue{psgbegriffe}

\newtoggle{gb-intro}\togglefalse{gb-intro}



%%%%%%%%%%%%%%%%%%%%%%%%%%%%%%%%%%%%%%%%%%%%%%%%%%%%
%%%             Preamble's End                   
%%%%%%%%%%%%%%%%%%%%%%%%%%%%%%%%%%%%%%%%%%%%%%%%%%%% 

\begin{document}
	

%%%% ue-loesung
%%%% true: Übung & Lösungen (slides) / false: nur Übung (handout)
%	\toggletrue{ue-loesung}

%%%% ha-loesung
%%%% true: Hausaufgabe & Lösungen (slides) / false: nur Hausaufgabe (handout)
%	\toggletrue{ha-loesung}

%%%% toc
%%%% true: TOC am Anfang von Slides / false: keine TOC am Anfang von Slides
\toggletrue{toc}

%%%% sectoc
%%%% true: TOC für Sections / false: keine TOC für Sections (StM handout)
%	\toggletrue{sectoc}

%%%% gliederung
%%%% true: Gliederung für Sections / false: keine Gliederung für Sections
%	\toggletrue{gliederung}


%%%%%%%%%%%%%%%%%%%%%%%%%%%%%%%%%%%%%%%%%%%%%%%%%%%%
%%%             Metadata                         
%%%%%%%%%%%%%%%%%%%%%%%%%%%%%%%%%%%%%%%%%%%%%%%%%%%%      

\title{Grundkurs Linguistik}

\subtitle{Lösungen -- Phonetik}

\author[A. Machicao y Priemer]{
	{\small Antonio Machicao y Priemer}
	\\
	{\footnotesize \url{http://www.linguistik.hu-berlin.de/staff/amyp}}
	%	\\
	%	\href{mailto:mapriema@hu-berlin.de}{mapriema@hu-berlin.de}}
}

\institute{Institut für deutsche Sprache und Linguistik}


% bitte lassen, sonst kann man nicht sehen, von wann die PDF-Datei ist.
%\date{ }

%\publishers{\textbf{6. linguistischer Methodenworkshop \\ Humboldt-Universität zu Berlin}}

%\hyphenation{nobreak}


%%%%%%%%%%%%%%%%%%%%%%%%%%%%%%%%%%%%%%%%%%%%%%%%%%%%
%%%             Preamble's End                  
%%%%%%%%%%%%%%%%%%%%%%%%%%%%%%%%%%%%%%%%%%%%%%%%%%%%      


%%%%%%%%%%%%%%%%%%%%%%%%%      
\huberlintitlepage[22pt]
\iftoggle{toc}{
	\frame{
		\begin{multicols}{2}
		\frametitle{Inhaltsverzeichnis}
		\tableofcontents
		%[pausesections]
	\columnbreak
	\textcolor{white}{
		\ea\label{ex:02sth}
		\ex\label{ex:02HA1}
		\ex\label{ex:02HA2}
		\ex\label{ex:02HA3}
		\z}
		\end{multicols}
	}
}


%%%%%%%%%%%%%%%%%%%%%%%%%%%%%%%%%%%
%%%%%%%%%%%%%%%%%%%%%%%%%%%%%%%%%%%
\section{Übungen}

%%%%%%%%%%%%%%%%%%%%%%%%%%%%%%%%%%
%% UE 1 - 02 Phonetik
%%%%%%%%%%%%%%%%%%%%%%%%%%%%%%%%%%

\begin{frame}
\frametitle{Übung -- Lösung}

Wie viele Laute haben die folgenden Wörter?

\begin{columns}
	\column{.30\textwidth}
	\begin{enumerate}
		\item \ab{Fische}
		\item \ab{Nixe}
		\item \ab{lang}
		\item \ab{Bearbeitung}
		\item[]
	\end{enumerate} 				
	\column{.35\textwidth}
	\begin{enumerate}
		\item<1-> \textipa{[ f \textsci{} \textesh{} @ ]}
		\item<3-> \textipa{[ n \textsci{} k s @ ]}
		\item<5-> \textipa{[ l a N ]}
		\item<7-> \textipa{[ b @ P a \textscr\  b \t{aI} t U N ]}
		\item<9->[] \textipa{[ b @ P a:  b \t{aI} t U N ]}
	\end{enumerate} 
	\column{.15\textwidth}
	\begin{enumerate}
		\item<2->[] 4
		\item<4->[] 5
		\item<6->[] 3
		\item<8->[] 10--11 % bearbeitung ai als ein
		% Laut oder als zwei gezählt
		\item<10->[] 9--10 % beabeitung ai als ein
		% Laut oder als zwei gezählt
	\end{enumerate}
\end{columns}


\bigskip
\begin{itemize}
	\item[]<11-> \textipa{[\t{aI}]} kann man als einen oder als zwei Laute zählen.
\end{itemize}         

\end{frame}


%%%%%%%%%%%%%%%%%%%%%%%%%%%%%%%%%%
%% UE 2 - 02 Phonetik
%%%%%%%%%%%%%%%%%%%%%%%%%%%%%%%%%%

\begin{frame}
\frametitle{Übung -- Lösung}

Welche der folgenden Laute sind stimmhaft und welche stimmlos?

\begin{exe}
\exr{ex:02sth} \textipa{[ d, z, f, v, g, k, P ]}
\end{exe}


\begin{multicols}{2}
	\begin{itemize}
		\item \textbf{stimmhaft:} \textipa{[ d, z, v, g ]}
		
		\ea \textipa{[ d ]}: \ab{Dampf}
		
		\ex \textipa{[ z ]}: \ab{Sinn}
		
		\ex \textipa{[ v ]}: \ab{Wald}
		
		\ex \textipa{[ g ]}: \ab{ganz}
		
		\z 
	\end{itemize}
	
	\columnbreak \pause
	
	\begin{itemize}
		\item \textbf{stimmlos:} \textipa{[ f, k, P ]}
		
		\ea \textipa{[ f ]}: \ab{Fass}
		
		\ex \textipa{[ k ]}: \ab{kalt}
		
		\ex \textipa{[ P ]}: \ab{vereisen}
		
		\z
	\end{itemize}
	
\end{multicols}

\end{frame}



%%%%%%%%%%%%%%%%%%%%%%%%%%%%%%%%%%
%% UE 3 - 02 Phonetik
%%%%%%%%%%%%%%%%%%%%%%%%%%%%%%%%%%

\begin{frame}
\frametitle{Übung -- Lösung}

\begin{table}
\scalebox{.9}{
	\begin{tabular}{llp{9cm}}
		1. Busch & \only<2->{[\textipa{bUS}]} & \only<3->{\textipa{b}: bilabialer, stimmhafter Plosiv; \textipa{S}:~postalveolarer, stimmloser Frikativ} \\
		
		2. malen & \only<4->{[\textipa{\textprimstress ma:l@n}]} & \only<5->{\textipa{m}:~bilabialer, stimmhafter Nasal; \textipa{n}:~alveolarer, stimmhafter Nasal, \textipa{l}:~alveolar, stimmhafter Lateral} \\
		
		3. Maus & \only<6->{[\textipa{m\texttoptiebar{aU}s}]} & \only<7->{\textipa{m}:~s.\,o.; \textipa{s}:~stimmloser, alveolarer Frikativ} \\
		
		4. Achtung & \only<8->{[\textipa{\textprimstress PaXtU\ng}]} & \only<9->{\textipa{P}:~glottaler, stimmloser Plosiv; \textipa{X}:~velarer, stimmloser Frikativ; \textipa{t}:~alveolarer, stimmloser Plosiv \textipa{\ng}: velarer, stimmhafter Nasal} \\

		5. Genie & \only<10->{[\textipa{Ze:\textprimstress ni:}]} & \only<11->{\textipa{Z}: postalveolarer, stimmhafter Frikativ, \textipa{n}:~s.\,o.} \\                
		                                                                                  
		6. zirpen & \only<12->{[\textipa{\texttoptiebar{ts}IKp@n}]} & \only<13->{\textipa{\texttoptiebar{ts}}: alveolare, stimmlose Affrikate; \textipa{K}:~uvularer, stimmhafter Frikativ; \textipa{p}:~bilabialer, stimmloser Plosiv; \textipa{n}: s.\,o.} \\

		7. wichtig & \only<14->{[\textipa{\textprimstress vI\c{c}tI\c{c}}]} & \only<15->{\textipa{v}: labiodentaler, stimmhafter Frikativ; \textipa{\c{c}}: palataler, stimmloser Frikativ \textipa{t}: s.\,o.} \\

		8. Wald & \only<16->{[\textipa{valt}]} & \only<17->{\textipa{v, l, t}: s.\,o.}
\end{tabular}
}
\end{table}

\end{frame}


%%%%%%%%%%%%%%%%%%%%%%%%%%%%%%%%%%
%% UE 4 - 02 Phonetik
%%%%%%%%%%%%%%%%%%%%%%%%%%%%%%%%%%

\begin{frame}
\frametitle{Übungen -- Lösung}

\begin{itemize}
	\item Bilden die folgenden Vokalabfolgen Diphthonge?
	\item[] Zeit, naiv, Haus
	
	\item \textcolor{red}{Ja: \textipa{[\texttoptiebar{ts}\texttoptiebar{aI}t]} , \textipa{[h\texttoptiebar{aU}s]}}
	\item \textcolor{red}{Nein: \textipa{[na.Pi:f]}}
\end{itemize}

\end{frame}


%%%%%%%%%%%%%%%%%%%%%%%%%%%%%%%%%%
%% UE 5 - 02 Phonetik
%%%%%%%%%%%%%%%%%%%%%%%%%%%%%%%%%%

\begin{frame}
\frametitle{Lösung: Transkription}

\begin{itemize}

\item Transkribieren Sie die folgenden Wörter nach einer standarddeutschen Aussprache:

%\begin{columns}
%	\column{.40\textwidth}
%	\begin{enumerate}
\ea 
\settowidth\jamwidth{XXXXXXXXXXXXXXXXXXXXXXXXXX}

		\ea Bergsteiger
		\loesung{1}{\textipa{[bE͡5k.St\t{aI}.g5]}}
		
		\ex Quotennote
		\loesung{2}{\textipa{[kvo:.t@n.no:.t@]}}
		
		\ex vielfaches
		\loesung{3}{\textipa{[fi:l.fa\.x@s]}}
		
		\ex Päckchenannahme
		\loesung{4}{\textipa{[pEk.\c{c}@n.Pan.na:.m@]}}
		
		\ex beenden
		\loesung{5}{\textipa{[b@.PEn.d@n]}}
		
		\ex verreisen
		\loesung{6}{\textipa{[fE͡5.\textscr \t{aI}.z@n]}}
		
		\ex vereisen
		\loesung{7}{\textipa{[fE͡5.P\t{aI}.z@n]}}
		
		\ex Einzahlung
		\loesung{8}{\textipa{[P\t{aI}n.\t{ts}a:.lUN]}}
		
		\ex gehen
		\loesung{9}{\textipa{[ge:.@n]}}
		
		\ex Gästebad
		\loesung{10}{\textipa{[gEs.t@.ba:t]}}
		
		\z 
\z		

\end{itemize}

\end{frame}



%%%%%%%%%%%%%%%%%%%%%%%%%%%%%%%%%%
%% UE 6 - 02 Phonetik
%%%%%%%%%%%%%%%%%%%%%%%%%%%%%%%%%%

\begin{frame}
\frametitle{Lösung: Text in IPA lesen}

\begin{itemize}
	\item {\footnotesize \textipa{aIns 'StKItn zI\c{c} \textprimstress nO5tvInt Un \textprimstress zOn@, v@5 f@n im \textprimstress baIdn vol d5 \textprimstress StE5k@K@ veK@, als aIn \textprimstress vand@K5, dE5 In aIn \textprimstress va5m \textprimstress mantl g@\textsecstress hylt va5, d@s \textprimstress veg@s da\textprimstress he5ka:m. zI vU5dn \textprimstress aInI\c{c}, das \textprimstress de5jenIg@ fy5 d@n \textprimstress StE5k@K@n \textsecstress gEltn zOlt@, dE5 d@n \textprimstress vand@K5 \textprimstress {tsvI\ng \ng}  {vy5d@}, zaIm \textprimstress mantl \textprimstress aptsU\textsecstress nemm. dE5 \textprimstress nO5tvIm \textprimstress blis mIt \textprimstress al5 \textprimstress maXt, ab5 je \textprimstress me5 E5 \textprimstress blis, dEsto \textprimstress fEst5 \textprimstress hylt@ zI\c{c} d5 \textprimstress vand@K5 In zaIm \textprimstress mantl aIn. \textprimstress EntlI\c{c} ga:p d5 \textprimstress nO5tvIn {d@\ng} \textprimstress kampf \textprimstress aUf. nun E5\textprimstress vE5mt@ dI \textprimstress zOn@ dI \textprimstress lUfp mIt i5n \textprimstress fKOIntlI\c{c}n \textprimstress StKa:ln, Un SonaX {\textprimstress venIg\ng} {\textprimstress aUg\ng \textsecstress blIk\ng} tsok d5 \textprimstress vand@K5 zaIm \textprimstress mantl aUs. da mUst@ d5 \textprimstress nO5tvIn \textprimstress tsugebm, das dI \textprimstress zOn@ f@n im \textprimstress baIdn d5 \textprimstress StE5k@K@ va5.}}
	
	\item \textcolor{red}{\footnotesize Einst stritten sich Nordwind und Sonne, wer von ihnen beiden wohl der Stärkere wäre, als ein Wanderer, der in einen warmen Mantel gehüllt war, des Weges daherkam. Sie wurden einig, dass derjenige für den Stärkeren gelten sollte, der den Wanderer zwingen würde, seinen Mantel abzunehmen. Der Nordwind blies mit aller Macht, aber je mehr er blies, desto fester hüllte sich der Wanderer in seinen Mantel ein. Endlich gab der Nordwind den Kampf auf. Nun erwärmte die Sonne die Luft mit ihrem freundlichen Strahlen, und schon nach wenigen Augenblicken zog der Wanderer seinen Mantel aus. Da musste der Nordwind zugeben, dass die Sonne von ihnen beiden der Stärkere war.}
	
	\hfill	(\citealp{Pompino95a}; \citealp[88--89]{Kohler99a})
	
\end{itemize}

\end{frame}



%%%%%%%%%%%%%%%%%%%%%%%%%%%%%%%%%%%
%%%%%%%%%%%%%%%%%%%%%%%%%%%%%%%%%%%
\section{Hausaufgaben}

%%%%%%%%%%%%%%%%%%%%%%%%%%%%%%%%%%
%% HA 1 - 02 Phonetik
%%%%%%%%%%%%%%%%%%%%%%%%%%%%%%%%%%

\begin{frame}
\frametitle{Hausaufgabe -- Lösung}

\begin{itemize}
	\item[1.] {Transkribieren Sie folgende Wörter des Deutschen mit dem IPA:}

		\begin{exe}
		\exr{ex:02HA1}
		\settowidth\jamwidth{XXXXXXXXXXXXXXXXXXXXXXXXXXXXXXXX} 
		\begin{xlist}
			\ex arbeiten\jambox{\only<2->{\textcolor{red}{
						\textipa{['PaK.b\texttoptiebar{aI}.t@n], ['Pa\;R.b\texttoptiebar{aI}.t@n], ['P\texttoptiebar{a5}.b\texttoptiebar{aI}.t@n]}
					}}}
			\ex Giebel\jambox{\only<3->{\textcolor{red}{
						\textipa{['gi:.b@l]}
					}}}
			\ex sagen\jambox{\only<4->{\textcolor{red}{
						\textipa{['za:.g@n]}
					}}}
			\ex fröhlich\jambox{\only<5->{\textcolor{red}{
						\textipa{['fr\o:.lI\c{c}]}
					}}}
			\ex Enge\jambox{\only<6->{\textcolor{red}{
						\textipa{['PE\.N@]}
					}}}
			\ex Dampfschiff\jambox{\only<7->{\textcolor{red}{
						\textipa{['dam\texttoptiebar{pf}.SIf]}
					}}}
		\end{xlist}
	\end{exe}

\end{itemize}

\end{frame}	


%%%%%%%%%%%%%%%%%%%%%%%%%%%%%%%%%%%
\begin{frame}
\frametitle{Hausaufgabe -- Lösung}

\begin{itemize}
	\item[2.] {Schreiben Sie für die folgenden Lautbeschreibungen das passende IPA-Symbol auf:}
	
	\begin{exe}
		\exr{ex:02HA2} 
		\settowidth\jamwidth{XXXXXX} 
		\begin{xlist}
			\ex bilabialer stimmloser Plosiv\jambox{\only<2->{\textcolor{red}{
						\textipa{[p]}
					}}}
			\ex hoher vorderer ungerundeter gespannter Vokal\jambox{\only<3->{\textcolor{red}{
					\textipa{[i:]}
				}}}
			\ex velarer dorsaler stimmloser Frikativ\jambox{\only<4->{\textcolor{red}{
						\textipa{[x]}
			}}}
			\ex glottaler stimmloser Plosiv\jambox{\only<5->{\textcolor{red}{
						\textipa{[P]}
			}}}
			\ex halbhoher fast vorderer ungerundeter ungespannter Vokal\jambox{\only<6->{\textcolor{red}{
					\textipa{[I]}
				}}}
			\ex postalveolarer stimmhafter Frikativ\jambox{\only<7->{\textcolor{red}{
						\textipa{[Z]}
					}}}
			\ex halbtiefer zentraler ungerundeter Vokal\jambox{\only<8->{\textcolor{red}{
						\textipa{[5]}
					}}}
			\ex obermittelhoher hinterer gerundeter gespannter Vokal\jambox{\only<9->{\textcolor{red}{
						\textipa{[o:]}
					}}}
			\ex alveolare koronale stimmlose Affrikate\jambox{\only<10->{\textcolor{red}{
						\textipa{[\texttoptiebar{ts}]}
			}}}
			\ex mittlerer zentraler ungerundeter Vokal\jambox{\only<11->{\textcolor{red}{
						\textipa{[@]}
			}}}
		\end{xlist}
	\end{exe}
	
\end{itemize}

\end{frame}


%%%%%%%%%%%%%%%%%%%%%%%%%%%%%%%%%%
\begin{frame}
\frametitle{Hausaufgabe -- Lösung}

\begin{itemize}
	\item[3.] {Beschreiben Sie folgende Laute des Deutschen mit relevanten phonetischen Merkmalen:}
	
	\begin{exe}
		\exr{ex:02HA3}
		\settowidth\jamwidth{XXXXXXXXXXXXXXXXXXXXXXXXXXXXXXXXXXXXXX}
		\begin{xlist}
			\ex \textipa{[y:]}\jambox{\only<2->{\textcolor{red}{
						hoher vorderer gerundeter gespannter Vokal
					}}}
			\ex \textipa{[x]}\jambox{\only<3->{\textcolor{red}{
						velarer dorsaler stimmloser Frikativ
					}}}
			\ex \textipa{[\o:]}\jambox{\only<4->{\textcolor{red}{
						obermittelhoher hinterer gerundeter gespannter Vokal
					}}}
			\ex \textipa{[Z]}\jambox{\only<5->{\textcolor{red}{
						postalveolarer koronaler stimmhafter Frikativ
					}}}
			\ex \textipa{[t]}\jambox{\only<6->{\textcolor{red}{
						alveolarer koronaler stimmloser Plosiv
					}}}
			\ex \textipa{[E]}\jambox{\only<7->{\textcolor{red}{
						untermittelhoher vorderer ungerundeter ungespannter Vokal
					}}}
			\ex \textipa{[m]}\jambox{\only<8->{\textcolor{red}{
						bilabialer stimmhafter Nasal
					}}}
			\ex \textipa{[O]}\jambox{\only<9->{\textcolor{red}{
						untermittelhoher hinterer gerundeter ungespannter Vokal
					}}}
		\end{xlist}
	\end{exe}
	
\end{itemize}

\end{frame}



%% -*- coding:utf-8 -*-

%%%%%%%%%%%%%%%%%%%%%%%%%%%%%%%%%%%%%%%%%%%%%%%%%%%%%%%%%


\def\insertsectionhead{\refname}
\def\insertsubsectionhead{}

\huberlinjustbarfootline


\ifpdf
\else
\ifxetex
\else
\let\url=\burl
\fi
\fi
\begin{multicols}{2}
{\tiny
%\beamertemplatearticlebibitems

\bibliography{gkbib,bib-abbr,biblio}
\bibliographystyle{unified}
}
\end{multicols}





%% \section{Literatur}
%% \begin{frame}[allowframebreaks]
%% \frametitle{Literatur}
%% 	\footnotesize

%% \bibliographystyle{unified}

%% 	%German
%% %	\bibliographystyle{deChicagoMyP}

%% %	%English
%% %	\bibliographystyle{chicago} 

%% 	\bibliography{gkbib,bib-abbr,biblio}
	
%% \end{frame}



\end{document}