%%%%%%%%%%%%%%%%%%%%%%%%%%%%%%%%%%
%% UE 4 - 03c Phonologie
%%%%%%%%%%%%%%%%%%%%%%%%%%%%%%%%%%

\begin{frame}
\frametitle{Übung -- Lösung}

\begin{itemize}
\item Silbifizieren Sie folgende Segmentsequenzen \textbf{in zwei Schritten}:
\begin{itemize}
	\item Onsetmaximierungsprinzip
	\item Sonoritätsprinzip
\end{itemize}

\item Stellen Sie fest, ob alle Silben wohlgeformt sind.\\
Falls nicht, benennen Sie die Verletzungen.

\begin{exe}
	\exr{ex:otling}
	\textipa{[o:tlIN5mSplag\textscr e:hOn]}\\
	\alertred{zuerst Onset"=Maximierung: \textipa{o: . tlI . \ng {\textturna} . mSpla . g\textscr e: . hOn}\\
	dann Anwendung des Sonoritätsprinzips: \textipa{o: . tlI\. \ng \textturna mS . pla . g\textscr e: . hOn}}
\end{exe}

% Otlingamsplagrehon

\begin{exe}
	\exr{ex:blumen}
	Blumentopferde\\ \pause
	\alertred{zuerst Onset-Maximierung: \textipa{blu: . m@ . ntO . pfE . {\textscr}d@}\\
	dann Awendung des Sonoritätsprinzips: \textipa{blu: . m@n . tO . pfE{\textscr} . d@}}
\end{exe}

\end{itemize}

% blu.men.to.pfer.de

%\ea
%Urinstinkt
%\z	

\end{frame}


% ' Stahl , tische     Hauptbetonung auf Stahl, Nebenbetonung auf Tische

