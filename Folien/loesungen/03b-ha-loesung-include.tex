%%%%%%%%%%%%%%%%%%%%%%%%%%%%%%%%%%
%% HA 1 - 03b Phonologie
%%%%%%%%%%%%%%%%%%%%%%%%%%%%%%%%%%

\begin{frame}
\frametitle{Hausaufgabe -- Lösung}

\begin{itemize}
	\item[1.]{Geben Sie die standarddeutsche \textbf{phonetische Transkription} für folgende Wörter an:}
	
	\begin{exe}
		\exr{ex:03bHA1}
		\begin{xlist}
		\settowidth\jamwidth{XXXXXXXXXXXXXXXXXXXXXXXXXXXXXX}
			\ex Spitzenschuhe \loesung{2}{\textipa{['SpI\textsubdot{\t{ts}}@n.Su:.@]}}
			\ex Endausscheidung \loesung{3}{\textipa{['PEnt.P\texttoptiebar{aU}s.S\texttoptiebar{aI}.dUN]}}
			\ex Platzanweiser \loesung{4}{\textipa{['pla\texttoptiebar{ts}.Pan.v\texttoptiebar{aI}.z5]}}
			\ex verzweifeln \loesung{5}{\textipa{[fE5.'\texttoptiebar{ts}v\texttoptiebar{aI}.f@ln]}}
			\ex abverlangen \loesung{6}{\textipa{['Pap.fE5.la\.N@n]}}
			\ex Überarbeitung \loesung{7}{\textipa{[Py:.b5.'Pa\;R.b\texttoptiebar{aI}.tUN]}}
			\ex Zugeständnis \loesung{8}{\textipa{['\texttoptiebar{ts}u:.g@.StEnt.nIs]}}
		\end{xlist}
	\end{exe}

\end{itemize}

\end{frame}
%%%%%%%%%%%%%%%%%%%%%%%%%%%%%%%%%

\begin{frame}{Hausaufgabe -- Lösung}

\begin{itemize}
	\item[2.]{Erläutern Sie anhand der folgenden Beispiele, unter welchen Bedingungen die \textbf{Auslautverhärtung} im Deutschen stattfindet.}

	\begin{exe}
		\exr{ex:03bHA2}
		\begin{xlist}
		\settowidth\jamwidth{XXXXXXXXXXXXXXXXXXXXXXXXXXXXXXX}
			\ex Wand -- Wände \loesung{2}{sth. Plosive am Wortende}
			\ex lesen -- lesbar \loesung{3}{generell am Silbenende}
			\ex sagen -- sagst \loesung{4}{betrifft \emph{alle} sth. Plosive in der Koda}
			\ex Roggen \loesung{5}{jedoch keine Silbengelenke}
		\end{xlist}
	\end{exe}

\end{itemize}

\end{frame}


%%%%%%%%%%%%%%%%%%%%%%%%%%%%%%%%%%
\begin{frame}{Hausaufgabe -- Lösung}

\begin{itemize}
\item[3.]{Geben Sie fünf verschiedene \textbf{phonetische oder phonologische Prozesse} an, die in dem folgenden Satz -- teilweise nur bei schnellerem Sprechen -- beobachtet werden können.} 

\begin{exe}
	\exr{ex:03bHA3}
	\begin{quote}
	Um die fünf Haken in regelmäßigen Abständen an die Wand schrauben zu können, sollten Sie sich Bohrmaschine, Wasserwaage, Zollstock und Dübel bereitgelegt haben und auf keinen Fall die Nerven verlieren, bevor Sie nicht befestigt sind.
	\end{quote}
\end{exe}


\begin{description}
	\item[\alertgreen{\textbf{Beispiele:}}] ~

\alertgreen{
	regressive Nasalassimilation in \emph{fünf}: \textipa{[fY\textbf{m}f]}\\
	progressive Nasalassimilation nach Schwa-Elision (feeding) in \emph{Haken}: \textipa{[hak\textbf{N}]}\\
	Auslautverhärtung in \emph{Wand}: \textipa{[van\textbf{t}]}\\
	progressive Nasalassimilation nach Schwa-Elision (feeding) in \emph{schrauben}: \textipa{[S\;R\t{aU}b\textbf{m}]}\\
	g-Spirantisierung in \emph{befestigt}: \textipa{[b@fEstI\textbf{\c{c}}t]}\\
	r-Vokalisierung in \emph{Bohrmaschine}: \textipa{[bo\textbf{5}maSi:n@]}
}						
		\end{description}

\end{itemize}

\end{frame}


%%%%%%%%%%%%%%%%%%%%%%%%%%%%%%%%%%%
\begin{frame}{Hausaufgabe -- Lösung}

\begin{itemize}	
\item[4.]{Illustrieren  Sie den deutschen phonemischen Kontrast der folgenden Phoneme durch \textbf{Minimalpaare}, wobei der Kontrast (wenn möglich) ein Mal initial,\\ ein Mal final vorkommen soll.

Beispiel: \textipa{[p]} -- \textipa{[f]} Paul -- faul (Initialposition), Laub -- Lauf (Finalposition)}

\begin{exe}
	\exr{ex:03bHA4}
	\settowidth\jamwidth{XXXXXXXXXXXXXXXXXXXXXXXXXXXXXXXXXXXX}
	\begin{xlist}
		\ex \textipa{[m]} -- \textipa{[n]} \loesung{2}{muss -- Nuss, beim -- Bein}
		\ex \textipa{[p]} -- \textipa{[b]} \loesung{3}{Pass -- Bass}
			\loesung{3}{(wegen Auslautverhärtung kein finaler Kontrast möglich)}
		\ex \textipa{[h]} -- \textipa{[v]} \loesung{4}{heiß -- weiß (\textipa{[h]} kommt nicht final vor)}
		\ex \textipa{[n]} -- \textipa{[N]} \loesung{5}{Sinn -- sing (\textipa{[N]} kommt nicht initial vor)}
		\ex \textipa{[f]} -- \textipa{[v]} \loesung{6}{Fass -- was}
			\loesung{6}{(wegen Auslautverhärtung kein finaler Kontrast möglich)}
	\end{xlist}
\end{exe}
		
\end{itemize}

\end{frame}
