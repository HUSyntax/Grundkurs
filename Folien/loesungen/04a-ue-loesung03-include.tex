%%%%%%%%%%%%%%%%%%%%%%%%%%%%%%%%%%
%% UE 3 - 04a Graphematik
%%%%%%%%%%%%%%%%%%%%%%%%%%%%%%%%%%

\begin{frame}
\frametitle{Übung -- Lösung}

\begin{itemize}
	\item Warum schreibt man \ab{dehnen} mit \ab{h}, obwohl das erste \ab{e} in einer offenen Silbe steht und daher nach silbischen Prinzipien sowieso lang gesprochen werden müsste?
		\item[] \alertgreen{Morphemkonstanz, da bei Flexionsformen wie \ab{dehnst} geschlossene Silbe}
	\item Warum schreibt man \ab{mann} und \ab{ball}, obwohl nach silbischen Prinzipien die Geminate einen ambisyllabischen Konsonanten anzeigt?
		\item<2->[] \alertgreen{Morphemkonstanz, da Pluralform Silbengelenk hat}
	\end{itemize}


\end{frame}


%%%%%%%%%%%%%%%%%%%%%%%%%%
\begin{frame}
\frametitle{Übung -- Lösung}

\begin{itemize}
	\item Warum sind die Wörter \ab{(du) ziehst}, \ab{säubern} und \ab{(er) fällt} Beispiele für morphologisches Schreiben?
		\item<2->[] \alertgreen{\ab{h} im Infinitiv silbentrennend, Singular-Flexionsformen sind jedoch einsilbig;}
		\item<3->[] \alertgreen{\ab{ä} zeigt die Verwandschaft zu \ab{sauber}: laut PGK schriebe man \textipa{[O\texttoptiebar{}I]} \ab{eu};}
		\item<4->[] \alertgreen{Konsonantenverdopplung wegen Silbengelenk im Infinitiv, Singular-Flexionsformen sind jedoch einsilbig,\\ \ab{ä} wegen \ab{a} im Infinitiv: \textipa{[E]} wäre nach PGK \ab{e}}
	\item Wie hätte eine Person, die \ab{Rad} und \ab{König} als Beispiele für das morphologische Prinzip anführt, \gqq{phonographisches Schreiben} verstanden?
		\item<5->[] \alertgreen{Verschriftlichung der \emph{phonetischen} (richtig wäre: \emph{phonologische}) Repräsentation}
\end{itemize}


\end{frame}

