%%%%%%%%%%%%%%%%%%
%06c Syntax ha-loesung
%%%%%%%%%%%%%%%%%%

\begin{frame}
\frametitle{Hausaufgabe -- Lösung}

\begin{itemize}
	\item Bestimmen Sie den Satzmodus der folgenden Sätze, geben Sie dabei die Merkmale zur Bestimmung des Satztyps, sowie den möglichen Funktionstyp an. 
\end{itemize}

\begin{exe}
\exr{ex:Rechnung} Wir haben unsere Rechnungen bezahlt.

\pause 
	\begin{itemize}
		\item \alertred{Satzmodus: Deklarativsatz}
		\item \alertred{Satztyp: V2"=Aussagesatz (Kein W"=Fragewort, Indikativ, Intonation: fallend)}
		\item \alertred{Funktionstyp: unmarkierte Mitteilung}
	\end{itemize}
%
%\exr{ex:krank} Ob ich morgen noch krank bin? 
%
%\alertred{E-Interogativsatz (VL + ob), Frage, Antwort wird verlangt}
%
%\ex \label{ex:Wagen} Er hätte einen Wagen kaufen können. 
%	
%	\alertred{Deklarativsatz (V2), unmarkierte Vermutung}
%
\exr{ex:Folien} Hast du endlich die Folien fertig?

\pause	
\begin{itemize}
		\item \alertred{Satzmodus: E-Interrogativsatz}
		\item \alertred{Satztyp: V1"=Fragesatz (Kein Fragewort, Indikativ, Intonation: steigend)}
		\item \alertred{Funktionstyp: auffordernde Frage (\ras Beeil dich!), Antwort wird verlangt}
\end{itemize}

\end{exe}

\end{frame}


%%%%%%%%%%%%%%%%%%%%%%%%%%%%%%%%%%%%%%%
\begin{frame}
\frametitle{Hausaufgabe -- Lösung} 

\begin{exe}

\exr{ex:Iss} Iss!

\pause
\begin{itemize}
	\item \alertred{Satzmodus: Imperativsatz}
	\item \alertred{Satztyp: V1"=Imperativsatz (kein W"=Fragewort, Tilgung des Subjekts in 2.Sg., V1, Verbmodus: Imperativ, fallende Intonation)}
	\item \alertred{Funktionstyp: Aufforderung/Befehl}
\end{itemize}

\exr{ex:Geld} Wenn ich nur Geld hätte!

\pause
\begin{itemize}
	\item \alertred{Satzmodus: Optativ}
	\item \alertred{Satztyp: VL + wenn, kein W"=Fragewort, Verwendung von \emph{nur}, Konjunktiv, fallende Intonation}
	\item \alertred{Funktionstyp: irrealer Wunsch}
\end{itemize}
%
%\exr{ex:spät} Kannst du mir sagen, wie spät es ist? 
%
%\alertred{E-Interrogativ (V1) + K-Interrogativ (VL), wenn die erste Teilantwort \gqq{ja} ist, dann folgt eine Antwort auf das Fragewort (\gqq{Ja, es ist \dots\ Uhr.})}
%
%\ea \label{ex:Störung} Verzeihen Sie die Störung.
%
%\alertred{Imperativ (V1), Bitte}
%
\end{exe} 

\end{frame}


%%%%%%%%%%%%%%%%%%%%%%%%%%%%%%%%%%%%%%%
\begin{frame}
\frametitle{Hausaufgabe -- Lösung} 

\begin{exe}
	
\exr{ex:geschlagen} Wen hast du geschlagen?

\pause 
\begin{itemize}
	\item \alertred{Satzmodus: K"=Interrogativ}
	\item \alertred{Satztyp: V2"=Fragesatz, W"=Fragewort im VF, Indikativ, steigende Intonation}
	\item \alertred{Funktionstyp: Antwort auf Fragewort wird verlangt}
\end{itemize}


\exr{ex:gewonnen} Ich habe gewonnen!

\pause
\begin{itemize}
	\item \alertred{Satzmodus: Exklamativ}
	\item \alertred{Satztyp: V2, keine Negation, Indikativ, fallende Intonation}
	\item \alertred{Funktionstyp: Ausdruck einer Überraschung}
\end{itemize}
%
%\exr{ex:Prüfung} Wenn ich doch die Prüfung bestehe, kaufe ich mir ein Auto.
%
%\alertred{Deklarativ (V2), an Bedingung geknüpfte Mitteilung}
%
%\exr{ex:Baum} Was für einen tollen Baum hat er gemalt? 
%
%\alertred{K-Interrogativ, Verständnis-Nachfrage}
%
%\exr{ex:leise} Sie sind jetzt aber leise!
%
%\alertred{Imperativ (V2), Befehl/ Aufforderung}
%
%\exr{ex:Verständnis} Wir bitten um Verständnis.
%
%\alertred{Deklarativ (V2), Aufforderung}
%
\end{exe}

\end{frame}


%%%%%%%%%%%%%%%%%%%%%%%%%%%%%%%%%%%%%%%
\begin{frame}
	\frametitle{Hausaufgabe (Lösung)} 
	
\begin{itemize}
	\item Geben Sie eine Analyse der folgenden Sätze (\textbf{inkl. Nebensätze}) nach dem topologischen Feldermodell.
\end{itemize}
	
\begin{exe}
	\exr{ex:Top1} Wenn ich die Prüfung bestehe, werde ich mir ein Buch von Chomsky kaufen.
	\exr{ex:Top2} Werde ich mir das Buch kaufen, wenn ich bestehe?
\end{exe}
	
	
	
\begin{table}
	\centering
	\scalebox{0.8}{
		\alertred{\begin{tabular}{p{3.5cm}|l|p{4cm}|p{1.5cm}|p{3cm}}
				\textbf{VF} & \textbf{LSK} & \textbf{MF} & \textbf{RSK} & \textbf{NF} \\ 
				\hline
				Wenn ich die Prüfung bestehe, & werde & ich mir ein Buch von Chomsky & kaufen. &\\
				\hline
				&Wenn& ich die Prüfung 							& bestehe & \\
				\hline
				\hline 
				&Werde& ich mir das Buch						&kaufen,	& wenn ich bestehe? \\
				\hline 
				&wenn& ich												&bestehe	&\\
		\end{tabular}}
	}
\end{table}
	
	
\end{frame}