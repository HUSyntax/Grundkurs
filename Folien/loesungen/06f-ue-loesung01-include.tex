%%%%%%%%%%%%%%%%%%%%%%%%%%%%%%%%%%%
%06f Syntax ue-loesung01
%%%%%%%%%%%%%%%%%%%%%%%%%%%%%%%%%

\begin{frame}
\frametitle{Übung -- Lösung}


\begin{minipage}[b]{0.29\textwidth}
	\centering
	\scalebox{0.55}{
		\begin{forest}
			MyP edges,
			[CP
			[DP$ _{k} $[Das Kind, roof]]
			[\MyPxbar{C} 
			[\zerobar{C}[küsst$ _{i} $]]
			[IP
			[\alertgreen{t$_{k}$}]
			[\MyPxbar{I}
			[VP [\MyPxbar{V}
			[DP[die Mama, roof]]
			[\zerobar{V}[t$ _{i} $]]
			]
			]
			[\zerobar{I}[t$ _{i} $]]
			]
			]
			]
			]
		\end{forest}
		}
\end{minipage}
%
%
\begin{minipage}[b]{0.31\textwidth}
	\centering
	\scalebox{0.55}{
		\begin{forest}
			MyP edges,
			[CP
			[DP$ _{k} $[Das Kind, roof]]
			[\MyPxbar{C} 
			[\zerobar{C}[küsst$ _{i} $]]
			[IP
			[DP[die Mama, roof]]
			[\MyPxbar{I}
			[VP [\MyPxbar{V}
			[\alertgreen{t$_{k}$}]
			[\zerobar{V}[t$ _{i} $]]
			]
			]
			[\zerobar{I}[t$ _{i} $]]
			]
			]
			]
			]
		\end{forest}
		}
\end{minipage}
%%
%%
\begin{minipage}[b]{.38\textwidth}


\begin{itemize*}
	\alertgreen{\item {\small \MyPobj{das Kind} und \MyPobj{die Mama} sind im Akkusativ und im Nominativ formgleich (Synkretismus).}}
	\alertgreen{\item {\small Im Dt. kann eine Phrase in die SpecCP-Position bewegt werden.}}
	\alertgreen{\item {\small Wegen des Synkretismus' ist nicht ersichtlich, ob \MyPobj{das Kind} sich aus der SpecIP- oder aus der Schwesterposition von V\MyPup{0} bewegt hat. }}
\end{itemize*}

\end{minipage}
\end{frame}
