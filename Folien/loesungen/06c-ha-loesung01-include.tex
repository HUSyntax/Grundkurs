%%%%%%%%%%%%%%%%%%
%06c Syntax ha-loesung01
%%%%%%%%%%%%%%%%%%

\begin{frame}
\frametitle{Hausaufgabe -- Lösung}

%\begin{itemize}
%	\item Bestimmen Sie den Satzmodus der folgenden Sätze, geben Sie dabei die Merkmale zur Bestimmung des Satztyps, sowie den möglichen Funktionstyp an. 
%\end{itemize}
\ea \label{ex:Rechnung2} Wir haben unsere Rechnungen bezahlt. 
\pause 
	\begin{itemize}
		\item \alertgreen{Satzmodus: Deklarativsatz}
		\item \alertgreen{Satztyp: V2"=Aussagesatz (Kein W"=Fragewort, Indikativ, Intonation: fallend)}
		\item \alertgreen{Funktionstyp: unmarkierte Mitteilung}
	\end{itemize}
%\ex \label{ex:Wagen} Er hätte einen Wagen kaufen können. 
%	
%	\alertgreen{Deklarativsatz (V2), unmarkierte Vermutung}	
\ex \label{ex:Folien2} Hast du endlich die Folien fertig?
\pause	
\begin{itemize}
		\item \alertgreen{Satzmodus: E-Interrogativsatz}
		\item \alertgreen{Satztyp: V1"=Fragesatz (Kein Fragewort, Indikativ, Intonation: steigend)}
		\item \alertgreen{Funktionstyp: auffordernde Frage (\ras Beeil dich!), Antwort wird verlangt}
\end{itemize}
\z

\end{frame}

%%%%%%%%%%%%%%%%%%%%%%%%%%%%%%%%%%%%%%%

\begin{frame}
\frametitle{Hausaufgabe -- Lösung} 

%\ea \label{ex:krank} Ob ich morgen noch krank bin? 
%
%\alertgreen{E-Interogativsatz (VL + ob), Frage, Antwort wird verlangt}

\ea \label{ex:Iss2} Iss!
\pause
\begin{itemize}
	\item \alertgreen{Satzmodus: Imperativsatz}
	\item \alertgreen{Satztyp: V1"=Imperativsatz (kein W"=Fragewort, Tilgung des Subjekts in 2.Sg., V1, Verbmodus: Imperativ, fallende Intonation)}
	\item \alertgreen{Funktionstyp: Aufforderung/Befehl}
\end{itemize}
\ex \label{ex:Geld2} Wenn ich nur Geld hätte!
\pause
\begin{itemize}
	\item \alertgreen{Satzmodus: Optativ}
	\item \alertgreen{Satztyp: VL + wenn, kein W"=Fragewort, Verwendung von \emph{nur}, Konjunktiv, fallende Intonation}
	\item \alertgreen{Funktionstyp: irrealer Wunsch}
\end{itemize}
%\ex \label{ex:spät} Kannst du mir sagen, wie spät es ist? 
%
%\alertgreen{E-Interrogativ (V1) + K-Interrogativ (VL), wenn die erste Teilantwort \gqq{ja} ist, dann folgt eine Antwort auf das Fragewort (\gqq{Ja, es ist \dots\ Uhr.})}
\z 

\end{frame}

%%%%%%%%%%%%%%%%%%%%%%%%%%%%%%%%%%%%%%%

\begin{frame}
\frametitle{Hausaufgabe -- Lösung} 

%\ea \label{ex:Störung} Verzeihen Sie die Störung.
%
%\alertgreen{Imperativ (V1), Bitte}

\ea \label{ex:geschlagen2} Wen hast du geschlagen?
\pause 
\begin{itemize}
	\item \alertgreen{Satzmodus: K"=Interrogativ}
	\item \alertgreen{Satztyp: V2"=Fragesatz, W"=Fragewort im VF, Indikativ, steigende Intonation}
	\item \alertgreen{Funktionstyp: Antwort auf Fragewort wird verlangt}
\end{itemize}


\ex \label{ex:gewonnen2} Ich habe gewonnen!

\pause
\begin{itemize}
	\item \alertgreen{Satzmodus: Exklamativ}
	\item \alertgreen{Satztyp: V2, keine Negation, Indikativ, fallende Intonation}
	\item \alertgreen{Funktionstyp: Ausdruck einer Überraschung}
\end{itemize}


%\ex \label{ex:Prüfung} Wenn ich doch die Prüfung bestehe, kaufe ich mir ein Auto.
%
%\alertgreen{Deklarativ (V2), an Bedingung geknüpfte Mitteilung}

\z 

\end{frame}