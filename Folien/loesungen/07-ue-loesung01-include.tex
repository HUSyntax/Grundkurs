%%%%%%%%%%%%%%%%%%
%Lösung 07 Semantik
%%%%%%%%%%%%%%%%

	
	\begin{frame}
		\frametitle{Übung -- Lösung}

Bestimmen Sie die Sinnrelationen bzw.\ die Ambiguitätsarten in den folgenden Wortpaaren.

\settowidth\jamwidth{Homonymie (Homographie und -phonie)}
\begin{exe}	
	\exr{ex:Rel1} Ballkleid -- Kleid \pause \jambox{\alertgreen{Hyponym/ Hyperonym}}
	\exr{ex:Rel2} Bank -- Bank \pause \jambox{\alertgreen{Homonymie (Homographie und -phonie)}}
	\exr{ex:Rel3} Schraubenzieher -- Zange \pause \jambox{\alertgreen{Kohyponymie}}
	\exr{ex:Rel4} gro\ss{} -- klein \pause \jambox{\alertgreen{Konträre Antonymie}}
	\exr{ex:Rel5} Henkel -- Tasse \pause \jambox{\alertgreen{Meronymie}}
	\exr{ex:Rel6} Ahorn -- Baum \pause \jambox{\alertgreen{Hyponym/ Hyperonym}}
\end{exe}

\end{frame}

%%%%%%%%%%%%%%%%%%%%%%%%%%%%%%%%%%%%%%%

\begin{frame}
	\frametitle{Übung -- Lösung}
	
Bestimmen Sie die Sinnrelationen bzw.\ die Ambiguitätsarten in den folgenden Wortpaaren.
	
\settowidth\jamwidth{Homonymie (Homographie und -phonie)}
\begin{exe}
	\exr{ex:Rel7} essen -- verzehren \pause \jambox{\alertgreen{Synonymie}}
	\exr{ex:Rel8} gerade natürliche Zahl -- ungerade  natürliche Zahl \pause \jambox{\alertgreen{Kontradiktorische Antonymie}}
	\exr{ex:Rel9} Stimme (Votum) -- Stimme (Sprachfähigkeit) \pause \jambox{\alertgreen{Homonymie (Homographie und -phonie)}}
	\exr{ex:Rel10} Wände -- Wende \pause \jambox{\alertgreen{Homonymie (Homophonie)}}
	\exr{ex:Rel11} Lache -- Lache \pause \jambox{\alertgreen{Homonymie (Homographie)}}
\end{exe}

\end{frame}