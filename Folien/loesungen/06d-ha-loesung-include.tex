\begin{frame}
\frametitle{Hausaufgabe (Lösung)} 

\begin{itemize}
	\item Bestimmen Sie den Kopf der folgenden \textbf{markierten} Phrasen und begründen Sie Ihre Entscheidung:
\end{itemize}
	
	\ea Es geht um [\textbf{wirklich von dieser Sache überzeugte und engagierte junge Schüler, die sich dennoch über das übliche und akzeptable Ausmaß hinaus daneben benommen haben}].
	\z

\pause 

	
\begin{itemize}
	\item \alertgreen{Kopf: Schüler (vorläufig)}
	\item \alertgreen{Interpretation: Es geht um \emph{Schüler}}
	\item \alertgreen{Distribution: Innerhalb einer PP wird eine (DP/)NP selegiert. Der Kopf dieser NP ist das Nomen.}
	\item \alertgreen{Phrasenaufbau: Alle anderen Modifikatoren beziehen sich auf das Nomen. }
\end{itemize}

	
\end{frame}


\begin{frame}
\frametitle{Hausaufgabe (Lösung)} 

\begin{itemize}
	\item Bestimmen Sie den Kopf der folgenden Phrase und begründen Sie Ihre Entscheidung:
\end{itemize}

\ea Wir warteten auf [\textbf{den von sich sehr überzeugten Redner}].
\z 

\pause 

\begin{itemize}
	\item \alertgreen{Kopf: Redner (vorläufig)}
	\item \alertgreen{Interpretation: Es geht um \emph{Redner}}
	\item \alertgreen{Distribution: Innerhalb einer PP wird eine (DP/)NP selegiert. Der Kopf dieser NP ist das Nomen.}
	\item \alertgreen{Phrasenaufbau: Alle anderen Modifikatoren beziehen sich auf das Nomen.}
\end{itemize}
	
	
\end{frame}


\begin{frame}
\frametitle{Hausaufgabe (Lösung)} 

\begin{itemize}
	
	\item Geben Sie für die folgenden Wörter den Subkategorisierungsrahmen (in dem besprochenen Format) und ein(en) Beispiel(satz), der den von Ihnen angegebenen Subkategorisierungsrahmen illustriert, an:
	
	\ea übergeben: \pause \alertgreen{DP$_{\textsc{nom,quelle}}$ (DP$_{\textsc{dat,ziel}}$) DP$_{\textsc{akk,th}}$ $\underline{\qquad}$ [Lesart 1]}
	
	\alertgreen{(dass) ich Peter die Briefe \emph{übergebe}}
	\pause 
	
	\ex stolz: \pause \alertgreen{PP$_{auf+\textsc{akk,th}}$ $\underline{\qquad}$ (DP$_{\textsc{exp}}$)}
	
	\alertgreen{die auf ihre Tochter \emph{stolze} Mutter}
	\pause 
	
	\ex donnern: \pause \alertgreen{DP$_{(es),\textsc{nom}}$ $\underline{\qquad}$}
	
	\alertgreen{(dass) es \emph{donnert}}
	\z 
	
	%	wissen, regnen, scheinen, mit, ängstigen, fürchten, Banane, Tatsache, warten, Angst, wohnen, bewohnen, auf, malen, krank, bemalen, fragen
	
\end{itemize}

\end{frame}


\begin{frame}
\frametitle{Hausaufgabe (Lösung)} 

\begin{itemize}

\item Geben Sie für die folgenden Wörter den Subkategorisierungsrahmen (in dem besprochenen Format) und ein(en) Beispiel(satz), der den von Ihnen angegebenen Subkategorisierungsrahmen illustriert, an:

\ea Frage: \pause \alertgreen{DP$_{\textsc{gen,ag}}$ $\underline{\qquad}$ PP$_{nach+\textsc{dat,th}}$}

\alertgreen{die Frage nach dem Schatz}
\pause 

\ex erschrecken: \pause \alertgreen{DP$_{\textsc{nom,causer}}$ DP$_{\textsc{akk,exp}}$ $\underline{\qquad}$ [Lesart 1]}

\alertgreen{(dass) Maria Peter erschreckt}
\pause 

\ex bemalen: \pause \alertgreen{DP$_{\textsc{nom,ag}}$ DP$_{\textsc{akk,pat/th}}$ $\underline{\qquad}$}

\alertgreen{(dass) ich die Wand bemale}
\z 

%	wissen, regnen, scheinen, mit, ängstigen, fürchten, Banane, Tatsache, warten, Angst, wohnen, bewohnen, auf, malen, krank, bemalen, fragen

\end{itemize}

\end{frame}

\begin{frame}
\frametitle{Hausaufgabe (Lösung)} 

\begin{itemize}
	
	\item Bestimmen Sie in den folgenden Sätzen, welche Phrasen Argumente und welche Modifikatoren des Verbs sind, und begründen Sie Ihre Entscheidung.
	
\end{itemize}

\ea Maria bearbeitete die Folien mit sehr viel Kreativität.

\pause 
\alertgreen{Arg.: Maria, die Folien}

\alertgreen{Mod.: mit sehr viel Kreatitvität}

\alertgreen{Begründung: \dots }

\pause 


\ex Maria arbeitete an den Folien den ganzen Tag.

\pause 
\alertgreen{Arg.: Maria, an den Folien}

\alertgreen{Mod.: den ganzen Tag}

\alertgreen{Begründung: \dots }

\z 

\end{frame}


\begin{frame}
\frametitle{Hausaufgabe (Lösung)} 

\begin{itemize}

\item Bestimmen Sie in den folgenden Sätzen, welche Phrasen Argumente und welche Modifikatoren des Verbs sind, und begründen Sie Ihre Entscheidung.

\end{itemize}

\ea Peter wirkte auf seinen Sohn stolz.

\pause 
\alertgreen{Arg.: Peter, auf seinen Sohn stolz}

\alertgreen{Mod.: $\emptyset$}

\alertgreen{Begründung: \dots }

\pause 


\ex Peter wirkte auf seinen Sohn stolz.

\pause 
\alertgreen{Arg.: Peter, auf seinen Sohn, stolz}

\alertgreen{Mod.: $\emptyset$}

\alertgreen{Begründung: \dots }
\z 


\end{frame}