%%%%%%%%%%%%%%%%%%%%%%%%%%%%%%%%%%
%% UE 4 - 04a Graphematik
%%%%%%%%%%%%%%%%%%%%%%%%%%%%%%%%%%

\begin{frame}
\frametitle{Übung -- Lösung}

\begin{itemize}
	\item Welche graphematischen Prinzipien (abgesehen von der phonographischen Schreibung) erklären die Schreibung der folgenden Wörter?
	
\begin{exe}
	\exr{ex:04azimmer}
	\settowidth\jamwidth{XXXXXXXXXXXXXXXXXXXXXXXXXXXXXXXXXXX}
	\begin{xlist}
		\ex \ab{Zi\rotul{mm}er} \loesung{1}{silbisches Prinzip: Silbengelenk}
		\ex \ab{W\rotul<2->{a}ise} \loesung{2}{Differenzierung homophoner Formen: zu \ab{weise}}
		\ex \ab{We\rotul<3->{h}en} \loesung{3}{silbisches Prinzip: silbentrennend}
		\ex \ab{Ru\rotul<4->{h}m} \loesung{4}{silbisches Prinzip: Dehnungs-h}
		\ex \ab{\rotul<5->{S}paß} \loesung{5}{ästhetische Schreibung: kein \ab{schp}}
		\ex \ab{A\rotul<6->{llee}} \loesung{6}{silbisches Prinzip: Silbengelenk \ab{ll}, Gespanntheit \ab{ee}}
		%          \ex \ab{Gras}
	\end{xlist}
\end{exe}
	
	\item Welche graphematische Funktion erfüllt das \ab{h} in den folgenden Wörtern?
	
\begin{exe}
	\exr{ex:04anacht}
	\settowidth\jamwidth{XXXXXXXXXXXXXXXXXXXXXXXXXXXXXXXXXXX}
	\begin{xlist}
		\ex \ab{Nacht} \loesung{7}{Teil von Digraph \ab{ch}}
		\ex \ab{Hilfe} \loesung{8}{Phonem-Graphem-Korrespondenz zu \textipa{[h]}}
		\ex \ab{sehen} \loesung{9}{silbentrennend}
		\ex \ab{Mehl} \loesung{10}{Dehnungs-h}
	\end{xlist}
\end{exe}
	
    \end{itemize}

\end{frame}


%%%%%%%%%%%%%%%%%%%%%%%%%%%%%%%%%%%%%%%
\begin{frame}{Übung -- Lösung}

	\begin{itemize}
	\item Wie würden die folgenden Wörter in phonographischer Schreibung aussehen? Geben Sie zunächst eine phonologische Transkription an (Notation mit / ~ /) und schreiben Sie anschließend phonographisch (Notation in \ab{ ~ }).

\begin{exe}
	\exr{ex:04avasenstück}
	\begin{multicols}{3}
	\begin{xlist}
		\ex \ab{Handy} 
		
		\ex \ab{Vasenstück}
		
		\ex \ab{Wannenbad}
		
%		\exi{} 
		\exi{} \only<2->{\alertgreen{\textipa{/hEn.di/}}}
		\exi{} \only<4->{\alertgreen{\textipa{/va:.z@n.StYk/}}}
		\exi{} \only<6->{\alertgreen{\textipa{/va\.n@n.ba:d/}}}

%		\exi{} 
		\exi{} \only<3->{\alertgreen{\ab{hendi}}}		
		\exi{} \only<5->{\alertgreen{\ab{wasenschtük}}}		
		\exi{} \only<7->{\alertgreen{\ab{wanenbad}}}		
	\end{xlist}
	\end{multicols}
\end{exe}
	
	\end{itemize}

\end{frame}

