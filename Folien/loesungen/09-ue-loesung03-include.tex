%%%%%%%%%%%%%%%%%%%%%%%%%%%%%%%%%%
%% UE 3 - 09 Übungen
%%%%%%%%%%%%%%%%%%%%%%%%%%%%%%%%%%

\begin{frame}
\frametitle{Übung: Morphologie -- Lösung}

\begin{itemize}
	\item[8.] Welche Wortbildungsprozesse haben hier stattgefunden?
	
	\settowidth \jamwidth{\alertgreen{\only<2->{Derivation (Präfigierung) oder Partikelverbbildung}}}
	
	\eal
	\ex übersetz(-en)  \jambox{\alertgreen{\only<2->{Derivation (Präfigierung) oder Partikelverbbildung}}}
	\ex bleifrei \jambox{\alertgreen{\only<3->{Rektionskompositum}}}
	\ex Tanz \jambox{\alertgreen{\only<4->{Konversion}}}
	\ex Bearbeitung \jambox{\alertgreen{\only<5->{1. Derivation (Präfigierung),}}}
			\jambox{\alertgreen{\only<5->{2. Derivation (Suffigierung)}}}
	\zl
	
\end{itemize}

\end{frame}

%%%%%%%%%%%%%%%%%%%%%%%%%%%%%%%%%%
\begin{frame}
	
\begin{itemize}
	
%	\only<1-1>{
		\item[9.] Geben Sie die Konstituentenstruktur der folgenden Wörter an und bestimmen Sie die Wortbildungstypen an jedem Knoten des Baumes so genau wie möglich.
%	}
	
%	\only<1-1>{
	\eal
	\ex \label{ex:Konstituente1}{Unbeweisbarkeitsannahmen}
	\ex \label{ex:Konstituente2}{(mit den) Blickbewegungsmessern}
	\zl
%	}

\end{itemize}

\end{frame}

%%%%%%%%%%%%%%%%%%%%%%%%%%%%%%%%%%

\begin{frame}

\begin{itemize}
	\item Analyse mit Fugenelement \\
	(\ref{ex:Konstituente1}): Unbeweisbarkeitsannahmen

	\scalebox{.68}{
	\alertgreen{
	\begin{forest} MyP edges,
		[N, name=N1
			[N, name=N2
				[N
					[N, name=N3
						[A, name=A1
							[A\MyPup{af} [un-, tier=word]]
							[A, name=A2
								[V, name=V3
									[A\MyPup{af} [be-, tier=word]]
									[V [weis, tier=word]]
								]
								[A\MyPup{af} [-bar, tier=word]]
							]
						]
						[N\MyPup{af} [-keit, tier=word]]
					]
					[FE [-s, tier=word]]
				]
				[N, name=N4
					[V, name=V1
						[V\MyPup{af} [an-, tier=word]]
						[V, name=V2 [nahm/nehm, tier=word]]
					]
					[N [-e, tier=word]]
				]
			]
			[FI [-n, tier=word]]
		]
	\draw[<-, HUgreen] (N1.west)--++(-12.5em,0pt)
	node[anchor=east,align=center]{Flexion (KEIN Wortbildungsprozess)};
	\draw[<-, HUgreen] (N2.west)--++(-13.5em,0pt)
	node[anchor=east,align=center]{Rektionskompositum};
	\draw[<-, HUgreen] (N3.west)--++(-9em,0pt)
	node[anchor=east,align=center]{Derivation};
	\draw[<-, HUgreen] (A1.west)--++(-4.7em,0pt)
	node[anchor=east,align=center]{Derivation};
	\draw[<-, HUgreen] (A2.east)--++(2.5em,0pt)--++(0pt,-22.5ex)
	node[anchor=north,align=center]{Derivation};
	\draw[<-, HUgreen](N4.east)--++(3.5em,0pt)--++(0pt,-43.5ex)
	node[anchor=north,align=center]{Derivation};
	\draw[<-, HUgreen](V1.west)--++(-2.5em,0pt)--++(0pt,-36.7ex)
	node[anchor=north,align=center]{Derivation};
	\draw[<-, HUgreen](V2.west)--++(-2em,0pt)--++(0pt,-32ex)
	node[anchor=north,align=center]{implizite Derivation};
	\draw[<-, HUgreen](V3.west)--++(-2em,0pt)--++(0pt,-15.5ex)
	node[anchor=north,align=center]{Derivation};
	\end{forest} 
	} }
	
\end{itemize}

\end{frame}

%%%%%%%%%%%%%%%%%%%%%%%%%%%%%%%%%%

\begin{frame}

\begin{itemize}
	\item Analyse mit Kompositionsstammform \\
	(\ref{ex:Konstituente1}): Unbeweisbarkeitsannahmen
	
	\scalebox{.75}{
	\alertgreen{
	\begin{forest} MyP edges,
		[N, name=N1
			[N, name=N2
				[N, name=N3
					[A, name=A1
						[A\MyPup{af} [un-, tier=word]]
						[A, name=A2
							[V, name=V3
								[A\MyPup{af} [be-, tier=word]]
								[V [weis, tier=word]]
							]
							[A\MyPup{af} [-bar, tier=word]]
						]
					]
					[N\MyPup{af} [-keit(-s), tier=word]]
				]
				[N, name=N4
					[V, name=V1
						[V\MyPup{af} [an-, tier=word]]
						[V, name=V2 [nahm/nehm, tier=word]]
					]
					[N [-e, tier=word]]
				]
			]
			[FI [-n, tier=word]]
		]
	\draw[<-, HUgreen] (N1.west)--++(-13em,0pt)
	node[anchor=east,align=center]{Flexion (KEIN Wortbildungsprozess)};
	\draw[<-, HUgreen] (N2.west)--++(-13.5em,0pt)
	node[anchor=east,align=center]{Rektionskompositum};
	\draw[<-, HUgreen] (N3.west)--++(-11em,0pt)
	node[anchor=east,align=center]{Derivation};
	\draw[<-, HUgreen] (A1.west)--++(-6.5em,0pt)
	node[anchor=east,align=center]{Derivation};
	\draw[<-, HUgreen] (A2.east)--++(2.5em,0pt)--++(0pt,-22.5ex)
	node[anchor=north,align=center]{Derivation};
	\draw[<-, HUgreen](N4.east)--++(3.5em,0pt)--++(0pt,-36.5ex)
	node[anchor=north,align=center]{Derivation};
	\draw[<-, HUgreen](V1.west)--++(-2.5em,0pt)--++(0pt,-29.7ex)
	node[anchor=north,align=center]{Derivation};
	\draw[<-, HUgreen](V2.west)--++(-2em,0pt)--++(0pt,-25ex)
	node[anchor=north,align=center]{implizite Derivation};
	\draw[<-,HUgreen](V3.west)--++(-2em,0pt)--++(0pt,-15.5ex)
	node[anchor=north,align=center]{Derivation};
	\end{forest} 
	} }

\end{itemize}

\end{frame}

%%%%%%%%%%%%%%%%%%%%%%%%%%%%%%%%%%

\begin{frame}

\begin{itemize}
	\item Analyse mit Fugenelement \\
	(\ref{ex:Konstituente2}): (mit den) Blickbewegungsmessern
			
	\scalebox{.85}{
	\alertgreen{
	\begin{forest} MyP edges,
		[N, name=N1
			[N, name=N2
				[N
					[N, name=N3
						[N [blick, tier=word]]
						[N, name=N4
							[V [beweg, tier=word]]
							[N\MyPup{af} [-ung, tier=word]]
						]
					]
					[FE [-s, tier=word]]
				]
				[N, name=N5
					[V [mess, tier=word]]
					[N\MyPup{af} [-er, tier=word]]
				]
			]
			[FI [-n, tier=word]]
		]
	\draw[<-, HUgreen](N1.west)--++(-11.8em,0pt)
	node[anchor=east,align=center]{Flexion (KEIN Wortbildungsprozess)};
	\draw[<-, HUgreen](N2.west)--++(-14.5em,0pt)
	node[anchor=east,align=center]{Rektionskompositum};
	\draw[<-, HUgreen](N3.west)--++(-7em,0pt)
	node[anchor=east,align=center]{Rektionskompositum};
	\draw[<-, HUgreen](N4.east)--++(2.5em,0pt)--++(0pt,-16.2ex)
	node[anchor=north,align=center]{Derivation};
	\draw[<-, HUgreen](N5.east)--++(2em,0pt)--++(0pt,-30ex)
	node[anchor=north,align=center]{Derivation};
	\end{forest}
	} }

\end{itemize}

\end{frame}	

%%%%%%%%%%%%%%%%%%%%%%%%%%%%%%%%%%

\begin{frame}

\begin{itemize}
	\item Analyse mit Kompositionsstammform \\
	(\ref{ex:Konstituente2}): (mit den) Blickbewegungsmessern
	
	\scalebox{.9}{
	\alertgreen{
	\begin{forest} MyP edges,
		[N, name=N1
			[N, name=N2
				[N, name=N3
					[N [blick, tier=word]]
					[N, name=N4
						[V [beweg, tier=word]]
						[N\MyPup{af} [-ung(-s), tier=word]]
					]
				]
				[N, name=N5
					[V [mess, tier=word]]
					[N\MyPup{af} [-er, tier=word]]
				]
			]
			[FI [-n, tier=word]]
		]
	\draw[<-, HUgreen](N1.west)--++(-11em,0pt)
	node[anchor=east,align=center]{Flexion (KEIN Wortbildungsprozess)};
	\draw[<-, HUgreen](N2.west)--++(-13em,0pt)
	node[anchor=east,align=center]{Rektionskompositum};
	\draw[<-, HUgreen](N3.west)--++(-7.7em,0pt)
	node[anchor=east,align=center]{Rektionskompositum};
	\draw[<-, HUgreen](N4.east)--++(3.5em,0pt)--++(0pt,-16.2ex)
	node[anchor=north,align=center]{Derivation};
	\draw[<-, HUgreen](N5.east)--++(2em,0pt)--++(0pt,-23ex)
	node[anchor=north,align=center]{Derivation};
	\end{forest}
	} }
	
\end{itemize}

\end{frame}
%%%%%%%%%%%%%%%%%%%%%%%%%%%%%%%%%%

\begin{frame}
	
\begin{itemize}
	
	\item[10.] Geben Sie Beispiele für die folgenden Kompositionsarten an:
	
	\settowidth \jamwidth{\alertgreen{Romanleser, Tierkennerin, Konfliktbewältigung, \dots}}
	
	\eal
	\ex Determinativkomposition \\ 
	\only<2->{\jambox{\alertgreen{Apfelsaft, Taschenlampe, Hundeleine, \dots}}}
\medskip	
	\ex Rektionskomposition \\ 
	\only<3->{\jambox{\alertgreen{Romanleser, Tierkennerin, Konfliktbewältigung, \dots}}}
\medskip	
	\ex Possessivkomposition \\ 
	\only<4->{\jambox{\alertgreen{Dickkopf, Grünschnabel, Milchgesicht, \dots}}}
\medskip	
	\ex Kopulativkomposition \\ 
	\only<5->{\jambox{\alertgreen{nordost, Berlin-Brandenburg, blaugrau, \dots}}}
	\zl
	
\end{itemize}

\end{frame}
%%%%%%%%%%%%%%%%%%%%%%%%%%%%%%%%%%

\begin{frame}

\begin{itemize}
	\item[11.] Geben Sie je ein Beispiel für einen Stamm, für eine Wurzel, für eine Basis und für ein unikales Morphem an.
	
	\settowidth \jamwidth{Handyvertrags-laufzeit [hinsichtl. Komposition]X}
	
	\eal
	\ex Stamm: \jambox{\only<2->{\alertgreen{Handyvertrags}-\alertgreen{laufzeit} [hinsichtl. Komposition]}} 
	
	\jambox{\only<2->{\alertgreen{Handyvertragslaufzeit}-en [hinsichtl. Flexion]}}
	
	\ex Wurzel: \jambox{\only<3->{un-be-\alertgreen{weis}-bar-es}}
	
	\ex Basis: \jambox{\only<4->{un-\alertgreen{beweisbar}}}
	
	\ex unikales Morphem: \jambox{\only<5->{ver-\alertgreen{letz}-en}}
	\zl
	
\end{itemize}

\end{frame}
%%%%%%%%%%%%%%%%%%%%%%%%%%%%%%%%%%