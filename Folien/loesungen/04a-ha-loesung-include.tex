%%%%%%%%%%%%%%%%%%%%%%%%%%%%%%%%%%
%% HA 1 - 04a Graphematik
%%%%%%%%%%%%%%%%%%%%%%%%%%%%%%%%%%

\begin{frame}%[allowframebreaks]
\frametitle{Hausaufgabe -- Lösung}

\begin{itemize}
	
	\item[1.] Kreuzen Sie die korrekten Aussagen an.
	
	\begin{itemize}
		\item[$\circ$] Die Orthographie ist eine linguistische Teildisziplin, die beschreibt wie man schreibt. Die Graphematik ist dagegen keine Teildisziplin der Linguistik, sondern eine \gqq{willkürliche} (normierende) Festlegung.
		
		\item[\alertgreen{$\checkmark$}] \alertgreen{Die Graphematik sollte intuitiv beherrschbar sein und das Lesen und Schreiben vereinfachen.}
		
		\item[$\circ$] Das Wort \ab{kalt} ist eine graphematisch \gqq{nackte} Silbe.
		
		\item[$\circ$] Es gibt im Deutschen eine eindeutige 1-zu-1-Korrespondenz zwischen Buchstaben und Lauten.
		
		\item[\alertgreen{$\checkmark$}] \alertgreen{Das Wort \ab{mächtig} wird aufgrund des morphologischen Prinzips (auch Prinzip der Schemakonstanz, Stammprinzip oder Verwandtschaftsprinzip) mit \ab{ä} geschrieben (vgl. \ab{Macht}).}
	\end{itemize}
\end{itemize}
\end{frame}


%%%%%%%%%%%%%%%%%%%%%%%%%%%%%%%%%%	
\begin{frame}
	\frametitle{Hausaufgabe -- Lösung}

\begin{itemize}
\item[2.] Ordnen Sie die graphematischen Prinzipien links den passenden Beispielen für die entsprechenden Prinzipien rechts zu.

NB: Beachten Sie bitte nicht die Großschreibung.

\vspace{.5cm}

\begin{minipage}{0.45\textwidth}
	\centering
	\begin{tabular}{|l|}
		\hline
		(A) Etymologische Schreibung\\
		\hline
		(B) Homonymievermeidung\\
		\hline
		(C) Morphologisches Prinzip\\
		\hline
		(D) Silbische Prinzip\\
		\hline
		(E) Phonographisches Prinzip\\
		\hline
	\end{tabular}
\end{minipage}
\hfill%
\begin{minipage}{0.45\textwidth}
	\centering
	\begin{tabular}{|p{0.075\textwidth}|l|}
		\hline
		\only<2->{\alertgreen{C}} & Bad, Bäder \\
		\hline
		\only<3->{\alertgreen{D}} & gehen \\
		\hline
		\only<4->{\alertgreen{A}} & Cello, *Tschello \\
		\hline
		\only<5->{\alertgreen{B}} & Wahl, Wal\\
		\hline
		\only<6->{\alertgreen{E}} & Flasche \\
		\hline
	\end{tabular}
\end{minipage}

\end{itemize}
\end{frame}


%%%%%%%%%%%%%%%%%%%%%%%%%%%%%%%%%%	
\begin{frame}%[allowframebreaks]
	\frametitle{Hausaufgabe -- Lösung}
	
\begin{itemize}
\item[3.] Betrachten Sie die unten angegebenen Kontexte. Diskutieren Sie kurz anhand dieser Beispiele, ob es sich bei der Groß- und Kleinschreibung des markierten Buchstabens um unterschiedliche Grapheme handeln kann oder nicht.

\begin{exe}
	\exr{ex:04aHA3}
	\begin{xlist}
	\ex Dieser \underline{W}eg ist sehr steil.
	\ex \underline{W}ege, die ich nicht bewandert habe, gibt es viele.
	\ex Meine Schlüssel sind \underline{w}eg.
	\ex \gqq{\underline{W}eg!}, schrie sie mich an und knallte mir die Tür vor der Nase zu.
	\ex Geh \underline{w}eg!
	\end{xlist}
\end{exe}
\end{itemize}
\end{frame}


%%%%%%%%%%%%%%%%%%%%%%%%%%%%%%%%%%	
\begin{frame}%[allowframebreaks]
	\frametitle{Hausaufgabe -- Lösung}

\begin{itemize}
%\pause
\item[\alertgreen{--}] \alertgreen{Graphem: Kleinste bedeutungsunterscheidende Einheit im schriftlichen System}

%\pause

\item[\alertgreen{--}] \alertgreen{\ab{Weg} und \ab{weg} kann als Minimalpaar angesehen werden, und \ab{W} und \ab{w} als unterschiedliche Grapheme, da sie bedeutungsunterscheidend sind (vgl.\ a und e). Es gibt darüber hinaus weitere Beispiele, die diese Tendenz zu belegen scheinen \ab{Reisen} \vs \ab{reisen}, \ab{Sie} \vs \ab{sie}, \ab{Gut} \vs \ab{gut}.}

%\pause

\item[\alertgreen{--}] \alertgreen{Andererseits kann die Großschreibung durch andere Prinzipien bedingt werden (\zB Satzanfang) und verliert somit den bedeutungsunterscheidenden Charakter (vgl.\ d und e).}

%\pause

\item[\alertgreen{--}] \alertgreen{Unter Berücksichtigung der gegebenen Beispiele könnte man zunächst vermuten, dass \ab{W} und \ab{w} unterschiedliche Grapheme  (vgl.\ Minimalpaare (a) und (c)). Die Groß- und  Kleinschreibung hat jedoch eine andere Funktion im Schriftsystem des Deutschen (\zB Markierung von Nomina und Satzanfängen) und wirkt sich somit nicht notwendigerweise bedeutungsunterscheidend aus.}
\end{itemize}

\end{frame}


%%%%%%%%%%%%%%%%%%%%%%%%%%%%%%%%%%		
\begin{frame}
	\frametitle{Hausaufgabe -- Lösung}

\begin{itemize}
\item[4.] Erläutern Sie stichpunktartig, welche (graphematische) Funktionen der Buchstabe \ab{h} in den folgenden Kontexten annimmt:

\begin{exe}
\exr{ex:04aHA4}
\settowidth\jamwidth{XXXXXXXXXXXXXXXXXXXXXXXXXXXXXXXXXX}
\begin{xlist}
	\ex Ha\underline{h}n: \loesung{2}{Dehnungs-h}
	
	\ex nä\underline{h}en: \loesung{3}{Silbentrennendes \ab{h}}
	
	\ex bein\underline{h}alten: \loesung{4}{Korrespondenz zu Phonem \textipa{/h/}}
	
	\ex Gesc\underline{h}ichte: \loesung{5}{Teil eines Trigraphen \ab{sch}} \loesung{5}{(Nicht Teil eines Lauts, sondern eines Graphems!)}
	
	\ex Geschic\underline{h}te: \loesung{6}{Teil eines Digraphen \ab{ch}} \loesung{6}{(Nicht Teil eines Lauts, sondern eines Graphems!)}
	
	\ex Dip\underline{h}thong: \loesung{7}{Teil eines Fremddigraphen \ab{ph}}
	
	\ex Dipht\underline{h}ong: \loesung{8}{Teil eines Fremddigraphen \ab{th}}
\end{xlist}
\end{exe}

\end{itemize}

\end{frame}


%%%%%%%%%%%%%%%%%%%%%%%%%%%%%%%%%%		
\begin{frame}
	\frametitle{Hausaufgabe -- Lösung}

\begin{itemize}
\item[5.] Geben Sie die \textbf{phonologische} Transkription, die \textbf{phonetische} Transkription und die \textbf{phonographische} Schreibung (nach der Phonem-Graphem-Korrespondenz) des folgenden Wortes an.

\begin{exe}
	\exr{ex:04aHA5} Abstellkammer
\end{exe}

\settowidth\jamwidth{XXXXXXXXXXXXXXXXXXXXXXXXXXXXXXXXX}
\item[] phonologisch: \loesung{2}{\textipa{/abStElkam@\textscr /}}

\item[] phonetisch: \loesung{3}{\textipa{[PapStElkam5]}}

\item[] phonographisch: \loesung{4}{\ab{abschtelkamer}}

\item[] \only<5->{\alertgreen{Hier erkennt man, dass es sich bei der phonographischen Trankskription um eine Phonem-Graphem-Korrespondenz (und nicht um eine Phon-Graphem-Korrespondenz) handelt.}}

\end{itemize}
\end{frame}