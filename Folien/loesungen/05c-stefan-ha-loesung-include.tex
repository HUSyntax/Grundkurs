%%%%%%%%%%%%%%%%%%%%%%%%%%%%%%%%%%
%% HA 1 - 05c Morphologie Stefan
%%%%%%%%%%%%%%%%%%%%%%%%%%%%%%%%%%


\begin{frame}
	\frametitle{Lösungen}

\begin{enumerate}
	\item Kreuzen Sie die korrekten Aussagen an: %\hfill(0,5 Punkte pro Aussage)\\
	
	\begin{itemize}
	\item[$\circ$] Die Graphemkette abarbeiten ist ein einzelnes phonologisches Wort im Deutschen.
	\item[$\circ$] \emph{Morphologieeinführungsbuch} ist ein orthographisch-graphemisches Wort des Deutschen, sowie \emph{introductory morphology book} ein orthographisch-graphemisches Wort des Englischen ist.
	\item[$\circ$] Ein Morphem ist die kleinste bedeutungsunterscheidende Einheit in einem bestimmten Sprachsystem.
	\item[\alertgreen{$\checkmark$}] \alertgreen{\ab{Brot} und \ab{Bröt} sind Allomorphe eines einzelnen Morphems.}
\end{itemize}
	
	\item Erklären Sie das Prinzip der Rechtsköpfigkeit in der Morphologie des Deutschen. Verwenden Sie bei Ihrer Erklärung die unten angegebenen Beispiele.%\hfill(4 Punkte)\\
	
	\settowidth\jamwidth{XXXXXXXXXXXXXXXXXXXXXXt}
		\eal
			\ex lichtblau, Blaulicht \loesung{2}{\ras Wortart}
			\ex die Fotowelt, das Weltfoto \loesung{3}{\ras Genus und Semantik}
			\ex der Bücherrücken/die Bücherrücken, das Rückenbuch/die Rückenbücher \loesung{4}{\ras Pluralflexion}
		\zl		
\end{enumerate}

\end{frame}

%%%%%%%%%%%%%%%%%%%%%%%%%%%%%%%%%%%%%%%%%%%%%%%%%%%%%%%%%%%%

\begin{frame}
\frametitle{Lösungen}
\begin{itemize}
	\item[3.] Geben Sie Argumente für oder gegen die Behandlung von \emph{ver-} in den folgenden Wörtern als Morphem an. Wenn es sich um ein Morphem handelt, ist das immer das gleiche Morphem? %(4 Punkte)
	
	\eal \label{ver}
	\ex\label{vera}  \emph{Ver}zweiflung
	\ex\label{verb} \emph{Ver}s
	\ex \label{verc} \emph{ver}kaufen
	\ex\label{verd}  \emph{ver}schreiben
	\ex\label{vere} \emph{ver}fahren
	\zl

	\begin{itemize}

		\item[] \alertgreen{Morphem: Kleinste bedeutungstragende Einheit im Sprachsystem.}

		\begin{itemize}{\alertgreen
			\item[] \gqq{ver} in (\ref{verb}) ist kein Morphem, sondern Bestandteil des Stammes.
			\item[] \gqq{ver} in (\ref{ver} a,c,d,e) sind Morpheme, aber unterschiedliche Morpheme, weil sie unterschiedliche Bedeutungen tragen
			\item[] \gqq{ver} in (\ref{ver} d,e) trägt die Bedeutung \gq{X falsch machen} (d.h. \gq{falsch schreiben/fahren})
			\item[] \gqq{ver} in (\ref{verc}) kehrt die Bedeutung von X um (kaufen \ras verkaufen)
			\item[] \gqq{ver} in (\ref{vera}) trägt eine intensivierende(?) Bedeutung
		}
		\end{itemize}
	
	\end{itemize}


\end{itemize}
\end{frame}



%%%%%%%%%%%%%%%%%%%%%%%%%%%%%%%%%%%%%%%%%%%%%%%%%%%%%%%%%%%%

\begin{frame}
\frametitle{Lösungen}

\begin{itemize}
	\item[4.] Ordnen Sie die Wortbildungsprozesse links den passenden Beispielen rechts zu (dazu müssen Sie nur den entsprechenden Buchstaben neben das passende Beispiel schreiben). %(0,5 Punkte pro Aussage)
\end{itemize}

	\begin{table}[h!]
	\begin{minipage}{0.4\linewidth}
		\centering
		\begin{tabular}{|l|p{0.1\textwidth}|}
			\hline 
			Determinativkompositum & (A)\\
			\hline
			Konversion & (B)\\
			\hline
			Zirkumfigierung (Derivation) & (C)\\
			\hline
			Rektionskompositum & (D)\\
			\hline
			Possessivkompositum & (E)\\
			\hline 
		\end{tabular}
		
	\end{minipage}\hfill%
	\begin{minipage}{0.4\linewidth}
		\centering
		\begin{tabular}{|p{0.1\textwidth}|r|}
			\hline 
			\alertgreen{C} & \emph{Gerede} \\
			\hline
			\alertgreen{E} & \emph{Milchgesicht}\\
			\hline
			\alertgreen{B} & \emph{Lauf} \\
			\hline
			\alertgreen{A} & \emph{Kettenraucher}  \\
			\hline
			\alertgreen{D} & \emph{Klausurbesprechung}  \\
			\hline 
		\end{tabular}
	\end{minipage}
\end{table}
\end{frame}



%%%%%%%%%%%%%%%%%%%%%%%%%%%%%%%%%%%%%%%%%%%%%%%%%%%%%%%%%%%

\begin{frame}
\frametitle{Lösungen}
\begin{itemize}
	\item[5.] Warum sind die Wörter unter (\ref{kauf}) grammatisch und die unter (\ref{fenster}) ungrammatisch? %(4 Punkte)
	\eal
	\ex\label{kauf} kaufbar, trinkbar
	\ex\label{fenster} *fensterbar, *helfbar, *schönbar
	\zl
	
	\alertgreen{
		Das Suffix \gqq{-bar} hat die folgenden Beschränkungen bzgl. der Basis X, mit der es sich verbindet:
	}

		\begin{itemize}\alertgreen{
			\item[] X muss ein Verb sein (nicht Nomen oder Adjektiv)
			\item[] X muss transitiv sein (nicht wie \gqq{helfen})
		}
		\end{itemize}

\end{itemize}

\end{frame}


\begin{frame}
\frametitle{Lösungen}
\begin{itemize}
	\item [6.] Sind die folgenden Verben Präfixverben oder Partikelverben? Begründen Sie Ihre Entscheidungen. %(3 Punkte)
	
	\eal
	\ex auskennen
	\ex erkennen
	\ex aberkennen
	\zl
	
	\alertgreen{
		Partikelverb: 1) morphologisch trennbar (\emph{aus-ge-kannt}, \emph{ab-zu-erkennen}), 2) syntaktisch trennbar (\gqq{Peter \emph{kennt} sich \emph{aus}}, \gqq{Die Frau \emph{erkennt} die Urkunde \emph{ab}}) und 3) die Partikel trägt die Hauptbetonung (\emph{AUSkennen} und \emph{ABerkennen}).
		Präfixverb: weder morphologisch noch syntaktisch trennbar (*\emph{ergekannt}, \gqq{*\emph{Peter kannte ihn er}}), Hauptbetonung liegt auf der Basis (\emph{erKENnen}).\\
		\bigskip
		EXTRA: \gqq{aberkennen} ist ein Partikelverb, welches aus einem Präfixverb und einer Partikel besteht (ab+erkennen).\\
	}
\end{itemize}

\end{frame}

%%%%%%%%%%%%%%%%%%%%%%%%%%%%%%%%%%%%%%%%%%%%%%%%%%%%%%%%%%%

\begin{frame}
	\frametitle{Lösungen}
\begin{itemize}
	\item [7.] Geben Sie für das folgende Wort eine morphologische Konstituentenstruktur (inklusive Konstituentenkategorien (N, N\textsuperscript{af}, V, V\textsuperscript{af}, \dots)) an, und bestimmen Sie für jeden Knoten den Wortbildungstyp. %(6,5 Punkte)
\ea
Wahlkampfberaterinnen
\z

\scalebox{.6}{
\alertgreen{
	%	\begin{figure}[h]
	\begin{forest} MyP edges,
		[N, name=N1
		[N, name=N2
		[N, name=N3
		[N, name=N4
		[N, name=N6 [V[wahl/wähl]]]
		[N, name=N7 [V[kampf/kämpf]]]]
		[N, name=N5[V, name=V1	[V\textsubscript{af}[be-]]
		[V[rat]]]
		[N\textsuperscript{af}[-er]]]]
		[N\textsuperscript{af}[-in]]]
		[Fl[-nen]]]	
		{
			\draw[<-, HUgreen] (N1.west)--++(-10em,0pt)
			node[anchor=east,align=center]{Flexion (KEIN Wortbildungsporzess)};
			\draw[<-, HUgreen] (N2.west)--++(-12em,0pt)
			node[anchor=east,align=center]{Derivation (Movierung)};
			\draw[<-, HUgreen] (N3.west)--++(-8em,0pt)
			node[anchor=east,align=center]{Determinativkompositum};
			\draw[<-, HUgreen] (N4.west)--++(-3em,0pt)
			node[anchor=east,align=center]{Determinativkompositum};
			\draw[<-, HUgreen] (N5.west)--++(-2em,0pt)
			node[anchor=east,align=center]{Derivation};
			\draw[<-, HUgreen] (N6.west)--++(-2em,0pt)
			node[anchor=east,align=center]{Implizite Derivation};
			\draw[<-, HUgreen] (N7.east)--++(2.5em,0pt)--++(0em,-18ex)%--++(2em,0pt)
			node[anchor=north,align=center]{Implizite Derivation};
			\draw[<-, HUgreen] (V1.east)--++(1.5em,0pt)--++(0em,-14ex)--++(2em,0pt)
			node[anchor=west,align=center]{Derivation};
		}	
	\end{forest}	
	%	\end{figure}
}
}
\end{itemize}
\end{frame}


%%%%%%%%%%%%%%%%%%%%%%%%%%%%%%%%%%%%%%%%%%%%%%%%%%%%%%%%%%%

\begin{frame}
\frametitle{Lösungen}

\begin{itemize}
	
	\item [8.] Paraphrasieren Sie das folgende komplexe Wort so, dass es der angegebenen Struktur entspricht (auch wenn Sie selbst eine andere Struktur plausibler finden sollten). %(2 Punkte)
	
	\begin{forest}sn edges,
		[N
		[N[N[Reserve]]
		[N[V[lehr]][N\textsuperscript{af}[-er]]]]
		[N[zimmer]]
		]
	\end{forest}
	
	\item[] \alertgreen{
		ein Zimmer für Reservelehrer
	}
\end{itemize}

\end{frame}


%%%%%%%%%%%%%%%%%%%%%%%%%%%%%%%%%%%%%%%%%%%%%%%%%%%%%%%%%%%

\begin{frame}
\frametitle{Lösungen}

\begin{itemize}

\item [9.] Geben Sie für die folgende Wortform die Flexionskategorien an, nach denen sie flektiert ist.%\\
%\hfill(3 Punkte)\\
\ea
bestehe
\z

\item [] \alertgreen{
	1. \ras 1.P. / Sg. / Präsens / Indikativ / Aktiv\\
	2. \ras 1.P. / Sg. / Präsens / Konjunktiv I / Aktiv\\
	3. \ras 3.P. / Sg. / Präsens / Konjunktiv I / Aktiv\\
	4. \ras 2.P. / Sg. / Präsens / Imperativ / Aktiv\\
}
\end{itemize}

\end{frame}

