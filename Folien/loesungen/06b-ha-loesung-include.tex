%%%%%%%%%%%%%%%%%%%%%%%%%%%%%%%%
%06b Syntax ha-loesung
%%%%%%%%%%%%%%%%%%%%%%%%%%%%%%


\begin{frame}
\frametitle{Hausaufgabe -- Lösung}

\begin{itemize}
	\item Geben Sie die Wortart der folgenden Einheiten an:
\end{itemize}

\begin{columns}
	\column{.30\textwidth}
	\begin{enumerate}
		\item Maria
		\item Ach!
		\item kauft
		\item den
		\item an
		\item weil
		\item beachten
		\item obwohl
	\end{enumerate}
	
	\column{.50\textwidth}

\pause
	
	\begin{enumerate}
		\item[\ras] \alertgreen{Nomen (Substantiv)}
		\item[\ras] \alertgreen{Interjektion}
		\item[\ras] \alertgreen{Verb}
		\item[\ras] \alertgreen{Determinierer}
		\item[\ras] \alertgreen{Präposition}
		\item[\ras] \alertgreen{Complementizer/Subjunktion}
		\item[\ras] \alertgreen{Verb}
		\item[\ras] \alertgreen{Complementizer/Subjunktion}
	\end{enumerate}
\end{columns}

\end{frame}
	
	
	%%%%%%%%%%%%%%%%%%%%%%%%%%%%%%%%%%
	\begin{frame}
	\frametitle{Hausaufgabe -- Lösung}
	
	
	\begin{itemize}
		\item Testen Sie mithilfe von mindestens zwei Konstituententests, ob die fettgedruckten Wortfolgen eine oder mehrere Konstituenten sind.
		
		\vspace{.5cm}
		
		\item[] Maria \textbf{stolperte über} den Stein.
	\end{itemize}
	
	\pause 
	
	\ea Weglasstest \& Fragetest \ras \alertgreen{keine Konstituente}
	\ea[*]{Maria [stolperte über] den Stein und Peter [\sout{stolperte über}] den Ast.}
	\ex[*]{[Was] Maria den Stein?}
	\z 
	\z
	
\end{frame}


%%%%%%%%%%%%%%%%%%%%%%%%%%%%%%%%%%
\begin{frame}
\frametitle{Hausaufgabe -- Lösung}

\begin{itemize}
\item[] Der Minister wird in \textbf{der nächsten Woche} die Aussage wiederholen.
\end{itemize}

\pause 

\ea Ersetzungstest, Koordinationstest, Fragetest, Vorfeldtest \ras \alertgreen{keine Konstituente}

\ea[??]{Der Minister wird in [dieser] die Aussage wiederholen.}
\ex[?]{Der Minister wird in [der nächsten Woche] und [dem nächsten Treffen] die Aussage wiederholen.}
\ex[*]{[Wann] wird der Minister in die Aussage wiederholen?}
\ex[*]{[Der nächsten Woche] wird der Minister in die Aussage wiederholen.}
\z 
\z

\end{frame}


%%%%%%%%%%%%%%%%%%%%%%%%%%%%%%%%%%
\begin{frame}
\frametitle{Hausaufgabe -- Lösung}

\begin{itemize}
\item[] Die Besucher beobachteten \textbf{in der Werkstatt bemaltes Porzellan}.

\item[] Die Besucher beobachteten \textbf{in der Werkstatt} bemaltes Porzellan.
\end{itemize}

\pause 

\ea Vorfeldtest, Pronominalisierungstest \ras \alertgreen{1 Konstituente} 
\ea {[}In der Werkstatt bemaltes Porzellan{]} beobachteten die Besucher.
\ex Die Besucher beobachteten [es].
\z 

\pause 

\ex Verschiebetest, Pronominalisierungstest, Fragetests \ras \alertgreen{2 Konstituenten}
\ea Die Besucher beobachteten [bemaltes Porzellan] [in der Werkstatt]. 
\ex Die Besucher beobachteten [es] [dort].
\ex {[}Was{]} beobachteten die Besucher [in der Werkstatt]?
\ex {[}Wo{]} beobachteten die Besucher [bemaltes Porzellan]?
\z
\z

\end{frame}


%%%%%%%%%%%%%%%%%%%%%%%%%%%%%%%%%%%%%%%%%%%%%%%%%%%%%%%

\begin{frame}
\frametitle{Hausaufgabe -- Lösung}

\begin{itemize}
\item[] Erika traf die \textbf{Lehrerin mit den roten Schuhen}.
\end{itemize}

\pause 

\ea Weglasstest, Vorfeldtest, Fragetest \ras \alertgreen{keine Konstituente}
\ea[*]{Erika traf die und begrüßte die [Lehrerin mit den roten Schuhen].}
\ex[*]{[Lehrerin mit den roten Schuhen] traf Erika die.}
\ex[*]{[Wen] traf Erika die?}
\z 
\z
\end{frame}


%%%%%%%%%%%%%%%%%%%%%%%%%%%%%%%%%%%%%%%%%%%%%%%%%%%%%%%

\begin{frame}
\frametitle{Hausaufgabe -- Lösung}

\begin{itemize}
\item[] Helmut hat sehr lange \textbf{auf Maria gewartet}.

\item[] Helmut hat sehr lange \textbf{auf Maria} gewartet.
\end{itemize}

\pause 

\ea Vorfeldtest, Fragetest, Verschiebetest \ras \alertgreen{1 aber auch 2 Konstituenten}
\ea[]{[Auf Maria gewartet] hat Helmut sehr lange.}
\ex[]{[Was] hat Helmut sehr lange?}
\ex[??]{Helmut hat [auf Maria gewartet] sehr lange.}
\ex[]{Helmut hat [auf Maria] sehr lange [gewartet].}
\ex[]{[Auf Maria] hat Helmut sehr lange gewartet.} 
\z
\z 

\end{frame}


%%%%%%%%%%%%%%%%%%%%%%%%%%%%%%%%%%%%%%%%%%%%%%%%%%%%%%

\begin{frame}
\frametitle{Hausaufgabe -- Lösung}

\begin{itemize}
\item[] \textbf{Maria wird} nach diesem Kurs Syntax lieben.
\end{itemize}

\pause 

\ea Verschiebetest, Fragetest, Weglasstest \ras \alertgreen{keine Konstituente}
\ea[*]{Nach diesem Kurs [Maria wird] Syntax lieben.}
\ex[*]{[Wer] nach diesem Semester Syntax lieben?}
\ex[]{[Maria wird] nach diesem Kurs Syntax lieben und [\sout{Maria wird}] nur noch an Bäume denken.}
\z 
\z

\end{frame}