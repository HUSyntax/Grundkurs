\begin{frame}
\frametitle{Hausaufgabe -- Lösung}

\begin{itemize}
\item Welche Bedeutungsrelationen bzw.\ Ambiguitätsarten bestehen zwischen den folgenden Wortpaaren? Nennen Sie diese.
\end{itemize}

\ea 
	\ea betrunken -- nüchtern \pause 
	\hfill \alertgreen{konträre Antonymie}
	
	\ex Orange -- Apfelsine \pause 
	\hfill \alertgreen{Synonymie}
	
	\ex Vogel -- Feder \pause 
	\hfill \alertgreen{Meronymie (\textit{Feder} ist ein Meronym zu \textit{Vogel})}
	
	
	\ex volljährig -- minderjährig \pause 
	\hfill \alertgreen{kontradiktorische Antonymie}
	
	\ex mehr -- Meer \pause 
	\hfill \alertgreen{Homonymie (genauer: Homophonie)}
	\z 
\z 

\end{frame}

%%%%%%%%%%%%%%%%%%%%%%%%%%%%%%%%%%%%%%%%%%%%%

\begin{frame}
\frametitle{Hausaufgabe -- Lösung}

\begin{itemize}
	\item Welche semantischen Relationen bestehen zwischen den folgenden Sätzen? Definieren Sie diese.
\end{itemize}

\ea 
\ea Auf dem Tisch liegt eine Rose.
\ex Auf dem Tisch liegt eine Blume.

\pause 

\hfill \alertgreen{a impliziert b}
\z 

\pause 

\ex 	
\ea Alle Vögel können fliegen.
\ex Kein Vogel kann nicht fliegen.

\pause 

\hfill \alertgreen{Paraphrase (synonyme Sätze)}
\z 

\pause 

\ex 	
\ea Einige Tiere haben Federn.
\ex Alle Tiere haben Federn.

\pause 

\hfill \alertgreen{b impliziert a}
\z 	
\z 

\end{frame}

%%%%%%%%%%%%%%%%%%%%%%%%%%%%%%%%%%%%%%%%

\begin{frame}
\frametitle{Hausaufgabe -- Lösung}

\begin{itemize}
	\item Überprüfen Sie die Richtigkeit der folgenden Aussagen.
	
	\vspace{1em}
	
	\begin{itemize}
		\item Die komplexe Aussage (\ref{ex:Tau3}) ist \textbf{tautologisch}.
		
		\begin{exe}
			\exr{ex:Tau3} $\lnot (p \land \lnot p)$
		\end{exe}
	\end{itemize}	
	
\end{itemize}

\begin{table}
	\centering	
	\begin{tabular}{c|c|c|c}
		$p$& $\lnot p$ & $p \land \lnot p$ & $\lnot (p \land \lnot p)$ \\ 
		\hline 
		0 & 1 & 0& \alertgreen{1}\\ 
		\hline 
		1 & 0 & 0& \alertgreen{1}\\
	\end{tabular} 
\end{table} 

\alertgreen{Die komplexe Aussage ist tautologisch (Wahrheitswert immer 1).}

\end{frame}

%%%%%%%%%%%%%%%%%%%%%%%%%%%%%%%%%%
\begin{frame}
\frametitle{Hausaufgabe -- Lösung}

\begin{itemize}
\item Überprüfen Sie die Richtigkeit der folgenden Aussagen.

\vspace{1em}

\begin{itemize}	
	\item Die komplexe Aussage (\ref{ex:Kon2}) ist \textbf{kontradiktorisch}.
	
	\begin{exe}
		\exr{ex:Kon2} $\lnot ((p \lor q) \leftrightarrow (q \lor p))$
	\end{exe}		
\end{itemize}	

\end{itemize}

\begin{table}
\centering	
\scalebox{.9}{\begin{tabular}{c|c|c|c|c|c}
		$p$ & $q$ & $p \lor q$ & $q \lor p$ & $(p \lor q) \leftrightarrow (q \lor p)$ & $\lnot ((p \lor q) \leftrightarrow (q \lor p))$ \\ 
		\hline 
		1 & 1 & 1 & 1 & 1 & \alertgreen{0}\\ 
		\hline 
		1 & 0 & 1 & 1 & 1 & \alertgreen{0} \\
		\hline
		0 & 1 & 1 & 1 & 1 & \alertgreen{0} \\
		\hline
		0 & 0 & 0 & 0 & 1 & \alertgreen{0} \\
\end{tabular} }
\end{table} 

\alertgreen{Die komplexe Aussage ist kontradiktorisch (Wahrheitswert immer 0).}

\end{frame}

%%%%%%%%%%%%%%%%%%%%%%%%%%%%%%%%%%
\begin{frame}
\frametitle{Hausaufgabe -- Lösung}

\begin{itemize}
\item Überprüfen Sie die Richtigkeit der folgenden Aussagen.

\vspace{1em}

\begin{itemize}
\item Die komplexe Aussage (\ref{ex:Con2}) ist \textbf{kontingent}.

\begin{exe}
	\exr{ex:Con2} $((p \rightarrow q) \leftrightarrow (q \rightarrow p))$
\end{exe}

\end{itemize}	

\end{itemize}

\begin{table}
\centering	
\begin{tabular}{c|c|c|c|c}
$p$ & $q$ & $p \ras q$ & $q \ras p$ & $(p \ras q) \leftrightarrow (q \ras p)$ \\ 
\hline 
1 & 1 & 1 & 1 & \alertgreen{1} \\ 
\hline 
1 & 0 & 0 & 1 & \alertgreen{0} \\
\hline
0 & 1 & 1 & 0 & \alertgreen{0} \\
\hline
0 & 0 & 1 & 1 & \alertgreen{1} \\
\end{tabular} 
\end{table} 

\alertgreen{Die komplexe Aussage ist kontigent (Wahrheitswert von der Welt abhängig).}

\end{frame}

%%%%%%%%%%%%%%%%%%%%%%%%%%%%%%%%%%%%%%%%%%

\begin{frame}
\frametitle{Hausaufgabe -- Lösung}

\begin{itemize}
	\item Geben Sie den Wahrheitswert der folgenden Formeln in einer Welt/Situation an, in der $p=0$ und $q=1$ sind.
\end{itemize}

\begin{exe}
	\exr{ex:Wert1} $(p \land q)$ \pause  \alertgreen{= 0}
	\exr{ex:Wert2} $(p \rightarrow (q \lor p))$ \pause \alertgreen{= 1}
	\exr{ex:Wert3} $((q \land q) \lor (p \land q))$ \pause \alertgreen{= 1}
\end{exe}
\end{frame}
