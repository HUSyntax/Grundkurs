%%%%%%%%%%%%%%%%%%%%%%%%%%%%%%%%%%
%% HA 1 - 02 Phonetik
%%%%%%%%%%%%%%%%%%%%%%%%%%%%%%%%%%

\begin{frame}
\frametitle{Hausaufgabe -- Lösung}

\begin{itemize}
	\item[1.] {Transkribieren Sie folgende Wörter des Deutschen mit dem IPA:}

		\begin{exe}
		\exr{ex:02HA1}
		\settowidth\jamwidth{XXXXXXXXXXXXXXXXXXXXXXXXXXXXXXXX} 
		\begin{xlist}
			\ex arbeiten\loesung{2}{
						\textipa{['PaK.b\texttoptiebar{aI}.t@n], ['Pa\;R.b\texttoptiebar{aI}.t@n], ['P\texttoptiebar{a5}.b\texttoptiebar{aI}.t@n]}
					}
			\ex Giebel\loesung{3}{
						\textipa{['gi:.b@l]}
					}
			\ex sagen\loesung{4}{
						\textipa{['za:.g@n]}
					}
			\ex fröhlich\loesung{5}{
						\textipa{['f\;R\o:.lI\c{c}]}
					}
			\ex Enge\loesung{6}{
						\textipa{['PE\.N@]}
					}
			\ex Dampfschiff\loesung{7}{
						\textipa{['dam\texttoptiebar{pf}.SIf]}
					}
		\end{xlist}
	\end{exe}

\end{itemize}

\end{frame}	


%%%%%%%%%%%%%%%%%%%%%%%%%%%%%%%%%%%
\begin{frame}
\frametitle{Hausaufgabe -- Lösung}

\begin{itemize}
	\item[2.] {Schreiben Sie für die folgenden Lautbeschreibungen das passende IPA-Symbol auf:}
	
	\begin{exe}
		\exr{ex:02HA2} 
		\settowidth\jamwidth{XXXXXX} 
		\begin{xlist}
			\ex bilabialer stimmloser Plosiv\loesung{2}{
						\textipa{[p]}
					}
			\ex hoher vorderer ungerundeter gespannter Vokal\loesung{3}{
					\textipa{[i:]}
				}
			\ex velarer dorsaler stimmloser Frikativ\loesung{4}{
						\textipa{[x]}
			}
			\ex glottaler stimmloser Plosiv\loesung{5}{
						\textipa{[P]}
			}
			\ex halbhoher fast vorderer ungerundeter ungespannter Vokal\loesung{6}{
					\textipa{[I]}
				}
			\ex postalveolarer stimmhafter Frikativ\loesung{7}{
						\textipa{[Z]}
					}
			\ex halbtiefer zentraler ungerundeter Vokal\loesung{8}{
						\textipa{[5]}
					}
			\ex obermittelhoher hinterer gerundeter gespannter Vokal\loesung{9}{
						\textipa{[o:]}
					}
			\ex alveolare koronale stimmlose Affrikate\loesung{10}{
						\textipa{[\texttoptiebar{ts}]}
			}
			\ex mittlerer zentraler ungerundeter Vokal\loesung{11}{
						\textipa{[@]}
			}
		\end{xlist}
	\end{exe}
	
\end{itemize}

\end{frame}


%%%%%%%%%%%%%%%%%%%%%%%%%%%%%%%%%%
\begin{frame}
\frametitle{Hausaufgabe -- Lösung}

\begin{itemize}
	\item[3.] {Beschreiben Sie folgende Laute des Deutschen mit relevanten phonetischen Merkmalen:}
	
	\begin{exe}
		\exr{ex:02HA3}
		\settowidth\jamwidth{XXXXXXXXXXXXXXXXXXXXXXXXXXXXXXXXXXXXXX}
		\begin{xlist}
			\ex \textipa{[y:]}\loesung{2}{
						hoher vorderer gerundeter gespannter Vokal
					}
			\ex \textipa{[x]}\loesung{3}{
						velarer dorsaler stimmloser Frikativ
					}
			\ex \textipa{[\o:]}\loesung{4}{
						obermittelhoher vorderer gerundeter gespannter Vokal
					}
			\ex \textipa{[Z]}\loesung{5}{
						postalveolarer koronaler stimmhafter Frikativ
					}
			\ex \textipa{[t]}\loesung{6}{
						alveolarer koronaler stimmloser Plosiv
					}
			\ex \textipa{[E]}\loesung{7}{
						untermittelhoher vorderer ungerundeter ungespannter Vokal
					}
			\ex \textipa{[m]}\loesung{8}{
						bilabialer stimmhafter Nasal
					}
			\ex \textipa{[O]}\loesung{9}{
						untermittelhoher hinterer gerundeter ungespannter Vokal
					}
		\end{xlist}
	\end{exe}
	
\end{itemize}

\end{frame}

