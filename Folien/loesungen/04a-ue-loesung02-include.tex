%%%%%%%%%%%%%%%%%%%%%%%%%%%%%%%%%%
%% UE 2 - 04a Graphematik
%%%%%%%%%%%%%%%%%%%%%%%%%%%%%%%%%%

\begin{frame}
\frametitle{Übung -- Lösung}

\begin{itemize}	
	\item Versuchen Sie, graphematische Regularitäten und Prinzipien zu finden, die die Unterscheidung lang \vs kurz bei Vokalen anzeigen. Gibt es Ausnahmen?
	
	\ea\label{ex:mutter}
	
	\begin{multicols}{2}
		\ea Mutter
		\ex Mehl
		\ex See
		\ex Nase
		\ex dehnen
		\ex gehen
		\ex Bier
		\ex Moor
		\ex an
		\ex zum
		\ex Mann
		\ex man
		\ex Herbst
		\ex Laub
		\ex sehr
		\ex Bohrer
		\z
	\end{multicols}
	
	\z
	
	
	
\end{itemize}

\end{frame}

