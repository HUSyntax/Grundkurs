%%%%%%%%%%%%%%%%%%%%%%%%%%%%%%%%%%
%% UE 2 - 04a Graphematik
%%%%%%%%%%%%%%%%%%%%%%%%%%%%%%%%%%

\begin{frame}
\frametitle{Übung -- Lösung}

\begin{itemize}	
	\item Versuchen Sie, graphematische Regularitäten und Prinzipien zu finden, die die Unterscheidung lang \vs kurz bei Vokalen anzeigen. Gibt es Ausnahmen?\\
	
%	\vspace{-.3cm}
	
	\begin{exe}
		\exr{ex:04amutter}
		\begin{xlist}
			\begin{multicols}{4}
				\ex Mutter
				\ex Mehl
				\ex See
				\ex Nase
				\ex dehnen
				\ex gehen
				\ex Bier
				\ex Moor
				\ex rot
				\ex zum
				\ex Mann
				\ex man
				\ex Herbst
				\ex Laub
				\ex sehr
				\ex Bohrer
			\end{multicols}
		\end{xlist}
	\end{exe}
	
\end{itemize}

%\vspace{-.3cm}


\begin{multicols}{2}
\begin{itemize}
\item[\alertgreen{--}] \alertgreen{Diphthonge: lang (n.)}
\item[\alertgreen{--}] \alertgreen{Doppelvokale: lang (c., h.), Sonderfall bei \ab{i}: \ab{ie} (g.)}
\item[\alertgreen{--}] \alertgreen{vor Dehnungs-h: lang (b., e., o., p.)}
\item[\alertgreen{--}] \alertgreen{vor Doppelkonsonant: kurz (a., k.)}
\item[\alertgreen{--}] \alertgreen{vor Konsonantencluster: kurz (m.)}
\item[\alertgreen{--}] \alertgreen{offene Schreibsilben: lang bei Hauptbetonung (d., f.),
\\sonst kurz (d.)}
\item[\alertgreen{--}] \alertgreen{einfach geschlossene Schreibsilben: uneindeutig bei Hauptbetonung\\(i., j., l.), sonst kurz (a., e., f., p.)}
\end{itemize}
\end{multicols}

\end{frame}

