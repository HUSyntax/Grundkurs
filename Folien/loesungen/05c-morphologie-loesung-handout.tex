%%%%%%%%%%%%%%%%%%%%%%%%%%%%%%%%%%%%%%%%%%%%%%
%% Compile: XeLaTeX BibTeX XeLaTeX XeLaTeX
%% Loesung-Handout: Antonio Machicao y Priemer
%% Course: GK Linguistik
%%%%%%%%%%%%%%%%%%%%%%%%%%%%%%%%%%%%%%%%%%%%%%

%\documentclass[a4paper,10pt, bibtotoc]{beamer}
\documentclass[10pt,handout]{beamer}

%%%%%%%%%%%%%%%%%%%%%%%%
%%     PACKAGES      
%%%%%%%%%%%%%%%%%%%%%%%%

%%%%%%%%%%%%%%%%%%%%%%%%
%%     PACKAGES       %%
%%%%%%%%%%%%%%%%%%%%%%%%



%\usepackage[utf8]{inputenc}
%\usepackage[vietnamese, english,ngerman]{babel}   % seems incompatible with german.sty
%\usepackage[T3,T1]{fontenc} breaks xelatex
\usepackage{lmodern}

\usepackage{amsmath}
\usepackage{amsfonts}
\usepackage{amssymb}
%% MnSymbol: Mathematische Klammern und Symbole (Inkompatibel mit ams-Packages!)
%% Bedeutungs- und Graphemklammern: $\lsem$ Tisch $\rsem$ $\langle TEXT \rangle$ $\llangle$ TEXT $\rrangle$ 
\usepackage{MnSymbol}
%% ulem: Strike out
\usepackage[normalem]{ulem}  

%% Special Spaces (s. Commands)
\usepackage{xspace}				
\usepackage{setspace}
%	\onehalfspacing

%% mdwlist: Special lists
\usepackage{mdwlist}	

\usepackage[noenc,safe]{tipa}

% maybe define \textipa to use \originalTeX to avoid problems with `"'.
%
%	\ex \textipa{\originalTeX [pa.pa."g\t{aI}]}

%

\usepackage{etex}		%For Forest bug

%
%\usepackage{jambox}
%


%\usepackage{forest-v105}
%\usepackage{modified-langsci-forest-setup}

\usepackage{xeCJK}
\setCJKmainfont{SimSun}


%\usepackage{natbib}
%\setcitestyle{notesep={:~}}




% for toggles
\usepackage{etex}



% Fraktur!
\usepackage{yfonts}

\usepackage{url}

% für UDOP
\usepackage{adjustbox}


%% huberlin: Style sheet
%\usepackage{huberlin}
\usepackage{hu-beamer-includes-pdflatex}
\huberlinlogon{0.86cm}


%% Last Packages
%\usepackage{hyperref}	%URLs
%\usepackage{gb4e}		%Linguistic examples

% sorry this was incompatible with gb4e and had to go.
%\usepackage{linguex-cgloss}	%Linguistic examples (patched version that works with jambox

\usepackage{multirow}  %Mehrere Zeilen in einer Tabelle
%\usepackage{array}
\usepackage{marginnote}	%Notizen




%%%%%%%%%%%%%%%%%%%%%%%%%%%%%%%%%%%%%%%%%%%%%%%%%%%%
%%%          Commands                            %%%
%%%%%%%%%%%%%%%%%%%%%%%%%%%%%%%%%%%%%%%%%%%%%%%%%%%%

%%%%%%%%%%%%%%%%%%%%%%%%%%%%%%%%
% German quotation marks:
\newcommand{\gqq}[1]{\glqq{}#1\grqq{}}		%double
\newcommand{\gq}[1]{\glq{}#1\grq{}}			%simple


%%%%%%%%%%%%%%%%%%%%%%%%%%%%%%%%
% Abbreviations in German
% package needed: xspace
% Short space in German abbreviations: \,	
\newcommand{\idR}{\mbox{i.\,d.\,R.}\xspace}
\newcommand{\su}{\mbox{s.\,u.}\xspace}
%\newcommand{\ua}{\mbox{u.\,a.}\xspace}       % in abbrev
%\newcommand{\zB}{\mbox{z.\,B.}\xspace}       % in abbrev
%\newcommand{\s}{s.~}
%not possibel: \dh --> d.\,h.


%%%%%%%%%%%%%%%%%%%%%%%%%%%%%%%%
%Abbreviations in English
\newcommand{\ao}{a.o.\ }	% among others
\newcommand{\cf}[1]{(cf.~#1)}	% confer = compare
\renewcommand{\ia}{i.a.}	% inter alia = among others
\newcommand{\ie}{i.e.~}	% id est = that is
\newcommand{\fe}{e.g.~}	% exempli gratia = for example
%not possible: \eg --> e.g.~
\newcommand{\vs}{vs.\ }	% versus
\newcommand{\wrt}{w.r.t.\ }	% with respect to


%%%%%%%%%%%%%%%%%%%%%%%%%%%%%%%%
% Dash:
\newcommand{\gs}[1]{--\,#1\,--}


%%%%%%%%%%%%%%%%%%%%%%%%%%%%%%%%
% Rightarrow with and without space
\def\ra{\ensuremath\rightarrow}			%without space
\def\ras{\ensuremath\rightarrow\ }		%with space


%%%%%%%%%%%%%%%%%%%%%%%%%%%%%%%%
%% X-bar notation

%% Notation with primes (not emphasized): \xbar{X}
\newcommand{\MyPxbar}[1]{#1$^{\prime}$}
\newcommand{\xxbar}[1]{#1$^{\prime\prime}$}
\newcommand{\xxxbar}[1]{#1$^{\prime\prime\prime}$}

%% Notation with primes (emphasized): \exbar{X}
\newcommand{\exbar}[1]{\emph{#1}$^{\prime}$}
\newcommand{\exxbar}[1]{\emph{#1}$^{\prime\prime}$}
\newcommand{\exxxbar}[1]{\emph{#1}$^{\prime\prime\prime}$}

% Notation with zero and max (not emphasized): \xbar{X}
\newcommand{\zerobar}[1]{#1$^{0}$}
\newcommand{\maxbar}[1]{#1$^{\textsc{max}}$}

% Notation with zero and max (emphasized): \xbar{X}
\newcommand{\ezerobar}[1]{\emph{#1}$^{0}$}
\newcommand{\emaxbar}[1]{\emph{#1}$^{\textsc{max}}$}

%% Notation with bars (already implemented in gb4e):
% \obar{X}, \ibar{X}, \iibar{X}, \mbar{X} %Problems with \mbar!
%
%% Without gb4e:
\newcommand{\overbar}[1]{\mkern 1.5mu\overline{\mkern-1.5mu#1\mkern-1.5mu}\mkern 1.5mu}
%
%% OR:
\newcommand{\MyPibar}[1]{$\overline{\textrm{#1}}$}
\newcommand{\MyPiibar}[1]{$\overline{\overline{\textrm{#1}}}$}
%% (emphasized):
\newcommand{\eibar}[1]{$\overline{#1}$}
\newcommand{\eiibar}[1]{\overline{$\overline{#1}}$}

%%%%%%%%%%%%%%%%%%%%%%%%%%%%%%%%
%% Subscript & Superscript: no italics
\newcommand{\MyPdown}[1]{$_{\textrm{#1}}$}
\newcommand{\MyPup}[1]{$^{\textrm{#1}}$}


%%%%%%%%%%%%%%%%%%%%%%%%%%%%%%%%
% Objekt language marking:
%\newcommand{\obj}[1]{\glqq{}#1\grqq{}}	%German double quotes
%\newcommand{\obj}[1]{``#1''}			%English double quotes
\newcommand{\MyPobj}[1]{\emph{#1}}		%Emphasising


%%%%%%%%%%%%%%%%%%%%%%%%%%%%%%%%
%% Semantic types (<e,t>), features, variables and graphemes in angled brackets 

%%% types and variables, in math mode: angled brackets + italics + no space
%\newcommand{\type}[1]{$<#1>$}

%%% OR more correctly: 
%%% types and variables, in math mode: chevrons! + italics + no space
\newcommand{\MyPtype}[1]{$\langle #1 \rangle$}

%%% features and graphemes, in math mode: chevrons! + italics + no space
\newcommand{\abe}[1]{$\langle #1 \rangle$}


%%% features and graphemes, in math mode: chevrons! + no italics + space
\newcommand{\ab}[1]{$\langle$#1$\rangle$}  %%same as \abu  
\newcommand{\abu}[1]{$\langle$#1$\rangle$} %%Umlaute

%%% Notizen
\renewcommand{\marginfont}{\singlespacing}
\renewcommand{\marginfont}{\footnotesize}
\renewcommand{\marginfont}{\color{black}}

\newcommand{\myp}[1]{%
	\marginnote{%
		\begin{spacing}{1}
			\vspace{-\baselineskip}%
			\color{red}\footnotesize#1
		\end{spacing}
	}
}
%%%%%%%%%%%%%%%%%%%%%%%%%%%%%%%%
%% Outputbox
\newcommand{\outputbox}[1]{\noindent\fbox{\parbox[t][][t]{0.98\linewidth}{#1}}\vspace{0.5em}}

%%%%%%%%%%%%%%%%%%%%%%%%%%%%%%%%
%% (Syntactic) Trees
% package needed: forest
%
%% Setting for simple trees
\forestset{
	MyP edges/.style={for tree={parent anchor=south, child anchor=north}}
}

%% this is taken from langsci-setup file
%% Setting for complex trees
%% \forestset{
%% 	sn edges/.style={for tree={parent anchor=south, child anchor=north,align=center}}, 
%% background tree/.style={for tree={text opacity=0.2,draw opacity=0.2,edge={draw opacity=0.2}}}
%% }

\newcommand\HideWd[1]{%
	\makebox[0pt]{#1}%
}


%%%%%%%%%%%%%%%%%%%%%%%%%%%%%%%%%%%%%%%%%%%%%%%%%%%%
%%%          Useful commands                     %%%
%%%%%%%%%%%%%%%%%%%%%%%%%%%%%%%%%%%%%%%%%%%%%%%%%%%%

%%%%%%%%%%%%%%%%%%%%%
%% FOR ITEMS:
%\begin{itemize}
%  \item<2-> from point 2
%  \item<3-> from point 3 
%  \item<4-> from point 4 
%\end{itemize}
%
% or: \onslide<2->
% or: \pause

%%%%%%%%%%%%%%%%%%%%%
%% VERTICAL SPACE:
% \vspace{.5cm}
% \vfill

%%%%%%%%%%%%%%%%%%%%%
% RED MARKING OF TEXT:
%\alert{bis spätestens Mittwoch, 18 Uhr}

%%%%%%%%%%%%%%%%%%%%%
%% RESCALE BIG TABLES:
%\scalebox{0.8}{
%For Big Tables
%}

%%%%%%%%%%%%%%%%%%%%%
%% BLOCKS:
%\begin{alertblock}{Title}
%Text
%\end{alertblock}
%
%\begin{block}{Title}
%Text
%\end{block}
%
%\begin{exampleblock}{Title}
%Text
%\end{exampleblock}


\newtoggle{uebung}
\newtoggle{loesung}
\newtoggle{toc}

% The toc is not needed on Handouts. Safe trees.
\mode<handout>{
\togglefalse{toc}
}

\newtoggle{hpsgvorlesung}\togglefalse{hpsgvorlesung}
\newtoggle{syntaxvorlesungen}\togglefalse{syntaxvorlesungen}

%\includecomment{psgbegriffe}
%\excludecomment{konstituentenprobleme}
%\includecomment{konstituentenprobleme-hinweis}

\newtoggle{konstituentenprobleme}\togglefalse{konstituentenprobleme}
\newtoggle{konstituentenprobleme-hinweis}\toggletrue{konstituentenprobleme-hinweis}

%\includecomment{einfsprachwiss-include}
%\excludecomment{einfsprachwiss-exclude}
\newtoggle{einfsprachwiss-include}\toggletrue{einfsprachwiss-include}
\newtoggle{einfsprachwiss-exclude}\togglefalse{einfsprachwiss-exclude}

\newtoggle{psgbegriffe}\toggletrue{psgbegriffe}

\newtoggle{gb-intro}\togglefalse{gb-intro}



%%%%%%%%%%%%%%%%%%%%%%%%%%%%%%%%%%%%%%%%%%%%%%%%%%%%
%%%             Preamble's End                   
%%%%%%%%%%%%%%%%%%%%%%%%%%%%%%%%%%%%%%%%%%%%%%%%%%%% 

\begin{document}
	
	
%%%% ue-loesung
%%%% true: Übung & Lösungen (slides) / false: nur Übung (handout)
%	\toggletrue{ue-loesung}

%%%% ha-loesung
%%%% true: Hausaufgabe & Lösungen (slides) / false: nur Hausaufgabe (handout)
%	\toggletrue{ha-loesung}

%%%% toc
%%%% true: TOC am Anfang von Slides / false: keine TOC am Anfang von Slides
\toggletrue{toc}

%%%% sectoc
%%%% true: TOC für Sections / false: keine TOC für Sections (StM handout)
%	\toggletrue{sectoc}

%%%% gliederung
%%%% true: Gliederung für Sections / false: keine Gliederung für Sections
%	\toggletrue{gliederung}


%%%%%%%%%%%%%%%%%%%%%%%%%%%%%%%%%%%%%%%%%%%%%%%%%%%%
%%%             Metadata                         
%%%%%%%%%%%%%%%%%%%%%%%%%%%%%%%%%%%%%%%%%%%%%%%%%%%%      

\title{Grundkurs Linguistik}

\subtitle{Lösungen -- Morphologie III}

\author[A. Machicao y Priemer]{
	{\small Antonio Machicao y Priemer}
	\\
	{\footnotesize \url{http://www.linguistik.hu-berlin.de/staff/amyp}}
	%	\\
	%	\href{mailto:mapriema@hu-berlin.de}{mapriema@hu-berlin.de}}
}

\institute{Institut für deutsche Sprache und Linguistik}


% bitte lassen, sonst kann man nicht sehen, von wann die PDF-Datei ist.
%\date{ }

%\publishers{\textbf{6. linguistischer Methodenworkshop \\ Humboldt-Universität zu Berlin}}

%\hyphenation{nobreak}


%%%%%%%%%%%%%%%%%%%%%%%%%%%%%%%%%%%%%%%%%%%%%%%%%%%%
%%%             Preamble's End                  
%%%%%%%%%%%%%%%%%%%%%%%%%%%%%%%%%%%%%%%%%%%%%%%%%%%%      


%%%%%%%%%%%%%%%%%%%%%%%%%      
\huberlintitlepage[22pt]
\iftoggle{toc}{
	\frame{
		\begin{multicols}{2}
			\frametitle{Inhaltsverzeichnis}
			\tableofcontents
			%[pausesections]
			\columnbreak
			\textcolor{white}{
				\ea \label{ex:05cHA2}
					\ea \label{ex:05cHA2a}
					\ex \label{ex:05cHA2b}
					\ex \label{ex:05cHA2c}
					\z
				\ex \label{ex:05cHA3}
					\ea \label{ex:05cHA3a}
					\ex \label{ex:05cHA3b}
					\ex \label{ex:05cHA3c}
					\ex \label{ex:05cHA3d}
					\ex \label{ex:05cHA3e}
					\z
				\ex \label{ex:05cHA5}
					\ea \label{ex:05cHA5a}
					\ex \label{ex:05cHA5b}
					\z
				\ex \label{ex:05cHA6}
				\ex \label{ex:05cHA7}
				\ex \label{ex:05cHA9}
				\z
			}
		\end{multicols}
	}
}


%%%%%%%%%%%%%%%%%%%%%%%%%%%%%%%%%%%
%%%%%%%%%%%%%%%%%%%%%%%%%%%%%%%%%%%
%\section{Übungen}


%%%%%%%%%%%%%%%%%%%%%%%%%%%%%%%%%%%
%%%%%%%%%%%%%%%%%%%%%%%%%%%%%%%%%%%
\section{Hausaufgaben}

%%%%%%%%%%%%%%%%%%%%%%%%%%%%%%%%%%
%% HA 1 - 05c Morphologie
%%%%%%%%%%%%%%%%%%%%%%%%%%%%%%%%%%

\begin{frame}{Hausaufgabe -- Lösung}

\begin{enumerate}
	\item Kreuzen Sie die korrekten Aussagen an %\hfill(0,5 Punkte pro Aussage)\\

\begin{itemize}
	\item[$\circ$] Die Graphemkette abarbeiten ist ein einzelnes phonologisches Wort im Deutschen.
	\item[$\circ$] \emph{Morphologieeinführungsbuch} ist ein orthographisch-graphemisches Wort des Deutschen, sowie \emph{introductory morphology book} ein orthographisch-graphemisches Wort des Englischen ist.
	\item[$\circ$] Ein Morphem ist die kleinste bedeutungsunterscheidende Einheit in einem bestimmten Sprachsystem.
	\alertred{
			\item[\alertred{$\checkmark$}] \ab{Brot} und \ab{Bröt} sind Allomorphe eines einzelnen Morphems.
		}
\end{itemize}

\end{enumerate}
\end{frame}


%%%%%%%%%%%%%%%%%%%%%%%%%%%%%%%%%%
\begin{frame}{Hausaufgabe -- Lösung}

\begin{enumerate}
	\item[2.] Erklären Sie das Prinzip der Rechtsköpfigkeit in der Morphologie des Deutschen. Verwenden Sie bei Ihrer Erklärung die unten angegebenen Beispiele. %\hfill(4 Punkte)\\

\begin{exe}
	\exr{ex:05cHA2}
	\begin{xlist}
		\ex lichtblau, Blaulicht
		\ex die Fotowelt, das Weltfoto
		\ex	die Bücherrücken, die Rückenbücher
	\end{xlist}
\end{exe}

\pause

\alertred{Der Kopf eines Wortes ist immer rechtsperipher. Er bestimmt die morphosyntaktischen Eigenschaften eines Wortes sowie viele semantische Aspekte, \zB}

	\begin{itemize}
		\item[\alertred{--}] \alertred{(\ref{ex:05cHA2a}): Wortart (A \vs N)}
		\item[\alertred{--}] \alertred{(\ref{ex:05cHA2b}): Genus (f \vs n)}
		\item[\alertred{--}] \alertred{(\ref{ex:05cHA2c}): Pluralflexion (endungslos \vs \emph{-er})}
		\item[\alertred{--}] \alertred{Bei Determinativkomposita bildet das Kompositum eine Unterart des Kopfes, bspw. geht es in (\ref{ex:05cHA2c}) im ersten Fall um einen bestimmen Blauton und im zweiten um eine bestimmte Art von Licht.}
	\end{itemize}

\end{enumerate}
\end{frame}


%%%%%%%%%%%%%%%%%%%%%%%%%%%%%%%%%

\begin{frame}{Hausaufgabe -- Lösung}

\begin{enumerate}
\item[3.] Geben Sie Argumente für oder gegen die Behandlung von \emph{ver-} in den folgenden Wörtern als Morphem an. Wenn es sich um ein Morphem handelt, ist das immer das gleiche Morphem?%\\
%\hfill(4 Punkte)\\

\begin{exe}
	\exr{ex:05cHA3}
	\begin{xlist}
		\ex \emph{Ver}zweiflung
		\ex \emph{Ver}s
		\ex \emph{ver}kaufen
		\ex \emph{ver}schreiben
		\ex \emph{ver}fahren
	\end{xlist}
\end{exe}

\pause

\alertred{Morphem: Kleinste bedeutungstragende Einheit im Sprachsystem.}

\begin{itemize}
	\item[\alertred{--}] \alertred{\emph{ver-} in (\ref{ex:05cHA3b}) ist kein Morphem, sondern Bestandteil des Stammes.}
	\item[\alertred{--}] \alertred{\emph{ver-} in (\ref{ex:05cHA3d}) und (\ref{ex:05cHA3e}) ist ein Morphem mit der Bedeutung \gq{X falsch machen}.}
	\item[\alertred{--}] \alertred{\emph{ver-} in (\ref{ex:05cHA3a}) und (\ref{ex:05cHA3c}) sind auch Morpheme, aber zwei andere Morpheme, weil sie jeweils abweichende Bedeutungen tragen:}
		\begin{itemize}
			\item[] \alertred{\emph{ver-} in (\ref{ex:05cHA3c}) kehrt die Bedeutung von X um.}
			\item[] \alertred{\emph{ver-} in (\ref{ex:05cHA3a}) trägt eine intensivierende(?) Bedeutung.}
		\end{itemize}
\end{itemize}

\end{enumerate}
\end{frame}


%%%%%%%%%%%%%%%%%%%%%%%%%%%%%%%%%
\begin{frame}{Hausaufgabe -- Lösung}

\begin{enumerate}
\item[4.] Ordnen Sie die Wortbildungsprozesse links den passenden Beispielen rechts zu (dazu müssen Sie nur den entsprechenden Buchstaben neben das passende Beispiel schreiben). %\\
%\hfill(0,5 Punkte pro Aussage)\\

\begin{table}[h!]
	\begin{minipage}{0.4\linewidth}
		\centering
		\begin{tabular}{l|p{0.1\textwidth}|}
			Determinativkompositum & (A)\\
			\hline
			Konversion & (B)\\
			\hline
			Zirkumfigierung (Derivation) & (C)\\
			\hline
			Rektionskompositum & (D)\\
			\hline
			Possessivkompositum & (E)\\
		\end{tabular}
	
\end{minipage}\hfill%
\begin{minipage}{0.4\linewidth}
\centering
		\begin{tabular}{|p{0.1\textwidth}|r}
			\only<2->{\alertred{C}} & \emph{Gerede} \\
			\hline
			\only<3->{\alertred{E}} & \emph{Milchgesicht}\\
			\hline
			\only<4->{\alertred{B}} & \emph{Lauf} \\
			\hline
			\only<5->{\alertred{A}} & \emph{Kettenraucher}  \\
			\hline
			\only<6->{\alertred{D}} & \emph{Klausurbesprechung}  \\
		\end{tabular}
	\end{minipage}
\end{table}


\item[5.] Geben Sie für die folgende Wortform die Flexionskategorien an, nach denen sie flektiert ist. %\\
%\hfill(3 Punkte)\\
\end{enumerate}

\begin{columns}
\column[t]{.28\textwidth}
	\begin{exe}
		\exr{ex:05cHA5} bestehe
	\end{exe}

\column[t]{.6\textwidth}
\visible<7->{%
\textcolor{red}{%
1. / Sg. / Präsens / Indikativ / Aktiv\\
1. / Sg. / Präsens / Konjunktiv I / Aktiv\\
3. / Sg. / Präsens / Konjunktiv I / Aktiv\\
2. / Sg. / Präsens / Imperativ / Aktiv
}
}
\end{columns}



\end{frame}


%%%%%%%%%%%%%%%%%%%%%%%%%%%%%%%

\begin{frame}{Hausaufgabe -- Lösung}

\begin{enumerate}
	\item[6.] Warum sind die Wörter unter (\ref{ex:05cHA6a}) grammatisch und die unter (\ref{ex:05cHA6b}) ungrammatisch? %(4 Punkte)
	
	\begin{exe}
		\exr{ex:05cHA6}
		\begin{xlist}
			\ex kaufbar, trinkbar
			\ex *fensterbar, *helfbar, *schönbar
		\end{xlist}
	\end{exe}
	
\pause

\alertred{Das Suffix \emph{-bar} hat die folgenden Beschränkungen bzgl. der Basis X, mit der es sich verbindet:}
		\begin{itemize}
			\item[\alertred{--}] \alertred{X muss ein Verb sein (nicht Nomen oder Adjektiv)}
			\item[\alertred{--}] \alertred{X muss transitiv sein (nicht wie \emph{helfen})}
		\end{itemize}

\end{enumerate}
\end{frame}


%%%%%%%%%%%%%%%%%%%%%%%%%%%%%%%

\begin{frame}{Hausaufgabe -- Lösung}

\begin{enumerate}
\item[7.] Sind die folgenden Verben Präfixverben oder Partikelverben? Begründen Sie Ihre Entscheidungen. %\hfill(3 Punkte)\\

\begin{exe}
	\exr{ex:05cHA7}
	\begin{xlist}
		\ex auskennen
		\ex erkennen
		\ex aberkennen
	\end{xlist}
\end{exe}

\pause

\alertred{Partikelverben sind:}
\begin{itemize}
	\item[\alertred{--}] \alertred{morphologisch trennbar (\emph{aus-ge-kannt}, \emph{ab-zu-erkennen}). }
	\item[\alertred{--}] \alertred{syntaktisch trennbar (\emph{Peter kennt sich aus.}, \emph{Die Frau erkennt die Urkunde ab.}). }
	\item[\alertred{--}] \alertred{betont (\emph{\textprimstress auskennen} und \emph{\textprimstress aberkennen}).}
\end{itemize}
		
\pause
\medskip
		
\alertred{Präfixverben sind:}
\begin{itemize}
	\item[\alertred{--}] \alertred{weder morphologisch noch syntaktisch trennbar (*\emph{ergekannt}, *\emph{Sie kannte ihn er.}). }
	\item[\alertred{--}] \alertred{nicht betont (\emph{er}\textprimstress \emph{kennen}). }
\end{itemize}

\pause

\alertred{\emph{aberkennen} ist ein Partikelverb, welches aus einem Präfixverb und einer Partikel besteht (ab$+$erkennen).}

\end{enumerate}
\end{frame}


%%%%%%%%%%%%%%%%%%%%%%%%%%%%%%%%%

\begin{frame}{Hausaufgabe -- Lösung}

\begin{enumerate}
\item[8.] Geben Sie für das folgende Wort eine morphologische Konstituentenstruktur (inklusive Konstituentenkategorien (N, N\textsuperscript{af}, V, V\textsuperscript{af}, \dots)) an, und bestimmen Sie für jeden Knoten den Wortbildungstyp. %\hfill(6,5 Punkte)\\

\begin{exe}
	\exr{ex:05cHA8} Wahlkampfberaterinnen
\end{exe}

\end{enumerate}

\vspace{-.25cm}

\begin{figure}
\centering

\scalebox{.6}{

\textcolor{red}{
\begin{forest} MyP edges,
	[N, name=N1
	[N, name=N2
	[N, name=N3
	[N, name=N4
	[N, name=N6 [V[wahl/wähl]]]
	[N, name=N7 [V[kampf/kämpf]]]]
	[N, name=N5[V, name=V1	[V\textsubscript{af}[be-]]
	[V[rat]]]
	[N\textsuperscript{af}[-er]]]]
	[N\textsuperscript{af}[-in]]]
	[Fl[-nen]]]	
{
	\draw[<-, red] (N1.west)--++(-12em,0pt)
	node[anchor=east,align=center]{Flexion (KEIN Wortbildungsporzess)};
	\draw[<-, red] (N2.west)--++(-14em,0pt)
	node[anchor=east,align=center]{Derivation (Movierung)};
	\draw[<-, red] (N3.west)--++(-9.5em,0pt)
	node[anchor=east,align=center]{Determinativkompositum};
	\draw[<-, red] (N4.west)--++(-4em,0pt)
	node[anchor=east,align=center]{Determinativkompositum};
	\draw[<-, red] (N5.west)--++(-3em,0pt)
	node[anchor=east,align=center]{Derivation};
	\draw[<-, red] (N6.west)--++(-3em,0pt)
	node[anchor=east,align=center]{Implizite Derivation};
	\draw[<-, red] (N7.east)--++(2.5em,0pt)--++(0em,-18ex)%--++(2em,0pt)
	node[anchor=north,align=center]{Implizite Derivation};
	\draw[<-, red] (V1.east)--++(1.5em,0pt)--++(0em,-14ex)--++(2em,0pt)
	node[anchor=west,align=center]{Derivation};
}
\end{forest}	
%\begin{itemize}
%	\item[]N1: Flexion (KEIN Wortbildungsporzess)
%	\item[]N2: Derivation (Movierung)
%	\item[]N3: Determinativkompositum
%	\item[]N4: Determinativkompositum
%	\item[]N5: Derivation
%	\item[]N6: Implizite Derivation
%	\item[]N7: Implizite Derivation
%	\item[]V1: Derivation
%\end{itemize}
}
}

\end{figure}
\end{frame}


%%%%%%%%%%%%%%%%%%%%%%%%%%%%%%%%%%%

\begin{frame}{Hausaufgabe -- Lösung}

\begin{enumerate}
\item[9.] Paraphrasieren Sie das folgende komplexe Wort so, dass es der angegebenen Struktur entspricht (auch wenn Sie selbst eine andere Struktur plausibler finden sollten). %\\
%\hfill(2 Punkte)\\


\begin{forest}sn edges,
	[N
	[N[N[Reserve]]
	[N[V[lehr]][N\textsuperscript{af}[-er]]]]
	[N[zimmer]]
	]
\end{forest}

\pause

\textcolor{red}{Ein Zimmer für Reservelehrer}

\end{enumerate}
\end{frame}



%% -*- coding:utf-8 -*-

%%%%%%%%%%%%%%%%%%%%%%%%%%%%%%%%%%%%%%%%%%%%%%%%%%%%%%%%%


\def\insertsectionhead{\refname}
\def\insertsubsectionhead{}

\huberlinjustbarfootline


\ifpdf
\else
\ifxetex
\else
\let\url=\burl
\fi
\fi
\begin{multicols}{2}
{\tiny
%\beamertemplatearticlebibitems

\bibliography{gkbib,bib-abbr,biblio}
\bibliographystyle{unified}
}
\end{multicols}





%% \section{Literatur}
%% \begin{frame}[allowframebreaks]
%% \frametitle{Literatur}
%% 	\footnotesize

%% \bibliographystyle{unified}

%% 	%German
%% %	\bibliographystyle{deChicagoMyP}

%% %	%English
%% %	\bibliographystyle{chicago} 

%% 	\bibliography{gkbib,bib-abbr,biblio}
	
%% \end{frame}



\end{document}
