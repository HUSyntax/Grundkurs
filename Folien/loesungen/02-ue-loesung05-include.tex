%%%%%%%%%%%%%%%%%%%%%%%%%%%%%%%%%%
%% UE 5 - 02 Phonetik
%%%%%%%%%%%%%%%%%%%%%%%%%%%%%%%%%%

\begin{frame}
\frametitle{Lösung: Transkription}

\begin{itemize}

\item Transkribieren Sie die folgenden Wörter nach einer standarddeutschen Aussprache:

%\begin{columns}
%	\column{.40\textwidth}
%	\begin{enumerate}
\ea 
\settowidth\jamwidth{XXXXXXXXXXXXXXXXXXXXXXXXXX}

		\ea Bergsteiger
		\loesung{1}{\textipa{[bE͡5k.St\t{aI}.g5]}}
		
		\ex Quotennote
		\loesung{2}{\textipa{[kvo:.t@n.no:.t@]}}
		
		\ex vielfaches
		\loesung{3}{\textipa{[fi:l.fa\.x@s]}}
		
		\ex Päckchenannahme
		\loesung{4}{\textipa{[pEk.\c{c}@n.Pan.na:.m@]}}
		
		\ex beenden
		\loesung{5}{\textipa{[b@.PEn.d@n]}}
		
		\ex verreisen
		\loesung{6}{\textipa{[fE͡5.\textscr \t{aI}.z@n]}}
		
		\ex vereisen
		\loesung{7}{\textipa{[fE͡5.P\t{aI}.z@n]}}
		
		\ex Einzahlung
		\loesung{8}{\textipa{[P\t{aI}n.\t{ts}a:.lUN]}}
		
		\ex gehen
		\loesung{9}{\textipa{[ge:.@n]}}
		
		\ex Gästebad
		\loesung{10}{\textipa{[gEs.t@.ba:t]}}
		
		\z 
\z		

\end{itemize}

\end{frame}

