%%%%%%%%%%%%%%%%%%%%%%%%%%%%%%%%%%
%% UE 2 - 09 Übungen
%%%%%%%%%%%%%%%%%%%%%%%%%%%%%%%%%%

\begin{frame}
\frametitle{Übung: Graphematik -- Lösung}

\begin{itemize}
	\item[6.] Geben Sie Beispiele für die Anwendung der folgenden graphematischen Prinzipien an:
	
	\begin{exe}
		\exr{ex:06}
		
		\begin{xlist}
			\ex \only<1->{Prinzip der Morphemkonstanz} \\
			\alertgreen{\only<2->{\item[-] Silbengelenke, wegen zugehöriger Pluralformen: \zB \textit{Ba\underline{ll}},}}
			\alertgreen{\only<2->{\item[-] Dehnungs-h, wegen zugehöriger Flexionsformen: \zB \textit{de\underline{h}nen, weil du dehnst}}}
	
			\ex \only<1->{Homonymieprinzip} \\
			\alertgreen{\only<3->{\item[-] Differenzierung homophoner Formen: \zB \textit{L\underline{ee}re vs. L\underline{eh}re}}}
	
			\ex \only<1->{Silbisches Prinzip} \\
			\alertgreen{\only<4->{\item[-] Silbengelenk: \zB \textit{Wa\underline{ss}er},}}
			\alertgreen{\only<4->{\item[-] Silbentrennendes h: \zB S\textit{chu\underline{h}e},}}
			\alertgreen{\only<4->{\item[-] Dehnungs-h: \zB \textit{Sa\underline{h}ne},}}
			\alertgreen{\only<4->{\item[-] Gespanntheit: \zB \textit{M\underline{oo}s},}}
		\end{xlist}
	
	\end{exe}
		
\end{itemize}

\end{frame}

%%%%%%%%%%%%%%%%%%%%%%%%%%%%%%%%%%

\begin{frame}
	
\begin{itemize}
	\item[7.] Geben Sie die rein phonographische Schreibung der folgenden Wörter an:	
	
	\begin{exe}
		\exr{ex:07}
	
		\begin{xlist}		
			\ex sprachbegabt \loesung{2}{\ab{schprachbegabt}}
	
			\ex Sträuchersee  \loesung{3}{\ab{schtreucherse}}
		\end{xlist}
	
	\end{exe}
\end{itemize}

\end{frame}
%%%%%%%%%%%%%%%%%%%%%%%%%%%%%%%%%%