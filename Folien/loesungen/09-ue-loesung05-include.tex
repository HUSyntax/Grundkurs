%%%%%%%%%%%%%%%%%%%%%%%%%%%%%%%%%%
%% UE 5 - 09 Übungen
%%%%%%%%%%%%%%%%%%%%%%%%%%%%%%%%%%

\begin{frame}
\frametitle{Übung: Semantik/Pragmatik -- Lösung}

\begin{itemize}
	\item[15.] Geben Sie die Bedeutungsrelationen (so genau wie möglich) zwischen den folgenden Wörtern an.
	
	\settowidth \jamwidth{\alertgreen{kontradiktorische Antonymie}}
	
	\eal
	\ex satt -- hungrig \only<2->{\jambox{\alertgreen{konträre Antonymie}}}
	\ex erwerben -- kaufen \only<3->{\jambox{\alertgreen{(partielle) Synonymie}}}
	\ex Haare -- Kopf \only<4->{\jambox{\alertgreen{Meronymie}}}
	\ex schuldig -- nicht schuldig \only<5->{\jambox{\alertgreen{kontradiktorische Antonymie}}}
	\ex heute -- Häute \only<6->{\jambox{\alertgreen{Homophonie}}}
	\ex Tiger -- Katze \only<7->{\jambox{\alertgreen{Hyponymie}}}
	\ex fruchtbar -- unfruchtbar \only<8->{\jambox{\alertgreen{kontradiktorische Antonymie}}}
	\zl
	
\end{itemize}

\end{frame}

%%%%%%%%%%%%%%%%%%%%%%%%%%%%%%%%%%

\begin{frame}
	
\begin{itemize}
	\item[16.] Illustrieren Sie die Begriffe Satzbedeutung, Äußerungsbedeutung und Sprecherbedeutung mithilfe des folgenden Satzes.
	
	\eal Ich glaube, du gehst jetzt! \\
	\ras Peter zu Klaus am 25. August 2020 um 20:30 Uhr.
	\zl
	
	\item Satzbedeutung: \\ \only<2->{\alertgreen{Der Sprecher des Satzes glaubt (zum Zeitpunkt der Äußerung), dass der Adressat der Äußerung geht.}}
	\item Äußerungsbedeutung: \\ \only<3->{\alertgreen{Klaus glaubt, dass Peter am 25. August 2020 um 20:30 Uhr geht.}}
	\item Sprecherbedeutung: \\ \only<4->{\alertgreen{Peter fordert Klaus bestimmt auf (\zB nach einer Auseinandersetzung) sofort zu gehen.}}
	
\end{itemize}

\end{frame}

%%%%%%%%%%%%%%%%%%%%%%%%%%%%%%%%%%

\begin{frame}

\begin{itemize}
	\item[17.] Geben Sie die Bedeutungsrelationen zwischen den folgenden Sätzen an.
		
		\settowidth \jamwidth{\only<2->{\alertgreen{\ras a impliziert b}}}
		
		\eal 
		\ex Hinter dem Baum steht ein Bär. \jambox{\only<2->{\alertgreen{Implikation}}}
		\ex Hinter dem Baum steht ein Tier. \jambox{\only<2->{\alertgreen{\ras a impliziert b}}}
		\zl
		
		\eal 
		\ex Peter fängt an zu arbeiten. \jambox{\only<3->{\alertgreen{Paraphrase}}}
		\ex Peter nimmt die Arbeit auf.
		\zl
		
		\eal 
		\ex Sandra ist groß. \jambox{\only<4->{\alertgreen{Kontradiktion}}}
		\ex Sandra ist nicht-groß.
		\zl
		
		\eal 
		\ex Ich habe alle Studenten gesehen. \jambox{\only<5->{\alertgreen{Paraphrase}}}
		\ex Ich habe nicht einen Studenten nicht gesehen.
		\zl
		
		\eal 
		\ex Maria geht wandern. \jambox{\only<6->{\alertgreen{Inkompatibilität}}}
		\ex Maria macht eine Kreuzfahrt.
		\zl
		
		\eal 
		\ex Gert ist verletzt. \jambox{\only<7->{\alertgreen{Implikation}}}
		\ex Gert hat ein gebrochenes Bein. \jambox{\only<6->{\alertgreen{\ras b impliziert a}}}
		\zl
		
\end{itemize}
	
\end{frame}

%%%%%%%%%%%%%%%%%%%%%%%%%%%%%%%%%%

\begin{frame}
	
\begin{itemize}
		
	\item[18.] Geben Sie eine Wahrheitswerttabelle für den folgenden aussagenlogischen Ausdruck an und bestimmen Sie, ob es sich dabei um eine tautologische, eine kontradiktorische oder eine kontingente Aussage handelt.
		\ea $((p \rightarrow q) \lor q)$
		\z
		
		\begin{table}
			\centering
			\scalebox{.95}{
				\only<2->{\alertgreen{
					\begin{tabular}{c|c|c|c}
					p & q & $(p \rightarrow q)$ & $((p \rightarrow q) \lor q)$ \\
					\hline
					1 & 1 & 1 & 1 \\
					\hline
					1 & 0 & 0 & 0 \\
					\hline
					0 & 1 & 1 & 1 \\
					\hline
					0 & 0 & 1 & 1 \\
					\end{tabular}
					}}}
		\end{table}

\medskip
	
	\only<2->{\item \alertgreen{Bei dem vorangehenden aussagenlogischen Ausdruck handelt es sich um eine kontingente Aussage.}}
				
\end{itemize}
	
\end{frame}

%%%%%%%%%%%%%%%%%%%%%%%%%%%%%%%%%%

\begin{frame}

\begin{itemize}
		\item[19.] Markieren Sie alle deiktischen und anaphorischen Elemente in den folgenden Sätzen und spezifizieren Sie diese.
		
	\eal
	\ex Sie haben diese Tür nicht geschlossen.
	\ex Gestern war mir das Wetter echt zu kalt!
	\ex Peter wusste, dass er es sich dort gemütlich machen würde.
	\zl

\end{itemize}

\end{frame}

%%%%%%%%%%%%%%%%%%%%%%%%%%%%%%%%%%

\begin{frame}

\begin{itemize}
		\item[19.] Markieren Sie alle deiktischen und anaphorischen Elemente in den folgenden Sätzen und spezifizieren Sie diese.
	
	\eal
	\ex \alertgreen{Sie} haben \alertgreen{diese} Tür nicht geschlossen.
	\ex \alertgreen{Gestern} war \alertgreen{mir} das Wetter echt zu kalt!
	\ex Peter wusste, dass \alertblue{er} es \alertblue{sich} \alertgreen{dort} gemütlich machen würde. 
	%	\ex \gqq{Ich bin sehr glücklich, mich wieder für das WTA-Finale qualifiziert zu haben. Ich freue mich darauf, dort anzutreten und gegen die Besten der Welt zu spielen}, sagte die 27-Jährige, die im vergangenen Jahr nur als Ersatzspielerin mitfahren durfte.
	\zl

\end{itemize}	
	
	\begin{minipage}[t]{0.40\textwidth}

	\begin{itemize}
		\item []
		\only<2->{\alertgreen{\item Deiktische Elemente:}}
		
			\begin{itemize}
				\only<3->{\alertgreen{
				 \item Sie: Sozialdeixis
				 \item diese: Objektdeixis
				 \item gestern: Temporaldeixis
				 \item mir: Personaldeixis
				 \item dort: Lokaldeixis
			}}
			\end{itemize}		

	\end{itemize}
	
	\end{minipage}
	\begin{minipage}[t]{0.50\textwidth}
	
	\begin{itemize}
		\item [] 
		\only<2->{\alertblue{\item Anaphorische Elemente:}}
		
			\begin{itemize}
				\only<3->{\alertblue{
				 \item er: anaphorischer Ausdruck zu \textit{Peter}
				 \item sich: anaphorischer Ausdruck zu \textit{er}
			}}
			\end{itemize}
		
	\end{itemize}

	\end{minipage}


\end{frame}

%%%%%%%%%%%%%%%%%%%%%%%%%%%%%%%%%%

\begin{frame}
	
\begin{itemize}
	\item[20.] Bestimmen Sie die Art von Folgerung (Implikation, Präsupposition, Implikatur), die zwischen dem ersten und den folgenden Sätzen besteht:
		
		\settowidth \jamwidth{\only<2->{\alertgreen{konversationalle Implikatur}}}
		
		\ea Sogar Peter hat zwei Kinder.
		\ea Peter hat nicht mehr als zwei Kinder. \jambox{\only<2->{\alertgreen{konversationalle Implikatur}}}
		\ex Es gibt ein Individuum namens Peter. \jambox{\only<3->{\alertgreen{Präsupposition}}}
		\ex Peter ist Vater. \jambox{\only<4->{\alertgreen{semantische Implikation}}}
		\ex Peter hat vier Kinder. \jambox{\only<5->{\alertgreen{keine Folgerung}}}
		\ex Überraschenderweise hat Peter Kinder. \jambox{\only<6->{\alertgreen{konventionelle Implikatur}}}
		\z
		\z 
		
\end{itemize}
	
\end{frame}

%%%%%%%%%%%%%%%%%%%%%%%%%%%%%%%%%%

\begin{frame}

\begin{itemize}
	\item[21.] Bestimmen Sie jeweils eine semantische Implikation aus den folgenden Sätzen:
	
	\eal 
	\ex \label{ex:Implikation1}{In einem Schuhkarton gibt es Platz für zwei Schuhe.}
	\ex \label{ex:Implikation2}{Saskia hat eine Schwedin geheiratet.}
	\zl
	
\end{itemize}
	
	\only<2->{\alertgreen{(\ref{ex:Implikation1}): $\vDash$ In einem Schuhkarton gibt es Platz für einen Schuh.}} ~\\
\medskip
	\only<3->{\alertgreen{(\ref{ex:Implikation2}): $\vDash$ Saskia hat eine Nordeuropäerin geheiratet.}}
	

\end{frame}

%%%%%%%%%%%%%%%%%%%%%%%%%%%%%%%%%%

\begin{frame}
	
\begin{itemize}
		
	\item[22.] Bestimmen Sie jeweils eine Präsupposition aus den folgenden Sätzen:
	
		\eal
		\ex \label{ex:Präsupposition1}{Ich freue mich darüber, dass wir die Klausur bestanden haben.}
		\ex \label{ex:Präsupposition2}{Auch Maria ist schwanger.}
		\ex \label{ex:Präsupposition3}{Alle Geiseln wurden gerettet.}
		\ex \label{ex:Präsupposition4}{Sie mögen immer noch Syntax.}
		\zl
		
\end{itemize}

	\only<2->{\alertgreen{(\ref{ex:Präsupposition1}): \prspp Wir haben die Klausur bestanden.}} ~\\
\medskip
	\only<3->{\alertgreen{(\ref{ex:Präsupposition2}): \prspp Mindestens eine weitere Entität neben Maria ist schwanger.}} ~\\
\medskip
	\only<4->{\alertgreen{(\ref{ex:Präsupposition3}): \prspp Die Geiseln waren in Gefahr.}} ~\\
\medskip
	\only<5->{\alertgreen{(\ref{ex:Präsupposition4}): \prspp Sie mochten bisher Syntax.}}
	
\end{frame}

%%%%%%%%%%%%%%%%%%%%%%%%%%%%%%%%%%

\begin{frame}
	
\begin{itemize}
	\item[23.] Geben Sie an, ob eine Maxime (scheinbar) verletzt oder befolgt wurde und um welche es sich handelt, um die angegebene Implikatur zu erhalten.
	
	\ea \label{ex:Maxime1}Wir haben einige Personen entlassen.\\
	$+>$ Es wurden nicht alle entlassen.
	\z
	
	\ea \label{ex:Maxime2}A: Wie war das Bewerbungsgespräch?\\
	B: Das Wetter ist ja super heute!\\
	$+>$ Es war furchtbar!
	\z
	
\end{itemize}

	\only<2->{\alertgreen{(\ref{ex:Maxime1}): Befolgung der Quantitätsmaxime}} ~\\
\medskip
	\only<3->{\alertgreen{(\ref{ex:Maxime2}): (scheinbare) Verletzung der Relevanzmaxime}}

\end{frame}

%%%%%%%%%%%%%%%%%%%%%%%%%%%%%%%%%%