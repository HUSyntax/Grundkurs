%% -*- coding:utf-8 -*-
\subsection{Bemerkungen zum Spracherwerb}



\frame{
\frametitle{Theorievarianten}

\begin{itemize}[<+->]
\item Die Strukturen, die Sie bisher gesehen haben, werden in vielen verschiedenen Theorien für das
  Deutsche angenommen.
\item Manche Wissenschaftler nehmen noch komplexere Strukturen an.

\item Um die Motivation verstehen zu können, müssen wir über Spracherwerb reden.

\end{itemize}



}


\frame{
\frametitle{Spracherwerb}


Man kann verschiedene Phasen im Spracherwerb beobachten:
\begin{itemize}
\item Neugeborene sind bereits in der Lage, menschliche Sprache von anderen Lauten zu unterschieden.
\pause
\item Lallphase: Ausprobieren diverser produzierbarer Laute,\\
      Vergessen solcher Laute, die in der Umgebungssprache nicht vorkommen\\
 (\zb Schnalzlaute im Deutschen).
\pause
\item Fähigkeit Wörter zu verstehen.
\pause
\item Produktion erster Wörter.
\pause
\item Wortschatzspurt (\emph{Vocabulary Spurt}):\\
      Rapider Anstieg der verstandenen und produzierten Wörter
\pause
\item Zweiwortphase, Templates
\pause
\item Mehrwortphase, Generalisierung, Abstraktion, Regeln
\end{itemize}


}


\frame{
\frametitle{Konkurrierende Ansätze}



\begin{itemize}[<+->]
\item Nativismus
\begin{itemize}
\item Sprachliches Wissen ist angeboren, da es ansonsten nicht erwerbbar wäre.
\end{itemize}
\item Kognitive Ansätze
\begin{itemize}
\item Sprachliches Wissen kann erworben werden, wenn entsprechende Fähigkeiten in anderen kognitiven
  Bereichen erworben worden sind.
\end{itemize}
\item Interaktionistische Ansätze
\begin{itemize}
\item Sprache wird in Interaktion mit anderen Individuen erworben.\\
      Der Erwerb ist sehr stark vom Input abhängig.
\end{itemize}
\end{itemize}

}

\subsubsection{Nativismus}

\frame[shrink]{
\frametitle{Chomskys Nativismus}

\begin{itemize}
\item Chomsky geht/ging davon aus, dass sprachliches Wissen angeboren ist.
\pause
\item Behauptung:\\
      Erwerb eines solch komplexen Systems ist anders nicht möglich.
\pause
\item Da Kinder mit englischen Eltern auch Chinesisch lernen können,\\
      wenn sie in einer chinesischen Umgebung aufwachsen,\\
      muss das, was angeboren ist, für alle Sprachen gleich sein,\\
      also eine angeborene Universalgrammatik.

\pause
\item Spracherwerb: Prinzipien und Parameter\\
Menschen verfügen über ein vorangelegten Satz grammatischer Kategorien und syntaktischer Strukturen.

\pause
\item In Abhängigkeit vom sprachlichen Input, den Kinder bekommen, setzen sie bestimmte Parameter
  und je nach Art der Parametersetzung ergibt sich dann die Grammatik des Deutschen, Englischen oder
  Japanischen.


\end{itemize}

}

\frame{
\frametitle{Beispiel: Englisch vs.\ Japanisch}


\begin{itemize}
\item Kopfposition ist initial $\to$ Englisch oder final $\to$ Japanisch.
\eal
\ex 
\gll be showing pictures of himself\\
     ist zeigend Bilder von sich\\
\ex
\gll zibun -no syasin-o mise-te iru\\
     sich  von Bild     zeigend sein\\
\zl

\end{itemize}

}


\frame{
\frametitlefit{Argumente für die Existenz angeborenen sprachlichen Wissens}


%Weitere Fakten wurden zur Stützung der Nativismus"=Hypothese herangezogen:
\begin{itemize}
\item Syntaktische Universalien 
\pause
\item die Tatsache, dass es eine "`kritische"' Periode für den Spracherwerb gibt
\pause
\item Fast alle Kinder lernen Sprache, aber nichtmenschliche Primaten nicht.
\pause
\item Kinder regularisieren spontan Pidgin-Sprachen.
\pause
\item Lokalisierung in speziellen Gehirnbereichen\nocite{MMGRRBW2003a}
\pause
\item Angebliche Verschiedenheit von Sprache und allgemeiner Kognition:
\begin{itemize}
\item Williams-Syndrom 
\pause
\item KE-Familie mit FoxP2-Mutation\nocite{LFHVM2001a}\nocite{Dabrowska2004a}
\end{itemize}
\pause
\item Poverty of the Stimulus
\end{itemize}

Siehe \zb \citew{Pinker94a} und Kritik von \citew{Tomasello95a,Dabrowska2004a}.

%Zusammengefasst in \citew{MuellerGTBuch2} bzw.\ \citew{MuellerGT-Eng1}.

\nocite{Goldberg2003b,Goldberg2006a}
}


\frame{
\frametitle{Angeborenes sprachspezifisches Wissen?}


\begin{itemize}
\item Die Existenz angeborenen sprachspezifischen Wissens ist heiß umkämpft.

\item Viele Argumente sind nicht stichhaltig und inzwischen weiß man auch mehr darüber, was und wie
  etwas aus Daten gelernt werden kann.

\item Biologen sagen uns, dass es unrealistisch ist, anzunehmen,\\
      dass reichhaltiges linguistisches Wissen in unserem Genom kodiert ist.

\item Die Diskussion können Sie in \citew{MuellerGTBuch2,MuellerGT-Eng1} nachlesen.

\item Wir kommen noch darauf zurück.
\end{itemize}


}

\subsubsection{Spracherwerb und \xbart}

\frame{
\frametitle{Spracherwerb und \xbart}


\begin{itemize}[<+->]
\item Annahme: Es gibt ein Problem bei der Erklärung der Spracherwerbs.
\item Spracherwerb wäre einfacher, wenn Lerner (Kinder) schon wüssten,\\
      dass alle sprachlichen Ausdrücke eine bestimmte Struktur haben.
\item Annahme: Alle sprachlichen Ausdrücke (die zum Kernbereich der Sprache gehören) gehorchen den Gesetzmäßigkeiten der \xbart.

\item Wichtiger Grundsatz: Wenn wir die UG erforschen und aus einer Sprache Aufschlüsse über die UG
  gewinnen, dann muss das natürlich auch für alle anderen Sprachen gelten.

(Achtung: Schlussfolgerung gilt nur, wenn Annahme richtig ist.)

\end{itemize}

}


\subsubsection{\xbart für Englisch}

\frame{
\frametitle{Die Struktur englischer Sätze}


\begin{itemize}
\item In früheren Arbeiten zum Englischen gab es für Sätze Regeln wie: 
\eal
\ex S $\to$ NP VP
\ex S $\to$ NP Infl VP
\zl

\centerline{%
\begin{forest}
sm edges
[S
  [NP   [Ann,roof]]
  [INFL [will]]
  [VP
    [V$'$
      [\vnull [read]]
      [NP [the newspaper,roof]]]]]
\end{forest}
}

\pause
\item Diese Regeln entsprechen nicht dem \xbar-Schema.


\end{itemize}

}


\subsubsubsection{TP und VP im Englischen}

\frame{
\frametitle{Exkurs: TP und VP im Englischen: Hilfsverben}

\centerline{%
\begin{forest}
sm edges
[TP
  [NP   [Ann,roof]]
  [T$'$
    [\tnull [will]]
    [VP
      [V$'$
        [\vnull [read]]
        [NP [the newspaper,roof]]]]]]
\end{forest}
}


\begin{itemize}
\item Statt dessen T (Tense) als Kopf, der eine VP als Komplement nimmt.
\item Hilfsverben stehen in \tnull (=~Aux, =~Infl).
\item Satzadverbien können zwischen Hilfsverb und Vollverb stehen.
\end{itemize}

}






\frame{
\frametitle{IP und VP im Englischen: Sätze ohne Hilfsverb}

\centerline{%
\begin{forest}
sm edges
[TP
  [NP   [Ann,roof]]
  [T$'$
    [\tnull [read$_k$ s]]
    [VP
      [V$'$
        [\vnull [\trace$_k$]]
        [NP [the newspaper,roof]]]]]]
\end{forest}
}


\begin{itemize}
\item Hilfsverben stehen in \tnull (=~Aux, =~Infl).
\item Position kann leer bleiben.\\
Wird dann mit der flektierten Form des finiten Verbs verknüpft.

%Früher: In \tnull stand das Affix \suffix{s}, das Verb bewegt sich in die \inull-Position.
\end{itemize}

}




\frame{
\frametitle{Englische Sätze mit Komplementierer}


\hfill\scalebox{0.73}{%
\begin{forest}
sm edges
[CP
  [C$'$
    [C [that]]
    [TP
      [NP   [Ann,roof]]
      [T$'$
        [\tnull [read$_k$ s]]
        [VP
          [V$'$
            [\vnull [\trace$_k$]]
            [NP [the newspaper,roof]]]]]]]]
\end{forest}
}\hfill\hfill\mbox{}

\begin{itemize}
\item Der Komplementierer (\emph{that}, \emph{because}, \ldots) verlangt eine TP.
\end{itemize}

}

\subsubsubsection{CP, TP und VP im Englischen}


\frame{
\frametitle{CP, TP und VP im Englischen: Fragesätze}
\small

\vfill

\hfill\scalebox{0.7}{%
\begin{forest}
sm edges
[TP
  [NP   [Ann,roof]]
  [T$'$
    [\tnull [will]]
    [VP
      [V$'$
        [\vnull [read]]
        [NP [the newspaper,roof]]]]]]]]
\end{forest}
}\hfill\hfill\mbox{}


\begin{itemize}
\item Ja/nein-Fragen werden durch Voranstellung des Hilfsverbs gebildet:
\ea
Will Ann read the newspaper?
\z

\end{itemize}

\vfill

}

\frame{
\frametitle{CP, TP und VP im Englischen: Fragesätze}
\small

\vfill
\hfill\scalebox{0.7}{%
\begin{forest}
sm edges
[CP
  [XP [\trace]]
  [C$'$
    [C [will$_k$]]
    [TP
      [NP   [Ann,roof]]
      [T$'$
        [\tnull [\trace$_k$]]
        [VP
          [V$'$
            [\vnull [read]]
            [NP [the newspaper,roof]]]]]]]]
\end{forest}
}\hfill\hfill\mbox{}


\begin{itemize}
\item Ja/nein-Fragen werden durch Voranstellung des Hilfsverbs gebildet:
\ea
Will Ann read the newspaper?
\z
\item Umstellung des Hilfsverbs erfolgt in Position, die sonst der Komplementierer innehat.
\end{itemize}

\vfill

}


\frame{
\frametitle{CP, TP und VP im Englischen: Fragesätze}
\small

\vfill
\hfill\scalebox{0.7}{%
\begin{forest}
sm edges
[CP
  [NP$_j$ [what,roof]]
  [C$'$
    [C [will$_k$]]
    [TP
      [NP   [Ann,roof]]
      [T$'$
        [\tnull [\trace$_k$]]
        [VP
          [V$'$
            [\vnull [read]]
            [NP [\trace$_j$]]]]]]]]
\end{forest}
}\hfill\hfill\mbox{}


\begin{itemize}
\item \emph{wh}-Fragen werden durch zusätzliche Voranstellung vor das Hilfsverb gebildet:
\ea
What will Ann read?
\z
\end{itemize}

\vfill

}



\frame{
\frametitle{Keine Evidenz für TP im Deutschen}

\begin{itemize}
\item In diversen verschiedenen Theorien wird für das Deutsche keine TP angenommen, weil sich diese
  nicht nachweisen lässt.
\item Wenn man sie trotzdem annimmt,\\
      muss man das mit Universalgrammatik begründen.
\item Das ist jedoch fragwürdig. (später mehr)

\end{itemize}

}

\frame{
\frametitle{Grundstruktur für das Deutsche parallel zum Englischen}
  
\vfill
%% \hfill\scalebox{0.7}{%
%% \begin{forest}
%% sm edges
%% [CP
%%   [XP [what$_j$]]
%%   [C$'$
%%     [C [will$_k$]]
%%     [TP
%%       [NP   [Ann,roof]]
%%       [T$'$
%%         [\tnull [\trace$_k$]]
%%         [VP
%%           [V$'$
%%             [\vnull [read]]
%%             [NP [\trace$_j$]]]]]]]]
%% \end{forest}
%% }
\hfill
\scalebox{0.7}{%
\begin{forest}
sm edges
[CP
  [C$'$
    [C [dass]]
    [TP
      [NP   [Ann,roof]]
      [T$'$
        [VP
          [V$'$
            [NP [den Aufsatz,roof]]
            [\vnull [lies-]]]]
        [\tnull [-t]]]]]]
\end{forest}
}
\hfill\mbox{}

\begin{itemize}
\item Subjekt ist nicht mehr in der VP, sondern in der TP

\pause
\item Verbstamm steht in \vnull.
\pause
\item Verbstamm bewegt sich zu \tnull.


\end{itemize}

}


\frame{
\frametitle{Entscheidungsfragesätze im Deutschen (V1)}
  
\vfill
\hfill
\scalebox{0.7}{%
\begin{forest}
sm edges
[CP
  [C$'$
    [C [lies-$_j$ -t$_k$]]
    [TP
      [NP   [Ann,roof]]
      [T$'$
        [VP
          [V$'$
            [NP [den Aufsatz,roof]]
            [\vnull [\trace$_j$]]]]
        [\tnull [\trace$_k$]]]]]]
\end{forest}
}
\hfill\mbox{}

\begin{itemize}
\item Das finite Verb geht von V zu T und wird dann samt Suffix nach C vorangestellt.
\end{itemize}

}

\frame{
\frametitle{Aussagesätze im Deutschen (V2)}
  
\vfill
\hfill
\scalebox{0.7}{%
\begin{forest}
sm edges
[CP
  [NP [Den Aufsatz$_i$,roof]]
  [C$'$
    [C [lies-$_j$ -t$_k$]]
    [TP
      [NP   [Ann,roof]]
      [T$'$
        [VP
          [V$'$
            [NP [\trace$_i$]]
            [\vnull [\trace$_j$]]]]
        [\tnull [\trace$_k$]]]]]]
\end{forest}
}
\hfill\mbox{}

\begin{itemize}
\item Eine beliebige Phrase wird vor das finite Verb (nach SpecCP) gestellt.
\end{itemize}

}


\subsection{NP oder DP?}

\frame{
\frametitle{NP oder DP?}

\begin{itemize}
\item Was ist der Kopf in Nominalstrukturen?
\ea
der Mann
\z
\item Wir haben bisher gesagt: das Nomen.
\item Es könnte aber auch der Determinator sein.
\item Kriterien für Kopfstatus
\begin{itemize}
\item Kopf kann nicht weggelassen werden, Nicht-Köpfe schon

Im Deutschen kann man sowohl Determinator als auch Nomen weglassen.

Im Englischen kann des Nomen weniger gut weggelassen werden als im Deutschen, was für das Nomen als
Kopf spricht.

\item Kopf bestimmt die Form der anderen Elemente

In der Nominalgruppe liegt Kongruenz vor $\to$ Kriterium hilft nicht weiter.

Genus ist eine inhärente Eigenschaft des Nomens,\\
was eher für das Nomen als Kopf spricht.

\end{itemize}
\end{itemize}


}

\frame{
\frametitle{Was nu?}


\begin{itemize}
\item Es gibt unterschiedliche Auffassungen. 
\begin{itemize}
\item 0--1987: N ist der Kopf
\item 1977-heute: D ist der Kopf\\
      \citep{VH77a-u,Hellan86a,Abney87a,Netter94,Netter98a}
\item 2000-heute: vielleicht ist ja doch N der Kopf\\
      (\citealp{vanLangendonck94a}; \citealp[\page 49]{ps2}; \citealp{Demske2001a};
\citealp[Section~6.6.1]{MuellerLehrbuch1}; \citealp{Hudson2004a}; \citealp{Bruening2009a})

\end{itemize}
\end{itemize}

}


\frame{
\frametitle{Parallelität von DP und TP}

\vfill

\hfill
\begin{forest}
sm edges
[DP
  [D$'$
    [\dnull [the]]
    [NP
      [N$'$
        [\nnull [sister]]
        [PP [of my aunt,roof]]]]]]
\end{forest}
\hfill
\begin{forest}
sm edges
[TP
  [DP   [Ann,roof]]
  [T$'$
    [\tnull [will]]
    [VP
      [V$'$
        [\vnull [read]]
        [DP [the newspaper,roof]]]]]]
\end{forest}
\hfill\mbox{}

\begin{itemize}
\item Ein sogenanntes funktionales Element \dnull, \tnull verbindet sich\\
mit einer Projektion eines sogenannten lexikalischen Elements (N, V).

\end{itemize}

\vfill



}

\frame{
\frametitle{Pro und Contra}

\begin{itemize}
\item Pro:
\begin{itemize}
\item Vereinheitlichung der Strukturen
\end{itemize}

\item Contra:
\begin{itemize}
\item TP ist für das Deutsche nicht sinnvoll.
\item Wenn diese Einheitlichkeit irgendwie beim Spracherwerb helfen soll,\\
      müsste man annehmen,\\
      dass diese Strukturen genetisch vorgegeben sind.

Diese Annahme ist aber problematisch.
\end{itemize}
\end{itemize}


}


\subsubsection{The Poverty of the Stimulus}

\frame[shrink=20]{
\frametitle{The Poverty of the Stimulus}

\citet[S.\,29--33]{Chomsky71a-u}: Hilfsverbstellung im Englischen: Voranstellung vor das Subjekt:
\eal
\ex{
{}[The dog in the corner] \gruenbf{is} hungry.
}
\ex{
\gruenbf{Is} [the dog in the corner] hungry?
}
\zl
\pause

Mit den Daten sind die folgenden Hypothesen kompatibel:

\eal
\ex Stelle das erste Hilfsverb voran.
\ex Stelle das Hilfsverb vor das dazugehörige Subjekt.
\zl

\pause
Die erste Hypothese versagt jedoch bei (\mex{1}a). Sie würde (\mex{1}b) erzeugen:
\eal
\ex[]{
{}[The dog that \rotit{is} in the corner] \gruenbf{is} hungry.
}
\ex[*]{
\rotit{Is} [the dog that in the corner] \gruenbf{is} hungry?
}
\ex[]{
\gruenbf{Is} [the dog that is in the corner] hungry?
}
\zl

}


\frame{
\frametitle{Behauptungen und Schlüsse}


Behauptung:
Sprecher des Englischen hören in ihrem ganzes Leben nur sehr wenige
und eventuell sogar gar keine Belege wie (\mex{1}):\nocite{Piattelli-Palmarini80a-u}

\ea
\gruenbf{Is} [the dog that \rotit{is} in the corner] hungry?
\z

\pause
Schlußfolgerungen:
\begin{itemize}
\item Lerner haben keine Möglichkeit, die falsche Hypothese zu verwerfen.
\pause
\item Sie wissen aber,\\
      dass Umstellungen aus dem Relativsatz heraus ungrammatisch sind.
\pause
\item Es muss also angeborenes sprachspezifisches Wissen geben.
\end{itemize}


}

\frame{
\frametitle{Poverty of the Stimulus-Argumente}

\begin{itemize}[<+->]
\item Die Argumente wurden nie richtig geführt.\\
      Zu den Details siehe \citew{PS2002a,SP2002b}.
\item Für Chomskys Beispiel findet man in der Tat normale Beispiele mit Relativsätzen, die es
  angeblich nicht geben sollte.

\item Selbst wenn es die nicht gäbe, könnte man die Strukturen lernen,\\
      wie wir gleich sehen werden.

\end{itemize}

}


\frame[shrink=10]{
\frametitle{Poverty of the Stimulus und U"=DOP}

\begin{itemize}
\item U-DOP lernt aus Beispielen, die kein Hilfsverbumstellungen mit Relativsätzen enthalten \citep{Bod2009a}.

Sobald die korrekten Strukturen für (\mex{1}) erworben sind, können auch Sätzen mit
Hilfsverbumstellungen und Relativsätzen die korrekten Strukturen zugewiesen werden (S.\,778):
\eal
\ex \gruen{The man who is eating} is hungry.
\ex \blau{Is} the boy \blau{hungry}?
\zl
\ea
\blau{Is} \gruen{the man who is eating} \blau{hungry}?
\z

\pause

Um Strukturen für (\mex{-1}) zu lernen, reichen \zb die Sätze in (\mex{1}) aus:
\eal
\ex The man who is eating mumbled.
\ex The man is hungry.
\ex The man mumbled.
\ex The boy is eating.
\zl

\end{itemize}
\nocite{GG2010a-u}
}

\frame{
\frametitle{Poverty of the Stimulus und U"=DOP -- II}

\begin{itemize}
\item Algorithmus:

\begin{enumerate}
\item Berechne alle möglichen (binär verzweigenden) Bäume (ohne Kategoriesymbole)\\
      für eine Menge gegebener Sätze.
\pause
\item Berechne alle Teilbäume dieser Bäume.
\pause
\item Berechne den besten (wahrscheinlichsten) Baum für einen gegebenen Satz.
\end{enumerate}

\pause

\item Die erworbenen Grammatiken machen dieselben Fehler wie Kinder!

%% \pause
%% Das beeindruckt Linguisten sehr! (mich zumindest)

%% \pause
%% \item Was man für U-DOP braucht: Ist (binäres) Merge! 

%% Das ist wahrscheinlich Teil einer allgemeinen formalen Kompetenz\\
%% \citep{Fanselow92b,Chomsky2007a}.

\end{itemize}

}
\nocite{FPG2006a,FPAG2007a,FPG2009a}


\frame{
\frametitlefit{Mögliche binär verzweigende Bäume für \textit{Watch the dog} and \textit{The dog barks}}

~
\vfill


\hfill
\begin{forest}
sm edges
[X
	[X
		[watch]
		[the]]
	[dog]]
\end{forest}
\hfill
\begin{forest}
sm edges
[X
	[watch]
	[X
		[the]
		[dog]]]
\end{forest}
\hfill\mbox{}
\\[3ex]
\hfill\begin{forest}
sm edges
[X
	[X
		[the]
		[dog]]
	[barks]]
\end{forest}
\hfill
\begin{forest}
sm edges
[X
	[the]
	[X
		[dog]
		[barks]]]
\end{forest}
\hfill\mbox{}


\vfill

}




\frame{
\frametitle{Teilbäume}

\begin{columns}[T]
\begin{column}{50mm}
\scalebox{0.57}{%
\begin{tabular}{@{}ccc@{}}
~\\
\begin{forest}
sm edges
[X
	[X
		[watch]
		[the]]
	[dog]]
\end{forest}
&
\begin{forest}
sm edges
[X
	[X [,phantom ]]
	[dog]]
\end{forest}
&
\begin{forest}
[X
	[watch]
	[the]]
\end{forest}\\[3ex]
\hfill\begin{forest}
sm edges
[X
	[watch]
	[X
		[the]
		[dog]]]
\end{forest}
&
\begin{forest}
sm edges
[X
	[watch]
	[X [,phantom ]]]
\end{forest}
&
\gruen<2->{\begin{forest}
[X
	[the]
	[dog]]
\end{forest}}\hfill\mbox{}
\\[3ex]
\begin{forest}
sm edges
[X
	[X
		[the]
		[dog]]
	[barks]]
\end{forest}
&
\begin{forest}
sm edges
[X
	[X [,phantom ]]
	[barks]]
\end{forest}
&
\gruen<2->{\begin{forest}
[X
	[the]
	[dog]]
\end{forest}}\hfill\mbox{}
\\[3ex]
\hfill\begin{forest}
sm edges
[X
	[the]
	[X
		[dog]
		[barks]]]
\end{forest}
&
\begin{forest}
sm edges
[X
	[the]
	[X [,phantom ]]]
\end{forest}
&
\begin{forest}
[X
	[dog]
	[barks]]
\end{forest}
\end{tabular}
}
\end{column}
\begin{column}{65mm}
\begin{itemize}
\item Jeder Baum hat eine Wahrscheinlichkeit von 1/12.
\pause
\item \emph{the dog} kommt zweimal vor!\\
      Wahrscheinlichkeit = 2/12.
\end{itemize}
\end{column}
\end{columns}
}


\frame{
\frametitle{Analyse mit Teilbäumen und Wahrscheinlichkeiten}

\vfill


\hfill
\begin{tabular}{@{}ccccc@{}}
\rot<10>{%
\adjustbox{valign=c}{%
\begin{forest}
sm edges
[X
	[the]
	[X
		[dog]
		[barks]]]
\end{forest}}}
&
erzeugt aus
&
\adjustbox{valign=c}{%
\begin{forest}
sm edges
[X
	[the]
	[X
		[dog]
		[barks]]]
\end{forest}}
&
\visible<2->{und}
&
\visible<2->{%
\adjustbox{valign=c}{%
\begin{forest}
sm edges
[X
	[the]
	[X [,phantom ]]]
\end{forest}}
$\circ$
\adjustbox{valign=c}{%
\begin{forest}
[X
	[dog]
	[barks]]
\end{forest}}}
\\
\visible<5->{13/144} && \visible<3->{1/12 = 12/144}  & & \visible<4->{1/12 $\times$ 1/12 = 1/144}\\
\end{tabular}\hfill\mbox{}

\pause\pause\pause\pause\pause

\vfill

\hfill
\begin{tabular}{@{}ccccc@{}}
\gruen<10->{%
\adjustbox{valign=c}{%
\begin{forest}
sm edges
[X
	[X
		[the]
		[dog]]
	[barks]]
\end{forest}}}
&
erzeugt aus
&
\adjustbox{valign=c}{%
\begin{forest}
sm edges
[X
	[X
		[the]
		[dog]]
	[barks]]
\end{forest}}
&
\visible<7->{und}
&
\visible<7->{%
\adjustbox{valign=c}{%
\begin{forest}
sm edges
[X
	[X [,phantom ]]
	[barks]]
\end{forest}}
$\circ$
\adjustbox{valign=c}{%
\begin{forest}
[X
	[the]
	[dog]]
\end{forest}}}\\
\visible<10->{14/144} && \visible<8->{1/12 = 12/144}  & & \visible<9->{1/12 $\times$ \gruen<9>{2/12} = 2/144}\\
\end{tabular}

\pause\pause\pause\pause

\hfill\mbox{}

%% \oneline{
%% \begin{tabular}{@{}ccccc@{}}
%% \rot<10>{\begin{tabular}{@{}ccc@{}}
%% \multicolumn{3}{c}{\node{41}{X}}\\[2ex]
%% & \multicolumn{2}{c}{\node{42}{X}}\\[2ex]
%% \node{43}{the} & \node{44}{dog} & \node{45}{barks}\\
%% \end{tabular}
%% \nodeconnect{41}{43}\nodeconnect{41}{42}%
%% \nodeconnect{42}{44}\nodeconnect{42}{45}%
%% }
%% &
%% \begin{tabular}{l}
%% is\\
%% produced\\ 
%% by
%% \end{tabular}
%% &
%% \begin{tabular}{@{}ccc@{}}
%% \multicolumn{3}{c}{\node{241}{X}}\\[2ex]
%% & \multicolumn{2}{c}{\node{242}{X}}\\[2ex]
%% \node{243}{the} & \node{244}{dog} & \node{245}{barks}\\
%% \end{tabular}
%% \nodeconnect{241}{243}\nodeconnect{241}{242}%
%% \nodeconnect{242}{244}\nodeconnect{242}{245}%
%% &
%% \visible<2->{and}
%% &
%% \visible<2->{%
%% \begin{tabular}{@{}ccc@{}}
%% \multicolumn{3}{c}{\node{341}{X}}\\[2ex]
%% & \multicolumn{2}{c}{\node{342}{X}}\\[2ex]
%% \node{343}{the}\\
%% \end{tabular}
%% \nodeconnect{341}{343}\nodeconnect{341}{342}%
%% $\circ$
%% %%%%%%%%%%%%%%%%%%%%
%% \begin{tabular}{@{}ccc@{}}
%% & \multicolumn{2}{c}{\node{442}{X}}\\[2ex]
%%  & \node{444}{dog} & \node{445}{barks}\\
%% \end{tabular}
%% \nodeconnect{442}{444}\nodeconnect{442}{445}%
%% %
%% \\[3pt]
%% \visible<5->{13/144} && \visible<3->{1/12 = 12/144}  & & \visible<4->{1/12 $\times$ 1/12 = 1/144}\\
%% \end{tabular}
%% }
%% }
%% \vfill

%% \pause\pause\pause\pause\pause

%% \oneline{%
%% \begin{tabular}{@{}ccccc@{}}
%% \gruen<10->{\begin{tabular}{@{}ccc@{}}
%% \multicolumn{3}{c}{\node{31}{X}}\\[2ex]
%% \multicolumn{2}{c}{\node{32}{X}}\\[2ex]
%% \node{33}{the} & \node{34}{dog} & \node{35}{barks}\\
%% \end{tabular}
%% \nodeconnect{31}{32}\nodeconnect{31}{35}%
%% \nodeconnect{32}{33}\nodeconnect{32}{34}%
%% }
%% &
%% \begin{tabular}{l}
%% is\\
%% produced\\ 
%% by
%% \end{tabular}
%% &
%% \begin{tabular}{@{}ccc@{}}
%% \multicolumn{3}{c}{\node{231}{X}}\\[2ex]
%% \multicolumn{2}{c}{\node{232}{X}}\\[2ex]
%% \node{233}{the} & \node{234}{dog} & \node{235}{barks}\\
%% \end{tabular}
%% \nodeconnect{231}{232}\nodeconnect{231}{235}%
%% \nodeconnect{232}{233}\nodeconnect{232}{234}%
%% &
%% \visible<7->{and}
%% &
%% \visible<7->{\begin{tabular}{@{}ccc@{}}
%% \multicolumn{3}{c}{\node{331}{X}}\\[2ex]
%% \multicolumn{2}{c}{\node{332}{X}}\\[2ex]
%%      && \node{335}{barks}\\
%% \end{tabular}
%% \nodeconnect{331}{332}\nodeconnect{331}{335}%
%% $\circ$
%% \begin{tabular}{@{}ccc@{}}
%% \multicolumn{2}{c}{\node{432}{X}}\\[2ex]
%% \node{433}{the} & \node{434}{dog}\\
%% \end{tabular}
%% \nodeconnect{432}{433}\nodeconnect{432}{434}%
%% }
%% \\[3pt]
%% \visible<10->{14/144} && \visible<8->{1/12 = 12/144}  & & \visible<9->{1/12 $\times$ \gruen<9>{2/12} = 2/144}\\
%% \end{tabular}
%% }

\vfill

}



\frame{
\frametitle{Schlussfolgerungen}


\begin{itemize}[<+->]
\item Was für UDOP angeguckt wird, ist gemeinsames Vorkommen von Wörtern.
\item Wortarten lassen sich auch gut lernen und würden eine weiter Verbesserung bringen.

\bigskip

\item Kompatibel mit der revidierten Ansicht von Chomsky, dass nur sehr allgemeine Eigenschaften von
  Sprache Bestandteil unseres sprachspezifischen angeborenen Wissens sind \citep*{HCF2002a}.

\item Das hat Konsequenzen für unsere Theorien.

\item Wir können nicht mit einer TP im Englischen für eine TP im Deutschen argumentieren.

\item Wir können nicht mit Einheitlichkeit sprachübergreifend für eine DP argumentieren.

\bigskip

\item In diesem Kurs (und in der Klausur) sollen Sie aber dennoch von DPen ausgehen.

\item Wenn Sie sich jetzt ärgern, ist das gut.

\end{itemize}


}

\subsection{Beispiele mit DP und TP}

\frame{
\frametitle{Deutsch mit DP und TP}

\vfill
\hfill
\scalebox{0.6}{%
\begin{forest}
sm edges
[CP
  [DP$_i$ 
    [D$'$
      [\dnull [Den]]
        [NP
          [N$'$
            [\nnull [Aufsatz]]]]]]
  [C$'$
    [C [lies-$_j$ -t$_k$]]
    [TP
      [DP 
        [D$'$
          [\dnull [der]]
          [NP
            [N$'$
              [AP
                [A$'$
                  [\anull [kluge]]]]
              [N$'$
                [\nnull [Mann]]]]]]]
      [T$'$
        [VP
          [V$'$
            [DP [\trace$_i$]]
            [\vnull [\trace$_j$]]]]
        [\tnull [\trace$_k$]]]]]]
\end{forest}
}
\hfill\mbox{}


}


\frame{
\frametitle{Eingebettete Sätze (abgekürzte Struktur)}


\vfill
\hfill
\scalebox{0.4}{%
\begin{forest}
sm edges
[CP
  [DP$_i$ [Peter,roof]]
  [C$'$
    [C [glaub-$_j$ -t$_k$]]
    [TP
      [DP [\trace$_i$]]
      [T$'$
        [VP
          [V$'$
            [CP 
              [C$'$
                [\cnull [dass]]
                [TP
                  [DP [der Mann,roof]]
                  [T$'$
                    [VP
                      [V$'$
                        [DP [den Verbrecher,roof]]
                        [\vnull [verfolg-]]]]
                    [\tnull [-t]]]]]]
            [\vnull [\trace$_j$]]]]
        [\tnull [\trace$_k$]]]]]]
\end{forest}
}
\hfill\mbox{}

\begin{itemize}
\item Genau wie \emph{Peter kennt den Mann}, nur dass \emph{glauben} CP als Objekt nimmt und
  \emph{kennen} eine DP.
\end{itemize}


}


\frame{
\frametitle{Eingebettete Sätze (komplette Struktur)}


\vfill
\hfill
\scalebox{0.4}{%
\begin{forest}
sm edges
[CP
  [DP$_i$ 
    [D$'$
      [\dnull [\trace]]
        [NP
          [N$'$
            [\nnull [Peter]]]]]]
  [C$'$
    [C [glaub-$_j$ -t$_k$]]
    [TP
      [DP [\trace$_i$]]
      [T$'$
        [VP
          [V$'$
            [CP 
              [C$'$
                [\cnull [dass]]
                [TP
                  [DP
                    [D$'$
                      [\dnull [der]]
                      [NP
                        [N$'$
                          [\nnull [Mann]]]]]]
                  [T$'$
                    [VP
                      [V$'$
                        [DP
                          [D$'$
                            [\dnull [den]]
                            [NP
                              [N$'$
                                [\nnull [Verbrecher]]]]]]
                        [\vnull [verfolg-]]]]
                    [\tnull [-t]]]]]]
            [\vnull [\trace$_j$]]]]
        [\tnull [\trace$_k$]]]]]]
\end{forest}
}
\hfill\mbox{}





}


\frame{
\frametitle{Eingebettete Sätze (oder auch mal andersrum)}


\vfill
\hfill
\scalebox{0.55}{%
\begin{forest}
sm edges
[CP
  [CP$_i$ 
              [C$'$
                [\cnull [dass]]
                [TP
                  [DP
                    [D$'$
                      [\dnull [der]]
                      [NP
                        [N$'$
                          [\nnull [Mann]]]]]]
                  [T$'$
                    [VP
                      [V$'$
                        [DP
                          [D$'$
                            [\dnull [den]]
                            [NP
                              [N$'$
                                [\nnull [Verbrecher]]]]]]
                        [\vnull [\trace$_j$]]]]
                    [\tnull [verfolg-$_j$ -t]]]]]]
  [C$'$
    [C [glaub-$_k$ -t$_l$]]
    [TP
      [DP 
        [D$'$
          [\dnull [\trace]]
          [NP
            [N$'$
              [\nnull [Peter]]]]]]
      [T$'$
        [VP
          [V$'$
            [CP [\trace$_i$] ]
            [\vnull [\trace$_k$]]]]
        [\tnull [\trace$_l$]]]]]]
\end{forest}
}
\hfill\mbox{}





}


\frame{
\frametitle{Eingebettete Sätze (übersichtlicher mit abgekürzten DPen)}


\vfill
\hfill
\scalebox{0.7}{%
\begin{forest}
sm edges
[CP
  [CP$_i$ 
              [C$'$
                [\cnull [dass]]
                [TP
                  [DP [der Mann,roof]]
                  [T$'$
                    [VP
                      [V$'$
                        [DP [den Verbrecher,roof]]
                        [\vnull [verfolg-]]]]
                    [\tnull [-t]]]]]]
  [C$'$
    [C [glaub-$_j$ -t$_k$]]
    [TP
      [DP [Peter,roof]]
      [T$'$
        [VP
          [V$'$
            [CP [\trace$_i$] ]
            [\vnull [\trace$_j$]]]]
        [\tnull [\trace$_k$]]]]]]
\end{forest}
}
\hfill\mbox{}





}
