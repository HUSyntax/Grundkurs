\section{Morphologie}

\author{Stefan Müller (Anke Lüdeling)}

\frame{
\frametitle{Morphologie: Material}


\citew[Kapitel~7 und 8]{Luedeling2009a}, \citew{Haspelmath2002a}




}


\frame{
\frametitle{Morphologie}


\begin{itemize}
\item Die Morphologie beschäftigt sich mit dem Aufbau komplexer Wörter.
\ea
des Brunnenkressesüppchens \citep{Luedeling2009a}
\z

Das Wort in (\mex{0}) kann man wie folgt zerteilen (((Brunnen-kresse)-süpp)-chen)-s =

Genitivform (\suffix{s}) einer kleinen (\suffix{chen}) Suppe (\emph{süpp}) mit Brunnenkresse.
\pause

\item Es gibt morphologische Bestandteile, die frei (alleine) vorkommen können (\emph{Brunnen}, \emph{Kresse},
  \emph{Suppe})
\pause
\item Es gibt morphologische Bestandteile, die nicht frei vorkommen können (\suffix{chen},
  \suffix{s}).

\pause
\item Manche Bestandteile verändern in bestimmten Umgebungen ihre Form (\emph{Suppe} vor
  \suffix{chen} $\to$ \emph{süpp}).

\pause
\item Struktur spiegelt die Bedeutung eines komplexen Ausdrucks wider.
\end{itemize}



}

\frame{
\frametitle{Wortbildung und Flexion}


Teile des Wortes machen die Bedeutung aus und könnten einen Lexikoneintrag bilden:
\emph{Brunnenkressesüppchen}.

Diese Grundform oder auch Zitierform nennt man \blaubf{Lemma}.

Die anderen Teile bestimmen die grammatischen Eigenschaften:\\
\suffix{s} = Genitiv.

Der Teil der Morphologie, der sich mit der Bildung von Lemmata beschäftigt, heißt
\blaubf{Wortbildungslehre}.

Die grammatischen Formen werden in der \blaubf{Flexionsmorphologie} behandelt.


}


\subsection{Der Wortbegriff}



\frame{
\frametitle{Der Wortbegriff}

Obwohl Wörter eine zentrale Rolle in der Grammatikforschung spielen,\\
wird immer noch kontrovers diskutiert, was ein Wort ist.

Kriterien:
\begin{itemize}
\item orthographisch-graphemische
\item phonetisch-phonologische
\item morphologische
\item lexikalisch-semantische
\item syntaktische
\end{itemize}

Siehe \citew{Bussmann2002a}.

}

\subsubsection{Die orthographisch-graphemische Ebene}

\frame{
\frametitle{Die orthographisch-graphemische Ebene}

Wörter werden durch Leerzeichen voneinander getrennt.

\pause
Problem 1: Komposita im Englischen:
\eal
\ex summer school
\ex Sommerschule
\zl

\pause

Städtenamen im Deutschen:
\eal
\ex New York
\ex Berlin
\zl

}

%\begin{CJK*}{UTF8}{code2k}% Aus irgendwelchen Gründen zeigt gbsn die Punkte nicht an.


\frame{
\frametitle{Wörter sind durch Leerzeichen abgetrennt}

Problem 2: Chinesisch


%\begin{CJK*}{UTF8}{gbsn} % ist jetzt gefixt (SuSE 11.1)
近年来,``应用语言学''作为语言学的一个分支,在国内外都得到了较大的发展,但对于``什么是应用语言学'',
``应用语言学包括哪些研究领域'' 等最基本的问题,学者们却始终没有一个统一的看法。对于一门发展中的、涉及内容广泛的学科而言这是正常的,但长期下去,又会对学科的发展产生不利影响。
%\end{CJK*}

Chinesische Wörter können aus einem oder mehreren Symbolen bestehen.\\
Texte werden von oben nach unten geschrieben.\\
Auf Computern von links nach rechts.\\
Es gibt keine Leerzeichen zwischen Wörtern.

}



\frame{
\frametitle{Wörter sind durch Leerzeichen abgetrennt}

\begin{itemize}
\item Problem 3: Sprachen ohne Schriftsystem

Es gibt Sprachen, für die noch kein Schriftsystem erarbeitet wurde.

\pause

\item Problem 4: die Rechtschreibreform

Hat sich im Deutschen der Wortstatus bestimmter Buchstabenfolgen in den letzten Jahren mehrmals geändert?

\pause

Nein! Die Schriftsprache ist sekundär. 

\pause
Im besten Fall wurde das Schriftsystem von fähigen Linguisten entwickelt.
\pause

Im schlechtesten Fall spiegelt es verschiedene Stufen der historischen Entwicklung einer Sprache und diverse
Kompromisse von normierenden Institutionen wider.
\end{itemize}

}


\subsubsection{Die phonetisch-phonologische Ebene}

\frame{
\frametitle{Die phonetisch-phonologische Ebene}

Wörter sind kleinste, durch Wortakzent und Grenzsignale wie Pause, Knacklaut u.\,a.\ theoretisch
isolierbare Lautsegmente.


Das funktioniert nicht immer, da wir ohne "`Punkt und Komma"' reden.

In manchen Sprachen gibt es Phänomene wie Vokalharmonie,\\
die einen Rückschluss auf das Wortende erlauben.

}

\subsubsection{Die morphologische Ebene}

\frame{
\frametitle{Die morphologische Ebene}

Wörter sind als Grundeinheiten von grammatischen Paradigmen wie Flexion gekennzeichnet und zu
unterscheiden von den morphologisch charakterisierten Wortformen (\emph{schreiben} vs.\
\emph{schreibst}, \emph{schrieb}, \emph{geschrieben}).

\pause

Problem: Es gibt unflektierbare Wörter.


}


\subsubsection{Die lexikalisch-semantische Ebene}

\frame{
\frametitle{Die lexikalisch-semantische Ebene}

Wörter sind die kleinsten, relativ selbständigen Träger von Bedeutung,\\
die im Lexikon kodifiziert sind.

\pause
Problem: Unikale Elemente
\eal
\ex \blaubf{klipp} und klar
\ex auf \blaubf{Anhieb}
\zl

%\pause
%Außerdem: War das nicht die Definition für Morphem?

}


\subsubsection{Die syntaktische Ebene}

\frame{
\frametitle{Die syntaktische Ebene}

Wörter sind die kleinsten verschiebbaren und ersetzbaren Einheiten des Satzes.
\pause

Ist \emph{anfangen} ein Wort oder zwei?
\eal
\ex weil nächste Woche die Schule anfängt
\ex Nächste Woche fängt die Schule an.
\zl


}


\subsubsection{Ein Ausweg?}

\frame{
\frametitle{Ein Ausweg?}


Ein Ausweg besteht darin, das Wort \emph{Wort} an den Stellen nicht mehr zu verwenden, an denen
Mißverständnisse aufkommen könnten.

Statt dessen \blaubf{Morphem}, \blaubf{Lexem} und \blaubf{Wortform}.


}

\subsubsection{Lexem}

\frame{
\frametitle{Lexem}


\blaubf{Lexeme} sind die lexikalischen Einheiten der Sprache.

Lexeme können (je nach Wortart) ein Paradigma bilden:

\eal
\ex lach-: lache, lachst, lacht, lachen, lacht, lachen, lachte, \ldots
\ex Mann-: \begin{tabular}[t]{@{}l@{~}l@{~}l@{~}l@{}}
           Mann, & Mannes, & Mann(e), & Mann\\
           Männer, & Männer, & Männern, & Männer\\
           \end{tabular}
\zl

\pause

Ein \blaubf{Lemma} ist eine (möglichst sinnvolle) Bezeichnung für ein Lexem:\\
\emph{lachen} für (\mex{0}a), \dash Infinitivform bei Verben\\
\emph{Mann} für (\mex{0}b), \dash Nominativ Singular bei Nomen

\pause

Komplexe Einheiten wie (\mex{1})  werden als \blaubf{Mehrwortlexeme} bezeichnet.

\eal
\ex klipp und klar
\ex ins Gras beiß-
\zl


}

\subsubsection{Wortform}

\frame{
\frametitle{Wortform}

Die verschiedenen Formen, die zum Paradigma eines Lexems gehören, werden \blaubf{Wortformen} genannt.

}



\subsubsection{Morphem}

\frame{
\frametitle{Morphem (klassische Definition)}


Ein \blaubf{Morphem} ist die kleinste, nicht mehr reduzierbare bedeutungstragende sprachliche Einheit.

Lexeme sind lexikalische Morpheme im Gegensatz zu (nur) grammatikalischen Morphemen, wie \zb Flexionsmorphemen.



}


\frame{
\frametitle{Morphem (revidierte Definition)}


 Ein \blaubf{Morphem} ist die kleinste, in ihren verschiedenen Vorkommen
 als formal einheitlich identifizierbare Folge von Segmenten, der
 (wenigstens) eine als einheitlich identifizierbare
 außerphonologische Eigenschaft zugeordnet ist. (Wurzel 1984:38)

\pause

 Bedeutung ist eine außerphonologische Eigenschaft

\begin{tabular}{@{}ll@{}}
  Pluralbildung:         & \suffix{er}\\
  `wie ein':             & \suffix{lich}\\
\end{tabular}

\pause

 Andere grammatische Merkmale werden ebenfalls morphologisch
 ausgedrückt:

\begin{tabular}{@{}ll@{}}
  Infinitivbildung:     &  \suffix{en}
\end{tabular}

}

\subsubsection{Allomorphe}

\frame{
\frametitle{Morpheme und Allomorphe}


Mitunter gibt es zu einem Morphem mehrere Morphe:

\begin{tabular}{lllll}
Morphem & Morph & Morph & Morph & Morph\\
{\sc Tee} & $<$tee$>$\\
{\sc suppe} & $<$suppe$>$ & $<$süpp$>$ \\
           &        & wie in \emph{Süpp-chen}\\
{\sc Brot} & $<$brot$>$ & $<$bröt$>$\\
           &        & wie in \emph{Bröt-chen}\\
-{\sc chen} & $<$chen$>$\\
{\sc Plural} & $<$e$>$  & $<$en$>$ & $<$er$>$ & \ldots\\ 

\end{tabular}

Diese werden auch \blaubf{Allomorphe} genannt.

Man kann so vom Plural-Morphem reden,\\
obwohl es viele verschiedene Realisierungsmöglichkeiten gibt.

}


\subsubsection{Suppletion}

\frame{
\frametitle{Suppletion}

\eal
\ex schön -- schöner -- am schönsten
\ex gut -- besser -- am besten
\zl

Sind \emph{gut}, \emph{bess} und \emph{be} Allomorphe desselben Morphems?

\pause

\emph{gut}, \emph{besser}, \emph{am besten} und \emph{sein}, \emph{bin}, \emph{ist}, \emph{war} sind
historisch zu  erklären:\\
Zwei oder mehrere Flexionsparadigmen sind zusammengefallen.

\pause

Solche Muster sind als Ausnahmen zu behandeln.


}

\subsubsection{Freie und gebundene Morpheme, Affixe}

\frame{
\frametitle{Freie und gebundene Morpheme, Affixe}

Morpheme, die durch mindestens ein Morph realisiert werden,\\
das auch alleine vorkommen kann, nennt man \blaubf{freie Morpheme}.

\pause
Morpheme, die nur durch Morphe realisiert werden, die nicht alleine vorkommen können, nennt man
\blaubf{gebundene Morpheme} oder \blaubf{Affixe}.

Beispiel: -{\sc chen}.


}


\subsubsection{Affixe}

\frame{
\frametitle{Affixe}

\begin{itemize}
\item Affixe, die vor anderen Morphemen stehen, heißen \blaubf{Präfixe}.

Beispiel: {\sc ver}-

\pause
\item Affixe, die nach anderen Morphemen stehen, heißen \blaubf{Suffixe}.

Beispiel: -{\sc chen}

\pause

\item Affixe, die andere Morpheme einschließen, heißen \blaubf{Zirkumfixe}.

Beispiel: {\sc ge}- -{\sc e} in \emph{Gerenne}.

\end{itemize}


}


\subsubsection{Stamm}

%\author{Stefan Müller}
\frame{
\frametitle{Stamm}

% Todo: Stimmt nicht
%Für jedes Morphem gibt es ein Allomorph,\\
%das in der Wortbildung hauptsächlich verwendet wird.


Morpheme (\emph{schön}) oder Morphemkonstruktionen (\emph{un-schön}, \emph{Schön-heit}),\\
an die Flexionsendungen treten können, werden \blaubf{Stamm} genannt.


Nomina: identisch mit dem Nominativ Singular: \emph{Baum}, \emph{Katze}, \emph{Kind}

Adjektive: prädikative Form: \emph{blau}, \emph{schlau}, \emph{genau}

Verben: Infinitivform ohne Infinitivendung: \emph{lauf}, \emph{sing}

\pause

Stämme, die nicht zerlegt werden können, heißen \blaubf{Wurzel}.


}



\subsubsection{Simplizia und komplexe Lexeme}


\frame{
\frametitle{Simplizia und komplexe Lexeme}

\begin{itemize}
\item Lexeme, die nur aus einem Allomorph eines freien Morphems bestehen, nennt man
  \blaubf{Simplizia}.

Diese sind für die Morphologie uninteressant,\\
da sie nicht zerlegt werden können.

\pause
\item Komplexe Lexeme werden durch Anwendung eines Prozesses/einer Regel auf ein Grundmorphem
  erzeugt.

\pause
\item Einfachster Prozess ist Aneinanderhängen (Konkatenation).

\pause
\item Im Deutschen zwei konkatenative Wortbildungsprozesse:\\
      Komposition und Derivation
\end{itemize}

}
