%%%%%%%%%%%%%%%%%%%%%%%%
%%     PACKAGES       %%
%%%%%%%%%%%%%%%%%%%%%%%%



%\usepackage[utf8]{inputenc}
%\usepackage[vietnamese, english,ngerman]{babel}   % seems incompatible with german.sty
%\usepackage[T3,T1]{fontenc} breaks xelatex

\usepackage{lmodern}

\usepackage{amsmath}
\usepackage{amsfonts}
\usepackage{amssymb}
%% MnSymbol: Mathematische Klammern und Symbole (Inkompatibel mit ams-Packages!)
%% Bedeutungs- und Graphemklammern: $\lsem$ Tisch $\rsem$ $\langle TEXT \rangle$ $\llangle$ TEXT $\rrangle$ 
\usepackage{MnSymbol}
%% ulem: Strike out
\usepackage[normalem]{ulem}  

%% Special Spaces (s. Commands)
\usepackage{xspace}				
\usepackage{setspace}
%	\onehalfspacing

%% mdwlist: Special lists
\usepackage{mdwlist}	

\usepackage[
%noenc,
safe]{tipa}

%\DeclareFontSubstitution{T3}{cmss}{m}{n}

%\DeclareFontSubstitution{T3}{ptm}{m}{n}

%Das hilft nicht, Vorsicht bei dem "U"
%\usepackage{tipx}



%\usepackage{vowel}

% maybe define \textipa to use \originalTeX to avoid problems with `"'.
%
%	\ex \textipa{\originalTeX [pa.pa."g\t{aI}]}

%

\usepackage{jambox}



%\usepackage{forest-v105}
%\usepackage{langsci-forest-v105-setup}

\usepackage{xeCJK}
\setCJKmainfont{SimSun}


%\usepackage{natbib}
%\setcitestyle{notesep={:~}}




% for toggles, is loaded in hu-beamer-includes-pdflatex
%\usepackage{etex}



% Fraktur!
\usepackage{yfonts}

\usepackage{url}

% für UDOP
\usepackage{adjustbox}


%% huberlin: Style sheet
%\usepackage{huberlin}
\usepackage{hu-beamer-includes-pdflatex}
\huberlinlogon{0.86cm}

% %% % use this definition, if you want to see the outlines in the handout
\renewcommand{\outline}[1]{%
%\beamertemplateemptyfootbar%
\huberlinjustbarfootline
\frame{\frametitle{\outlineheading}#1}%
%\beamertemplatecopyrightfootframenumber%
\huberlinnormalfootline 
\huberlinpagedec
}



%% Last Packages
%\usepackage{hyperref}	%URLs
%\usepackage{gb4e}		%Linguistic examples

% sorry this was incompatible with gb4e and had to go.
%\usepackage{linguex-cgloss}	%Linguistic examples (patched version that works with jambox

\usepackage{multirow}  %Mehrere Zeilen in einer Tabelle
\usepackage{adjustbox} %adjusting tables
%\usepackage{array}
\usepackage{marginnote}	%Notizen


