%%%%%%%%%%%%%%%%%%%%%%%%%%%%%%%%%%%%%%%%%%%%%%%%
%% Compile the master file!
%% 		Slides: Antonio Machicao y Priemer
%% 		Course: GK Linguistik
%%%%%%%%%%%%%%%%%%%%%%%%%%%%%%%%%%%%%%%%%%%%%%%%


%%%%%%%%%%%%%%%%%%%%%%%%%%%%%%%%%%%%%%%%%%%%%%%%%%%%
%%%             Metadata                         
%%%%%%%%%%%%%%%%%%%%%%%%%%%%%%%%%%%%%%%%%%%%%%%%%%%%      

\title{Grundkurs Linguistik}

\subtitle{Formelles}

\author[aMyP]{
	{\small Antonio Machicao y Priemer}
	\\
	{\footnotesize \url{www.linguistik.hu-berlin.de/staff/amyp}}
%	\\
%	{\footnotesize \href{mailto:mapriema@hu-berlin.de}{mapriema@hu-berlin.de}}
}

\institute{Institut für deutsche Sprache und Linguistik}

\date{ }

%\publishers{\textbf{6. linguistischer Methodenworkshop \\ Humboldt-Universität zu Berlin}}

%\hyphenation{nobreak}


%%%%%%%%%%%%%%%%%%%%%%%%%%%%%%%%%%%%%%%%%%%%%%%%%%%%
%%%             Preamble's End                   
%%%%%%%%%%%%%%%%%%%%%%%%%%%%%%%%%%%%%%%%%%%%%%%%%%%%      


%%%%%%%%%%%%%%%%%%%%%%%%%      
\huberlintitlepage[22pt]

\iftoggle{toc}{
	\frame{
%		\begin{multicols}{2}
			\frametitle{Inhaltsverzeichnis}\tableofcontents
			%[pausesections]
%		\end{multicols}
	}
}


%%%%%%%%%%%%%%%%%%%%%%%%%%%%%%%%%%
%%%%%%%%%%%%%%%%%%%%%%%%%%%%%%%%%%
%%%%%LITERATURE:

%%% Allgemein
%\nocite{Glueck&Roedel16a}
%\nocite{Luedeling2009}
%\nocite{Meibauer&Co07a} 
%\nocite{Repp&Co15a} 

%% Morphologie
%\nocite{Eisenberg04}

%% Syntax
%\nocite{Adger04a}
%\nocite{Altmann&Hofmann08a} % Satztypen & Satzmodi
%\nocite{Altmann93a} % Satztypen & Satzmodi
%\nocite{Brandt&Co06a} 
%\nocite{Fanselow&Sascha87a}
%\nocite{Fanselow&Sascha93a}
%\nocite{Fries&MyP16b} % Akzeptabilität
%\nocite{Fries16a} % Grammatikalität
%\nocite{Fries&MyP16d} % Kompetenz vs Performanz
%\nocite{Fries&MyP16c} % GG
%\nocite{Fries&MyP16a} % X-Bar-Theorie
%\nocite{Fries16e} % Satztyp
%\nocite{Fries16d} % Satzmodus 
%\nocite{Grewendorf&Co91a} 
%\nocite{MyP17b} % Kerngrammatik
%\nocite{MyP18a} % Konstituententest
%\nocite{MyP18b} % Kopf
%\nocite{MyP18c} % Phrase
%\nocite{MyP18s} % Funktionale Kategorie
%\nocite{MyP18t} % Argumentstruktur
%\nocite{MuellerS13f} 
%\nocite{MuellerS15b}
%\nocite{Stechow&Sternefeld88a}
%\nocite{Sternefeld06a}
%\nocite{Sternefeld06b}
%\nocite{Woellstein10a} % Topologisches Feldermodell

%% Semantik & Pragmatik
%\nocite{Loebner15a} %% Semantics
%\nocite{Loebner15b} %% Semantics
%\nocite{Lohnstein11} %% Semantics
%\nocite{MyP16a} %% Bikonditional
%\nocite{Partee&Co93a} %% Semantics
%\nocite{ZimmermannT&Sternefeld13a} %% Semantics

	
%%%%%%%%%%%%%%%%%%%%%%%%%%%%%%%%%%%
%%%%%%%%%%%%%%%%%%%%%%%%%%%%%%%%%%%
%
\section{Kontakt}
\iftoggle{sectoc}{
\frame{
\begin{multicols}{2}
\frametitle{~}
	\tableofcontents[currentsection]
\end{multicols}
}
}
%%%%%%%%%%%%%%%%%%%%%%%%%%%%%%%%%%%

\begin{frame}{Kontakt}

\begin{itemize}
	\item \textbf{Dozent:} Antonio Machicao y Priemer \textipa{[ma.\t{tS}i.\textprimstress ka.o \textprimstress Pi \textprimstress pKi:.m5]}
	
	\item \textbf{Büro:} Dorotheenstraße 24, Raum: 3.305
	
	\item \textbf{Telefon:} +49 (30) 2093-9702
	
	\item \textbf{Webseite:} \url{www.linguistik.hu-berlin.de/staff/amyp}
	
	\item \textbf{E-Mail:} \href{mailto:mapriema@hu-berlin.de}{mapriema@hu-berlin.de}
	\item[]
	
	\item \textbf{Sprechstunde}: Mo. 10--12h (Anmeldung per E-Mail erforderlich!)
	\item[] \textbf{Keine Sprechstunden}: %14.01.--20.01.
\end{itemize}	

\end{frame}


%%%%%%%%%%%%%%%%%%%%%%%%%%%%%%%%%%%
%%%%%%%%%%%%%%%%%%%%%%%%%%%%%%%%%%%
\section{Sekretariat}
 \iftoggle{sectoc}{
 \frame{
 \begin{multicols}{2}
 \frametitle{~}
 	\tableofcontents[currentsection]
 \end{multicols}
 }
}
%%%%%%%%%%%%%%%%%%%%%%%%%%%%%%%%%%%

\begin{frame}{Sekretariat}
	
\begin{itemize}
	\item[] \textbf{Anina Klein}	
	\item \textbf{Büro:} Dorotheenstraße 24, Raum: 3.306
	\item \textbf{Telefon:} +49 (30) 2093-9639
	\item \textbf{E-Mail:} \href{mailto:Anina.Klein@cms.hu-berlin.de}{Anina.Klein@cms.hu-berlin.de}
\end{itemize}	

\end{frame}


%%%%%%%%%%%%%%%%%%%%%%%%%%%%%%%%%%%
%%%%%%%%%%%%%%%%%%%%%%%%%%%%%%%%%%%
\section{Moodle}	
\iftoggle{sectoc}{
\frame{
\begin{multicols}{2}
\frametitle{~}
	\tableofcontents[currentsection]
\end{multicols}
}
}
%%%%%%%%%%%%%%%%%%%%%%%%%%%%%%%%%%%

\begin{frame}{Moodle}

\begin{itemize}
	\item Alle \textbf{Folien} und \textbf{Materialien} werden über Moodle zur Verfügung gestellt.
	\item[]
	\item Wichtige \textbf{Hinweise} (Ausfälle, etc.) werden immer über Moodle bekannt gegeben.
	\item[]
	\item \textbf{Moodleseite des Kurses:} \url{https://moodle.hu-berlin.de/course/view.php?id=83975}\\
	\textbf{Moodleschlüssel:} Performanz
\end{itemize}		

\end{frame}


%%%%%%%%%%%%%%%%%%%%%%%%%%%%%%%%%%%
%%%%%%%%%%%%%%%%%%%%%%%%%%%%%%%%%%%
\section{Tutorien}
\iftoggle{sectoc}{
\frame{
\begin{multicols}{2}
\frametitle{~}
	\tableofcontents[currentsection]
\end{multicols}
}
}
%%%%%%%%%%%%%%%%%%%%%%%%%%%%%%%%%%%

\begin{frame}{Tutorien}

	\begin{itemize}
		\item \textbf{Online-Tutorium Linguistik} 
		\begin{itemize}
			\item Fragen mit automatischer Korrektur (über Moodle)!
			\item \textbf{Moodle:} \url{https://moodle.hu-berlin.de/course/view.php?id=38846}\\
			\textbf{Moodleschlüssel:} tutonline
		\end{itemize}
		

		\item[]
		\item \textbf{Präsenztutorien}
		
		\begin{itemize}
			\item \textbf{Lehrende:} Mareike Lisker, Henrike Prochno, Pia Linscheid
			\item \textbf{Raum:} 0.01 (SO22) $|$ Zeit: Mo. 18--20 oder Do. 08--10
			\item \textbf{Raum:} 1.102 (DOR24) $|$ Zeit: Di. 18--20 oder Mi. 08--10
			\item \textbf{Moodle:} \url{https://moodle.hu-berlin.de/course/view.php?id=85175}\\
				\textbf{Moodleschlüssel:} Suffix
			\item Die Präsenztutorien fangen erst \textbf{in der zweiten Woche} an!
		\end{itemize}
		
	\end{itemize}
	
\end{frame}


%%%%%%%%%%%%%%%%%%%%%%%%%%%%%%%%%%%
%%%%%%%%%%%%%%%%%%%%%%%%%%%%%%%%%%%
\section{Zu erbringende Leistungen}
\iftoggle{sectoc}{
\frame{
\begin{multicols}{2}
\frametitle{~}
	\tableofcontents[currentsection]
\end{multicols}
}
}
%%%%%%%%%%%%%%%%%%%%%%%%%%%%%%%%%%%

\begin{frame}{Zu erbringende Leistungen}

	\begin{itemize}
	\item[] Regelmäßige und \textbf{aktive!} Teilnahme (45 h)
	
    \item[+] \textbf{Vor-} und \textbf{Nachbereitung} 105 h (17 * 6 h 11 min)
	
	\item[+] Abgabe von 10 (aus 12) \textbf{Übungsaufgaben }
	
	(Abgabe ist \textbf{nur über Moodle} möglich)
	
	\item[+] \textbf{Erklärung} von 1--2 Hausaufgaben an der Tafel 
	
	\item[=] Voraussetzung für die \textbf{MAP-Zulassung}

\pause

	\item[+] \textbf{Modulabschlussprüfung} \ras GK Linguistik + UE Deutsche Grammatik

	Achtung: Klausurergebnis taucht bei allen auf dem Zeugnis auf!

	\item[]
	\item Klausurtermin: 20.02.2019, 14:00--16:00
              
	\end{itemize}
	
\end{frame}


%%%%%%%%%%%%%%%%%%%%%%%%%%%%%%%%%%%

\begin{frame}{Übungsaufgaben}

\begin{itemize}
	\item \textbf{Abgabedatum} der Hausaufgaben \ras Semesterplan
	
	\item Die Hausaufgaben werden \textbf{nur über Moodle angenommen}.
	
	 \item Sie können Ihre Hausaufgaben \textbf{in Gruppen} lösen, die Abgabe erfolgt aber \textbf{individuell}!
\end{itemize}
\end{frame}

%%%%%%%%%%%%%%%%%%%%%%%%%%%%%%%%%%%
\begin{frame}

Was wird noch erwartet\dots

\begin{itemize}
	\item Befassen Sie sich mit der angegebenen Literatur!
	
	\item Fragen und Diskussionen!
	
	\item Während des Seminars bitte kein Facebook, Twitter, Tumblr,
	Youtube, WhatsApp, Netflix, \dots
	
	\item Wenn Sie früher gehen müssen, sagen Sie mir bitte zu Beginn der
	Stunde Bescheid, und nehmen Sie bitte Platz in der Nähe der Tür.
	
\end{itemize}

\end{frame}


%%%%%%%%%%%%%%%%%%%%%%%%%%%%%%%%%%%
%%%%%%%%%%%%%%%%%%%%%%%%%%%%%%%%%%%
\section{Literatur}
\iftoggle{sectoc}{
\frame{
\begin{multicols}{2}
\frametitle{~}
	\tableofcontents[currentsection]
\end{multicols}
}
}
%%%%%%%%%%%%%%%%%%%%%%%%%%%%%%%%%%%

\begin{frame}{Literatur}

\begin{itemize}
	\item Für jede Sitzung wird die Literatur im Semesterplan (s.\ Handout bzw.\ Semesterplan in Moodle) vorausgesetzt
	
	\item Die Lektüre für jede Sitzung wird als PDF über Moodle bereitgestellt.
	
	\item Die \textbf{Lexika} \citet{Glueck&Roedel16a} und \citet{Schierholz&Co18} sind durch die Webseite der HU-Bibliothek kostenlos konsultierbar.
	
	\item Die \textbf{AM} \citep{Abramowski2016} erhalten sie kostenlos, eine PDF wird im Moodle bereitgestellt.
	
	\item Das \textbf{Einführungsbuch} von \citet{SchaeferR16a} ist OpenAccess veröffentlicht und somit kostenlos herunterladbar aus der angegebenen URL.
	
\bigskip

	\item Dieser Kurs basiert hauptsächlich auf \citet{Luedeling2009a}, \citet{Meibauer&Co07a} und \citet{Abramowski2016}.
\end{itemize}		

\end{frame}


%%%%%%%%%%%%%%%%%%%%%%%%%%%%%%%%%%%
%%%%%%%%%%%%%%%%%%%%%%%%%%%%%%%%%%%
\section{Verbesserungsvorschläge}
\iftoggle{sectoc}{
\frame{
\begin{multicols}{2}
\frametitle{~}
	\tableofcontents[currentsection]
\end{multicols}
}
}
%%%%%%%%%%%%%%%%%%%%%%%%%%%%%%%%%%%

\begin{frame}
\frametitle{Beschwerden, Verbesserungsvorschläge}

\begin{itemize}
      \item mündlich
      \item per Mail oder 
      \item anonym durch die Lehrevaluation am Ende des Jahres
\end{itemize}
\end{frame}


%%%%%%%%%%%%%%%%%%%%%%%%%%%%%%%%%%%
%%%%%%%%%%%%%%%%%%%%%%%%%%%%%%%%%%%
\section{Mailverkehr}
\iftoggle{sectoc}{
\frame{
\begin{multicols}{2}
\frametitle{~}
	\tableofcontents[currentsection]
\end{multicols}
}
}
%%%%%%%%%%%%%%%%%%%%%%%%%%%%%%%%%%%

\begin{frame}
\frametitle{Mailverkehr}

\begin{itemize}
	\item \textbf{HU-Mail-Adresse} verwenden (wegen Spam-Gefahr)
	
	\item Geben Sie Ihren \textbf{Vor- und Nachname} richtig an.
	
	\item Geben Sie an in welcher \textbf{Veranstaltung} Sie sind.

	\item Überlegen Sie zuerst, ob Ihre Kommilitonen Ihnen nicht eine Antwort auf die Frage geben können.
	
	\ras Benutzen Sie das \textbf{Forum} auf unserer Moodleseite.
	
	\item Bitte beachten Sie die gängigen Höflichkeitsregeln beim Mailverkehr.
\end{itemize}

\end{frame}
